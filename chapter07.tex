% Marieke Kuijjer
% 2013-02-15
% chapter 07

	%\documentclass[12pt,b5paper]{book}
	%\setcounter{secnumdepth}{0}
	%\setcounter{tocdepth}{1}
	%\usepackage[hidelinks]{hyperref}

%\begin{document}

%%% title page

\chapter{Identification of osteosarcoma driver genes by integrative analysis of copy number and gene expression data}\label{ch7}
\thispagestyle{empty}				%%% to remove page number from first page of chapter, must be placed after calling the chapter

\vfill

\vspace{0.5cm}
This chapter is based on the publication:
\underline{Kuijjer ML}, Rydbeck H, Kresse SH, Buddingh EP, Lid AB, Roelofs H, B{\"u}rger H, Myklebost O, Hogendoorn PCW, Meza-Zepeda LA, Cleton-Jansen AM. {\it Genes Chromosomes Cancer}. 2012 Jul;51(7):696-706

\newpage

%%% main document

\section{Abstract}\label{abstract7}
High-grade osteosarcoma is a tumor with a complex genomic profile, occurring primarily in adolescents with a second peak
at middle age. The extensive genomic alterations obscure the identification of genes driving tumorigenesis during osteosarcoma
development. To identify such driver genes, we integrated DNA copy number profiles (Affymetrix SNP 6.0) of 32 diagnostic
biopsies with 84 expression profiles (Illumina Human-6 v2.0) of high\hyp{}grade osteosarcoma as compared with its
putative progenitor cells, {\it i.e.} mesenchymal stem cells ($n=12$) or osteoblasts ($n=3$). In addition, we performed paired analyses between copy number and expression profiles of a subset of 29 patients for which both DNA and mRNA profiles were available. Integrative analyses were performed in Nexus Copy Number software and statistical language R. Paired analyses
were performed on all probes detecting significantly differentially expressed genes in corresponding {\it LIMMA} analyses. For
both nonpaired and paired analyses, copy number aberration frequency was set to $>35\%$. Nonpaired and paired integrative
analyses resulted in 45 and 101 genes, respectively, which were present in both analyses using different control sets. Paired
analyses detected $>90\%$ of all genes found with the corresponding nonpaired analyses. Remarkably, approximately twice as
many genes as found in the corresponding nonpaired analyses were detected. Affected genes were intersected with differentially expressed genes in osteosarcoma cell lines, resulting in 31 new osteosarcoma driver genes. Cell division related genes, such as {\it MCM4} and {\it LATS2}, were overrepresented and genomic instability was predictive for metastasis\hyp{}free survival, suggesting that deregulation of the cell cycle is a driver of osteosarcomagenesis.

\section{Introduction}\label{introduction7}
High-grade osteosarcoma is an aggressive primary
bone tumor, which mostly occurs during
adolescence, with a second peak at middle age,
at the metaphysis of long bones. The tumor is
characterized by aberrant production of osteoid
matrix and by very complex karyotypes~\cite{raymond2002conventional,cleton2005central}.
Since the introduction of DNA microarray technology,
recurrent DNA copy number changes in
human osteosarcoma tumor tissues have been
identified by comparative genomic hybridization
(CGH) and high\hyp{}density single nucleotide polymorphisms
(SNP) microarray analysis. There is a
general consensus about gain of chromosome
arms 6p, 8q, and 17p, but many additional regions
are reported as well~\cite{squire2003high,man2004genome,
atiye2005gene,yen2009identification,kresse2010evaluation}. The effects of copy number
alterations may be reflected by changes in expression
of genes in the affected chromosomal
regions. There are various publications on human
osteosarcoma gene expression, but few show robust
bioinformatics (as described by Kuijjer {\it et al}.~\cite{kuijjer2011mrna}). Often, small sample sizes and heterogeneity
within groups result in only a small number of
significant genes, on which usually no correction
for multiple testing is applied. Another problem
when studying osteosarcoma gene expression data
is the lack of an osteosarcoma benign precursor
lesion and its debated cell of origin---although it
becomes clearer that the mesenchymal stem cell
or its derivative is the progenitor of osteosarcoma~\cite{mohseny2009osteosarcoma,mohseny2011concise}. The disease seems to develop suddenly as
a full\hyp{}blown tumor, rendering it difficult to detect
early drivers of osteosarcomagenesis. We have
previously determined differential expression
related to specific clinical parameters~\cite{buddingh2011tumor,kuijjer2011mrna}. In addition, we
have compared osteosarcoma with osteoblastoma---a
benign tumor which develops at the
same site as osteosarcoma, but which does not
progress into the latter. This comparison of
human osteosarcoma with a control tissue showed
that cell cycle regulation is the most significantly
altered pathway in osteosarcoma~\cite{cleton2009profiling}.

There are advantages of integrating copy number
and expression data when aiming to identify
driver genes. First, copy number data analysis of
tumors with complex genomic profiles may return
numerous bystander or hitch\hyp{}hiker genes, as copy
number alterations may occur not only because
they are advantageous for the tumor but also as a
result of general genomic instability. Regions of
copy number alteration may therefore encompass
no driver gene at all, or may include additional
genes. Also, some genes with altered copy numbers
may not be expressed in the tumor due to
tissue\hyp{}specific expression. These aspects hamper
the identification of drivers of tumorigenesis,
especially when the number of recurrent genes in
such tumors is high. Second, at the mRNA level,
drivers affect downstream genes and switch on
feedback mechanisms, again rendering it difficult
to determine the real osteosarcoma drivers in a
pool of differentially expressed genes~\cite{lee2008integrative}. Integration of DNA copy number and
gene expression data filters out at least part of
such bystanders and of genes that act downstream
of drivers of tumorigenesis, because most
of these genes have altered copy numbers, but no
change in expression, or vice versa, while drivers
are both amplified and upregulated, or deleted
and downregulated. Particularly osteosarcoma is
genetically extremely unstable and therefore
genomic data analysis of this tumor type would
benefit from an approach that distinguishes driver
genes from the numerous more random genetic
events.

Nonpaired integrative analysis may be performed
by determining recurrent regions of copy
number alterations which have higher than
expected numbers of differentially expressed
genes. Paired integrative analysis is a more powerful
method, because the relationship between copy
number alterations and gene expression can be
inferred in each specific sample, instead of being
based on averaged quantities. A statistically correct
method for paired integrative analysis of these different
data types has not yet been defined. Paired
integrative analysis is usually performed by selecting
genes based on the correlation between gene
expression and copy number levels, such as is performed
by the recently published methods DR-Integrator~\cite{salari2010dr}
and Regularized dual
Canonical Correlation Analysis~\cite{soneson2010integrative}.
However, gains and losses may not necessarily
directly translate to the same quantity of change
in expression levels~\cite{lee2008integrative}, and important
genes may be overlooked this way. A method
where paired integrative analysis is detected for
specific chromosomal regions with altered genomic
and transcriptional status does exist~\cite{bicciato2009computational},
but this method is not optimal for tumors
such as osteosarcoma with highly unstable
genomes, since copy number values are normalized
to the mean copy number over each array, and this
mean value may be altered in such tumors. Two
methods, PARADIGM and CNAmet, combine different
types of data on a gene\hyp{}based level. In
PARADIGM, integration of different data types is
used to detect patient\hyp{}specific pathway activities~\cite{vaske2010inference}. CNAmet returns genes that
show differential expression between samples with
and without methylation and/or copy number
alteration~\cite{louhimo2011cnamet}. This
elegant approach may however hamper the identification
of genes that are regulated by other frequently
altered genes, such as {\it TP53} and {\it MDM2} in
osteosarcoma.

Aiming to identify osteosarcoma driver genes,
we performed both nonpaired and paired integrative
analyses on high\hyp{}grade osteosarcoma prechemotherapy
biopsy data. We combined results
from analyses as compared with different control
sets---mesenchymal stem cells (MSCs) and osteoblasts,
so that we did not exclude one of these
proposed progenitors as the cell of origin of osteosarcoma.
We show that the paired integrative
analysis returns more affected genes than the
nonpaired integrative analysis. There is an overrepresentation
of genes involved in genomic stability
in osteosarcoma samples. The identified
genes may be important drivers in
osteosarcomagenesis.

\section{Materials and methods}\label{methods7}
\subsection{Ethics statement}
All biological material was handled in a coded
fashion. Ethical guidelines of the individual European
partner institutions were followed and samples
and clinical data were handled in a coded
fashion and stored in the EuroBoNeT biobank.

\subsection{Patient material and cell lines}
Genome\hyp{}wide expression profiling was performed
on pretreatment diagnostic biopsies of 84
resectable high\hyp{}grade osteosarcoma patients from
the EuroBoNeT consortium (\url{www.eurobonet.eu}).
Clinicopathological details of these samples can
be found in Table~\ref{tab7.1}.
%
\begin{table}[htbp]
	\centering
	\small
		\begin{tabular}[c]{|lcc|}
		\hline
		Category & Patient characteristics & Number of biopsies (\%)\\
		\hline
		Institution & LUMC, Netherlands & 36 (42.9)\\
		& IOR, Italy & 12 (14.3)\\
		& LOH, Sweden & 3 (3.6)\\
		& Radiumhospitalet, Norway & 1 (1.2)\\
		& WWUM, Germany & 32 (38.1)\\
		Location primary tumor & Femur & 40 (47.6)\\
		& Tibia/Fibula & 28 (33.3)\\
		& Humerus & 11 (13.1)\\
		& Axial skeleton & 1 (1.2)\\
		& Unknown/other & 4 (4.8)\\
		Histological subtype & Osteoblastic & 52 (61.9)\\
		& Chondroblastic & 9 (10.7)\\
		& Fibroblastic & 7 (8.3)\\
		& Telangiectatic & 4 (4.8)\\
		& Minor subtype & 11 (13.1)\\
		& Unknown & 1 (1.2)\\
		Huvos grade & 1 or 2 & 38 (45.2)\\
		& 3 or 4 & 33 (39.3)\\
		& Unknown/NA & 14 (16.7)\\
		Metastasis at diagnosis & Yes & 14 (16.7)\\
		& No & 69 (82.1)\\
		& Unknown & 1 (1.2)\\
		Sex & Male & 54 (64.3)\\
		& Female & 29 (34.5)\\
		& Unknown & 1 (1.2)\\
		Age & $<20$ years & 64 (76.2)\\
		& $>=20$ years & 19 (22.6)\\
		& Unknown & 1 (1.2)\\
		\hline
		\end{tabular}
\caption{Clinicopathological details.}
	\label{tab7.1}  
\end{table}
%
Human bone\hyp{}marrow\hyp{}derived
MSCs were obtained from five osteosarcoma
patients and seven healthy donors. Osteoblasts
($n=3$) were derived from MSCs on
osteogenic differentiation. MSCs and osteoblasts
were characterized and handled as described~\cite{cleton2009profiling}. Copy number analysis
was performed on 32 pretreatment diagnostic
biopsies, of which 29 overlapped with the 84
samples used for expression analysis.

\subsection{Copy number microarray data analysis}
Affymetrix Genome\hyp{}Wide Human SNP 6.0
arrays (Affymetrix, Santa Clara, CA) were used
for SNP data analysis. Genomic DNA preparation,
labeling, hybridization, and scanning were
performed as described by Kresse {\it et al}.~\cite{kresse2010evaluation}.
Microarray data preprocessing was performed as
described previously~\cite{pansuriya2011genome}.
Hybridization quality was estimated by the genotype
call rate using the Birdseed genotype calling
algorithm in Genotyping Console (version 4.0,
Affymetrix). Samples of poor quality were
excluded from further analyses. We performed
copy number analysis in Nexus software version
5 (Biodiscovery, El Segundo, CA) using CNCHP
log ratio files generated by Genotyping Console
using 27 controls as a baseline, which is a subset
of the reference set of 29 samples which was
used by Pansuriya {\it et al}.~\cite{pansuriya2011genome}. We rejected two
samples based on results from the quality control
analysis in Genotyping Console. Circular Binary Segmentation (CBS)\hyp{}based SNPRank Segmentation
was used to identify aberrant genomic
regions. To be included as frequently aberrant, a
copy number alteration was called when detected
in at least 35\% of all cases. Correlation of copy
number alterations with clinical data was performed
in Nexus software, with correction for
multiple testing.

\subsection{Genome\hyp{}wide gene expression microarray data analysis}
Osteosarcoma tissue handling, RNA isolation,
synthesis of cDNA, cRNA amplification, hybridization
of cRNA onto the Illumina Human-6 v2.0
Expression BeadChips (Illumina, San Diego,
CA), and microarray data processing and quality
control in the statistical language R version 2.10~\cite{r2.10.0}
were performed
as described previously~\cite{buddingh2011tumor}.
High correlations between these microarray
data and corresponding qPCR results have
been demonstrated previously~\cite{buddingh2011tumor}.
Unsupervised hierarchical cluster analysis
was performed using R package pvclust with
default settings~\cite{suzuki2006pvclust}.

\subsection{Data deposition}
MIAME\hyp{}compliant copy number and gene
expression data have been deposited in the GEO
database (\url{www.ncbi.nlm.nih.gov/geo/}, superseries
accession number GSE33383).

\subsection{Detection of significantly differentially expressed genes}
We performed a {\it LIMMA} analysis~\cite{smyth2004linear}
in order to determine differential expression
between high\hyp{}grade osteosarcoma samples ($n=84$)
and control tissues---MSCs ($n=12$) and osteoblasts
($n=3$). Also, gene expression differences between
MSCs and osteoblasts were determined. We used a
Benjamini and Hochberg False Discovery Rate
(FDR) of 0.05 as cut-off for significance.

\subsection{Nonpaired integrative analysis}
Nonpaired integrative analysis was performed
by importing lists of differentially expressed
genes into the Copy Number module of Nexus
software version 5. Based on the length of the
gene list, Nexus software performs a Fisher's
exact test in order to determine whether the
number of differentially expressed genes in a
specific region with a significant copy number
alteration is larger than expected by chance.
Genes present in such regions of copy number
alteration with FDR\hyp{}adjusted p-values (Q-bounds
in Nexus software) $<0.05$ were returned from
this integrative analysis. We did not apply any
restrictions on the size of copy number aberrations.
A few small altered regions that did not
encompass an entire gene were detected, but
these regions did not return genes upon integration
with expression data. Nexus software only
reports genes which are both gained and overexpressed,
or both deleted and downregulated.

\subsection{Paired integrative analysis}
For the paired integrative analysis, copy number
data of all autosomal overlapping genes
between the copy number and gene expression
data were exported from Nexus software, and
converted into a binary matrix containing all
genes with a gain (1) and no gain (0), and a similar
binary matrix for losses. As in the nonpaired
integrative analysis, we did not apply any restrictions
on the size of copy number alterations.
Gene expression data of each probe for each sample
were normalized against average gene expression
values of the corresponding probes over all
control samples (either expression data from 12
MSCs or from three osteoblasts)---this was performed
by subtracting the average expression of
the control samples from the expression levels of
the sample of interest, since these are log\hyp{}transformed
expression values. For both analyses, only
genes that were significantly differentially
expressed between the 84 osteosarcoma samples
and the specific control set were analyzed, in
order to make sure that all genes returned from
the integrative analysis were significantly differentially
expressed. Subsequently, genes that overlapped
between the copy number binary matrices
and that matched the fold change of expression
(upregulation for genes with gains, and downregulation
for genes with losses) were returned as
two-way contingency tables using scripts in R.
Genes that were altered in two types of data
were further filtered by applying the sample recurrence
criterion of 35\%. This generated lists of
recurrent two-way altered genes. The odds ratios
for having both copy number and expression
changes were calculated for different combinations,
for instance gain and upregulation. We
used Bonferroni corrected Chi-square or Fisher's
exact p-values $<0.05$ to determine significance.

\subsection{Gene set enrichment}
GO term enrichment was tested using Bioconductor
package {\it topGO}~\cite{alexa2006improved}. Lists
of significantly affected genes were compared
with all genes eligible for the analysis. GO terms
with Fisher's exact p-values $<0.0001$, as calculated
by the {\it weight01} algorithm from {\it topGO}, were
defined significant.

\subsection{Genomic instability scores and survival analysis}
We calculated genomic instability scores for 83
(out of the 84) osteosarcoma biopsies (for one sample
no follow\hyp{}up data were available) and all controls,
as well as for two normal bone samples
(obtained from cancer patients at the Norwegian
Radium Hospital), 20 osteosarcoma xenografts
(Kresse {\it et al}., {\it unpublished data}, and Kuijjer {\it et al}.~\cite{kuijjer2011mrna}), 19 osteosarcoma cell lines~\cite{ottaviano2010molecular},
and the HeLa cervical cancer cell line. For
calculation of the genomic instability scores, we
refer to the article by Carter {\it et al}.~\cite{carter2006signature}. In short,
this method calculates per sample per probe the
expression of that particular probe minus the
mean expression of that probe over all samples.
For each sample, the sum of these values for all
probes present in the genomic instability signature
is calculated. This value is then compared
between all samples and thus gives a relative measure
of genomic instability. We used 24 genes of
the CIN25 signature, because for one gene no
probe was present on the Illumina v2.0 BeadChip.
For genes with multiple probes, we used the
probe that showed the highest variation in expression
levels. We determined metastasis\hyp{}free survival
using the Kaplan\hyp{}Meier method and performed a
Logrank test for trend using GraphPad Software
(La Jolla, CA, www.graphpad.com).

\section{Results}\label{results7}
\subsection{Recurrent chromosomal regions with copy number aberrations in high\hyp{}grade osteosarcoma}
Thirty two high\hyp{}grade osteosarcoma prechemotherapy
biopsies were hybridized to Af\-fy\-met\-rix
SNP 6.0 arrays in order to determine recurrent
copy number alterations. In total, 67 regions with
recurrent alterations were detected, of which 35
regions had copy number gain, and 32 copy number
loss (see Supporting Information Table 1 ({\it available online}~\cite{ch7additional})).
Recurrent gains were present on chromosome
arms 1p, 1q, 4p, 5p, 6p, and 8q, and losses on
chromosome arms 1p, 1q, 2q, 3q, 6q, 7q, 8p, 10p,
10q, 12p, 13q, 15q, 16p, and 16q. A genome\hyp{}wide
frequency plot of copy number alterations is
shown in Figure~\ref{fig7.1}.
%
\begin{figure}[htbp]
	\centering
	\includegraphics[width=1.0\textwidth]{figs07/fig1bw.pdf}	% OBS! print version bw
%	\includegraphics[width=1.0\textwidth]{figs07/fig1rgb.pdf}	% OBS! pdf version rgb
	\caption{Genome\hyp{}wide frequency plot of copy number alterations on chromosomes 1--22 in 32 high\hyp{}grade osteosarcoma prechemotherapy biopsies. Left of the chromosomes: loss, right: gain.}
	\label{fig7.1}
\end{figure}
%
No significant correlation was
detected for specific regions with copy number
alterations and clinical information (tested clinical
parameters are shown in Table~\ref{tab7.1}).

\subsection{Comparison of osteoblasts and MSCs}
Unsupervised hierarchical cluster analysis
resulted in separate clusters for biopsies and cell
lines. Within the cell line cluster, osteosarcoma
cell lines formed one subcluster, whereas MSCs
and osteoblasts formed a second subcluster (Supporting
Information Figure 1 ({\it available online}~\cite{ch7additional})). This indicates that
the control cell lines are more similar to one
another than to osteosarcoma cells. On the basis
of hierarchical clustering of gene expression data,
we cannot determine the cell of origin of osteosarcoma.
A total of $1,382$ genes were differentially
expressed between osteoblasts and MSCs. GO
term enrichment resulted in seven significant GO
terms, which are represented in Supporting Information
Figure 2 ({\it available online}~\cite{ch7additional}). In summary, GO term enrichment
showed differences in cellular structure,
proliferation, and apoptosis. Genes showing significant
differences between both control cell
types, however, can nonetheless be differentially
expressed between osteosarcoma samples and
both control cell types, thus can still be important
drivers of osteosarcomagenesis. We therefore set
out to select genes that showed differential
expression in osteosarcoma as compared with
both MSCs and osteoblasts.

\subsection{Gene expression signature of high\hyp{}grade osteosarcoma}
We detected $12,542$ and $2,939$ probes encoding
for genes that were significantly differentially
expressed between the 84 osteosarcoma biopsies
and MSCs and osteoblasts, respectively. MA
plots, showing log\hyp{}intensity ratios and log\hyp{}intensity
averages for both analyses, are depicted in
Supporting Information Figure 3 ({\it available online}~\cite{ch7additional}). A total of $1,679$
probes overlapped between both analyses, of
which $1,639$ were either up- or downregulated in
both. GO term analysis on the genes represented
by these $1,639$ probes showed an enrichment of
apoptosis and signal transduction genes. Antigen
processing and presentation, as well as angiogenesis
were also overrepresented (Supporting Information
Figure 4 ({\it available online}~\cite{ch7additional})).

\subsection{Paired integrative analysis is more sensitive than nonpaired integrative analysis}
Nonpaired integrative analysis was performed
on data from 32 samples hybridized on SNP
arrays and from 84 samples hybridized on gene
expression arrays, whereas paired analysis was
performed on a subset of 29 samples for which
both SNP and expression data were available. In
total, $16,105$ autosomal genes were represented
both on SNP and on gene expression arrays.
Nonpaired integrative analysis resulted in 253
significantly affected genes in osteosarcoma biopsies
versus mesenchymal stem cells, whereas 71
genes were detected when osteoblasts were used
as a control. A total of 45 genes were identified
in both analyses versus MSCs and versus osteoblasts
(Figure~\ref{fig7.2}).
%
\begin{figure}[htbp]
  \centering
  \begin{minipage}[b]{0.50\linewidth}
    \includegraphics[width=1\textwidth]{figs07/fig2bw.pdf}		% OBS! print version bw
%    \includegraphics[width=1\textwidth]{figs07/fig2rgb.pdf}	% OBS! pdf version rgb
  \end{minipage}
    \hfill
  \begin{minipage}[b]{0.46\linewidth}
    \caption{Venn diagram with numbers of affected genes in both nonpaired and paired analyses, and in osteosarcoma biopsies versus MSCs and versus osteoblasts. NP: nonpaired integrative analysis, P: paired integrative analysis, OB: analysis of osteosarcoma biopsies versus osteoblasts, MSC: analysis of osteosarcoma samples versus mesenchymal stem cells.}
     \label{fig7.2}
     \end{minipage}
\end{figure}
%
Of these 45 genes, 23 were also
detected in expression analyses of a panel of 19
osteosarcoma cell lines~\cite{ottaviano2010molecular}
versus MSCs and osteoblasts (Supporting Information
Figure 5A ({\it available online}~\cite{ch7additional})). For the paired integrative analyses,
we determined whether the number of genes
with gain combined with overexpression and with
loss combined with downregulation was higher
than expected per sample, based on the numbers
of copy number alterations and gene expression
changes in the whole genome. This was true for
most samples, as depicted in Figure~\ref{fig7.3}, in which the
odds ratios and significance of data dependencies
are shown.
%
\begin{figure}[htbp]
	\centering
	\includegraphics[width=1.0\textwidth]{figs07/fig3rgb.pdf}	% pdf version also rgb
	\caption{Dependence of gene copy number and gene expression data. The heatmaps depict odds ratios for the numbers of genes per sample which show gain and overexpression (overGain), gain and underexpression (underGain), loss and overexpression (overLoss), and loss and underexpression (underLoss). Chi-square tests, or, in case a group contained $<10$ genes, Fisher's exact tests, were performed in order to evaluate whether the number of genes reported from the integrative analysis was higher than expected by chance. $\ast$ Bonferroni\hyp{}corrected p-values $<0.05$. {\it A}, Osteosarcoma biopsies versus MSCs, {\it B}, versus osteoblasts.}
	\label{fig7.3}
\end{figure}
%
Paired integrative analysis resulted in
445 and 138 genes when compared with MSCs
and osteoblasts, respectively. A total of 101 genes
overlapped between these different analyses (Figure~\ref{fig7.2}), and of this set, 31 genes were also detected in
the cell line expression data (Supporting Information
Figure 5B ({\it available online}~\cite{ch7additional}), Table~\ref{tab7.2}).
%
\begin{table}[htbp]
	\centering
	\small
		\begin{tabular}[c]{|lcccc|}
		\hline
		Symbol & Cytoband & CNA & CNA freq (\%) & logFC\\
		\hline
		{\it CLCC1} & 1p13.3 & Gain & 41.4 & 1.24\\
		{\it MCM4} & 8q11.21 & Gain & 37.9 & 1.35\\
		{\it AKR1C3} & 10p15.1 & Loss & 37.9 & -1.94\\
		{\it AKR1C4} & 10p15.1 & Loss & 37.9 & -1.34\\
		{\it ARHGAP22} & 10q11.22 & Loss & 37.9 & -0.45\\
		{\it PGBD3} & 10q11.23 & Loss & 41.4 & -0.82\\
		{\it ARID5B} & 10q21.2 & Loss & 48.3 & -2.33\\
		{\it REEP3} & 10q21.3 & Loss & 48.3 & -0.51\\
		{\it HERC4} & 10q21.3 & Loss & 51.7 & -1.31\\
		{\it PBLD} & 10q21.3 & Loss & 48.3 & -0.29\\
		{\it RUFY2} & 10q21.3 & Loss & 48.3 & -0.20\\
		{\it KIAA1279} & 10q22.1 & Loss & 43.1 & -0.57\\
		{\it SRGN} & 10q22.1 & Loss & 43.1 & -2.26\\
		{\it AIFM2} & 10q22.1 & Loss & 44.8 & -0.52\\
		{\it CHST3} & 10q22.1 & Loss & 48.3 & -1.17\\
		{\it FAS} & 10q23.31 & Loss & 44.8 & -0.42\\
		{\it PCGF5} & 10q23.32 & Loss & 37.9 & -0.34\\
		{\it PPP1R3C} & 10q23.32 & Loss & 37.9 & -2.89\\
		{\it AVPI1} & 10q24.2 & Loss & 37.9 & -2.35\\
		{\it BLOC1S2} & 10q24.31 & Loss & 37.9 & -0.51\\
		{\it CASC2} & 10q26.11 & Loss & 44.8 & -0.18\\
		{\it FAM45A} & 10q26.11 & Loss & 39.7 & -0.78\\
		{\it ERCC6} & 13q11.23 & Loss & 41.4 & -0.52\\
		{\it WASF3} & 13q12.13 & Loss & 44.8 & -2.43\\
		{\it C13orf33} & 13q12.3 & Loss & 48.3 & -2.26\\
		{\it LHFP} & 13q14.11 & Loss & 48.3 & -1.89\\
		{\it WBP4} & 13q14.11 & Loss & 55.2 & -0.93\\
		{\it TSC22D1} & 13q14.11 & Loss & 58.6 & -1.39\\
		{\it RCBTB1} & 13q14.2 & Loss & 58.6 & -0.25\\
		{\it LATS2} & 13q21.11 & Loss & 44.8 & -0.96\\
		{\it DCUN1D3} & 16p12.3 & Loss & 37.9 & -1.39\\
		\hline
		\end{tabular}
\caption{Candidate osteosarcoma driver genes. All frequencies and fold changes are mean values of both integrative analyses---osteosarcoma biopsies versus MSCs and osteosarcoma biopsies versus osteoblasts. For genes for which more than one probe was present on the array, the probe with the highest fold change was used. Cytoband: UCSC cytogenetic band, CNA: copy number aberration, CNA freq: copy number aberration frequency (for $n=29$), logFC: log fold change in biopsies (negative means downregulation, positive means upregulation).}
	\label{tab7.2}
\end{table}
%
Hence, paired analyses
detected $>90\%$ of all genes found with corresponding
nonpaired analyses. In addition, approximately
twice as many genes as found in the
corresponding nonpaired analyses were detected
(Figure~\ref{fig7.2}, Supporting Information Figure 6 ({\it available online}~\cite{ch7additional})). Note that
in the paired analysis fewer samples are included.
Thus, paired analysis gives more robust results
despite the lower sample size. Changing the
threshold of FDR\hyp{}adjusted p-values in the nonpaired
integrative analysis from 0.05 to 0.15 ({\it data
not shown}) did not alter this ratio.

\subsection{Genomic instability genes play a role in osteosarcoma progression}
We calculated genome instability scores using
the method of Carter {\it et al}.~\cite{carter2006signature}, which compares
levels of gene expression of a previously
defined genomic instability signature between
samples in a dataset, for all osteosarcoma biopsies
and different control tissues and cell lines (Figure~\ref{fig7.4}A).
%
\begin{figure}[htbp]
	\centering
	\includegraphics[width=1.0\textwidth]{figs07/fig4bw.pdf}	% print version bw, pdf version bw
	\caption{Genomic instability scores and metastasis\hyp{}free survival. {\it A}, Genomic instability scores for high\hyp{}grade osteosarcoma biopsies, normal bone, osteosarcoma xenografts and cell lines, the HeLa cell line, and mesenchymal stem cells (MSC) and osteoblasts (OB), as calculated by the method of Carter {\it et al}.~\cite{carter2006signature}. {\it B}, Metastasis\hyp{}free survival Kaplan\hyp{}Meier curves for four quartiles of genomic instability scores. {\it C}, Metastasis\hyp{}free survival Kaplan\hyp{}Meier curves for the total amount of genes with copy number gains and losses, using a cut-off based on the median amount of genes per sample showing copy number aberration.}
	\label{fig7.4}
\end{figure}
%
The osteosarcoma biopsies showed highly
variable scores, whereas genomic instability scores
for the controls, normal bone, MSCs, and osteoblasts
were relatively low. High instability scores
were detected for osteosarcoma xenografts, cell
lines, and the HeLa cell line, in increasing order.
This signature predicted for metastasis\hyp{}free survival
in osteosarcoma samples as well (Figure~\ref{fig7.4}B),
with high scores correlating with shorter metastasis\hyp{}free survival (Logrank test for trend p-value $=0.0112$).
As expected, the total number of genes
with copy number gains or losses, which is a direct
measure of genomic instability from the SNP data,
was predictive for progression as well (Logrank
test p-value $=0.018$, Figure~\ref{fig7.4}C).

\subsection{Candidate osteosarcoma driver genes}
The 31 genes returned by the paired integrative
analysis on clinical samples that also were
differentially expressed in osteosarcoma cell lines
are shown in Table 2, together with their chromosomal
locations, aberration frequencies, and
log fold changes. A total of 22/31 genes have
been described to play a role in cancer. Interestingly,
one third of these 22 genes have a role in
cell cycle regulation, matching the importance of
cell cycle and replication in osteosarcomagenesis
as was found both using the genomic instability
scores of the expression data and the overall chromosomal
instability as detected in the copy number
data (Figure~\ref{fig7.4}).

\section{Discussion}\label{discussion7}
In this study, we report copy number and gene
expression alterations in high\hyp{}grade osteosarcoma
prechemotherapy biopsies, and then integrate
these data in order to detect osteosarcoma driver
genes. Copy number analyses, which were
obtained with high\hyp{}density SNP microarrays,
showed very high genomic instability in the osteosarcoma
biopsies. The pattern of aberrations is
in line with previous studies using aCGH and
SNP arrays, which show recurrent gains in chromosome
arms 1p, 6p, and 8q, and losses in chromosome
10. The previously reported recurrent
amplification on chromosome arm 17p~\cite{squire2003high,man2004genome,atiye2005gene,yen2009identification}
is not listed, because we used a
very strict cut-off for aberration frequency (35\%).
Aberration frequencies of 17\%~\cite{man2004genome}
and 26\%~\cite{yen2009identification} were previously found
on chromosome arm 17p, and a distinct amplification
in 17p with an aberration frequency of 21\%
can be seen in Figure 1. We chose such a high
cut-off for recurrent aberrations in order to enrich
for selected genetic events and exclude the
numerous haphazard alterations that can be
attributed to the high genomic instability of high\hyp{}grade
osteosarcoma. In addition, we previously
determined that this cut-off, as compared with
cut-offs of 15\% and 50\%, showed the most consistent
results in subsequent network and pathway
analyses on osteosarcoma cell line SNP data
({\it data not shown}). For genome\hyp{}wide gene expression
analyses, both MSCs and osteoblasts were
used as control cells, and we only considered
overlapping genes between both comparisons, in
order to make sure the affected genes were differentially
regulated in osteosarcoma when compared
with its putative progenitor cells. This
analysis identified a large number ($n=1,639$) of 
probes encoding for differentially expressed
genes. Many of these genes encode tissue type\hyp{}specific
proteins, as is shown in the GO term
enrichment analysis, and appear as upregulated in
osteosarcoma biopsies because the {\it in vitro} grown
control cells, MSCs and osteoblasts, lack surrounding
stroma and are nurtured under other
conditions. Antigen processing and presentation
as well as angiogenesis pathways were expected
to be upregulated, as macrophages and other
infiltrating cells are present in osteosarcoma tissue~\cite{buddingh2011tumor}, and as angiogenesis
plays a role in osteosarcoma progression~\cite{lee1999cell}.
Nevertheless, most stroma\hyp{}derived
gene expression is filtered out by integration with
copy number data, as this expression is not a
result of underlying copy number changes. In
addition to stroma\hyp{}related gene sets, GO term
analysis showed enrichment in apoptosis and signal
transduction genes, which are probably
altered in the osteosarcoma tumor cells and not
in the stroma. Because genes with concordant
changes in copy number and gene expression are
likely to be enriched in drivers of tumorigenesis,
we performed integrative analyses on both types
of data.

We found a remarkable increase in significant
differential genes in paired compared with nonpaired
analysis, {\it i.e.} 101 versus 45. In general,
paired integrative analysis was advantageous over
nonpaired integrative analysis, identifying roughly
twice as many genes, also when different aberration
frequency cut-offs or less stringent cut-offs
for significance were used in the nonpaired analysis.
Nonpaired analysis as performed in Nexus
software compares the number of differentially
expressed genes in a region of copy number aberration
with the expected number of differentially
expressed genes, which is based on the total
number of differentially expressed genes over the
whole genome. This method may be too rigorous,
because an altered copy number region may
encompass tissue\hyp{}specific genes, which may not
be expressed in the particular tumor tissue.
These genes then have altered copy number, but
no difference in expression. If an altered copy
number region contains a relatively large number
of such genes plus only a few candidate drivers,
the entire region will be removed from the output
of the analysis, which increases the amount
of false negatives. Moreover, in the cancer gene
expression profile, a large number of genes downstream
of drivers, {\it i.e.} directly or indirectly regulated
by drivers, or present in feedback loops will
be differentially expressed. This increases the
total number of differentially expressed genes,
which again lowers the chance that a specific
altered region is returned from the nonpaired
integrative analysis as significantly affected. Furthermore,
a single differentially expressed gene
in a certain region of copy number alteration may
still exert its driving function, and this driving
function usually does not depend on the proportion
of differentially expressed genes in the same
region. Because of this, and because our method
of paired integrative analysis is gene\hyp{}based and
not region\hyp{}based, we did not perform a correction
based on the total number of differentially
expressed genes when compared with the
affected copy number regions in the paired analysis
in R, and this may be an additional reason
why more genes are returned from the paired
analysis. However, in all samples, except for one
(L3438), the number of genes showing both copy
number alteration and differential expression was
higher than expected when compared with the
numbers of copy number alterations and differentially
expressed genes over the whole genome.
This was significant for the vast number of samples
(28/29, 23/29, 27/29, and 23/29, for combinations
gain and overexpression, loss and
underexpression in biopsies versus MSCs, and
gain and overexpression, loss and underexpression
in biopsies versus osteoblasts, respectively,
as shown in Figure~\ref{fig7.3}).

Genomic instability scores showed that the
instability in osteosarcoma tissues ranges from a
level comparable to that of the controls, to the
high instability levels of repeatedly passaged
tumors in xenografts and osteosarcoma cell lines.
We demonstrated both on copy number data, as
well as by applying a genomic instability gene
signature to genome\hyp{}wide gene expression data,
that high genomic instability in osteosarcoma is
correlated with poor metastasis\hyp{}free survival. This
suggests that genes playing a role in genomic
instability may be potent drivers of osteosarcoma
progression, as has been reported for various
other tumor types~\cite{carter2006signature}. Paired
integrative analysis confirmed this result, as one
third of the genes with a possible role in tumorigenesis
had a function connected to the cell
cycle. Of these genes, {\it MCM4} showed gain and
overexpression and was only detected by the
paired integrative analysis. {\it MCM4} is part of the
minichromosome maintenance complex, which
functions as a replication helicase, with a role in
maintaining genomic stability~\cite{aguilera2008genome}. This gene has been
reported overexpressed in various tumor types~\cite{freeman1999minichromosome,alison2002minichromosome,majid2010regulation}. Genes that were detected in both
nonpaired and paired analyses were all deleted
and underexpressed. {\it AVPI1}, or arginine vasopressin\hyp{}induced 1, may be involved in cell cycling~\cite{apweiler2011ongoing}. {\it ERCC6} is involved
in transcription\hyp{}coupled nucleotide excision
repair, which is a critical survival pathway protecting
against cancer~\cite{fousteri2008transcription}. {\it RCBTB1}, a candidate tumor suppressor,
was recently shown to have growth inhibitory activity
in osteosarcoma cells by regulating pathways
of DNA damage/repair and apoptosis~\cite{zhou2010clld7}.
{\it LATS2}, or large tumor suppressor
homolog 2, plays a critical role in centrosome
duplication, maintenance of mitotic fidelity,
and genomic instability~\cite{visser2010lats}.
Positive feedback between the p53 and Lats2 tumor
suppressors prevents tetraploidization~\cite{aylon2006positive},
which could be an initiating step in
osteosarcomagenesis, leading to genomic instability~\cite{ganem2007limiting,ganem2007tetraploidy}. Also, a role of Lats2 in quenching of the
increased genomic instability of H-Ras\hyp{}induced
transformation has been identified~\cite{aylon2006positive}. {\it DCUN1D3} encodes for a UVC\hyp{}responsive
protein involved in cell cycle progression and cell
growth~\cite{ma2008dcun1d3}. Additional candidate
genes with no direct role in cell cycle regulation
include for example genes with a role in apoptosis
({\it AIFM2}, {\it BLOC1S2}, {\it FAS}) and metabolism
({\it AKR1C3} and {\it -4}). Some previously reported
genes with a driver role in osteosarcoma were not
identified, mainly because our high cut-off for
recurrence. For example, {\it CDKN2A}, {\it MDM2}, and
{\it E2F2} had recurrence frequencies of 28\%, 17\%,
and 34\%, respectively (in the dataset of 29 samples).
{\it CDKN2A} and {\it MDM2} were not significantly
differentially expressed, but {\it E2F2} was consistently
significantly overexpressed with log fold
changes $>1.50$ in all analyses (biopsies and cell
lines as compared with different controls). {\it TP53}
and {\it RB1} aberrations were present in $>35\%$ of all
samples (38\% and 69\%, respectively). {\it TP53} was
significantly downregulated in biopsies as compared
with both controls, but not in the osteosarcoma
cell line dataset. {\it RB1} showed significant
downregulation when compared with MSCs, but
not with osteoblasts, indicating a difference
between these controls in {\it RB1} signaling. We set
our cut-off for recurrence to 35\% and only
selected genes present both in osteosarcoma
biopsies as well as in cell lines as compared with
two different control sets, in order to select for
the most important osteosarcoma drivers. Using
this method, we were able to detect previously
unreported driver genes.

In summary, we have shown that an individual
gene\hyp{}based paired integrative analysis of copy
number and gene expression data performs better
than a region\hyp{}based nonpaired analysis. Several
osteosarcoma candidate driver genes, especially
genes playing a role in cell cycle progression,
have been identified. Additional research, particularly
functional studies, should reveal whether
these genes are early or late drivers in
osteosarcomagenesis.

%%% references

\begin{small}
\begin{singlespace}
\bibliographystyle{unsrtnatshort}		% sorted as referenced, was unsrtnat, but unsrtnatshort gives shorter output
\bibliography{biblio}
\end{singlespace}
\end{small}

%\end{document}