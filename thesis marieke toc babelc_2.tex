% Marieke Kuijjer
% 2013-02-15
% general file to combine all chapters together


%%%%% start

\documentclass[12pt,b5paper]{book}

% layout packages and commands
	%\usepackage{breakurl}
	\usepackage[small]{caption}
	\usepackage{setspace}
	\usepackage{units}
	\usepackage{geometry}
	\onehalfspacing					% 1.5-spacing
	\raggedbottom					% overcomes spacing problem
	\usepackage{fancyhdr}				% for nice headers, which are introduced at the start of each chapter (separate .tex files)
	\usepackage[T1]{fontenc}			% changes spacing between paragraphs (and more?)
	\geometry{verbose,a4paper,tmargin=15mm,bmargin=30mm,lmargin=30mm,rmargin=20mm}		% makes text less bold (due to reduction A4->A5?)
	\usepackage{appendix}

% language packages and commands
	\usepackage[english,dutch]{babel}		% loads hyphenation patterns for American English en Dutch
	\hyphenation{bot-ont-kal-king blok-ke-ren he-ma-to-poi-e-tic Euro-BoNeT Af-fy-met-rix fibro-blastic chondro-blastic osteo-blastic im-mu-no-stim-u-la-to-ry os-teo-sar-coom xe-no-trans-plan-ta-ten}	% hyphenation for undefined words, this works
	\usepackage{hyphenat}				% this package comes handy for rendering words combined with a hyphen available for hyphenation (except for hyphen part)
%		\overfullrule3mm				%%% OBS! % run this to check for overfull boxes, this should be REMOVED
	\usepackage[font=small,labelfont=bf]{caption}	% eg for hyphenation for languages with accented characters
	\usepackage{upgreek}

% packages for table of contents and commands
	\usepackage{tocloft} %to control overfull boxes in table of contents
	\setcounter{secnumdepth}{0} 			% do not number (sub)sections, but do list them in the table of contents
	\setcounter{tocdepth}{1}				% the number of section levels in the table of contents

% packages for figures and tables
	%\usepackage{graphics}
	\usepackage{graphicx}
	\usepackage{float}
	\usepackage{lscape}
	\usepackage{array}
	\usepackage{afterpage}				% to start a longtable at an even page
	\usepackage{longtable}				% for table over multiple pages
	\usepackage{booktabs}				% package for table with merged columns
	\usepackage{hvfloat} 				% for rotated table on same page as section heading

%%% mystrut for vertical lines table with horizontal lines stretching over several columns
\usepackage{bigstrut}
    \setlength\bigstrutjot{3pt}
\makeatletter
\newlength\mylena
\newlength\mylenb
\newcommand\mystrut[1][2]{%
    \setlength\mylena{#1\ht\@arstrutbox}%
    \setlength\mylenb{#1\dp\@arstrutbox}%
    \rule[\mylenb]{0pt}{\mylena}}
\makeatother

%%% new environment myTable for table 5.1, a horizontal longtable which needs singlespacing small
\newenvironment{myTable}%
{%
    \small
    \singlespace
    \setlength\tabcolsep{1.5ex}
}%

% packages and definitions for bibliographies and commands
%	\usepackage[sectionbib]{bibunits}		% references per chapter, at section level
%	\usepackage{cite}					% necessary for proper citing
	\usepackage[round,numbers,sort&compress,sectionbib]{natbib}	% natbib and chapterbib should together work for generating bibliographies in each chapter	
	\usepackage[sectionbib]{chapterbib}		% from ale's thesis
%	\defaultbibliographystyle{unsrt}			% default style references (ale heeft \usepackage[round,numbers,sort&compress]{natbib})
%	\renewcommand{\bibname}{References}	% OLD changes Bibliography into References
		%% OLD to compile, open each by latex created .aux files in latex and build bibtex.
		%% OLD close, then build latex as many times as there are reference sections

% packages for links and commands
	% \usepackage[hidelinks]{hyperref} % does not work here, does work in chapter files
	\usepackage[linktocpage]{hyperref}		% linktocpage nodig om te klikken op page#, daardoor line te kunnen breaken in toc
	\usepackage[all]{hypcap}

%%% fancyhdr should be called from main file. this returns the right chapter number, also at the final page of the chapter
  \pagestyle{fancy}
	\renewcommand{\chaptermark}[1]{\markboth{#1}{}}
	\renewcommand{\sectionmark}[1]{\markright{\thesection\ #1}} \fancyhf{}
	\fancyhead[LE,RO]{\bfseries\thepage}
	\fancyhead[LO]{\bfseries\ Chapter \thechapter}				% instead of Chapter 
	\fancyhead[RE]{\bfseries\ Chapter \thechapter}		% instead of Chapter \thechapter
	\renewcommand{\headrulewidth}{0.5pt}
	\renewcommand{\footrulewidth}{0pt}
	\addtolength{\headheight}{0.5pt}

%%% 2013-04-17 clearemptydouble page for first pages of thesis
\let\origdoublepage\cleardoublepage
\newcommand{\clearemptydoublepage}{%
  \clearpage
  {\pagestyle{empty}\origdoublepage}%
}

%%%


\begin{document}
\selectlanguage{english}

%%%%% begin first titlepage
\begin{titlepage}
	\thispagestyle{empty}
	{\centering    
		\vspace*{1.cm} 
	 	\vspace*{3.cm} 
		\begin{doublespace}	%%% 2013-04-09
	  	{\huge\sc A Systems Biology Approach to Study High-grade Osteosarcoma}\\
		\end{doublespace}		%%% 2013-04-09
}
\pagebreak 
	\thispagestyle{empty}
	\vspace*{\fill}
%		\\
\begin{minipage}[b]{0.55\linewidth}
	{\setlength{\parindent}{0cm}
	\begin{singlespace}
		\footnotesize{
			The work presented in this thesis was financially supported by the Dutch Cancer Society (2008-4060) and EuroBoNeT, a European Commission granted Network of Excellence for studying the pathology and genetics of bone tumors (LSHC-CT-2006-018814).\newline

			Publication of this thesis was financially supported by the Dutch Cancer Society and by the Department of Pathology, Leiden University Medical Center.\newline

			Cover art: M.L. Kuijjer, photographs of the contemporary artwork `Ritterschlacht' by Martin Honert, taken at the exhibition `Kinderkreuzzug', Hamburger Bahnhof, Berlin, Germany.\newline

			Printed by: W\"ohrmann Print Service, Zutphen, the Netherlands.
		}%footnotesize
	\end{singlespace}
	}%parindent for this minipage
\end{minipage}
%\end{titlepage}
\newpage
%
%%% begin second titlepage
%\begin{titlepage}
	\thispagestyle{empty}
	{\centering    
		\vspace*{1.cm} 
%	  	{UNIVERSITEIT LEIDEN}\\  
 	\vspace*{3.cm} 
		\begin{doublespace}	%%% 2013-04-09
	  	{\huge\sc A Systems Biology Approach to Study High-grade Osteosarcoma}\\
		\end{doublespace}		%%% 2013-04-09
	\vspace*{3.5cm}
			\huge Proefschrift\\
			\large
	  \vspace{1.5cm}
			ter verkrijging van\\
		  	de graad van Doctor aan de Universiteit Leiden,\\
			op gezag van Rector Magnificus prof.mr. C.J.J.M. Stolker,\\
			volgens besluit van het College voor Promoties\\
			te verdedigen op woensdag 26 juni 2013\\
			klokke 16.15 uur\\
  	\vspace{4.cm}
			door\\
	\vspace {0.5cm}
			Marieke Lydia Kuijjer\\	
%	\vspace {0.5cm}
			geboren te Zaanstad\\
			in 1982\\
}%centering
\pagebreak 
	\thispagestyle{empty}
	\vspace {1cm}
\begin{tabular}{ll}
	\multicolumn{2}{l}{{\bf Promotiecommissie}}\\
		\\
		Promotor: & Prof. dr. P.C.W. Hogendoorn\\
		\\
		Co-promotor: & Dr. A.M. Cleton-Jansen\\
		\\
		Overige leden: & Prof. dr. C.J. Cornelisse\\
		& Prof. dr. J. Kirpensteijn (Utrecht University, Utrecht)\\
		& Prof. dr. O. Myklebost (Norwegian Radium Hospital, Oslo, Norway)\\
		& Dr. J.M. Boer (Erasmus Medical Center, Rotterdam)\\
\end{tabular}

\end{titlepage}
%
%%%
%%%%%% test other page
%\newpage
%\pagestyle{empty}
%\cleardoublepage
%\clearpage{\cleardoublepage} %empty page won t have header

\thispagestyle{empty}
\vspace*{\fill} 
\begin{quote} 
\centering 
``Qui o altrove, tenente, siamo tutti da qualche parte per sbaglio''
\end{quote}
\begin{quote} 
\centering 
\textit{Il deserto dei Tartari, Dino Buzzati / Valerio Zurlini}
\end{quote}
\begin{quote} 
\centering 
\end{quote}
\vspace*{\fill}


%%%%% table of contents
\clearemptydoublepage
%\cleardoublepage{\pagestyle{empty}} %%% OBS! 2013-04-17 test
\newpage
\setcounter{tocdepth}{0}
\tableofcontents
\thispagestyle{empty}
\clearemptydoublepage % 2013-04-17
%\cleardoublepage


%%%%% include chapters
\newpage
\renewcommand{\bibname}{References}

% Marieke Kuijjer
% 2013-02-15
% chapter 01

	%\documentclass[12pt,b5paper]{book}
	%\setcounter{secnumdepth}{0}
	%\setcounter{tocdepth}{1}
	%\usepackage[hidelinks]{hyperref}

%	\pagestyle{fancy}
%	\renewcommand{\chaptermark}[1]{\markboth{#1}{}}
%	\renewcommand{\sectionmark}[1]{\markright{\thesection\ #1}} \fancyhf{}
%	\fancyhead[LE,RO]{\bfseries\thepage}
%	\fancyhead[LO]{\bfseries\ Chapter 1}
%	\fancyhead[RE]{\bfseries\ Chapter 1}
%	\renewcommand{\headrulewidth}{0.5pt}
%	\renewcommand{\footrulewidth}{0pt}
%	\addtolength{\headheight}{0.5pt}

%\begin{document}

%%% title page

\chapter{General introduction}\label{ch1}
\thispagestyle{empty}				%%% to remove page number from first page of chapter, must be placed after calling the chapter
%\vfill

\newpage

%%% main document

%
\section{Cancer genomics}\label{cancer1}
Cancer develops through the acquisition of genomic alterations, {\it i.e.} changes in the DNA sequence and chromosomal numerical content of a cell, such as point mutations, insertions, deletions, amplifications, and translocations. Such alterations may alter protein expression and/or function of oncogenes---genes promoting cancer---and tumor suppressors---genes protecting from cancer. Other factors, caused by mechanisms which do not change the underlying DNA sequence ({\it e.g.} DNA methylation and histone modification), may also alter the expression of genes, and may thereby play a role in cancer as well. As different biological processes, the so-called hallmarks of cancer~\cite{hanahan2000hallmarks,hanahan2011hallmarks}, need to be deregulated before a cancer can develop, a combination of events is needed to change a normal cell into a cancerous cell. Cancer is thought to arise through the stepwise acquisition of such events~\cite{fearon1990genetic}, although it has become clear that different alterations may be caused by a single event~\cite{stephens2011massive,yates2012evolution}, and particularly oncogene\hyp{}activating translocations seem to be sufficient for oncogenesis in some types of leukemias, lymphomas, and sarcomas~\cite{knudson2004human}.

In cancer genomics, germline and somatic aberrations, {\it i.e.} aberrations present in the germline of the patient and acquired aberrations, are studied in order to identify genes and biological processes which are important in the development and progression of cancer. Determining aberrations that are crucial for a cancer cell to survive, identifying defective tumor suppressors, and identifying biological processes which facilitate tumor progression is tremendously important for diagnostics and prognostics, and for the identification of targeted treatments. In the late 1990s, high\hyp{}throughput methods have been developed which can be utilized in studying cancer genomics---so-called microarrays. In this thesis, we have used these high\hyp{}throughput techniques in order to study high\hyp{}grade osteosarcoma genomics, aiming to learn more on osteosarcoma biology, and to identify possible targets for treatment.

%
\section{High-grade osteosarcoma}\label{high1}
High-grade osteosarcoma is a primary malignant tumor consisting of mesenchymal tumor cells producing osteoid. The tumor is rare, with an approximate incidence of 5--6 patients in a population of one million per year. The incidence is higher in adolescents and young adults, and shows a second peak at middle age~\cite{mirabello2009osteosarcoma}. Osteosarcoma developing later in adult life is thought to be partially secondary, and may be caused by previous treatment with radiation or by an underlying Paget's disease of bone. Males are more often affected by osteosarcoma than females (with a ratio of 3:2). High\hyp{}grade osteosarcoma most frequently develops in the long bones of patients, with the metaphysis as the most frequent (91\%), and the diaphysis as the second most frequent site ($<9\%$). Most often, the tumor develops in the region around the knee (distal femur and proximal tibia), followed by the proximal humerus~\cite{raymond2002conventional}. Osteosarcoma is rarely seen in the axial bones of the patient. The incidence pattern of osteosarcoma suggests a link between the development of the disease and growth~\cite{mirabello2011height} (this will be further discussed in Chapter~\ref{ch5}).

High-grade osteosarcoma is a very aggressive tumor. Patients are usually treated with several series of neoadjuvant chemotherapy consisting of a combination of different chemotherapeutic drugs, especially cisplatin, doxorubicin, and high\hyp{}dose methotrexate~\cite{raymond2002conventional}. The tumor is then removed by limb\hyp{}salvage surgery, although sometimes amputation is needed. Afterwards, a second series of adjuvant therapy is given to the patient. Despite this intensive treatment schedule, a significant number of patients die due to the development of distant metastases, which are most often pulmonary. The tumor metastasizes in approximately 45\% of all patients~\cite{pakos2009prognostic}. Overall survival of patients with resectable metastatic disease is roughly 20\%~\cite{buddingh2010prognostic}. Neoadjuvant treatment was introduced in the 1970s, and improved overall survival from 10--20\% to approximately 60\%. However, except for macrophage\hyp{}activating and recruiting agents, such as L-MTP-PE (discussed in Chapter~\ref{ch4} of this thesis), no new treatment options have been developed that can raise survival significantly. The many caveats and challenges hampering osteosarcoma research, which might explain why osteosarcoma patients still have no other treatment options, are discussed in Chapter~\ref{ch2}.

Known genes involved in osteosarcomagenesis have essential roles in cell cycle progression~\cite{cleton2005central}. The tumor suppressor {\it TP53}, which can induce cell cycle arrest or apoptosis in response to cellular stress, such as DNA damage, is mutated in approximately 20\% of high\hyp{}grade osteosarcomas and also often present in regions of copy number loss. {\it MDM2}, which targets the p53 protein for degradation, is amplified in 6--15\% of the tumors. {\it TP53} and {\it MDM2} aberrations have been described to be mutually exclusive~\cite{overholtzer2003presence}, although in our dataset, one sample (the osteosarcoma cell line HAL) had copy number loss of {\it TP53} and gain of {\it MDM2}. Inactivating somatic mutations of {\it RB1}, a negative regulator of the cell cycle, are also often found in osteosarcoma, and this gene is present in regions of copy number loss in over 60\% of osteosarcomas~\cite{thomas2003role,kuijjer2012identification}. Other players of the Rb pathway have been described in osteosarcoma as well, for instance {\it CDKN2A} deletions, which present homozygously and occur in approximately 25\% of all patients~\cite{mohseny2010small}. {\it TP53} and {\it RB1} mutations are not always somatic---a small percentage of osteosarcoma is hereditary, with mutations present in the germline of patients. The associated hereditary syndromes, Li\hyp{}Fraumeni and Retinoblastoma for mutations in {\it TP53} and {\it RB1}, respectively, give a strong predisposition to develop osteosarcoma. A third hereditary syndrome that is thought to predispose to osteosarcoma is Rothmund\hyp{}Thomson syndrome, where {\it REQL4}, a gene encoding for a DNA helicase, is mutated~\cite{calvert2012risk}, however, in contrast to {\it TP53} and {\it RB1}, the gene is not a frequent target for sporadic mutations in osteosarcoma~\cite{nishijo2004mutation}.

%
\section{The EuroBoNeT high-grade osteosarcoma database}\label{eurobonet1}
The aim of this thesis was to study osteosarcomagenesis by bioinformatics analysis of a high\hyp{}throughput dataset consisting of microarray data from high\hyp{}grade osteosarcoma specimens. A relatively large cohort of na\"{\i}ve, preoperative diagnostic osteosarcoma biopsies was collected as a collaborative effort by EuroBoNeT, a European Network of Excellence for studying primary bone tumors. This clinically well defined cohort consisted of samples from 84 patients. For most of these patients, clinical data were available on patient sex, age at diagnosis of the primary tumor (in months), tumor location, histological subtype of the tumor, and response to neoadjuvant chemotherapy (Huvos grade)~\cite{huvos1991bone}. Follow\hyp{}up data (metastasis\hyp{}free survival and overall survival, measured in months from diagnosis) was available for 83/84 patients. Clinical characteristics of this cohort can be found in Table~\ref{tab7.1}. In addition to the clinical samples, we used data from two osteosarcoma model systems---osteosarcoma cell lines (characterized and published by Ottaviano {\it et al}.~\cite{ottaviano2010molecular}) and xenografts~\cite{mayordomo2010tissue}, see Table~\ref{tab3.1} for clinical characteristics of the original tumors of these model systems. The entire osteosarcoma database consisted of data obtained from three different microarray platforms---genome\hyp{}wide gene expression data, data obtained with a kinome screen, and Single Nucleotide Polymorphism (SNP) microarrays. Table~\ref{tab1.1} illustrates the different data types, numbers of osteosarcoma and control samples, and the different comparative analyses which are described in this thesis. Raw and processed data are deposited in online databases~\cite{edgar2002gene,r2microarray}.
%
%%% table tab1.1
\newcolumntype{x}[1]{>{\raggedright\arraybackslash}p{#1}}
\begin{table}[htbp]
	\centering
	\small
		\begin{tabular}[c]{|ll >{\raggedright}p{1.2in} >{\raggedright}p{1.55in}|}
		\hline
		Data type & mRNA & Kinome & SNP\tabularnewline
		\hline
		Company & Illumina & PamGene & Affymetrix\tabularnewline
		Array & Human-6 v2.0 & Ser/Thr kinase PamChip & Genome\hyp{}wide Human SNP Array 6.0\tabularnewline
		Software & Bioconductor & BioNavigator, & Genotyping Console,\tabularnewline
		& & Bioconductor & Nexus Copy Number\tabularnewline
		OS samples & 84 diagnostic biopsies, & 2 cell lines & 32 diagnostic biopsies,\tabularnewline
		& 19 cell lines, & & 12 cell lines\tabularnewline
		& 18 xenografts & &\tabularnewline
		Control samples & 12 MSC cultures, & 12 MSC cultures & 27 normal samples\tabularnewline
		& 3 osteoblast cultures  & & \\
		Analysis methods & {\it LIMMA}, {\it pamr} & {\it LIMMA} & Cut-off for aberration frequency\tabularnewline
		Comparative & Clinical parameters, & Tumor {\it vs} controls & Clinical parameters,\tabularnewline
		analyses & Tumor {\it vs} controls  & & Tumor {\it vs} controls\tabularnewline
			\hline
		\end{tabular}
		\caption{Layout of the high-grade osteosarcoma database. MSC: mesenchymal stem cell.}
		\label{tab1.1}
	\end{table}
%

%
\section{High\hyp{}throughput platforms to study osteosarcoma}\label{platforms1}
Genome\hyp{}wide gene/mRNA expression profiling can be performed using RNA isolated from a sample, such as tumor tissue or a cell culture. Generally, cDNA or cRNA is prepared from the RNA and is labeled with a fluorescent dye. This is then hybridized to a microarray chip containing oligonucleotide probes, which are short sequences of DNA, complementary to most or all specific transcripts, capable of binding the labeled cDNA/cRNA. For measuring genome\hyp{}wide gene expression, single- and dual channel microarrays are available. With dual channel microarrays, samples, {\it e.g.} paired tumor samples and normal tissues, can be directly compared on one chip, by labeling the cDNA/cRNA with two different fluorescent dyes. For the research described in this thesis (Chapters~\ref{ch3}--\ref{ch8}), single channel microarrays were used, which means that control samples were hybridized on different chips. We used Illumina Human-6 v2.0 BeadChips (Illumina, San Diego, CA). These microarrays contain over $48,000$ probes, of which approximately half are recognized by well\hyp{}annotated Reference Sequence (RefSeq) genes~\cite{pruitt2002reference}. Illumina BeadChips have a special structure: probes are present on beads, which are randomly arranged on the chip. Every bead type is replicated on each chip with a mean of approximately 35--40 times~\cite{oliphant2002beadarray,barbosa2010re} (see Figure~\ref{fig1.1}A).
%
\begin{figure}[htbp]
	\centering
	\includegraphics[width=1.0\textwidth]{figs01/fig1bw.pdf}	% pdf version also bw
	\caption{Schematic overview of {\it A}, the Illumina BeadChip and {\it B}, the Affymetrix SNP 6.0 array. Figure adapted from Hup\'e, P., \url{http://commons.wikimedia.org}.}
	\label{fig1.1}
\end{figure}
%
Both the random position and the high amount of replicated beads make robust measurements possible~\cite{dunning2007beadarray}. The software designed by Illumina for data analysis, BeadStudio, does not take advantage of the large number of replications of beads present on these chips. Therefore, various methods have been specifically developed for analyzing Illumina BeadChips, such as Bioconductor~\cite{gentleman2004bioconductor} packages {\it beadarray}~\cite{dunning2007beadarray}, {\it beadarraySNP}~\cite{oosting2010beadarraysnp} (specifically for Illumina SNP data), and {\it lumi}~\cite{du2008lumi}, which will be described in the next section.

Peptide microarrays can be used for studying kinase activity in a sample. For the research performed in Chapter~\ref{ch6} of this thesis, we used PamGene\textregistered~serine/threonine (Ser/Thr) PamChips (PamGene, 's-Hertogenbosch, the Netherlands). These chips consist of porous membranes, which contain 142 different peptides derived from phosphorylation sites for Ser/Thr kinases of the human proteome. Cell or tissue lysates are supplemented with ATP and subsequently pumped through these membranes, so that kinases in the lysates have access to, and can phosphorylate the peptides on the chip. Phosphorylation is measured over a time span of 30 to 60 minutes by the detection of light emitted by fluorescently\hyp{}labeled, phospho\hyp{}specific antibodies. Figure~\ref{fig1.2} gives an overview of the experimental workflow of PamGene.
%
\begin{figure}[htbp]
  \centering
  \begin{minipage}[b]{0.50\linewidth}
%    \includegraphics[height=1\textheight]{figs01/fig2bw.pdf}	% OBS! print version bw
   \includegraphics[height=1\textheight]{figs01/fig2rgb.pdf}	% OBS! pdf version rgb
  \end{minipage}
    \hfill
  \begin{minipage}[b]{0.46\linewidth}
    \caption{Peptides can serve as substrates for kinases present in the sample. Phosphorylation is detected by fluorescently labeled phopho\hyp{}specific antibodies ({\it A}). The microarrays consist of a porous ceramic membrane ({\it B}), on which 142 different peptide substrates are present ({\it C}). Four arrays are combined into one chip ({\it D}). The phosphorylation reaction occurs by an up and down movement of the sample solution through the array, giving the kinases maximal opportunity to phosphorylate the peptides on each array ({\it E}). When the solution is underneath the array, the CCD camera in the workstation takes an image of each array, which is later used by the software to generate kinetic data curves ({\it F}). The incubation, washing, dispensing of reagents and imaging of the arrays is done in fully automated workstations ({\it G}). Figure adapted from PamGene\textregistered.}
     \label{fig1.2}
     \end{minipage}
\end{figure}
%

Single Nucleotide Polymorphisms (SNPs) are genetic changes or variations of a single base pair, which occur in at least 1\% of the population~\cite{gibbs2003international}. SNP microarrays contain so-called allele\hyp{}specific oligonucleotide probes (Figure~\ref{fig1.1}B), which are used to discriminate between specific SNPs in the sample, because of the different binding properties of the sample DNA, which is again labeled with a fluorescent dye. SNP microarrays can be employed to genotype a sample, which is used to identify small variations between genomes (to determine {\it e.g.} disease susceptibility), but can also be utilized to infer copy number aberrations and allelic states of regions in the genome. The SNP microarrays used in this thesis (Chapters~\ref{ch7}--\ref{ch8}) are Affymetrix Genome\hyp{}Wide Human SNP Array 6.0  chips (Affymetrix, Santa Clara, CA). These high\hyp{}density chips contain over $900,000$ SNPs and over $900,000$ probes for the detection of copy number variation.

%
\section{Microarray data preprocessing}\label{preprocessing1}
The three different platforms described above have in common that, after hybridization of DNA/cDNA/cRNA to the chip, or after phosphorylation of peptides on the microarray, a fluorescent signal is emitted, which is measured by a scanner. The image files that are returned by the scanner can be utilized for deducing intensity signals and the location of the specific spots/beads. This is usually performed directly by the software provided by the company which distributes the arrays, and generally overlays a grid and returns median intensity signals for each spot/bead. Alternatively raw image files can be analyzed (for example using {\it beadarray}~\cite{dunning2007beadarray}), thereby allowing additional methods of data processing. In the following paragraphs, we will discuss data preprocessing and subsequent data analysis of data generated with the above described microarray chips.

Preprocessing of microarray data is performed in order to correct for experimental bias and to reduce the signal to noise ratio. Numerous methods of microarray data preprocessing exist, and specific methods may differ per data type and platform. Preprocessing of microarray data can be performed using the software provided by the company that produced the arrays, or can be analyzed with open source programs, such as the statistical software R~\cite{r2.15.0}, for which several packages have been made available in the Bioconductor~\cite{gentleman2004bioconductor} framework to specifically analyze the raw data of various microarray platforms.

An optional start of preprocessing the raw data is a global or local background subtraction step. This can eliminate signals due to nonspecific binding, thereby reducing noise in the data. However, when applying this step, probes of low signal will be discarded, resulting in missing values. Some researchers convert these missing values into zero expression. Illumina's scanner software, BeadScan, automatically subtracts local background measures from the foreground intensities to generate bead level text files---files including intensities and location information obtained from the original .tiff files produced by the scanner software. These {\it bead level files} can be used for downstream data analysis. The standard local background subtraction method provided by Illumina results in a very low estimate of the background, which is thought to be mostly related to the optical properties of the array surface~\cite{dunning2008statistical}. Additional background subtracting methods can be applied, such as {\it background normalization} in BeadStudio, which subtracts the mean intensity of negative control beads from the foreground intensities. This method increases variability, and also introduces a significant number of negative values~\cite{dunning2008statistical}. Especially for small sample sizes it is crucial to achieve a homogeneous variance, and thus, as background subtraction introduces additional variation in the data, this may not be beneficial for the detection of differences between two or more groups~\cite{schmid2010comparison}. Apart from the local background subtraction by BeadScan (for mRNA expression data), we did not use other background subtraction methods in the preprocessing of our microarray data.

Data transformation is necessary because of the complicated error structure of microarray data, which is intensity\hyp{}dependent and nonlinear~\cite{durbin2002variance}. Often, a simple log transformation is used, but other methods exist that are milder in transforming signals near background, which are inflated by standard log transformations. Examples of such methods are variance stabilizing normalization ({\it vsn})~\cite{huber2002variance}, which both transforms the data and performs normalization of the data between the different arrays, and variance stabilizing transformation ({\it vst})~\cite{lin2008model}, a method similar to {\it vsn}, specifically developed for preprocessing Illumina BeadChips. {\it vst} has been shown to be advantageous over log transformation when large changes in expression are expected~\cite{dunning2008spike,du2010evaluation}. Normalization of the data is applied to reduce bias that may arise due to differences in sample preparation, and production (batch effects) and processing of the arrays. Various normalization methods exist, of which complete data methods, such as quantile normalization, are preferred over methods that use a baseline array in order to normalize the data~\cite{bolstad2003comparison}. We used {\it vst} and robust spline normalization ({\it rsn}), a normalization method specifically designed to normalize variance stabilization transformed data, on mRNA expression data (Chapters~\ref{ch3}--\ref{ch8}). Transformation and normalization of peptide chips (Chapter~\ref{ch5}) was performed using {\it vsn}, while SNP microarray data were $log_2$ transformed and quantile normalized. SNP microarray data (Chapters~\ref{ch7}--\ref{ch8}) were further corrected for the guanine\hyp{}cytosine (GC) content, as different percentages in GC content can cause waviness in the $log_2$ ratio data, which can increase false positive and false negative segment calls. We used the Regional GC correction algorithm in Genotyping Console to correct for this waviness~\cite{gcwaviness}.

%%%
\section{Quality control}\label{quality1}
A very important microarray data preprocessing step is outlier detection. When correctly performed, this step can significantly improve data quality and thereby improve the outcome of the experiment, {\it e.g.} the detection of differential expression~\cite{allison2006microarray,kauffmann2010microarray}. Defective probes from Affymetrix chips can be detected and subsequently removed~\cite{li2001model}. In Illumina data, spatial artifacts can be detected and removed using BeadArray Subversion of Harshlight, or {\it BASH}~\cite{cairns2008bash}. Although the detection of large spatial artifacts may be helpful for determining whole outlier chips, the {\it BASH} algorithm only improves results very mildly. This can be described to the extremely robust structure of the Illumina BeadChips (tested for Human-6 and GoldenGate BeadChips, Kuijjer {\it et al}., {\it unpublished results}). The more recently developed HumanHT-12 Expression BeadChips contain fewer replicates per bead type, and this preprocessing step may therefore be valuable for removing outliers in these newer chips. Other artifacts in Illumina data have been reported, such as particularly bright beads showing a bleed over effect on neighboring beads, raising their associated values~\cite{smith2010identification}. One can adjust for such spatial artifacts by masking affected beads using the {\it beadarray} package~\cite{dunning2007beadarray}.

Regularly, it is necessary to remove entire chips of poor quality, since such chips can impair overall statistical and biological significance~\cite{kauffmann2010microarray}. Poor quality chips can be identified by visually checking the scanner images, the distribution of both raw and normalized data ({\it e.g.} by plotting density plots, boxplots, and MA-plots), and by performing unsupervised hierarchical clustering or visualizing the data using principal components analysis (PCA, reducing the data dimensionality to {\it e.g.} its first two or three principle components). Such methods can for example be applied using Bioconductor package {\it arrayQualityMetrics}~\cite{kauffmann2009arrayqualitymetrics} (used for quality control of mRNA and kinome profiling in this thesis) or using quality control functions in the package {\it affy}~\cite{gautier2004affy}. Another method to control the influence of poor quality chips is assigning weights to all chips, so that arrays of better quality will have a higher influence on the analysis than poor quality arrays ({\it arrayWeights}~\cite{ritchie2006empirical}). Such an approach is, however, not intended to replace a quality check identifying catastrophically poor quality chips, and these should still be discarded. In a comparative study of removing poor quality chips with {\it arrayQualityMetrics}, assigning {\it arrayWeights} to the data, or applying both methods on the {\it LIMMA} analysis described in Chapter~\ref{ch4}, we determined more differentially expressed probes at a false\hyp{}discovery rate (FDR, see next section for an explanation) of 0.05 without assigning weights, but this depended on the FDR (for $0.05<$ FDR $\le0.1$ {\it arrayWeights} or a combination of both methods performed slightly better, Kuijjer {\it et al}., {\it unpublished results}).

In SNP microarray quality control, one can determine the ability of an experiment to resolve SNP signals into three genotype clusters (AA, AB, BB). The Affymetrix Genotyping Console Contrast Quality Control test metric is a measure for this ability~\cite{qualitycontrol}, and was used in this thesis (Chapters~\ref{ch7}--\ref{ch8}). This test uses $10,000$ random SNPs to measure the difference between peaks in the distributions of homozygote genotypes (AA and BB), and the valleys these distributions share with the heterozygote peak (AB). When this difference approaches zero, the experiment poorly distinguishes between homozygous and heterozygous genotypes. Such chips should be removed from further data analysis.

%
\section{Microarray data analysis}\label{analysis1}
After having performed the preprocessing steps necessary for the specific type and platform of microarray data, the actual data analysis can be performed.

Unsupervised hierarchical clustering of microarray data may not only be used as a quality check (as described in the previous section), but can also be applied to detect different subgroups of samples, which may be associated with a clinical feature. In a supervised approach, differences between groups of samples can be determined using a moderated t-test, such as the {\it LIMMA} analysis (used in this thesis for detection of differential expression and phosphorylation)~\cite{smyth2004linear}. Important to note is that with the testing of multiple hypotheses, the amount of true null hypotheses that are rejected will increase. In microarray experiments, often large numbers of probes/peptides are tested for differential expression or phosphorylation, and therefore, an excessive amount of false\hyp{}positives may be returned from conventional statistical tests. Hence, a correction for multiple testing should be performed~\cite{allison2006microarray}. Examples of such methods are conservative familywise error rate procedures, such as the Bonferroni method~\cite{weisstein2006bonferroni}, or the less stringent false discovery rate (FDR) controlling methods, {\it e.g.} the Benjamini and Hochberg~\cite{benjamini1995controlling}, and Benjamini and Yekutieli~\cite{benjamini2001control} approaches. Other methods use permutations to estimate the FDR, such as Significant Analysis of Microarrays ({\it SAM})~\cite{tusher2001significance}.

SNP data is analyzed in a different manner. Genotyping can be performed by specific genotyping algorithms, such as the Birdseed v2 algorithm in Genotyping Console, which uses unsupervised learning to fit the data, producing genotype calls and returning confidence scores for each SNP~\cite{genotypingconsole}. Copy number data analysis is performed by comparing the intensity signals for each marker and each sample against a reference genome, which usually consists of a set of in-house or publicly available control samples. A cut-off for gains and losses is used to determine whether probes are present in a region of amplification or deletion (in this thesis, an absolute log$_2$ ratio cut-off of 0.2, equivalent to an absolute fold change of approximately 1.15, was used). Using the genotyping information, calls can also be made for allelic ratios. In Nexus Copy Number software, this is done by determining the B-allele frequency. Regions on the genome which show LOH will not reveal any AB signals (a B-allele frequency of 0.5, at least in theory, if there are no normal cells present in the tumor tissue). This also makes the identification of allelic imbalance possible, which, over a genomic region, will show multiple B-allele frequencies in between 0 and 1, depending on the amounts of copy number of each allele.

A drawback in SNP data analysis is that copy number changes are detected relative to the overall DNA content in the sample~\cite{attiyeh2009genomic}. In addition, normal cell populations, such as stromal and inflammatory cells, and heterogeneity within the tumor itself can further impede the detection of the true copy number alterations in the tumor cell. In epithelial tumors, a DNA index can be determined by flow\hyp{}sorting tumor cells, which can separate these from mesenchymal cells, and which can identify subpopulations of tumor cells with different chromosomal aberrations. To infer true copy numbers and allelic states, the algorithm lesser allele intensity ratio (LAIR, included in {\it beadarraySNP}~\cite{oosting2010beadarraysnp}) integrates the DNA index in the analysis of SNP data~\cite{corver2008genome}. Unfortunately, this approach can not be applied to SNP data analysis of high\hyp{}grade osteosarcoma samples, as osteosarcoma is a mesenchymal tumor for which no specific markers are available. However, the amount of stroma in osteosarcoma is not as extensive as in epithelial tumors, and the percentage of stroma as determined by the pathologist could in principle be used in order to approximate the DNA index of these tumors.

SNP microarray data show a high degree of noise, and not all markers reflect the true copy number of the region. Segmentation is performed in order to identify the chromosomal segments with actual copy number aberrations. Most frequently used algorithms for segmentation are Circular binary segmentation (CBS)\hyp{}based~\cite{olshen2004circular} or Hidden Markov Model (HMM)\hyp{}based methods. CBS\hyp{}based methods divide the genome into always smaller segments until no region can be further segmented, taking into account a minimum amount of probes per segment. The SNPRank segmentation algorithm in Nexus Copy Number Software is CBS\hyp{}based, and ranks log ratio probe values and B-allele frequencies in a segment. If the distribution of these probe ranks is significantly different from those of an adjacent segment, the region is segmented out, meaning the region probably has a different median copy number than that of the adjacent segment. HMM\hyp{}based methods, such as the SNP-FASST segmentation algorithm in Nexus Copy Number software, perform faster than CBS\hyp{}based methods, but require an estimate of signal--copy number relationship, as it works with integer copy numbers. Because of the heterogeneity present in tumor samples, this is probably not an optimal way to segment tumor data~\cite{rasmussen2011allele}. We used SNPRank segmentation to segment the copy number data, with a minimum of 5 probes per segment. After segmentation, a cut-off for frequency of copy number changes can be set, so that the most recurrent alterations will be detected. One can also specifically look for focal or broad events, as is described in Chapter~\ref{ch9} of this thesis. As with the analysis of other microarray data types, permutations can be used to determine whether there are significant differences in copy number or LOH profiles of groups with different features ({\it e.g.} in Nexus Copy Number software).

%
\section{Downstream data analysis}\label{downstream1}
Deducing a biological interpretation from large lists of significant genes may be challenging, and validation of all significant genes is often very labor intensive. Several methods have been developed which determine whether specific signal transduction pathways, biological processes, or other groups of genes with similar functions, are affected. Genes making up such pathways or processes are often taken from public databases, such as the Gene Ontology (GO)~\cite{ashburner2000gene} or the Kyoto Encyclopedia of Genes and Genomes (KEGG)~\cite{kanehisa2000kegg}, or are available as commercial software, such as Ingenuity Pathways Analysis (IPA, Ingenuity Systems), which is manually curated. The hypergeometric test (a one\hyp{}tailed Fisher's exact test) is most often used to obtain information on the enrichment of significant genes in specific pathways or biological processes. This test determines whether there is more overlap between the list of significant genes and the set of genes of interest ({\it e.g.} the pathway) than would be expected by chance. The hypergeometric test can be applied on microarray data in IPA (used in Chapter~\ref{ch6}) and in the Bioconductor {\it topGO} package (used in Chapters~\ref{ch3} and~\ref{ch7})~\cite{alexa2006improved}. A disadvantage of this simple test is that it requires a hard definition of significance ({\it e.g.} a p-value cut-off), and discards information on the exact p-values of the genes tested. The hypergeometric test also assumes independence of genes, which is not accurately representing the biology of a cell, since the expression of functionally related genes is often correlated. Because of this assumption, the hypergeometric test may understate the true p-values. It is therefore recommended to use a very low p-value ({\it e.g.} $0.001$ or lower) as cut-off for significance when applying this test. Another problem of the hypergeometric test is that it assumes independence of categories. GO terms are certainly not independent, as these terms are set up in a hierarchical structure of nodes, with parent terms representing a broader GO term, and child terms a more specific subset of its parent terms~\cite{ashburner2000gene,rhee2008use}. Algorithms which can identify the GO term which better represents the biological situation (significantly affected genes) than other terms from its neighborhood have been developed, such as the {\it weight} algorithm in the {\it topGO} package~\cite{alexa2006improved}.

A method which takes into account a continuous measure of significance is gene set enrichment analysis ({\it GSEA})~\cite{subramanian2005gene}. This method ranks genes based on their associated p-values and subsequently determines an enrichment score based on the rank of the genes present and not present in a specific pathway or category. The significance of this enrichment score is subsequently tested by permuting phenotype labels to determine the null distribution of the enrichment score. 

Another approach to determine which biological pathways are significantly affected is the {\it globaltest} (used in Chapter~\ref{ch5}). Based on a logistic regression model, this test determines whether a prespecified group of genes is differentially expressed, and thus tests groups of genes instead of single genes~\cite{goeman2004global}. This test is particularly intended for identifying gene sets for which many genes are associated with a phenotype in a small way. Using this approach may be especially fruitful in case no overall differential expression is detected due to small sample sizes, as this approach significantly reduces the multiple testing problem~\cite{goeman2005testing}. The {\it globaltest} has much more power than self\hyp{}contained tests (tests which compare a gene set with its complement), such as the hypergeometric test~\cite{goeman2007analyzing}. To apply the {\it globaltest} on GO terms, Goeman {\it et al}. also developed a method that preserves the specific graph structure of the Gene Ontology~\cite{goeman2008multiple}. In addition, this algorithm can be used in combination with follow\hyp{}up data~\cite{goeman2005testing}.

A final method to extract biological information from lists of significantly affected genes is performing network analysis. Networks are assembled {\it de novo}, based on connectivity ({\it e.g.} binding or functional properties) between affected molecules. In IPA, networks are assembled using decreasingly connected molecules from the significant genes in the dataset which is analyzed, and are annotated with functional categories, which are manually curated. In contrast to pathway analysis, these IPA networks do not have directionality (but network analysis methods which include directionality between molecules also exist). We used network analysis to interpret differential gene expression between various histological subtypes of osteosarcoma (Chapter~\ref{ch3}).

%
\section{Supervised learning}\label{supervised1}
Generating a prediction profile which can classify tumors based on mRNA expression or specific copy number aberrations may also be used in microarray analysis of a cancer dataset. Classification may for example help to diagnose a tumor based on its microarray data profile, or may predict event\hyp{}free or overall survival of patients. Some examples of supervised learning approaches are nearest shrunken centroids classification ({\it e.g.} available in Bioconductor package {\it pamr}~\cite{tibshirani2002diagnosis}), support vector machine (SVM) learning ({\it e.g.} available in R package {\it e1701}~\cite{dimitriadou2008misc}), and random forest classification ({\it e.g.} available in R package {\it varSelRF}~\cite{diaz2007genesrf}).

In this thesis, we used nearest shrunken centroids classification to develop a classifier of the main histological subtype of conventional osteosarcoma. We validated this classifier on an independent dataset, and applied it on data obtained from osteosarcoma model systems (Chapter~\ref{ch3}). Nearest centroids classification determines centroids for each class by dividing average expression of a gene signature by the standard deviation. New samples are classified to that specific class, of which the centroid is closest---in squared distance---to the expression of the genes in the prediction profile. Nearest shrunken centroids is an adaptation of this method---it shrinks each centroid toward the overall centroid for all classes by a certain threshold. This shrinkage automatically selects genes and reduces the effect of noisy genes. The profile with the lowest prediction error is then selected as the final classifier. Internal cross\hyp{}validation, which divides the training set in different parts, is subsequently used to compute a cross\hyp{}validated error. This approach, however, leads to an underestimation of the error rate, as the same data is used to select features and to estimate the error rate. An extra external cross\hyp{}validation step would thus be appropriate, or, given that there is often only a limited number of samples available for training, the feature selection (genes to include in the profile) should be newly computed for each separate cross\hyp{}validation step~\cite{wood2007classification,ambroise2002selection}. External cross\hyp{}validation is performed in order to correct for overfitting of the data by the model. This can be done on an independent cross\hyp{}validation set, or by using methods such as one\hyp{}leave\hyp{}out cross\hyp{}validation~\cite{simon2003pitfalls}. Also regularization may be used to prevent overfitting, but this is not often used in microarray data analysis, and is therefore beyond the scope of this thesis.

In prediction profiling, the way the distance between the actual sample and the class is calculated may be very different, and this has important consequences for biological interpretation of the profile. In a prediction profile where the magnitude ({\it e.g.} of gene expression) is important, Euclidian distance is best used, while correlation ({\it e.g.} Pearson or Spearman) coefficients are more useful when the way the genes depend on each other, so the pattern of expression, is important for the specific gene list~\cite{quackenbush2006microarray}. This may be one of the reasons why the CINSARC profile, a gene expression signature which was generated on sarcomas~\cite{chibon2010validated} and which uses Spearman correlation as a measurement for distance, did not show significant results on our osteosarcoma dataset (centroids for classifier needed to be retrained, because we used data of a different platform than the original CINSARC signature, Kuijjer {\it et al}., {\it unpublished results}), while the Carter signature~\cite{carter2006signature}, which classifies data based on average expression of genomic instability genes, could predict for metastasis\hyp{}free survival in our data (as shown in Chapter~\ref{ch7}).

%
\section{Data integration}\label{integration1}
As explained in the next chapter, the integration of different data types is particularly relevant when studying a highly genomically unstable tumor. We used superimposed integration of mRNA expression and kinome profiling data in Chapter~\ref{ch6}. This approach was taken, because kinase activity usually does not have a direct downstream effect on mRNA expression (generally, there are several intermediate molecules which confer signaling), and the other way around. It may therefore be more relevant to determine how these data complement each other, instead of identifying only overlapping genes.

For integration of copy number and gene expression (Chapters~\ref{ch7}--\ref{ch8}) data, we identified genes with aberrations occurring in both data types, as the copy number state of a gene can have a direct effect on its expression. We specifically chose to identify cooccurrence and not correlation of copy number and expression signals, because these signals do not have to show a linear correlation, {\it i.e.} correlation will miss our genes which are also regulated at other dimensions, such as epigenetics and feedback mechanisms.

A conservative approach was taken---only genes which were significantly differentially expressed between osteosarcoma tumors and presumed osteosarcoma progenitors were analyzed, and the cut-off for recurrence was set to 35\%. We tested this approach in a paired and nonpaired way to determine cooccurrence of copy number aberrations and differential expression in Chapter~\ref{ch7}, and used paired analysis of cooccurrence of LOH, copy number gains, and differential expression in Chapter~\ref{ch8}.

%
\section{Aims and outline of this thesis}\label{aims1}
In this thesis, a systems biology approach to study high\hyp{}grade osteosarcoma is described. Chapter~\ref{ch1} starts with an introduction on cancer genomics and high\hyp{}grade osteosarcoma, and introduces the EuroBoNeT high\hyp{}grade osteosarcoma database, on which the research in the following chapters is based. In addition, different platforms used in this thesis are described, and different types of high\hyp{}throughput data analyses are explained (this chapter).

In Chapter~\ref{ch2}, published literature on microarray studies on high\hyp{}grade osteosarcoma is reviewed. This review also discusses challenges in high\hyp{}throughput data analysis of osteosarcoma and introduces different model systems which have been used in osteosarcoma research. In addition, information on different comparative analyses and a rationale for integrating different data types are given. The review concludes with a section on how bioinformatics can be translated into functional studies.

The following six chapters of the thesis describe the work which has been performed to answer different research questions regarding osteosarcoma biology and possible targets for therapy. Specifically, we aimed to study molecular differences between clinically different tumors, such as tumors of different histological subtypes, and of tumors with different metastasis\hyp{}free survival profiles. These research questions are answered in Chapters~\ref{ch3} and~\ref{ch4}, respectively. In addition, in Chapter~\ref{ch3}, a histological subtype\hyp{}specific gene expression profile is tested on osteosarcoma model systems. High\hyp{}grade osteosarcoma is also compared with controls, in order to detect what signal transduction pathways may be targeted in osteosarcoma to identify potential adjuvant drugs for treatment of this aggressive tumor (Chapters~\ref{ch5} and~\ref{ch6}). Chapter~\ref{ch5} reports on the analysis of gene expression data, while Chapter~\ref{ch6} determines active pathways based on kinome profiling, and integrates gene expression data with kinome profiling results. Finally, we performed integrative data analysis of SNP and gene expression data, to detect osteosarcoma driver genes (Chapters~\ref{ch7} and~\ref{ch8}). In Chapter~\ref{ch7}, copy number aberrations are integrated with overexpression and downregulation, while in Chapter~\ref{ch8} we specifically look at the combination of Loss of Heterozygosity (LOH), DNA copy number gain, and differential mRNA expression.

In Chapter~\ref{ch9}, results described in Chapters~\ref{ch3} to~\ref{ch8} are discussed and future perspectives for high\hyp{}throughput data analysis on high\hyp{}grade osteosarcoma are given. Chapter~\ref{ch10} includes a Dutch summary, Curriculum Vitae, and a list of publications.


%%% references

\begin{small}
\begin{singlespace}
\bibliographystyle{unsrtnatshort}		% sorted as referenced, was unsrtnat, but unsrtnatshort gives shorter output
\bibliography{biblio}
\end{singlespace}
\end{small}

%\end{document}
% Marieke Kuijjer
% 2013-02-15
% chapter 02

	%\documentclass[12pt,b5paper]{book}
	%\setcounter{secnumdepth}{0}
	%\setcounter{tocdepth}{1}
	%\usepackage[hidelinks]{hyperref}

%\begin{document}

%%% title page

\chapter{Genome\hyp{}wide analyses on high-grade osteosarcoma: making sense of a genomically most unstable tumor}\label{ch2}
\thispagestyle{empty}				%%% to remove page number from first page of chapter, must be placed after calling the chapter

\vfill

\vspace{0.5cm}
This chapter is based on the review:
\underline{Kuijjer ML}, Hogendoorn PCW, Cleton-Jansen AM. \emph{Int J Cancer}. 2013 Feb 22;Epub

\newpage

%%% main document

%
\section{Abstract}\label{abstract2}
High-grade osteosarcoma is an extremely genomically unstable tumor. This, together with other challenges, such as the
heterogeneity within and between tumor samples, and the rarity of the disease, renders it difficult to study this tumor on a
genome\hyp{}wide level. Now that most laboratories change from genome\hyp{}wide microarray experiments to Next\hyp{}Generation
Sequencing it is important to discuss the lessons we have learned from microarray studies. In this review, we discuss the
challenges of high\hyp{}grade osteosarcoma data analysis. We give an overview of microarray studies that have been conducted so far on both osteosarcoma tissue samples and cell lines. We discuss recent findings from integration of different data types,
which is particularly relevant in a tumor with such a complex genomic profile. Finally, we elaborate on the
translation of results obtained with bioinformatics into functional studies, which has lead to valuable findings, especially
when keeping in mind that no new therapies with a significant impact on survival have been developed in the past decades.

%
\section{Introduction}\label{introduction2}
\subsection{High-grade osteosarcoma, a rare, genomically complex and unstable tumor}
High-grade osteosarcoma is the most prevalent primary malignant
bone tumor. The disease occurs most often in children
and adolescents and is the sixth leading cause of death
in children under the age of 15 years. Notwithstanding,
osteosarcoma is a rare disease, with an incidence of five to
ten new cases per $1,000,000$ per year~\cite{fletcher1994cytogenetic,raymond2002conventional}. Osteosarcoma is
composed of extremely genomically complex and unstable
mesenchymal tumor cells, generally exhibiting both complex
clonal and numerous nonclonal aberrations~\cite{fletcher1994cytogenetic}, which are characterized
by the direct production of osteoid~\cite{raymond2002conventional,helman2003mechanisms}. The tumor is
highly aggressive, with distant metastases developing in
approximately 45\% of all patients~\cite{pakos2009prognostic} although patients are
treated with intensive neoadjuvant treatment consisting of
high doses of multiple chemotherapeutic drugs. Better surgery
has improved survival slightly but no other significant
improvement has been made since decades, and increasing
dose or the administration of more than three chemotherapeutic
regimens does not increase overall
survival~\cite{lewis2007improvement,eselgrim2006dose,anninga2011chemotherapeutic}. Hence, new
therapeutics are seriously needed. Studying the tumor
biology and pathology in a systematic manner can result in a
better understanding of osteosarcomagenesis and can potentially
identify new targets for treatment.

\subsection{Caveats and challenges}
Several challenges and caveats are encountered when studying
a rare, highly genomically unstable tumor on a genome\hyp{}wide
level. The first challenge is apparent when collecting
osteosarcoma tumor samples. Osteosarcoma is a rare disease
and therefore often large interinstitutional efforts have to be
achieved to collect the substantial amount of samples that is
needed for analyses in computational biology. For most purposes,
studying osteosarcoma pretreatment diagnostic biopsies
is preferred over using resection material of the primary
tumor. Presurgery chemotherapy causes substantial necrosis,
even in poor responders, thereby rendering the tissue unsuitable
for high quality nucleic acid retrieval. Moreover, biopsies
are more representative of the state of the tumor before any
treatment as chemotherapy changes the distribution of subclones
present in the primary tumor, and can cause clonal
evolution~\cite{ding2012clonal}. Biopsies are taken to establish a histopathological
diagnosis, and are unfortunately often very small and not
always available for research. In addition, material is often
collected retrospectively, which can introduce heterogeneity
owing to, for example, different treatment procedures, unless
patients are collected who have been enrolled in the same
clinical trial. Thus, the collection of clinical data and the
grouping of clinical parameters have to be carried out very
carefully. For a rare entity such as osteosarcoma, collaborations
are indispensable to collect significant cohorts, an
example of this being the European Network of Excellence
EuroBoNeT, in which various European institutes collaborated
to collect a large, homogeneous set of, among other
bone tumors, high\hyp{}grade osteosarcoma biopsies.

Primary osteosarcoma is subdivided into numerous different
low- and high\hyp{}grade subtypes~\cite{mohseny2008bone}. In this review, we concentrate
on high\hyp{}grade conventional osteosarcoma, which is by
far the most prevalent variant. Although there is often intratumor
heterogeneity, high\hyp{}grade conventional osteosarcoma
can be grouped into various histological subtypes, based on
the produced extracellular matrix of the tumor~\cite{mohseny2008bone}. Osteoblastic,
chondroblastic and fibroblastic osteosarcoma are the most
common histological subtypes of high\hyp{}grade conventional osteosarcoma.
Some correlation of the distinct histological subtypes
to specific clinical outcomes has been observed~\cite{hauben2002does,hauben2006clinico} and
it may thus be difficult to collect a homogeneous set of samples.
In fact, often it is not clearly described which exact histological
subtypes are used in a specific study, and in what
percentages these subtypes are present in the data set. In
addition, the subclassification is hindered by the occurrence
of mixed cases containing two different matrix types. Nonetheless,
a concordance of 98\% has been found between the
histological subtype of osteosarcoma biopsies and the corresponding
resections~\cite{hauben2002does}.

A general problem in studying tumor cell biology is that
the true cell of origin is often not defined, rendering it difficult
to select a representative control tissue or control cells.
Osteosarcoma cells are osteoblast\hyp{}like cells of mesenchymal
origin. Of the different histological subtypes that exist, multiple
subtypes can be present within a single tumor. Considering
the differentiation capacity of the mesenchymal stem cell
(MSC), this cell type is the most probable candidate for being
the osteosarcoma progenitor~\cite{mohseny2011concise,bovee2003skeletogenesis}. It was recently found that
osteosarcoma tumors can be spontaneously formed when
mouse MSCs are transferred into mice~\cite{tolar2006sarcoma,mohseny2009osteosarcoma} and zebrafish~\cite{mohseny2012osteosarcomazebrafish,mohseny2012osteosarcomamodels}.
This does, however, not exclude osteoblasts as putative progenitor
cells, as osteoblasts might redifferentiate into the
primitive osteoblast\hyp{}like tumor cells of osteosarcoma.

\subsection{Osteosarcoma models}
As collecting fresh frozen osteosarcoma tumor samples can
be a challenge, performing analyses on data derived from osteosarcoma
cell lines or xenografts may be a good alternative~\cite{mohseny2012osteosarcomamodels}. Osteosarcoma cell lines are frequently used in
biological studies, because they generally grow fast and are
easy to maintain in culture and hence osteosarcoma cell lines
are easily available. One caveat of using cell cultures is that
slight differences in culture conditions, for example the percentage
of cells in the culture dish or flask, or the medium
that is used, can lead to significant differences in protein
expression or signal transduction pathway activities, and
these specific conditions may differ per cell line. Using a
large panel of cell lines cultured under standard settings can
overcome this problem. Cell culture may furthermore
introduce additional mutations and genomic aberrations in
the cell genome, because of selection based on the {\it in vitro}
conditions~\cite{weinberg2007cells}, but in general, cell lines are reported to
adequately represent the tumor from which they are derived.
{\it In vitro}, they preserve the genetic aberrations of the parent
tumor, while acquiring additional locus\hyp{}specific alterations~\cite{greshock2007cancer}.

A panel of 19 osteosarcoma cell lines was recently characterized
genetically by MLPA on 38 tumor suppressor gene
loci~\cite{ottaviano2010molecular}. A screen for {\it TP53} mutations, {\it MDM2} amplification,
{\it CDKN2A/B} deletion and genomic deletions of 38 additional
tumor suppressor genes was performed on these cell lines. As
three cell lines of this panel---HOS, 143B and MNNG-HOS---
have common ancestry, we report the following percentages
based on 17 cell lines. Homozygous deletion of the
{\it CDKN2A/B} locus was detected in 35\%, whereas hemizygous
deletion of this locus was found in 24\% of osteosarcoma cell
lines. An additional homozygous deletion was found for
{\it TP73} in one cell line. Mutation in {\it TP53} was detected in 41\%,
whereas {\it MDM2} amplification was detected in 17\% of cell
lines. These percentages are higher than those in osteosarcoma
tumor tissues that are reported in the previously published
literature~\cite{cleton2005central}, which may be explained by an advantage
for primary tumor cells harboring such mutations to be effectively
immortalized, or by the acquisition of additional mutations
owing to long\hyp{}term culture. {\it MDM2} amplification and
{\it TP53} mutations were mutually exclusive in this cell line
panel. This has also been observed in osteosarcoma tumor
data~\cite{overholtzer2003presence}. The differentiation capacity of this cell line panel has
been determined as well~\cite{mohseny2011functional}. All 19 cell lines were able to differentiate
toward at least one of the three tested---osteoblastic,
chondroblastic and adipocytic---lineages. Most cell lines
(14/19) could differentiate to at least two lineages, whereas
3/19 cell lines had full differentiation capacity.

{\it In vivo} osteosarcoma model systems include transplantation
of a human tumor in mice~\cite{mayordomo2010tissue,kresse2011preclinical}, subcutaneous or orthotopically
injections of osteosarcoma cells or late\hyp{}passage
transformed MSCs into mice~\cite{mohseny2009osteosarcoma}
or zebrafish~\cite{mohseny2012osteosarcomazebrafish}. Transgenic
mouse models of osteosarcoma can be developed by overexpression
of {\it c-fos}~\cite{ruther1989c}, or conditional inactivation of {\it TP53} and
{\it RB1}~\cite{walkley2008conditional}. These different models have been shown to resemble
osteosarcoma phenotypically~\cite{mohseny2009osteosarcoma,mohseny2012osteosarcomazebrafish,mohseny2011functional,mayordomo2010tissue,ruther1989c,walkley2008conditional,kuijjer2011mrna,kresse2011preclinical}. For example, subcutaneous
and intramuscular injection of osteosarcoma cells
in nude mice resulted in high\hyp{}grade sarcoma, resembling
tumors which produced osteoid~\cite{mohseny2011functional} for 8/19 cells from the
above\hyp{}described panel. The {\it in vivo} lineage\hyp{}specific differentiation
capacity of these cells, however, was limited, reflecting
the importance of stomal or microenvironmental stimulation
for this process.

As with cell lines, xenograft tumor cells may acquire additional
changes owing to selection, and often, xenografts lose
matrix after several passages~\cite{mayordomo2010tissue}. This will probably not have a
significant effect on genomic profiles, but does influence
expression and methylation patterns. High\hyp{}resolution microassay\hyp{}based array comparative genomic hybridization
(aCGH) including nine osteosarcoma patient--xenograft pairs
showed that genomes of human tumors transplanted into
immunodeficient mice, which were repeatedly passaged in
new mice, where comparable to genomes of their tumor of
origin, with the acquisition of only a small number additional
significant changes in the xenograft genomes~\cite{kresse2011preclinical}. Different
microarray studies have shown that osteosarcoma cell lines
and xenografts resemble the primary tumor from which they
are derived. Gene expression profiling of a subset of the
EuroBoNeT cell line panel, for which the original histological
subtype of the primary tumor was known, and of osteosarcoma
xenografts and pretreatment biopsies showed that,
despite the lower amounts of matrix, histological subtypespecific
mRNA signatures are retained in these model
systems, and therefore may be a useful tool for expression
analysis (Chapter~\ref{ch3},~\cite{kuijjer2011mrna}). Despite the similarities between genome and
expression profiles of the model systems described above and
the tumors of origin, the absence (cell lines) or lower
amounts (xenografts) of stromal cells and extracellular
matrix, the absence of interaction with the immune system
(cell lines and some xenograft models) and the higher degree
of clonality remain important limitations for studying tumor
biology using these model systems.

\subsection{Genome wide profiling to study osteosarcoma}
In the next sections, we describe different methods to analyze
specific types of microarray data, and give examples of
how results from bioinformatics can be translated into functional
studies. This review is not aiming to give a comprehensive
overview of all genome\hyp{}wide studies on
osteosarcoma, but rather illustrates and summarizes the
major findings on DNA/RNA microarray reports. A summary
of these findings is provided in Table~\ref{tab2.1}.
%
%%% OBS! references in this table are placed there by hand, as it would be difficult to place them here, but use text numbering
\afterpage{
	 \clearpage% flush all other floats
	 \ifodd\value{page}
%	\else% uncomment this else to get odd/even instead of even/odd
	\expandafter\afterpage% put it on the next page if this one is odd
	\fi
    {
\begin{landscape}
	\centering
	\footnotesize
	\begin{singlespacing}
		\begin{longtable}[c]{|>{\raggedright}p{0.7in} >{\raggedright}p{0.6in} >{\raggedright}p{1.35in} >{\raggedright}p{1.2in} >{\raggedright}p{2.0in} >{\raggedright}p{3.0in}|}
		\hline
		Analysis & Data type & Study & Osteosarcoma samples & Comparison & Pathway/genes\tabularnewline
		\hline
		Single-way & mRNA & Kuijjer {\it et al}.$^{28}$ & 76 B, 13 X, 18 C & Histological subtypes & NF$\upkappa$B in fibroblastic, chondroid\hyp{}matrix\hyp{}associated genes in chondroblastic osteosarcoma. Primary tumor expression signatures are preserved in model systems\tabularnewline
		& & Buddingh {\it et al}.$^{30}$ & 53 B & Metastasis\hyp{}free survival & Macrophage\hyp{}associated genes correlate
with better MFS\tabularnewline
		& & Su {\it et al}.$^{31}$ & 3 C, 5 X & Capacity to metastasize & {\it IGFBP5} downregulation correlates with metastasis\tabularnewline
		& & Naml{\o}s {\it et al}.$^{33}$ & 12 B/T, 11 M & Tumor sample type & Immunological processes and chemokine
pattern upregulated in metastases\tabularnewline
		& & Cleton-Jansen {\it et al}.$^{32}$ \\ Kuijjer {\it et al}.$^{28}$ & 25 B \\ 69 B & Response to chemotherapy & No significant differential expression\tabularnewline
		& & Cleton-Jansen {\it et al}.$^{32}$ & 25 B & Control samples (osteoblastoma, MSC, osteoblast) &  Cell\hyp{}cycle regulation, DNA replication pathways\tabularnewline
		& & Sadikovic {\it et al}.$^{42}$ & 6 B & Control sample (osteoblast) &  DNA replication network\tabularnewline
		& & Kuijjer {\it et al}.$^{43}$ & 84 B & Control samples (MSC, osteoblast) &  Apoptosis, signal transduction\tabularnewline
		& & Kansara {\it et al}.$^{44}$ & 5 C & Treatment with demethylating agent &  WIF1 methylation and downregulation\tabularnewline
		& miRNA & Jones {\it et al}.$^{45}$ & 18 B & Control samples (normal bone) &  miR-16 Downregulation, miR-27a association with metastasis\tabularnewline
		& CN & Kresse {\it et al}.$^{25}$ & 9 T/M and their derived xenografts &  Tumor sample type & Xenografts are representative for primary tumors although some additional aberrations are observed\tabularnewline
		& & Squire {\it et al}.$^{50}$ \\ Man {\it et al}.$^{51}$ \\ Atiye {\it et al}.$^{52}$ \\ Yang {\it et al}.$^{53}$ \\ Kresse {\it et al}.$^{54}$ \\ Kuijjer {\it et al}.$^{43}$ \\ Lockwood {\it et al}.$^{55}$ \\ Yen {\it et al}.$^{56}$ \\ Smida {\it et al}.$^{58}$ \\ Pasic {\it et al}.$^{59}$ & 9B \\ 48 B/T/M \\ 22 C/TS/R \\ 20 B \\ 36 TS/M/X, 20 C \\ 32 B \\ 22 TS \\ 42 TS/R/M/C \\ 45 B \\ 27 B & Control samples & Overall high level of aneuploidy, which seems nonrandom. Regions described by three or more studies are gains on 1p, 6p, 8q, 12q and 17p and losses on 2q, 3q, 6q, 10, 13q and 17p\tabularnewline
		& & Kuijjer {\it et al}.$^{43}$ \\ Smida {\it et al}.$^{58}$ & 32 B \\ 45 B &  Metastasis/event\hyp{}free survival & Genomic alterations are prognostic predictors\tabularnewline
		& & Yen {\it et al}.$^{56}$ & 23 TS, 14 R/M & Tumor sample type & Identified deletions/amplifications which differ between TS and R/M\tabularnewline
		& & Kresse {\it et al}.$^{54}$ \\ Yen {\it et al}.$^{56}$ \\ Pasic {\it et al}.$^{59}$ & 36 TS/M/X, 20 C \\ 42 TS/R/M/C \\ 27 B & Control samples & Frequent deletion of {\it LSAMP}\tabularnewline
		& & Yang {\it et al}.$^{53}$ & 20 B & Control samples & Enrichment of VEGF pathway\tabularnewline
		Integrative & CN, mRNA & Kuijjer {\it et al}.$^{43}$ & 29 B & Control samples (MSC, osteoblast) & Set of 31 candidate drivers enriched in genes with a role in genomic instability\tabularnewline
		& & Lockwood {\it et al}.$^{55}$ & 22 TS, 8 X & Control samples (normal tissues) & Amplification and overexpression of cyclin E3\tabularnewline
		& miRNA, mRNA & Jones {\it et al}.$^{45}$ & 14 B & Control samples (normal bone) & Transcriptional regulation, cell cycle control and cancer signaling\tabularnewline
		& & Naml{\o}s {\it et al}.$^{46}$ & 19 C & Control samples (normal bone) & Pairs of miRNAs with 26 mRNAs\tabularnewline
		& CN, mRNA, methylation & Sadikovic {\it et al}.$^{48}$ & 2 C & Control sample (osteoblast) & Hypomethylation of genes connected to c-Myc\tabularnewline
		& & Sadikovic {\it et al}.$^{42}$ & 5 B & Control sample (osteoblast) & RUNX2 amplification and overexpression, {\it DOCK5} and {\it TNFRSF10A/D} loss and underexpression, hypomethylation, gain and overexpression of histone cluster 2 genes\tabularnewline
		& & Kresse {\it et al}.$^{49}$ & 19 C & Control samples (osteoblast, normal bone) & 350 genes with two aberration types, including {\it RUNX2} and {\it DLX5} amplification and overexpression\tabularnewline
		\hline
		\caption{Overview of genome\hyp{}wide data analyses in high\hyp{}grade osteosarcoma. The table gives an overview of single\hyp{}way and integrative analyses described in this review. For each study, the sample type and sample size is given under Osteosarcoma samples column and the comparison which is made in the bioinformatics analysis, for example, comparison with control tissue, is shown in the Comparison column. Several studies used different sample types in one group. When this was done, these sets are shown in the table as combined into one group as well. Groups of different sample types which have been used in separate analyses are shown as different groups. Not always, it is clear whether na{\"i}ve tumor biopsies, untreated primary tumor resections or resections of treated primary tumors were used. For such studies, we have used the abbreviation TS (for tumor sample). B: na{\"i}ve tumor biopsies, T: resections of primary tumors, R: resections of recurrences, M: metastatic resections, C: cell lines, X: xenografts.}
		\label{tab2.1}
		\end{longtable}
	\end{singlespacing}
\end{landscape}
}
}
%
With the
purpose to review bioinformatic analyses on osteosarcoma,
we only review studies where at least three samples were
included, and only refer to articles where robust statistical
analyses have been applied.

%
\section{Single platform analyses of osteosarcoma genome\hyp{}wide data}\label{single2}
\subsection{Different approaches for single\hyp{}way analyses}
In a typical supervised genome\hyp{}wide data analysis, significant
differences, for example significantly differentially expressed
genes/miRNAs or differential methylation, are determined
between two or more groups of samples. These groups can
exist of different clinical parameters, of tumor samples and
their nontumorigenic counterpart or of experimentally
induced and noninduced samples as shown in Figure~\ref{fig2.1}.
%
\begin{figure}[htbp]
  \centering
    \includegraphics[width=1\textwidth]{figs02/fig1bw.pdf}	% pdf version also bw
    \caption{Different supervised comparisons in genome\hyp{}wide data
analysis. Flow chart describing single\hyp{}way bioinformatic analyses
that are most typically performed on genome\hyp{}wide data. For mRNA,
miRNA and methylation data analysis, the comparative analysis
usually exists of tumor samples versus nontumorigenic counterparts,
of different groups of tumor samples, defined by clinical parameters
or samples which are experimentally altered compared to
samples which are not, although tumor samples of a specific
group are also sometimes compared to a pool of all samples (not
illustrated in this figure). Copy number data are most often compared
to a reference set, which may be an in-house, or a public
reference set, and which does not have to consist of the nontumorigenic
counterpart of the tumor that is studied. Additional comparative
analyses may determine the differences between different
subgroups within the samples that are studied. Although for
mRNA, miRNA and methylation data, often significant differential
expression/methylation is returned by statistical tests, for copy
number data researchers mostly look at frequency of the aberration
in the studied groups.}
     \label{fig2.1}
\end{figure}
%
Copy
number profiling data are analyzed somewhat differently, as
copy number profiles of tumor samples do not necessarily
have to be compared to their specific nontumorigenic counterparts,
but can be compared to, for example, a public reference
set, such as HapMap samples~\cite{gibbs2003international}. Usually, a cutoff for
frequency is used to determine whether an amplification or
deletion is recurrently present in a specific region. Unsupervised
analysis, on the other hand, can give information on
quality of the data, and on whether there are certain subgroups
within the tumor samples that behave differently.

Each of these distinct ways to analyze genome\hyp{}wide data
has been applied to high\hyp{}grade osteosarcoma data sets. An
overview of these different approaches in osteosarcoma on
gene expression, microRNA (miRNA), methylation and copy
number data is given in the following paragraphs. Functional
verification of the results obtained with these studies will be
discussed in a later section of this review.

\subsection{Genome\hyp{}wide gene expression data, comparison of clinical parameters}
Comparisons between different clinical subgroups of osteosarcoma
have resulted in a prediction profile that can classify
the main histological subtypes of conventional high\hyp{}grade osteosarcoma
in biopsy material, but also in cell lines and in
osteosarcoma xenografts (Chapter~\ref{ch3},~\cite{kuijjer2011mrna}). Protein interaction networks illustrated
that chondroid matrix\hyp{}associated proteins were overexpressed
in chondroblastic osteosarcoma, whereas NF$\upkappa$B--STAT5
signaling showed higher expression in fibroblastic osteosarcoma.
The absence of a specific network for osteoblastic
osteosarcoma indicates that the features of the main osteoblast\hyp{}like
cell and of the osteoid matrix are present in tumors
of all three main histological subtypes.

A second example of a comparison between different clinical
parameters is the comparison of samples with different
outcomes in event\hyp{}free survival or overall survival. It is important
to note that when designing an analysis for such a
study, a uniform set of clinical follow\hyp{}up parameters should
be employed, instead of directly comparing patients with or
without metastases, or patients who are alive or deceased. In
one study, differential expression was determined between biopsy
material of patients developing metastases within 5
years and patients who did not develop metastases within
this time frame. This study demonstrated that, in osteosarcoma,
an expression profile associated with macrophages correlated
with better overall survival (Chapter~\ref{ch4},~\cite{buddingh2011tumor}). To identify genes
playing a role in metastasis, comparisons between osteosarcoma
cell lines that can or cannot metastasize upon passaging
into mice have also been made. A recent study identified
downregulation of {\it IGFBP5}, or insulin\hyp{}like growth factor
binding protein 5, in the metastatic cell line MG63.2 and in
tumors derived from this cell line~\cite{su2011insulin}. Interestingly, this gene
was also significantly downregulated in our analysis, comparing
osteosarcoma biopsies with control tissues~\cite{cleton2009profiling}). Metastasis
progression can be studied by comparing metastatic resections
to the primary tumor. This has been performed in one
study, where higher expression of genes involved in immunological
processes was detected in the metastasis samples~\cite{namlos2012global}.
This may correlate with our findings that more CD14\textsuperscript{+} cells
are present in metastatic lesions than in pretreatment
biopsies~\cite{buddingh2011tumor}.

Another important clinical parameter that has been studied
in human osteosarcoma is response to chemotherapy,
which is predictive for overall survival~\cite{bacci1998predictive,huvos1991bone,bacci2006prognostic}. Differentially
expressed genes discriminating between good and poor responders
to chemotherapy have been detected by different
groups, but with little consensus in the gene lists. Most studies
did not use robust statistics with correction for multiple
testing, a shortcoming that is too often seen in biomedical
research~\cite{dupuy2007critical}. When differential expression was determined
between poor and good responders in two studies where correction
for multiple testing was applied (Chapter~\ref{ch3},~\cite{kuijjer2011mrna}, and~\cite{cleton2009profiling}), no significant
genes were detected although larger sample sizes and homogeneous
data sets were used (17 poor {\it vs} 8 good responders
in Cleton-Jansen {\it et al}.~\cite{cleton2009profiling} and 36 poor {\it vs} 33 good responders
in Kuijjer {\it et al}.~\cite{kuijjer2011mrna}). Although these sample sizes are not comparable
to what is often used for studying less rare tumor
types, the distribution of the nonadjusted p-values did not
show any trend for the lower p-values to be more prevalent
(Additional Figure~\ref{afig2.1}). This indicates that in a comparison
between two groups no effect is detected, and increasing
sample size will not lead to a significant increase in power~\cite{van2009relative}. A
major issue with comparing responders with nonresponders in
gene expression analysis is that resistance to chemotherapy may
be caused by the alteration of a single gene. A specific gene
causing resistance in a subset of samples will not be picked up
by a comparison of responders and nonresponders~\cite{borst2010predictive}.

In human osteosarcoma xenografts, significant differential
expression has been detected between good and poor responders
to single chemotherapeutic agents~\cite{bruheim2009gene}. A pitfall of this
study, however, was that the studied sample set included xenografts
derived from biopsies, resections as well as from metastases.
Surviving cells of pretreated tumors are resistant to
chemotherapy. Thus, the differences in gene expression
between poor and good responders to these chemotherapeutic
agents may actually reflect an effect of presurgery therapy.
It was indeed demonstrated that xenografts of these
implanted pretreated tumors often responded poorly to multiple
chemotherapeutic agents~\cite{bruheim2004human}.

\subsection{Genome\hyp{}wide gene expression data, comparison with control tissues}
mRNA expression levels in osteosarcoma samples can also be
compared to expression in control tissues. The control tissues
that have been used for this purpose are normal bone, osteoblastoma,
osteoblasts, MSCs, or, for example, a pool of different
cell lines. One comparison of high\hyp{}grade osteosarcoma
biopsy specimens with control samples is described in Cleton-Jansen {\it et al}.~\cite{cleton2009profiling}, who made different comparisons of 25 osteosarcoma
biopsies with five osteoblastomas, with five MSCs
and with five osteoblast cultures. Gene set enrichment
detected cell\hyp{}cycle regulation and DNA replication pathways
as the most significantly affected pathways in osteosarcoma.
A DNA replication network was also identified in an analysis
of gene expression microarrays of six osteosarcoma biopsies
as compared to one osteoblast culture although a caveat of
this study is the small sample size of the control set
($n=1$)~\cite{sadikovic2009identification}. A larger set of osteosarcoma biopsies ($n=84$) was
compared to 12 MSCs and separately with three osteoblast
cultures (Chapter~\ref{ch7},~\cite{kuijjer2012identification}). Intersection of the differentially expressed genes in
both analyses identified antigen processing and presentation
as well as angiogenesis as significantly different between tumor
samples and control cell lines, most probably because of
the amount of stroma present in the tumor samples. In addition,
altered apoptosis and signal transduction were detected.

\subsection{Genome\hyp{}wide gene expression data, experimentally induced differences}
We give a final example of genome\hyp{}wide gene expression
analyses in osteosarcoma, which is experimentally induced
differential expression. This is, for example, reported in the
study by Kansara {it et al}.~\cite{kansara2009wnt}, who compared a set of five human
osteosarcoma cells treated with a demethylating agent to
untreated cells, after having shown that demethylating agents
can induce growth arrest and differentiation in osteosarcoma.
The list of candidate genes was then filtered for expression in
human osteoblasts and loss of expression in primary osteosarcomas.
This screen identified {\it WIF1}, a Wnt inhibitory factor,
as a candidate tumor suppressor in osteosarcoma.

\subsection{microRNA expression data}
Several studies have been published, describing miRNA
microarray data analysis on osteosarcoma tissues or cell lines
as compared to osteoblasts or normal bone, but in most studies
no robust statistics were applied. Jones {\it et al}.~\cite{jones2012mirna} and
Naml{\o}s {\it et al}.~\cite{namlos2012modulation} published the only miRNA microarray studies
in which false discovery rate corrections were applied. In
the article by Jones {\it et al}.~\cite{jones2012mirna}, miRNA expression was compared
between 18 osteosarcoma resections or biopsies and 12 normal
bone samples, which lead to the detection of a downregulated
tumor suppressive miRNA and of a prometastatic
miRNA (these miRNAs will be discussed in the Integrative % OBS! add link here and to section
analyses section). Naml{\o}s {\it et al}.~\cite{namlos2012modulation} compared miRNA expression
in 19 osteosarcoma cell lines with expression in normal
bone ($n=4$) and integrated these results with mRNA expression
data. Results from this study will therefore be discussed
in the Integrative analyses section. Sarver {\it et al}.~\cite{sarver2010s} published % OBS! add link here and to section
an online accessible Sarcoma miRNA expression database
(S-MED), which includes 15 osteosarcoma samples and six
normal bone samples.

\subsection{Genome\hyp{}wide methylation data}
Only three studies have been published so far on genome\hyp{}wide
methylation in high\hyp{}grade osteosarcoma~\cite{sadikovic2009identification,sadikovic2008vitro,kresse2012integrative}. These
studies describe an integrative analysis with different data
types, without presenting conclusions on specific genes, or on
results obtained with gene set enrichment on single\hyp{}way
methylation analyses although Kresse {\it et al}.~\cite{kresse2012integrative} found overall
more hypermethylation in osteosarcoma cell lines than hypomethylation.
We will discuss the results from these studies
under the Integrative analysis section of this review. % OBS! add link here and to section

\subsection{Genomic copy number data}
The genomic instability of high\hyp{}grade osteosarcoma, which is
more pronounced in this tumor than in many other tumor
types, hampers the identification of specific genomic regions.
Several array comparative genomic hybridization (aCGH)
studies~\cite{kresse2011preclinical,squire2003high,man2004genome,atiye2005gene,yang2011genetic,kresse2009lsamp,lockwood2011cyclin} and single\hyp{}nucleotide polymorphism (SNP)
microarray studies~\cite{kuijjer2012identification,yen2009identification,kresse2010evaluation,smida2010genomic,pasic2010recurrent} on osteosarcoma specimens have
been published. Copy number profiles clearly show that
high\hyp{}grade osteosarcoma samples are characterized by a high
level of aneuploidy, and that there is heterogeneity between
different tumor samples. There is a general consensus about
copy number alterations for some regions, such as gains on
chromosome arms 6p, 8q and 17p, which have been detected
by classical karyotyping and conventional CGH as well~\cite{raymond2002conventional,lau2003frequent},
but it is difficult to directly compare studies as the definition
of a recurrent alteration varies.

In three separate studies, a focal deletion of the region
3q13.31, which harbors a putative tumor suppressor gene,
{\it LSAMP}, was detected~\cite{kresse2009lsamp,yen2009identification,pasic2010recurrent}. siRNA\hyp{}mediated silencing of
{\it LSAMP} promoted proliferation of normal osteoblasts~\cite{pasic2010recurrent}, and
low expression of the gene was associated with poor overall
survival in one of these studies~\cite{kresse2009lsamp}. Gene set enrichment on an
aCGH study showed an enrichment of amplified genes of the
VEGF signaling pathway of which {\it VEGFA} amplification also
correlated with poor prognosis, showing that gene set enrichment
on copy number data can identify pathways associated
with tumorigenesis~\cite{yang2011genetic}.

Copy number profiles of osteosarcoma cell lines roughly
resemble profiles of tumor biopsies, but show an increased
overall aneuploidy (Kuijjer {\it et al}., {\it unpublished data}) and
increased expression of genomic instability genes (Chapter~\ref{ch7},~\cite{kuijjer2012identification}). Also, as
described above, genomic profiles of xenografts are highly
similar to primary tumors although some deviations may
occur owing to additional genetic alterations during passaging,
or owing to general tumor progression~\cite{kresse2011preclinical}. In some data
sets, specific copy number alterations have not been detected
for different clinical groups of interest~\cite{kuijjer2012identification} although both a high
degree of genomic alterations and a loss of heterozygosity
were found to be associated with poor event\hyp{}free survival~\cite{kuijjer2012identification,smida2010genomic}.
Yen {\it et al}.~\cite{yen2009identification} found that specific aberrations were more frequent
in recurrences and metastases than in primary
tumors---deletion of 6q14.1-q22.31 and 8p23.2-p12 and
amplification of 8q21.12-q24.3 and 17p12---and vice versa---Xp11.22 gain and 13q31.3 deletion~\cite{yen2009identification}.

%
\section{Integrative analyses}\label{integrative2}
For high-grade osteosarcoma, integration of different data
types is of specific importance. Integrative analysis can narrow
down the large lists of significantly affected genes to a
gene list containing the major tumor driver genes. An integrative
approach on copy number and gene expression data,
for example, typically returns a more specific list of driver
genes because passenger- and tissue\hyp{}specific genes will be
largely eliminated~\cite{lee2008integrative}. Different methods exist for the integration
of different types of data. Figure~\ref{fig2.2} shows an overview of
direct dependencies between copy number, methylation,
miRNA and mRNA data.
%
\begin{figure}[htbp]
  \centering
  \begin{minipage}[b]{0.50\linewidth}
    \includegraphics[width=1\textwidth]{figs02/fig2bw.pdf}	% pdf version also bw
  \end{minipage}
    \hfill
  \begin{minipage}[b]{0.46\linewidth}
    \caption{Flow chart showing
direct dependencies between different data types, which can be
utilized for the interpretation of integrative analyses. Arrow\hyp{}headed
and bar\hyp{}headed lines show positive and negative influences,
respectively. DNA copy number positively affects miRNA and mRNA
copies, whereas miRNA expression can cause downregulation of
target mRNAs, and DNA methylation can inhibit transcription.}
     \label{fig2.2}
     \end{minipage}
\end{figure}
%
Comparison of data can be performed
nonpaired or paired, and by determining correlation
or cooccurrence.

Cooccurring genomic alterations and gene expression
changes have been recently determined to identify putative
driver genes in high\hyp{}grade osteosarcoma (Chapter~\ref{ch7},~\cite{kuijjer2012identification}). A paired integrative
analysis of 29 pretreatment biopsies returned a list of 31
genes with recurrence frequency of at least $35\%$, which showed an
overall significant upregulation as compared to control cell
lines in case of a gain, and downregulation in case of a deletion.
Genes affecting genomic stability were overrepresented,
which may point to a role of this process in osteosarcoma.
Nonpaired analysis on the same series, but extended with
more cases in both the SNP and the gene expression data
sets resulted in a smaller set of significantly affected genes,
with substantial overlap with the list of genes detected by the
paired analysis, thereby showing that the paired analysis was
more powerful on this data set. This is especially of interest
for the data analysis of osteosarcoma pretreatment biopsies
because these samples are rare. By performing a paired analysis,
fewer samples can be used. Nonpaired integrative analysis
of high\hyp{}level amplifications in 22 osteosarcoma specimens
with gene expression data of eight osteosarcoma xenografts
as compared to 19 normal tissue controls identified 43 genes
with high\hyp{}level amplification and overexpression in osteosarcoma.
{\it CCNE1}, the gene encoding for cyclin E1, showed correlation
of copy number levels and gene expression in an
additional panel of ten osteosarcoma cell lines, and therefore
could play an oncogenic role in osteosarcoma~\cite{lockwood2011cyclin}.

miRNA expression data can be integrated with mRNA
expression data to determine whether the miRNAs of interest
affect mRNA expression of their target genes. This is
generally performed by correlation of expression levels, as
was performed by Baumhoer {\it et al}.,~\cite{baumhoer2012microrna} Naml{\o}s {\it et al}.~\cite{namlos2012modulation} and Jones {\it et al}.~\cite{jones2012mirna} (discussed above). The latter subsequently performed
pathway analysis on target genes of the detected differentially
expressed miRNAs, which illustrated the effects
of these miRNAs on transcriptional regulation, cell\hyp{}cycle
control and known cancer signaling pathways~\cite{jones2012mirna}. In the
study by Naml{\o}s {\it et al}.~\cite{namlos2012modulation}, cell line miRNA data were integrated
with mRNA targets which were significant in both
osteosarcoma pretreatment biopsies and cell lines. Among
the inversely correlated miRNA/mRNA pairs, miRNAs regulating
{\it TGFBR2}, {\it IRS1}, {\it PTEN} and PI3K subunits were
detected. Methylation data are also typically integrated with
mRNA expression data to evaluate the effect of the methylation
on gene expression, but few studies described two\hyp{}way
comparisons of methylation and mRNA microarray data in
osteosarcoma. Kresse {\it et al}.~\cite{kresse2012integrative} detected hypermethylation and
underexpression of chemokine ligand 5 ({\it CXCL5}) by two\hyp{}way
comparison in both osteosarcoma cell lines and tumor
samples.

Integration of more than two different data types is
reported by Sadikovic {\it et al}.~\cite{sadikovic2008vitro} in two articles, where copy
number, methylation and gene expression data were integrated.
In one of these articles, the authors described cooccurrent
epigenetic, genomic and gene expression changes
in two osteosarcoma cell lines as compared to an osteoblast
culture, and detected a region of gain on chromosome 8q
encompassing the {\it c-MYC} oncogene, which was also detected
in a network analysis, confirming overexpression and hypomethylation
of genes connected to {\it c-MYC}. In the second article,
the authors used the same integrative approach to
perform a three\hyp{}way analysis on five osteosarcoma pretreatment
biopsies, and to compare gene regulation networks of
single\hyp{}way analyses including more samples. In this way, a
number of candidate genes were characterized, including
{\it RUNX2}, a transcription factor involved in osteoblastic differentiation~\cite{sadikovic2009identification}. A
shortcoming of both studies, however, is that
as a control for methylation and mRNA expression in
osteosarcoma, material from only one osteoblastic culture
was used. Another integrative analysis on copy number,
methylation and mRNA data reported 350 genes, showing
two types of aberrations ({\it e.g.} gain and overexpression, or hypermethylation
and underexpression). This set of genes was
enriched in genes with a function in skeletal system development
and extracellular matrix remodeling, such as {\it RUNX2}
and {\it DLX5}~\cite{kresse2012integrative}.

%
\section{Translating bioinformatics into functional studies}\label{translating2}
\subsection{Functional validation of candidate genes}
Several of the candidate tumor suppressor genes and oncogenes
that have been identified with microarray studies have
been functionally validated. {\it IGFBP5} was significantly downregulated
in metastatic cell lines and derivative tumors as compared
to nonmetastatic cell lines, and also showed lower
protein expression in metastatic lesions than in primary tumor
samples of osteosarcoma patients. The effects of overexpression
or knockdown of {\it IGFBP5} on cell proliferation,
migration, wound healing and invasion confirmed the role of
this IGF\hyp{}binding protein in preventing metastasis, which was
furthermore validated in a xenograft model~\cite{su2011insulin}.

The candidate tumor suppressor gene {\it WIF1} was found to
regulate differentiation and suppress cell growth {\it in vitro}.
{\it WIF1} knockout mice developed radiation\hyp{}induced osteosarcoma
earlier than their littermate controls~\cite{kansara2009wnt}. From miRNA
expression profiling studies, miR-16 was validated as a tumor
suppressive miRNA, whereas miR-27a was validated as a
prometastatic miRNA, using colony formation assays, and
wound healing and invasion assays, respectively. Overexpression
of these miRNAs {\it in vivo} resulted in smaller tumors for
miR-16, and in higher numbers of pulmonary metastases for
miR-27a~\cite{jones2012mirna}.

\subsection{Functional validation of pathway activity and enriched gene sets}
Pathways important in the development of bone biology have
been returned from gene expression analysis as compared to
controls. Genes upstream canonical Wnt signaling were, for
example, found to be downregulated as compared to osteoblasts~\cite{cleton2009profiling}. A
subsequent functional study, where nuclear $\upbeta$-catenin
staining was determined on osteosarcoma biopsies, and
Wnt luciferase activity and mRNA expression of the specific
downstream Wnt target gene Axin2 were measured in cell
lines, illustrated that canonical Wnt signaling is indeed often
downregulated in osteosarcoma~\cite{cai2010inactive}. Loss of canonical Wnt signaling
causes failure to commit to differentiation of MSCs, as
has been reported in malignant fibrous histiocytoma (also
undifferentiated pleomorphic sarcoma), which could be
reprogrammed by re\hyp{}establishing Wnt signaling~\cite{matushansky2007derivation}. Also in osteosarcoma,
reactivation of the Wnt signaling pathway with a
GSK3$\upbeta$ inhibitor triggered a more differentiated phenotype,
or a reduced proliferation capacity, depending on the osteosarcoma
cell line~\cite{cai2010inactive}.

These results seem contradictive to the finding that {\it WIF1}
can inhibit cell growth and increase differentiation in osteosarcoma
cells~\cite{kansara2009wnt}. A possible explanation for this discrepancy is
that {\it WIF1} inhibits both canonical and noncanonical Wnt signaling~\cite{malinauskas2011modular},
whereas GSK3$\upbeta$ also plays a role in additional signal
transduction pathways, such as NF$\upkappa$B signaling~\cite{tang2012glycogen}. However,
the role of Wnt signaling remains contradictory, as this pathway
was recently described to be active in multiple sarcoma
subtypes, which also included osteosarcoma~\cite{vijayakumar2011high}. The use of
different methods to assess active Wnt signaling may be the
cause for the discrepancies between these studies.

TGF-$\upbeta$/BMP signaling was found to be affected in osteosarcoma
by pathway analysis on mRNA expression data. Activity
of these pathways was validated by immunohistochemistry of
phosphorylated Smad1 and Smad2, and nuclear staining of
these intracellular effectors was detected in 70\% of all osteosarcoma
samples. Cases with very low or absent phosphorylated
Smad2 had worse overall survival. {\it In vitro} pathway modulation
did not affect proliferation or differentiation, but lower TGF$\upbeta$/BMP
activity might affect the prevention of metastasis in
these patients~\cite{mohseny2012activities}.

The macrophage signature that was prominent upon comparing
mRNA profiles of metastatic and nonmetastatic osteosarcoma
was confirmed by qPCR and immunohistochemistry,
and it was shown in additional cohorts that the sum of M1
and M2 types of macrophages was predictive for better overall
survival (Chapter~\ref{ch4},~\cite{buddingh2011tumor}). Treating patients with macrophage\hyp{}activating
agents may reduce metastases of osteosarcoma~\cite{cleton2012immunotherapy}. This is corroborated
by clinical trials in dogs and humans, where treatment
with mifamurtide, a macrophage\hyp{}activating agent, has
been reported to positively affect overall survival~\cite{kurzman1995adjuvant,meyers2008osteosarcoma}.

%
\section{Conclusions and future directions}\label{conclusions2}
In this review, we have presented and discussed the results of
studies on high\hyp{}grade osteosarcoma material using bioinformatic
analysis on microarray data of three or more samples.
Although studying such a very heterogeneous and genomically
unstable tumor remains challenging, and sample sizes
are often small owing to the rarity of the disease, structured
microarray data analysis has provided interesting results and
has given further insight into the biology and progression of
osteosarcoma. This information could not have been obtained
from functional studies only. Studying copy number aberrations,
differential expression, and epigenetics in a genome\hyp{}wide
manner and subsequent integration leads to new
hypotheses regarding tumor development and progression,
which can subsequently be validated in functional studies.
This provides a motivation to take the study of high\hyp{}grade
osteosarcoma to the next level, and to analyze this tumor
into further detail using Next Generation Sequencing methods,
such as whole\hyp{}genome, exome or transcriptome sequencing.
Whole\hyp{}genome sequencing has recently been performed
in a study of different cancer types, which showed that a subset
of osteosarcomas (three out of nine) undergo chromothripsis---a
single catastrophic genomic instability event,
resulting in hundreds of genomic rearrangements~\cite{stephens2011massive}. This may
explain the sudden onset of osteosarcoma and the complexity
and heterogeneity of the osteosarcoma genome. Next Generation
Sequencing will provide us with many forms of new information.
In addition to copy number changes, mutations,
translocations, unannotated genes, splicing variants, and so
on, can be detected in a high\hyp{}throughput manner. Transcriptome
sequencing exhibits higher sensitivity and increased
dynamic range than mRNA expression microarray data,
thereby providing higher power for the detection of differential
gene expression~\cite{oshlack2010rna}. Now that the first Next Generation
Sequencing studies including large numbers of high\hyp{}grade osteosarcoma
are ongoing or being planned, it is important to
reflect on the previous genome\hyp{}wide studies in osteosarcoma.
When keeping in mind the lessons we have learned on study
design in microarray data analysis---using a sufficient amount
of samples, defining homogeneous groups, and analyzing the
data with robust statistics---we will be given new opportunities
in unraveling the biology of this complex disease and in
providing future clinical trials with robust data to incorporate
into novel therapeutic strategies.

%%% references

\begin{small}
\begin{singlespace}
\bibliographystyle{unsrtnatshort}		% sorted as referenced, was unsrtnat, but unsrtnatshort gives shorter output
\bibliography{biblio}
\end{singlespace}
\end{small}

%%% appendix
% supporting information figure 1, now appendix figure 1
\begin{subappendices}
	\newpage
	\setcounter{figure}{0}
	\section{Additional Figures}
		\renewcommand{\figurename}{Additional Figure}
		%
		\begin{figure}[h]
		  \centering
		    \includegraphics[width=1\textwidth]{figs02/supfig1bw.pdf}	% pdf version also bw
		    \caption{This figure illustrates a histogram of nonadjusted, moderated p-values and the empirical
cumulative distribution of p-values for the studies of {\it A}, Cleton-Jansen {\it et al}.~\cite{cleton2009profiling} and {\it B}, Kuijjer {\it et al}.~\cite{kuijjer2011mrna}, both describing no significant difference in mRNA expression in pretreatment biopsies of patients with poor versus good response to chemotherapy. The figures were generated using Bioconductor package {\it SSPA}~\cite{van2009relative}.}
		     \label{afig2.1}
		\end{figure}
		%
\end{subappendices}

%\end{document}
% Marieke Kuijjer
% 2013-02-15
% chapter 03

	%\documentclass[12pt,b5paper]{book}
	%\setcounter{secnumdepth}{0}
	%\setcounter{tocdepth}{1}
	%\usepackage[hidelinks]{hyperref}

%\begin{document}

%%% title page

\chapter{mRNA expression profiles of primary high-grade
central osteosarcoma are preserved in cell lines
and xenografts}\label{ch3}
\thispagestyle{empty}				%%% to remove page number from first page of chapter, must be placed after calling the chapter

\vfill

\vspace{0.5cm}
This chapter is based on the publication:
\underline{Kuijjer ML}, Naml{\o}s HM, Hauben EI, Machado I, Kresse SH, Serra M, Llombart-Bosch A, Hogendoorn PCW, Meza-Zepeda LA, Myklebost O, Cleton-Jansen AM. {\it BMC Med Genomics}. 2011 Sep 20;4:66

\newpage

%%% main document

\section{Abstract}\label{abstract3}
\textbf{Background}: Conventional high-grade osteosarcoma is a primary malignant bone tumor, which is most prevalent
in adolescence. Survival rates of osteosarcoma patients have not improved significantly in the last 25 years. Aiming
to increase this survival rate, a variety of model systems are used to study osteosarcomagenesis and to test new
therapeutic agents. Such model systems are typically generated from an osteosarcoma primary tumor, but undergo
many changes due to culturing or interactions with a different host species, which may result in differences in
gene expression between primary tumor cells, and tumor cells from the model system. We aimed to investigate
whether gene expression profiles of osteosarcoma cell lines and xenografts are still comparable to those of the
primary tumor.

\textbf{Methods}: We performed genome\hyp{}wide mRNA expression profiling on osteosarcoma biopsies ($n=76$), cell lines ($n=13$), and xenografts ($n=18$). Osteosarcoma can be subdivided into several histological subtypes, of which
osteoblastic, chondroblastic, and fibroblastic osteosarcoma are the most frequent ones. Using nearest shrunken
centroids classification, we generated an expression signature that can predict the histological subtype of
osteosarcoma biopsies.

\textbf{Results}: The expression signature, which consisted of 24 probes encoding for 22 genes, predicted the histological
subtype of osteosarcoma biopsies with a misclassification error of 15\%. Histological subtypes of the two
osteosarcoma model systems, {\it i.e.} osteosarcoma cell lines and xenografts, were predicted with similar
misclassification error rates (15\% and 11\%, respectively).

\textbf{Conclusions}: Based on the preservation of mRNA expression profiles that are characteristic for the histological
subtype we propose that these model systems are representative for the primary tumor from which they are
derived.

\section{Background}\label{introduction3}
Conventional high\hyp{}grade osteosarcoma is the most frequent
primary malignant bone tumor, with a peak
occurrence in children and adolescents and a second
peak in patients older than 40 years. It is a highly
genetically unstable tumor, of which karyotypes often
show aneuploidy, high level amplification and deletion,
and translocations~\cite{cleton2005central}. No precursor lesion is known,
although part of the osteosarcomas in patients over 40
years is secondary, and is induced by radiation, chemicals,
or by an underlying history of Paget's disease of
bone~\cite{raymond2002conventional}. The leading cause of death of osteosarcoma
patients are distant metastases, which despite aggressive
chemotherapy regimens still develop in approximately
45\% of all patients~\cite{buddingh2010prognostic}. Overall survival of osteosarcoma
patients has increased from 10--20\% before the introduction
of preoperative chemotherapy in the 1970s, to
about 60\%~\cite{rozeman2006pathology}. However, survival has reached a plateau,
and treating with higher doses of chemotherapy does
not lead to better overall survival~\cite{lewis2007improvement}.

Osteosarcoma is a heterogeneous tumor type, which
can be subdivided into various subtypes~\cite{mohseny2008bone}. Conventional
high\hyp{}grade osteosarcoma is the most common
subtype, and can be further subdivided in different histological
subtypes, of which osteoblastic (50\%), chondroblastic
(25\%), and fibroblastic osteosarcoma (25\%) are
the most frequent ones. Other subtypes of conventional
high\hyp{}grade osteosarcoma, such as chondromyxoid
fibroma\hyp{}like, clear cell, epitheliod, sclerosing, and giant
cell rich osteosarcoma, are extremely rare~\cite{raymond2002conventional}. Often,
osteosarcoma tissue contains a mixture of morphologically
differing cell types, and the classification is based
on the most dominant type~\cite{hauben2002does}. The three main histological
subtypes have different survival profiles. Patients
with fibroblastic osteosarcoma have a significantly better
response to preoperative chemotherapy, which is a
known predictor for improved survival, than do osteoblastic
osteosarcoma patients~\cite{huvos1991bone}. Although patients with
chondroblastic osteosarcoma are relatively poor responders
to preoperative chemotherapy~\cite{hauben2002does,bacci1998predictive}, which is probably
caused by the impermeability of the chondroid
areas of the tumor, there is a trend for these patients to
have better 5-year survival profiles~\cite{hauben2002does}, but also a higher
risk for late relapse~\cite{hauben2006clinico}.

The search for new (targeted) therapies to treat osteosarcoma
is ongoing~\cite{hattinger2010emerging}. Because the disease is relatively
rare, large efforts need to be done in order to collect a
considerable amount of patient samples. Moreover,
material is usually scarce due to necrosis in resections
of the primary tumor, which is mostly present in tumors
of patients who respond fairly well to neoadjuvant chemotherapy.
No necrosis is present in prechemotherapy
biopsies, but these are often very small and are not
readily available for research because they are needed
for diagnosis. Because of these limitations, model systems
are widely used to study osteosarcomagenesis and
for preclinical testing of candidate drugs. Osteosarcoma
cell lines, especially SAOS-2 and U-2 OS are frequently
used as model systems, remarkably not only to study
osteosarcoma, but all types of {\it in vitro} cell biological
processes, as these cell lines grow fast and are relatively
easy to transfect. Recently, the EuroBoNeT (\url{www.eurobonet.eu}) osteosarcoma panel of 19 cell lines was
characterized, which allows us to study osteosarcoma in
a high\hyp{}throughput manner~\cite{ottaviano2010molecular}. This panel of osteosarcoma
cell lines has been shown to resemble osteosarcoma
phenotypically and functionally~\cite{mohseny2011functional}. Other
established model systems include xenografts from primary
tumors or osteosarcoma cell lines in immunodeficient
nude mice, which subsequently develop into
tumors resembling osteosarcoma~\cite{mohseny2011functional,mayordomo2010tissue,kresse2011preclinical}. Osteosarcomagenesis
can also be induced in mice by radiation or
orthotopically implanting chemical carcinogens~\cite{jones2011osteosarcomagenesis}. We
have previously shown that DNA copy number profiles
of xenografts resemble those of the corresponding primary
tumor, although some significant changes for
osteosarcoma were observed~\cite{kresse2011preclinical}.

Established cancer cell lines are often thought not to
be representative for the originating primary tumor.
Since there could have been a selection for their propensity
to grow in culture, they lack interaction with
stroma and may have acquired additional mutations in
culture~\cite{weinberg2007biology}. Xenografts do have tumor--host interactions,
but can lose matrix as well after several passages. It is
not clear whether such changes in matrix composition
of xenografts are caused by the tumor cells, or by
changes in mouse stroma~\cite{mayordomo2010tissue}. Despite these biological
differences, model systems are useful for studying signal
transduction pathways important in tumor biology, of
which mRNA expression, as measured by qPCR or
using gene expression microarrays, is frequently used as
a readout. It is therefore highly important to determine
whether gene expression levels of these model systems
are comparable to those of the corresponding primary
tumors, which we aimed to do in this study. We performed
gene expression analysis on a panel of 76 conventional
high\hyp{}grade osteosarcoma pretreatment
biopsies. We set out to recapitulate representative
expression profiles from primary untreated osteosarcoma
biopsies in corresponding models {\it i.e.} cell lines
and xenografts. We could demonstrate that both model
systems still express genes that are characteristic for the
specific histological subtype of the primary tumor. We
therefore endorse that, despite biological differences,
both xenografts and cell lines are representative model
systems for studying mRNA expression in high\hyp{}grade
osteosarcoma. Specific models may be identified that
would be appropriate to use for studies of specific subgroups
of osteosarcoma.

\section{Methods}\label{methods3}
\subsection{Ethics statement}
All biological material was handled in a coded fashion.
Ethical guidelines of the individual European partners
were followed and samples and clinical data were stored
in the EuroBoNet biobank. For xenograft experiments,
informed consent and sample collection were approved
by the Ethical Committee of Southern Norway (Project
S-06132) and the Institutional Ethical Committee of
Valencia University.

\subsection{Patients cohorts}
Genome\hyp{}wide expression profiling was performed on
pretreatment diagnostic biopsies of 76 resectable highgrade
osteosarcoma patients from the EuroBoNet consortium
(\url{www.eurobonet.eu}). Clinicopathological
details of these 76 samples can be found in Table~\ref{tab3.1}.
Samples with a main histological subtype ($n=66$) were
selected for subsequent subtype analyses. These 66 samples
included 50 osteoblastic, 9 chondroblastic, and 7
fibroblastic osteosarcomas. Five additional osteosarcoma
biopsies (1 chondroblastic and 4 osteoblastic osteosarcomas),
12 mesenchymal stem cell (MSC) and 3 osteoblast
cultures, and 12 chondrosarcoma biopsies were used for
validation.
%
\begin{landscape}
	\begin{table}[htbp]
		\centering
		\small
		\begin{tabular}[c]{|p{4.5in}cccc|} % OBS! 4.5in is entire page length % 4.2in gives same length as table 3.2
			\hline
			Category & Patient characteristics & Biopsies (\%) & Cell lines (\%) & Xenografts (\%)\\
			\hline
			Total nr of samples & & 76 (100) & 13 (100) & 18 (100)\\
			Institution & LUMC, Netherlands & 29 (38.2) & 0 (0) & 0 (0)\\
			& IOR, Italy & 11 (14.5) & 7 (53.8) & 0 (0)\\
			& LOH, Sweden & 3 (3.9) & 0 (0) & 0 (0)\\
			& Radiumhospitalet, Norway & 1 (1.3) & 3 (23.1) & 12 (66.7)\\
			& UV, Spain & 0 (0) & 0 (0) & 6 (33.3)\\
			& WWUM, Germany & 32 (42.1) & 0 (0) & 0 (0)\\
			& Other & 0 (0) & 3 (23.1) & 0 (0)\\
			Origin & Biopsy & 76 (100) & 0 (0) & 0 (0)\\
			& Resection & 0 (0) & 7 (53.8) & 11 (61.1)\\
			& Metastasis & 0 (0) & 3 (23.1) & 1 (5.6)\\
			& Unknown & 0 (0) & 3 (23.1) & 6 (33.3)\\
			Location of primary tumor & Femur & 36 (47.4) & 0 (0) & 10 (55.6)\\
			& Tibia/fibula & 26 (34.2) & 0 (0) & 2 (11.1)\\
			& Humerus & 10 (13.2) & 0 (0) & 2 (11.1)\\
			& Axial skeleton & 1 (1.3) & 0 (0) & 1 (5.6)\\
			& Unknown/other & 3 (3.9) & 13 (100) & 3 (16.7)\\
			Histological subtype & Osteoblastic & 50 (65.8) & 9 (69.2) & 15 (83.3)\\
			& Chondroblastic & 9 (11.8)&  0 (0) & 3 (16.7)\\
			& Fibroblastic & 7 (9.2) & 4 (30.8) & 0 (0)\\
			& Minor & 10 (13.2) & 0 (0) & 0 (0)\\
			Histological response to preoperative chemotherapy in the primary tumor & Good response & 33 (43.4) & 0 (0) & 0 (0)\\
			& Poor response & 36 (47.4) & 0 (0) & 0 (0)\\
			& Unknown/NA & 7 (9.2) & 13 (100) & 18 (100)\\
			Sex & Male & 52 (68.4) & 9 (69.2) & 9 (50)\\
			& Female & 24 (31.6) & 4 (30.8) & 3 (16.7)\\
			& Unknown & 0 (0) & 0 (0) & 6 (33.3)\\
			\hline
		\end{tabular}
		\caption{Clinicopathological details of patients with conventional high\hyp{}grade osteosarcoma, including all patients from the biopsy, cell line, and xenograft datasets.}
		\label{tab3.1}
	\end{table}
\end{landscape}
%

\subsection{Osteosarcoma cell lines}
Out of the EuroBoNeT panel of 19 cell lines, 13 cell
lines were recorded to belong to a main histological
subtype. This set of 13 cell lines contained 4 cell lines
derived from fibroblastic, and 9 cell lines derived from
osteoblastic osteosarcomas. The 13 osteosarcoma cell
lines IOR/MOS, IOR/OS10, IOR/OS14, IOR/OS15,
IOR/OS18, IOR/OS9, IOR/SARG, KPD, MG-63, MHM,
OHS, OSA, and ZK-58 were maintained in RPMI 1640
(Invitrogen, Carlsbad, CA, USA) supplemented with 10\%
fetal calf serum and 1\% Penicillin/Streptomycin (Invitrogen)
as previously described~\cite{ottaviano2010molecular}. Clinical details of the
tissue from which these cell lines were derived are
shown in Table~\ref{tab3.1} and are described previously~\cite{ottaviano2010molecular}.

\subsection{Osteosarcoma xenografts}
The osteosarcoma xenograft model is described in
Kresse {\it et al}.~\cite{kresse2011preclinical}. In short, human tumors were
implanted directly from patient samples and successively
passaged subcutaneously in nude mice. Eighteen different
xenografts were used, of which 3 were derived from
chondroblastic, and 15 from osteoblastic osteosarcomas.
Clinical data on primary tumor samples and xenograft
passages that were used are shown in Table~\ref{tab3.1}.

\subsection{Determination of histological subtypes}
Histological subtyping was performed by two pathologists
(PCWH, EH) on hematoxylin and eosin (HE)
stained slides of all biopsies and of all primary tumors
from which the osteosarcoma cell lines and xenografts
were derived. Osteoblastic, chondroblastic, and fibroblastic
osteosarcoma samples were selected for further
study. Other subtypes (anaplastic, chondromyxoid
fibroma\hyp{}like, fibroblastic MFH-like, giant cell rich, pleomorphic,
and sclerosing osteosarcoma) were excluded
because these subtypes are rare.

\subsection{RNA isolation, cDNA synthesis, cRNA amplification, and Illumina Human-6 v2.0 Expression BeadChip hybridization}
Osteosarcoma and xenograft tissue was handled as previously
described~\cite{buddingh2011tumor}. Osteosarcoma cell lines were prepared
as in Ottaviano {\it et al}.~\cite{ottaviano2010molecular}. RNA isolation,
synthesis of cDNA, cRNA amplification, and hybridization
of cRNA onto the Illumina Human-6 v2.0 Expression
BeadChips were performed as previously described~\cite{buddingh2011tumor}.

\subsection{Microarray data preprocessing}
Microarray data processing and quality control were
performed using the statistical language R~\cite{r2.10.0} as
described previously~\cite{buddingh2011tumor}. MIAME\hyp{}compliant data have
been deposited in the GEO database (\url{www.ncbi.nlm.nih.gov/geo/}, accession number GSE30699). High
correlations between these microarray data and corresponding
qPCR results have been demonstrated previously~\cite{buddingh2011tumor}.

\subsection{Detection of significantly differentially expressed genes}
We performed {\it LIMMA} analyses~\cite{smyth2004linear} in order to determine
differential expression for the following clinical
parameters: sex (52 male {\it vs} 24 female), tumor location
(36 femur, 10 humerus, 26 fibula/tibia), response to preoperative
chemotherapy (36 poor responders, or Huvos
grade 1--2, {\it vs} 33 good responders, or Huvos grade 3--4),
and histological subtype (an analysis comparing
50 osteoblastic, 9 chondroblastic, and 7 fibroblastic
osteosarcomas). Genes that play a role in metastasis\hyp{}free
survival are described in Buddingh {\it et al}.~\cite{buddingh2011tumor}. Probes
with Benjamini and Hochberg False discovery rate\hyp{}adjusted
p-values (adjP) $< 0.05$ were considered to be
significantly differentially expressed.

\subsection{Prediction analysis}
The gene expression profile was generated on the dataset
of biopsies using Bioconductor~\cite{gentleman2004bioconductor} package {\it pamr}~\cite{tibshirani2002diagnosis}. Internal cross\hyp{}validation was performed 50 times.
A threshold was selected where the error rate of the
prediction profile was minimal. The minimum error rate
was representative of 50 independent simulations. In
order to minimize optimization bias~\cite{wood2007classification}, we validated
the profile on an independent dataset of osteosarcoma
biopsies ($n=5$), containing 1 chondroblastic
osteosarcoma and 4 osteoblastic osteosarcomas. In addition,
we applied the profile on datasets containing positive
controls---mesenchymal stem cells (MSC, $n=12$),
osteoblasts ($n=3$), and chondrosarcoma biopsies ($n=12$, previously published in~\cite{hallor2009genomic}, GEO accession number
GSE12532). We subsequently applied the validated prediction
profile to two independent datasets, the first
consisting of gene expression data of osteosarcoma cell
lines, the second of xenografts. Expression of the probes
that composed the prediction profile was verified using
a {\it LIMMA} analysis, comparing chondroblastic,
fibroblastic, and osteoblastic osteosarcoma biopsy
samples.

\subsection{Gene set enrichment}
Network analysis was performed using Ingenuity Pathways
Analysis (IPA, Ingenuity Systems, \url{www.ingenuity.com}). For both chondroblastic\hyp{}specific and
fibroblastic\hyp{}specific analyses, data for all reference
sequences containing expression values and FDR\hyp{}adjusted
p-values were uploaded into the application.
Each identifier was mapped to its corresponding object
in Ingenuity's Knowledge Base. An adjP cut-off of 0.05
was set to select genes whose expression was significantly
differentially regulated. The Network Eligible
molecules were overlaid onto a global molecular network
developed from information contained in Ingenuity's
Knowledge Base. Networks of Network Eligible
Molecules were then algorithmically generated based on
their connectivity. GO term enrichment was tested
using Bioconductor package {\it topGO}~\cite{alexa2006improved}. Lists of significantly
affected genes were compared with all genes eligible
for the analysis. GO terms with Fisher's exact p-values
$< 0.001$, as calculated by the {\it weight} algorithm
from {\it topGO}, were defined significant.

\section{Results}\label{results3}
\subsection{Histological subtypes of osteosarcoma biopsies have different gene expression profiles}
We determined differential expression for different clinical
parameters. Of all comparisons of clinical parameters
only histological subtypes appeared to give a sufficient
number of differentially expressed genes to design a prediction
profile. {\it LIMMA} analyses resulted in one location\hyp{}specific differentially expressed gene: {\it HOXD4},
which was overexpressed in tumors at the humerus versus
at fibula/tibia and femur. Between tumors from
male and female patients, 18 genes were significantly
differentially expressed, all belonging to X- and Y\hyp{}chromosome\hyp{}specific genes, which are not considered as
representative for osteosarcoma, yet this finding validates
the analysis. No significantly affected genes were
returned with regards to response to preoperative chemotherapy.
To determine differential expression
between the three main histological subtypes, we
excluded all samples with unknown or rare subtypes.
This resulted in a dataset of 66 conventional high\hyp{}grade
osteosarcoma biopsies with a main histological subtype.
Using a {\it LIMMA} analysis, we determined $1,338$
significantly differentially expressed genes (adjP $<0.05$)
that were specific for a certain main histological subtype
(depicted in a Venn diagram in Figure~\ref{fig3.1}).% figure title is `Subtype\hyp{}specific genes'
%
\begin{figure}[htbp]
  \centering
  \begin{minipage}[b]{0.50\linewidth}
%    \includegraphics[width=1\textwidth]{figs03/fig1bw.pdf}		% OBS! print version bw
   \includegraphics[width=1\textwidth]{figs03/fig1rgb.pdf}	% OBS! pdf version rgb
  \end{minipage}
    \hfill
  \begin{minipage}[b]{0.46\linewidth}
%			%%% OBS! caption for fig3.1 print version:
%     \caption{Venn diagram representing numbers of fibroblastic-, \mbox{chondroblastic-,} and osteoblastic\hyp{}specific differentially expressed genes (shaded regions) obtained with {\it LIMMA} analysis, considering chondroblastic versus osteoblastic (chondro {\it vs} osteo), fibroblastic versus osteoblastic (fibro {\it vs} osteo), and chondroblastic versus fibroblastic (chondro {\it vs} fibro) analyses. Subtype\hyp{}specific genes are genes that are either both upregulated or both downregulated in the subtype of interest in the different comparisons.}
%			%%% OBS! caption for fig3.1 pdf version:
    \caption{Venn diagram representing numbers of fibroblastic- (green), chondroblastic- (red), and osteoblastic (blue)\hyp{}specific differentially expressed genes obtained with {\it LIMMA} analysis, considering chondroblastic versus osteoblastic (chondro {\it vs} osteo), fibroblastic versus osteoblastic (fibro {\it vs} osteo), and chondroblastic versus fibroblastic (chondro {\it vs} fibro) analyses. Subtype\hyp{}specific genes are genes that are either both upregulated or both downregulated in the subtype of interest in the different comparisons.}
     \label{fig3.1}
     \end{minipage}
\end{figure}
%
%
%
%
A subtype\hyp{}specific
probe was defined as a probe that had the same
sign of log fold change in both analyses, {\it e.g.} the probe
was upregulated in chondroblastic samples versus osteoblastic,
and in chondroblastic versus fibroblastic
samples.

\subsection{Gene set enrichment shows specific sets of genes are affected in fibroblastic and chondroblastic osteosarcoma}
Network analysis using IPA showed that fibroblastic
osteosarcoma\hyp{}specific networks most\-ly had a role in cellular
growth and proliferation, which was also the most
significant biological function as detected by IPA (see
Additional File 1 ({\it available online}~\cite{ch3additional}) for all affected networks and biological
functions). The most significant network is illustrated in
Additional File 2A ({\it available online}~\cite{ch3additional}) and shows that mRNA of various
genes with a connection to the NF-$\upkappa$B pathway and
STAT5A signaling are upregulated in fibroblastic osteosarcoma
biopsies, as compared with both osteoblastic
and chondroblastic osteosarcoma. The most significant
network specific for the chondroblastic subtype consisted
of genes important in skeletal connective tissue
development and function (Additional File 2B ({\it available online}~\cite{ch3additional})), and
shows that, also on the gene expression level, chondroblastic
osteosarcoma is mainly distinguished from osteoblastic
and fibroblastic osteosarcoma based on the
composition of the extracellular matrix of the tumor
(Additional File 1 ({\it available online}~\cite{ch3additional}) shows all affected networks and biological
functions).

Results from network analysis were confirmed using
{\it topGO}, a gene set enrichment approach analyzing the significance
of GO terms in a specific dataset. These analyses
resulted in two significant fibroblastic specific GO terms
in osteosarcoma: regulation of tyrosine phosphorylation of
Stat5 protein (GO:0042522, p-value $=4.8\cdot10^{-4}$) and regulation of
survival gene product expression (GO:0045884, p-value $=8.2\cdot10^{-
4}$). Significantly differentially expressed genes from both
GO terms partly overlap the fibroblastic osteosarcoma\hyp{}specific
network detected with IPA. Two GO terms were
significant in the chondroblastic\hyp{}specific analysis as well:
skeletal system development (GO:0001501), and extracellular
matrix organization (GO:0030198), which strengthen
the results found in the IPA network analyses. GO term
subgraphs of the five most significant GO terms for both
analyses are shown in Additional File 3 ({\it available online}~\cite{ch3additional}).

Gene set enrichment on genes specific for osteoblastic
osteosarcoma was not performed, because only one
osteoblastic osteosarcoma\hyp{}specific probe was detected
that distinguishes the osteoblastic subtype from fibroblastic
and chondroblastic. This probe matches to
{\it UNQ1940}, or {\it FAM180A}, a protein\hyp{}coding gene with a
yet unknown function.

\subsection{Generation and validation of the prediction profile}
Because we could not directly compare subtype\hyp{}specific
genes between our different model systems due to small
sample sizes, we generated a profile that could predict
the histological subtype of osteosarcoma. The prediction
profile was generated on 66 high\hyp{}grade conventional
osteosarcoma prechemotherapy biopsies, using nearest
shrunken centroids classification. Optimal control of
error rate in the prediction profile was found at delta
thresholds of $4.9-5.1$ (Figure~\ref{fig3.2}A), where merely 10 out
of 66 samples (15\%) in the training set were wrongly
assigned to a specific histological subtype.
%
\begin{figure}[htbp]
	\centering
%	\includegraphics[width=1.0\textwidth]{figs03/fig2bw.pdf}	% OBS! print version bw
	\includegraphics[width=1.0\textwidth]{figs03/fig2rgb.pdf}	% OBS! pdf version rgb
%	\caption{{\it A}, Illustration of training the {\it pamr} prediction profile on osteosarcoma biopsies. At thresholds of $4.9-5.1$, the misclassification error rate was minimal. {\it B}, True versus predicted values from the nearest shrunken centroid fit. {\it C}, Probabilities of each biopsy to belong to any of the three histological subtypes. Samples are separated (dotted lines) based on their true subtypes. Cross\hyp{}validated probabilities for each histological subtype are shown on the y-axis, so that for every sample three open circles are present (black, gray, and light gray circles for \mbox{osteo-,} chondro-, and fibroblastic osteosarcoma, respectively). A sample is classified into a specific subtype if the probability to belong to that specific subtype is higher than the probabilities to belong to the other subtypes. {\it D}, The FDR plotted against different thresholds of the prediction profile. At a threshold of 5.0, 24 genes are included in the prediction profile. These 24 genes have an FDR $< 5\%$.} % OBS! bw caption for print file
	 	\caption{{\it A}, Illustration of training the {\it pamr} prediction profile on osteosarcoma biopsies. At thresholds of $4.9-5.1$, the misclassification error rate was minimal. {\it B}, True versus predicted values from the nearest shrunken centroid fit. {\it C}, Probabilities of each biopsy to belong to any of the three histological subtypes. Samples are separated (dotted lines) based on their true subtypes. Cross\hyp{}validated probabilities for each histological subtype are shown on the y-axis, so that for every sample three open circles are present (blue, red, and green circles for \mbox{osteo-,} chondro-, and fibroblastic osteosarcoma, respectively). A sample is classified into a specific subtype if the probability to belong to that specific subtype is higher than the probabilities to belong to the other subtypes. {\it D}, The FDR plotted against different thresholds of the prediction profile. At a threshold of 5.0, 24 genes are included in the prediction profile. These 24 genes have a FDR $< 5\%$.} % OBS! rgb caption for pdf file
	\label{fig3.2}
\end{figure}
%
This error
rate was representative for a set of 50 simulations,
which resulted in error rates between 13.5\% and 15\%.
Subtype\hyp{}specific error rates were 22\%, 43\%, and 10\% for
chondroblastic, fibroblastic, and osteoblastic subtypes,
respectively (Figure~\ref{fig3.2}B). Probabilities of each sample to
belong to any of the three histological subtypes are
shown in Figure~\ref{fig3.2}C. At a threshold delta of 5.0, the prediction
profile consisted of 24 probes encoding for 22
genes. All genes were below a FDR threshold of 5\% (Figure~\ref{fig3.2}D). Expression of the 24 probes of the profile were
verified in a {\it LIMMA} analysis which was corrected
for multiple testing. All 24 probes were confirmed
to be significantly differentially expressed in the
{\it LIMMA} analysis as well. Results from {\it pamr} and
{\it LIMMA} analyses are shown in Table~\ref{tab3.2}. A supervised
heatmap depicting expression of the 24 probes in all
samples is shown in Additional File 4 ({\it available online}~\cite{ch3additional}). The prediction
profile was validated at threshold delta of 5.0 in an independent
dataset of osteosarcoma biopsies and positive
controls. Histological subtypes of biopsies had a prediction
error of 0\% (0/5). Mesenchymal stem cells and
osteoblasts all fitted in the osteoblastic group, while
11/12 chondrosarcoma samples were best corresponding to
the group of chondroblastic osteosarcoma. The remaining
chondrosarcoma sample was a dedifferentiated
chondrosarcoma and was predicted in the fibroblastic
group, probably because of the high amount of spindle
cells present in the biopsy. Additional File 5 ({\it available online}~\cite{ch3additional}) shows prediction
probabilities for each subtype of these additional
datasets.
%
\subsection{A prediction profile based on osteosarcoma biopsy data can predict histological subtypes of cell lines and xenografts}
Unsupervised clustering of all biopsies, xenografts, and
cell lines demonstrated that xenografts and cell lines
show different overall expression profiles from most
biopsies, and that there are no subtype\hyp{}specific clusters
based on overall expression (Additional File 6 ({\it available online}~\cite{ch3additional})). In order
to determine whether histological subtypes of cell lines
and xenografts could be predicted as well with the 24-gene prediction profile, we applied this profile to two
independent datasets. In the first dataset, consisting of
osteosarcoma cell line data, 2 out of 13 samples (15\%,
Figure~\ref{fig3.3}A) were wrongly classified.
%
These samples were
MG63, a cell line derived from a fibroblastic osteosarcoma,
which was subtyped as being osteoblastic, and
IOR/OS18, derived from an osteoblastic osteosarcoma,
which was subtyped by the prediction profile as a fibroblastic
osteosarcoma. Interestingly, HOS, HOS-MNNG,
and HOS-143B, all cell lines derived from the HOS cell
line, which is derived from fibroblastic and epithelial
osteosarcoma and therefore was not added to our set of
13 osteosarcoma cell lines, were all predicted as fibroblastic
osteosarcoma ({\it data not shown}). Two out of 18
xenograft samples (11\%, Figure~\ref{fig3.3}B) were wrongly
classified. One of these samples was OKx, a xenograft
derived from a chondroblastic tumor, which was subtyped
as an osteoblastic osteosarcoma. The other sample
was KPDx, a xenograft derived from an osteoblastic
tumor, which was subtyped as fibroblastic. The KPD cell
line was subtyped rightly as an osteoblastic
osteosarcoma.
%
\begin{landscape}
	\begin{table}[htbp]
		\centering
		\small
		\begin{tabular}{|l l >{\raggedleft}p{0.8in} >{\raggedleft}p{0.8in} >{\raggedleft}p{0.8in} l l l l|}
			\hline
			probeID & symbol & {\it LIMMA} logFC C{\it vs}F & {\it LIMMA} logFC C{\it vs}O & {\it LIMMA} logFC F{\it vs}O & {\it LIMMA} adjP & {\it pamr} chondro\hyp{}score & {\it pamr} fibro\hyp{}score & {\it pamr} osteo\hyp{}score\\
			\hline
			5910377 & {\it ACAN} & 2.42 & 2.24 & -0.18 & 0.0000 & 0.9294 & 0 & -0.0147\\
			3390678 & {\it NFE2L3} & -1.74 & -0.02 & 1.71 & 0.0000 & 0 & 0.9184 & 0\\
			1990523 & {\it COL9A1} & 3.49 & 3.01 & -0.47 & 0.0000 & 0.6011 & 0 & 0\\
			360553 & {\it SCRG1} & 4.55 & 3.51 & -1.04 & 0.0001 & 0.4571 & 0 & 0\\
			3310368 & {\ itID3} & 1.87 & -0.29 & -2.16 & 0.0003 & 0 & -0.4053 & 0\\
			10561 & {\it ITIH5L} & 0.68 & 0.65 & -0.04 & 0.0001 & 0.295 & 0 & 0\\
			5050110 & {\it MGC34761} & 0.93 & 0.83 & -0.09 & 0.0002 & 0.2818 & 0 & 0\\
			4780368 & {\it ACAN} & 1.34 & 1.19 & -0.14 & 0.0004 & 0.2716 & 0 & 0\\
			7150719 & {\it COL2A1} & 4.82 & 4.36 & -0.47 & 0.0016 & 0.183 & 0 & 0\\
			3830341 & {\it LYRM1} & -1.23 & -0.18 & 1.06 & 0.0007 & 0 & 0.1677 & 0\\
			3990500 & {\it MATN4} & 1.96 & 1.69 & -0.27 & 0.0012 & 0.151 & 0 & 0\\
			4280370 & {\it POPDC3} & -0.88 & -0.08 & 0.80 & 0.0009 & 0 & 0.0909 & 0\\
			6520487 & {\it UNQ830} & 4.10 & 2.90 & -1.20 & 0.0016 & 0.0817 & 0 & 0\\
			2850202 & {\it COL11A2} & 1.37 & 1.10 & -0.27 & 0.0014 & 0.0735 & 0 & 0\\
			4220452 & {\it C11ORF41} & -0.89 & -0.03 & 0.86 & 0.0011 & 0 & 0.0721 & 0\\
			4560091 & {\it COL9A3} & 1.14 & 1.21 & 0.07 & 0.0018 & 0.0698 & 0 & 0\\
			5890452 & {\it LOC652881} & 0.43 & 0.37 & -0.06 & 0.0001 & 0.0666 & 0 & 0\\
			3990259 & {\it PPP2R2B} & -1.00 & 0.10 & 1.10 & 0.0016 & 0 & 0.0603 & 0\\
			5340392 & {\it MAN2A1} & -1.42 & -0.22 & 1.20 & 0.0018 & 0 & 0.0477 & 0\\
			3360139 & {\it DLX5} & 1.84 & -0.20 & -2.04 & 0.0033 & 0 & -0.0358 & 0\\
			2630762 & {\it C14ORF78} & -1.07 & 1.45 & 2.52 & 0.0011 & 0 & 0 & -0.0307\\
			3460037 & {\it UNQ1940} & 0.44 & 1.71 & 1.27 & 0.0018 & 0 & 0 & -0.0219\\
			6110722 & {\it COL2A1} & 1.22 & 1.44 & 0.22 & 0.0032 & 0.0087 & 0 & 0\\
			6980164 & {\it ALPL} & 2.52 & -0.67 & -3.19 & 0.0038 & 0 & -0.0036 & 0\\
			\hline
		\end{tabular}
		\caption{Comparison of the 24 genes obtained with {\it pamr} prediction with a {\it LIMMA} analysis between the three different histological subtypes (CvsF: chondroblastic {\it vs} fibroblastic, CvsO: chondroblastic {\it vs} osteoblastic, FvsO: fibroblastic {\it vs} osteoblastic osteosarcoma), for which log fold changes (logFC) are shown for the different coefficients of the analysis. Note that the adjP shows the significance for the whole {\it LIMMA} analysis, and does not reflect the adjPs per subanalysis.}
		\label{tab3.2}
	\end{table}
\end{landscape}
%

\begin{figure}[htbp]
	\centering
%	\includegraphics[width=1.0\textwidth]{figs03/fig3bw.pdf}	% OBS! print version bw
	\includegraphics[width=1.0\textwidth]{figs03/fig3rgb.pdf}	% OBS! pdf version rgb
	\caption{Probabilities of {\it A}, cell lines and {\it B}, xenografts to belong to any of the three histological subtypes. For an explanation of what is represented by these graphs, see Figure~\ref{fig3.2}{\it C}.}
	\label{fig3.3}
\end{figure}

\section{Discussion}\label{discussion3}
In this study, we aimed to compare gene expression
profiles of osteosarcoma biopsies with cell lines and
xenografts, in order to investigate whether these model
systems are representative for the primary tumor. We
have determined differential gene expression for different
clinical parameters important in high\hyp{}grade osteosarcoma
on a dataset consisting of 76 conventional
high\hyp{}grade osteosarcoma samples. Importantly, pretreatment
biopsies were used instead of resected specimens,
because preoperative chemotherapy may cause
tumor necrosis in responsive patients, thus altering gene
expression and hampering the generation of high quality
mRNA. We intended to generate a gene expression profile
that could not only predict a specific clinical parameter
in biopsies, but in osteosarcoma cell lines and
xenografts as well. The metastasis/survival profile is
described previously and may serve as a tool to predict
prognosis and as a target for therapy~\cite{buddingh2011tumor}. However,
since most of the genes associated with osteosarcoma
metastasis were macrophage associated, and no stroma
or infiltrate is present in cell lines, this profile could not
be applied to osteosarcoma cell lines. We therefore
compared gene expression profiles of these different
sample sets based on other clinical parameters. No significant
differentially expressed genes were found
between poor and good responders to chemotherapy.
Several reports on genome\hyp{}wide expression profiling in
osteosarcoma have been published describing detection
of differential expression between poor and good
responders of preoperative chemotherapy~\cite{mintz2005expression,ochi2004prediction,man2005expression,salas2009molecular}. However,
the cohorts used in these studies are all relatively
small ($n=$ 13--30), and, more importantly, the reported
p-values were not corrected for multiple testing in these
studies. Remarkably, only two of the genes that were
found to correlate with response to chemotherapy in
these studies overlap, and one of these two genes was
upregulated in poor responders in one study, whereas it
was upregulated in good responders in the other study~\cite{mintz2005expression,salas2009molecular}. Another report described differential expression
between a metastatic and a nonmetastatic cell line, for
which metastatic capacity correlates with response to
chemotherapy~\cite{walters2008identification}. In that particular study, four genes
out of 252 were found to overlap with a patient study
by Mintz {\it et al}.~\cite{mintz2005expression}. However, the up- and downregulation
of these four genes were not consistent between
the two studies. We clearly show in a large cohort that
there are no differences between these groups of
patients, as the most significant probe had an adjP of
0.9998. These results are in line with our previous findings
obtained by analyzing an osteosarcoma cohort on a
different platform~\cite{cleton2009profiling}. The parameter `histological subtypes'
resulted in a considerable number of differentially
expressed genes. Our prediction profile is not directly
applicable to other platforms, but there is no real need
to have a prediction profile for primary osteosarcoma
histological subtype, since pathologists are very well able
to assess this on an HE-section, even on a biopsy, with
a concordance of 98\% between histological subtype of
biopsies and corresponding resections~\cite{hauben2002does}. Yet, we here
show a quite important use of this profile, {\it i.e.} to determine
the histological subtype of cell lines and xenografts.
{\it In vitro} 2\hyp{}dimensional growing cells lack extra
cellular matrix formation, which is the characteristic feature
to distinguish histological subtypes in high\hyp{}grade
central osteosarcoma.

The gene expression profiles as detected by analyzing
osteosarcoma biopsy data show a large number of subtype\hyp{}specific differentially expressed genes. In particular,
fibroblastic osteosarcoma differed most from the two
other main subtypes. Using gene set enrichment, we
detected a network of genes upregulated in fibroblastic
osteosarcoma, with a role in cellular growth and proliferation,
and connection to the NF-$\upkappa$B pathway. This
may be a readout of the high cellularity and low matrix
composition of fibroblastic osteosarcoma in comparison
with osteoblastic and chondroblastic osteosarcoma~\cite{raymond2002conventional}.
GO term enrichment analysis confirmed these results.
These pathways may explain why it is comparatively
easy to culture fibroblastic osteosarcoma cells, which
also may explain why the percentage of fibroblastic
osteosarcoma is relatively high in our cell line dataset
(31\%, compared to 11\% in the biopsy dataset). Next to
this link to cellular growth and proliferation, the most
significant network with fibroblastic\hyp{}specific upregulated
genes showed a connection to the immune system. GO
analysis of the five most significant GO terms pointed
to involvement of the immune system as well (GO term
GO:0006955, p-value $=3.9\cdot10^{-3}$, see Additional File 3 ({\it available online}~\cite{ch3additional}) for GO
term subgraphs). Forty\hyp{}four genes in this GO term were
significant, of which 43 were upregulated in fibroblastic
osteosarcoma. An elevated immune response might be
the reason why patients with fibroblastic osteosarcoma
tend to have better survival profiles, as a proinflammatory
environment has an important role in osteosarcoma
metastasis\hyp{}free survival. This profile is different from the
previously found macrophage\hyp{}specific profile which was
associated with better metastasis\hyp{}free survival of osteosarcoma
patients~\cite{buddingh2011tumor}. The overrepresentation of pathways
involved in chondrogenesis in the chondroblastic
profile is in line with the high amount of chondroid
matrix in this subtype. We only detected one osteoblastic\hyp{}specific gene, {\it UNQ1940}, or {\it FAM180A}, with a yet
unknown function. Since 50 osteoblastic osteosarcoma
samples were compared with only 9 chondroblastic and
7 fibroblastic osteosarcoma samples, we suggest that
fibroblastic and chondroblastic osteosarcoma have specific
characteristics that distinguishes these tumors from
osteoblastic osteosarcoma, and that the latter does not
have such an extra feature in comparison with chondro- and
fibroblastic osteosarcoma.

Our histological subtype prediction profile consists of
24 probes encoding for 22 genes, all with a specific
score which depends on the significance of each gene.
The genes that make up the chondroblastic\hyp{}specific part
of this expression profile are mostly chondroid matrixassociated
genes, such as {\it ACAN}, {\it COL2A1}, and {\it MATN4},
and are all upregulated in chondroblastic osteosarcoma.
Fibroblastic\hyp{}specific genes that make up the prediction
profile are up- or downregulated. An example of a gene
upregulated in fibroblastic osteosarcoma is {\it NFE2L3}, a
transcription factor which heterodimerizes with small
musculoaponeurotic fibrosarcoma factors and for which
a protective role was suggested in hematopoietic malignancies~\cite{chevillard2011nfe2l3}. {\it DLX5}, a transcription factor involved in
bone formation, is downregulated in fibroblastic osteosarcoma,
and reflects the lower amounts of matrix present
in fibroblastic osteosarcoma. No known function is
yet available for the two osteoblastic\hyp{}specific genes. The
misclassification error of the prediction profile in the
training set of biopsies was 15\%. Cell lines and xenografts
were predicted with misclassification errors of
15\% and of 11\%, respectively. It seems most likely that
these prediction errors are inherent to the error rate of
the prediction profile, which is also 15\%. Thus, because
these misclassification errors are in the same range, we
suggest that gene expression of these model systems is
highly similar to gene expression of the tumor from
which they are derived. This is especially of interest for
studies in cell lines, since no stroma is present on which
subtyping can be performed, but repeatedly passaged
xenografts often lose stroma as well. Most genes of the
prediction profile are matrix\hyp{}associated genes. Even
though these cell lines do not secrete any matrix, and
xenografts have diminished amounts of matrix, we can
still detect histological subtype markers on an mRNA
level, and are able to distinguish different histological
subtypes of cell lines and xenografts using this profile.

\section{Conclusions}\label{conclusions3}
As osteosarcoma xenografts and cell lines still express
histological subtype\hyp{}specific mRNAs that are characteristic
of the primary tumor, these model systems are
representative for the primary tumor from which they
are derived, and will be useful tools for studying mRNA
expression and pathways important in high\hyp{}grade
osteosarcoma.

%%% references

\begin{small}
\begin{singlespace}
\bibliographystyle{unsrtnatshort}		% sorted as referenced, was unsrtnat, but unsrtnatshort gives shorter output
\bibliography{biblio}
\end{singlespace}
\end{small}

%\end{document}
% Marieke Kuijjer
% 2013-02-15
% chapter 04

%	\documentclass[12pt,b5paper]{book}
%	\setcounter{secnumdepth}{0}
%	\setcounter{tocdepth}{1}
%	\usepackage[hidelinks]{hyperref}

%\begin{document}
%%%

%%% title page

\chapter{Tumor\hyp{}infiltrating macrophages are associated with metastasis suppression in high\hyp{}grade osteosarcoma: a rationale for treatment with macrophage activating agents}\label{ch4}
\thispagestyle{empty}				%%% to remove page number from first page of chapter, must be placed after calling the chapter

\vfill

\vspace{0.5cm}
This chapter is based on the publication:
Buddingh EP$\dagger$, \underline{Kuijjer ML}$\dagger$, Duim RA, B{\"u}rger H, Agelopoulos K, Myklebost O, Serra M, Mertens F, Hogendoorn PCW, Lankester AC, Cleton-Jansen AM. {\it Clin Cancer Res}. 2011 Apr 15;17(8):2110-9
\begin{small}
$\dagger$Shared first authorship
\end{small}

\newpage

%%% main document

\section{Abstract}\label{abstract4}
\textbf{Purpose}: High-grade osteosarcoma is a malignant primary bone tumor with a peak incidence in adolescence. Overall survival (OS) of patients with resectable metastatic disease is approximately 20\%. The exact mechanisms of development of metastases in osteosarcoma remain unclear. Most studies focus on tumor cells, but it is increasingly evident that stroma plays an important role in tumorigenesis and metastasis. We investigated the development of metastasis by studying tumor cells and their stromal context.

\textbf{Experimental Design}: To identify gene signatures playing a role in metastasis, we carried out genome\hyp{}wide gene expression profiling on prechemotherapy biopsies of patients who did ($n=34$) and patients who did not ($n=19$) develop metastases within 5 years. Immunohistochemistry (IHC) was performed on pretreatment biopsies from 2 additional cohorts ($n=63$ and $n=16$) and corresponding postchemotherapy resections and metastases.

\textbf{Results}: A total of 118/132 differentially expressed genes were upregulated in patients without metastases. Remarkably, almost half of these upregulated genes had immunological functions, particularly related to macrophages. Macrophage\hyp{}associated genes were expressed by infiltrating cells and not by osteosarcoma cells. Tumor\hyp{}associated macrophages (TAM) were quantified with IHC and associated with significantly better overall survival (OS) in the additional patient cohorts. Osteosarcoma samples contained both M1- (CD14/HLA-DR$\upalpha$ positive) and M2-type TAMs (CD14/CD163 positive and association with angiogenesis).

\textbf{Conclusions}: In contrast to most other tumor types, TAMs are associated with reduced metastasis and improved survival in high\hyp{}grade osteosarcoma. This study provides a biological rationale for the adjuvant treatment of high\hyp{}grade osteosarcoma patients with macrophage activating agents, such as muramyl tripeptide.

\section{Introduction}\label{introduction4}
High-grade osteosarcoma is a malignant bone tumor
characterized by the production of osteoid. The highest
incidence is in adolescent patients, with a second peak in
patients older than 40 years~\cite{raymond2002conventional}. Despite wide\hyp{}margin
surgery and intensification of chemotherapeutic treatment,
overall survival (OS) rates have reached a plateau at about
60\%~\cite{lewis2007improvement,bacci2006prognostic,bielack2002prognostic}. Novel administration modalities are needed,
but data on critical biological mechanisms allowing the
development of novel therapeutic agents are scarce for this
relatively rare tumor. In addition to conventional chemotherapeutic
agents, recent trials have explored immunostimulatory
strategies. The ongoing EURAMOS-1 trial
randomizes for treatment with IFN-$\upalpha$ in patients with good
histological response to neoadjuvant chemotherapy~\cite{marina2009international}. A
recently published clinical trial has shown improved OS for
osteosarcoma patients treated with the macrophage activating
agent muramyl tripeptide (MTP) added to the standard
chemotherapy regimen~\cite{meyers2008osteosarcoma}. However, only limited information
on macrophage infiltration and activation in osteosarcoma
is available~\cite{kleinerman1992unique}.

Tumor\hyp{}associated macrophages (TAM) may promote
tumorigenesis through immunosuppression, expression
of matrix\hyp{}degrading proteins and support of angiogenesis.
In numerous cancer types, high numbers of M2 or `alternatively
activated' TAMs are associated with a worse prognosis~\cite{hagemann2006ovarian,lee2008prognostic,lissbrant2000tumor,volodko1998tumour,jensen2009macrophage,van2009anti}. M2 macrophages have important functions
in wound healing and angiogenesis, express high levels of
the immunosuppressive cytokines interleukin (IL)-10 and
TGF-$\upbeta$, and express scavenger receptors such as CD163~\cite{sica2008macrophage,qian2010macrophage}.
`Classical activation' of macrophages by IFN-$\upgamma$ or
microbial products results in polarization toward M1-type
macrophages. M1 macrophages express high levels of
proinflammatory cytokines such as IL-12, IL-1, and IL-6
and have potent antitumor efficacy, both by reactive oxygen
species and cytokine\hyp{}induced cytotoxicity and by
induction of natural killer (NK) and T cell activity~\cite{mosser2008exploring}.
Rarely, high numbers of TAMs are associated with better
prognosis~\cite{kim2008high,forssell2007high}. In these cases, TAMs are presumably
polarized toward an M1 phenotype, although macrophage
subtypes were not reported in these two studies. Alternatively,
macrophages may directly phagocytose tumor cells, as has
been shown in acute myeloid leukemia~\cite{jaiswal2010macrophages}.

To investigate the role of stroma and stroma--tumor
interactions important in metastasis of osteosarcoma, we
investigated the development of metastasis by studying
tumor cells and their stromal context. By using genome\hyp{}wide
expression analysis, we showed that high expression
of macrophage\hyp{}associated genes in pretreatment biopsies
was associated with a lower risk of developing metastases.
In addition, we quantified and characterized TAMs in two
independent cohorts, including pretreatment biopsies,
postchemotherapy resections, and metastatic lesions. In
contrast to the tumor\hyp{}supporting role for TAMs in most
epithelial tumor types, higher numbers of infiltrating TAMs
correlated with better survival in osteosarcoma. Our findings
suggest that macrophages have direct or indirect antiosteosarcoma
activity and provide a possible explanation
for the beneficial effect of treatment with macrophage
activating agents in osteosarcoma.

\section{Materials and methods}\label{methods4}
\subsection{Patient cohorts}
Genome\hyp{}wide expression profiling was performed on
snap\hyp{}frozen pretreatment diagnostic biopsies containing
viable tumor material of 53 resectable high\hyp{}grade osteosarcoma
patients from the EuroBoNet consortium (\url{www.eurobonet.eu}; cohort 1). For immunohistochemical
validation, a tissue microarray containing 145 formalin\hyp{}fixed
paraffin\hyp{}embedded (FFPE) samples of 88 consecutive
high\hyp{}grade osteosarcoma patients with primary resectable
disease (cohort 2) and 29 FFPE samples of a cohort of 20
consecutive high\hyp{}grade osteosarcoma patients with resectable
disease were used (cohort 3), including material from
pretreatment biopsies, postchemotherapy resections, and
metastatic lesions~\cite{mohseny2009osteosarcoma}. Clinicopathological details can be
found in Supplemental Table S1 ({\it available online}~\cite{ch4additional}). All biological material
was handled in a coded fashion. Ethical guidelines of the
individual European partners were followed and samples
and clinical data were stored in the EuroBoNet biobank.

\subsection{Cell lines}
The 19 osteosarcoma cell lines 143B, HAL, HOS,
IOR/MOS, IOR/OS10, IOR/OS14, IOR/OS15, IOR/OS18,
IOR/OS9, IOR/SARG, KPD, MG-63, MHM, MNNG-HOS, OHS, OSA,
SAOS-2, U-2 OS, and ZK-58 were maintained in RPMI
1640 (Invitrogen) supplemented with 10\% fetal calf serum
and 1\% penicillin/streptomycin (Invitrogen) as previously
described~\cite{ottaviano2010molecular}.

\subsection{RNA isolation, cDNA synthesis, cRNA amplification,
and Illumina Human-6 v2.0 Expression BeadChip
hybridization}
Osteosarcoma tissue was snap\hyp{}frozen in 2-methylbutane
(Sigma\hyp{}Aldrich) and stored at 70$^\circ$C. By using a cryostat,
20mm sections from each block were cut and stained with
hematoxylin and eosin to ensure at least 70\% tumor
content and viability. RNA was isolated with TRIzol (Invitrogen),
followed by RNA cleanup using the QIAGEN
Rneasy mini kit with on\hyp{}column DNase treatment. RNA
quality and concentration were measured using an Agilent
2100 Bioanalyzer and Nanodrop ND-1000 (Thermo Fisher
Scientific), respectively. Synthesis of cDNA, cRNA amplification,
and hybridization of cRNA onto the Illumina
Human-6 v2.0 Expression BeadChips was carried out as
per manufacturer's instructions.

\subsection{Reverse transcriptase quantative PCR}
Reverse transcriptase quantative PCR (qPCR) analysis
of selected target genes was performed as previously
described~\cite{rozeman2005absence}. Each experiment was conducted in duplicate
by using an automated liquid\hyp{}handling system (Tecan,
Genesis RSP 100). Data were normalized by geometric
mean expression levels of 3 reference genes, {\it i.e.} {\it SRPR},
{\it CAPNS1}, and {\it TBP} using geNorm (\url{medgen.ugent.be/~jvdesomp/genorm/}). Primer sequences can be found in
Supplemental Table S2 ({\it available online}~\cite{ch4additional}).

\subsection{Enzymatic and fluorescent immunostainings}
Enzymatic and fluorescent immunostainings were performed
on 4mm sections of FFPE tissue as previously
described~\cite{mohseny2009osteosarcoma}. Details regarding antibodies and procedures
can be found in Supplemental Table S3 ({\it available online}~\cite{ch4additional}). In case of
double immunohistochemistry (IHC), incubation with
anti\hyp{}CD45 and development with DAB+ (Dako) occurred
first, followed by a second antigen retrieval before incubation
with either anti\hyp{}CD163 or anti\hyp{}HLA-DR$\upalpha$ and development
using the alkaline phosphatase substrate Vector
Blue (Vector Labs). In case of double immunofluorescent
(IF) stainings, primary antibodies were coincubated overnight.
As a positive control, normal and formic acid decalcified
tonsil was used, and as a negative control, no
primary antibody was added. Tissue microarray slides were
scanned using the MIRAX SCAN slide scanner and software
(Zeiss, Mirax 3D Histech). Numbers of positively stained
cells and vessels were counted using ImageJ (National
Institutes of Health, Bethesda, MD) and averaged per
0.6mm core. IF and double IHC images were acquired
using a Leica DM4000B microscope fitted with a CRI
Nuance spectral analyzer (Cambridge Research and Instrumentation,
Inc.) and analyzed using the supplied colocalization
tool to determine the percentage of single and
double positive pixels per region of interest.

\subsection{Microarray data analysis}
Gene expression data were exported from BeadStudio
version 3.1.3.0 (Illumina) in GeneSpring probe profile
format and processed and analyzed using the statistical
language R~\cite{r2.9.0}. As Illumina identifiers are not stable and
consistent between different chip versions, raw oligonucleotide
sequences were converted to nuIDs~\cite{du2007nuid}. Data
were transformed using the variance stabilizing transformation
algorithm to take advantage of the large number of
technical replicates available on the Illumina BeadChips~\cite{lin2008model}.
Transformed data were normalized using robust
spline normalization, an algorithm combining features
of quantile and loess normalization, specifically designed
to normalize variance\hyp{}stabilized data. All microarray data
processing was carried out by Bioconductor package {\it lumi}~\cite{gentleman2004bioconductor,du2008lumi}.
Quality control was performed using Bioconductor
package {\it arrayQualityMetrics}~\cite{kauffmann2009arrayqualitymetrics}. MIAME (minimum
information about a microarray experiment) compliant
data have been deposited in the GEO database (\url{www.ncbi.nlm.nih.gov/geo/}, accession number GSE21257).

\subsection{Statistical analysis}
Differential expression between patients who did ($n=34$) and did not ($n=19$) develop metastases within 5 years
from diagnosis of the primary tumor was determined using
linear models for microarray data ({\it LIMMA}~\cite{smyth2004linear}), applying
a Benjamini and Hochberg false discovery rate\hyp{}adjusted p-value cutoff of 0.05. Other univariate statistical
analyses were performed using GraphPad Prism Software
(version 5.01). Multivariate survival analyses were carried
out according to the Cox proportional hazards model in
SPSS (version 16.0.2). Two\hyp{}sided p-values $<0.05$ were determined to be significant; p-values between 0.05 and
0.15 were defined to be a trend.

\section{Results}\label{results4}
\subsection{High expression of macrophage\hyp{}associated genes in osteosarcoma biopsies of patients who did not develop metastases within 5 years from diagnosis (cohort 1)}
Comparison of genome\hyp{}wide gene expression in tumors
of patients who did and did not develop metastases within
5 years resulted in 139 significantly differentially expressed
(DE) probes, of which 125 corresponded to 118 upregulated
and 14 to downregulated genes in patients who did
not develop metastases. A summary of DE genes and
detailed descriptions of all probes can be found in Table~\ref{tab4.1}
and Supplemental Table S4 ({\it available online}~\cite{ch4additional}), respectively. Two DE genes
were specific for macrophages ({\it CD14} and {\it MSR1}) and 30/132 of the DE genes were associated with macrophage
functions such as antigen processing and presentation
({\it e.g.} {\it HLA-DRA} and {\it CD74}) or pattern recognition ({\it e.g.}
{\it TLR4} and {\it NLRP3}). Overall, approximately 20\% of the
upregulated probes corresponded to genes that were associated
with macrophage function and development and an
additional 25\% of the upregulated probes corresponded to
genes with other immunological functions, such as cytokine
production and phagocytosis. Four genes were
selected for validation of the microarray data using qPCR:
{\it CD14}, {\it HLA-DRA}, {\it CLEC5A}, and {\it FCGR2A}. Expression
levels as determined by qPCR correlated well with
expression levels obtained by microarray analysis (Supplemental
Figure S1 ({\it available online}~\cite{ch4additional})). Metastases\hyp{}free survival curves of the same
cohort, generated using median expression of the probe of
interest as a cutoff determining low and high expression,
are shown in Figure~\ref{fig4.1}B and Supplemental Figure 2 ({\it available online}~\cite{ch4additional}). Cox
proportional hazards analysis revealed expression of
macrophage\hyp{}associated genes {\it CD14} and {\it HLA-DRA} to be
independently associated with metastasis\hyp{}free survival
(Supplemental Table S5 ({\it available online}~\cite{ch4additional})).
%
\begin{landscape}
	\begin{table}[htbp]
		\centering
		\small
		\begin{tabular}{|>{\raggedright}p{2.3in} >{\raggedleft}p{0.6in} >{\raggedleft}p{0.6in} >{\raggedright}p{3in} >{\raggedleft}p{0.6in} >{\raggedleft}p{0.6in} l|}
			\hline
			& \multicolumn{3}{c}{Higher expression in patients without metastases} & \multicolumn{3}{l|}{Lower expression in these patients}\\
			\cmidrule(r){2-4}\cmidrule(r){5-7}
			\\[-3em]\mystrut
		& Number of probes & Number of genes & Examples & Number of probes & Number of genes & Examples\\
			\hline
			\rule{-4pt}{3ex} Pattern recognition receptor or signaling & 18 & 17 & {\it MSR1}, {\it CD14}, {\it NLRP3}, {\it TLR7}, {\it TLR8}, {\it TLR4}, {\it NAIP}, {\it IL1B}, {\it PYCARD}, {\it NLRC4} & 0 & 0 & \\
			Immunological & 16 & 15 & {\it CD86}, {\it C1QA}, {\it LY9}, {\it CD37}, {\it LY86} & 0 & 0 & \\
			HLA class II & 12 & 12 & {\it HLA-DMB}, {\it HLA-DRA}, {\it CD74}, {\it HLA-DQA1} & 0 & 0 & \\
			Hematopoietic cells & 11 & 10 & {\it HMHA1}, {\it MYO1G}, {\it LST1} & 0 & 0 & \\
			Cytokines and cytokine signaling & 7 & 6 & {\it CXCL16}, {\it CSF2RA}, {\it IFNGR1}, {\it IL10RA} & 1 & 1 & {\it MAP2K7}\\
			Metabolism & 9 & 9 & {\it PFKFB2}, {\it SLC2A9}, {\it CECR1}, {\it ALOX5} & 0 & 0 & \\
			Fc receptor & 6 & 4 & {\it FCGR2B}, {\it FCGR2A}, {\it FGL2}, {\it PTPN6} & 0 & 0 & \\
			Cytoskeleton & 5 & 5 & {\it HCLS1}, {\it WAS}, {\it IQGAP2} & 1 & 1 & {\it DNAI2}\\
			(An)ion transporters and channels & 4 & 4 & {\it SLCO2B1}, {\it SLC11A1} & 1 & 1 & {\it SLC24A4}\\
			AKT pathway & 3 & 3 & {\it PIK3IP1}, {\it PKIB} & 0 & 0 & \\
			Endocytosis & 3 & 3 & {\it APPL2}, {\it NECAP2} & 0 & 0 & \\
			Apoptosis, cell cycle control, and proliferation & 4 & 4 & {\it TMBIM4}, {\it TNFRSF1B}, {\it OGFRL1} & 1 & 1 & {\it BCCIP}\\
			Signaling & 4 & 4 & {\it RGS10}, {\it MFNG}, {\it FHL2}, {\it PILRA} & 0 & 0 & \\
			Growth hormone signaling & 0 & 0 & & 1 & 1 & {\it GHR}\\
			Morphogenesis & 0 & 0 & & 1 & 1 & {\it HOXC4}\\
			Others & 7 & 6 & {\it CUGBP2}, {\it CYP2S1}, {\it VAV1}, {\it GGN} & 2 & 2 & {\it NSUN5}, {\it MRPL4}\\
			Unknown & 16 & 16 & {\it VMO1}, {\it MICALCL}, {\it MS4A6A} & 6 & 6 & {\it NHN1}, {\it BRWD1}\\
			Total & 125 & 118 & & 14 & 14 & \\
			\hline
		\end{tabular}
		\caption{DE genes and probes by category comparing high\hyp{}grade osteosarcoma patients with and without metastases within 5 years by genome\hyp{}wide expression profiling (cohort 1).}
		\label{tab4.1}
	\end{table}
\end{landscape}
%

\subsection{Macrophage\hyp{}associated genes are expressed by infiltrating he\-ma\-to\-poi\-e\-tic cells and not by tumor cells}
The most probable source of expression of the DE macrophage\hyp{}associated genes was infiltrating immune cells and
not osteosarcoma cells. To confirm this, we performed qPCR
of {\it CD14} and {\it HLA-DRA} on osteosarcoma cell lines ($n=19$) and biopsies ($n=45$, a subset of cohort 1). {\it CD14} and
{\it HLA-DRA} expression was variable in osteosarcoma biopsies,
but almost undetectable in cell lines. This indicates that
these macrophage\hyp{}associated genes were not expressed by
tumor cells but by infiltrating cells because only osteosarcoma
biopsies contain macrophage infiltrate, whereas RNA
from cell lines is exclusively from tumor cells (Figure~\ref{fig4.1}A,
Mann\hyp{}Whitney U test p-value $<0.0001$).
%
\begin{figure}[htbp]
	\centering
	\includegraphics[width=1.0\textwidth]{figs04/fig1rgb.pdf}	% pdf version also rgb
	\caption{Macrophage\hyp{}associated genes are not expressed by osteosarcoma tumor cells. {\it A}, qPCR of osteosarcoma cell lines and biopsies of {\it CD14} and {\it HLA-DRA} demonstrating lack of expression by osteosarcoma cells. Mann\hyp{}Whitney U test p-value $<0.0001$, $\ast\ast\ast$. {\it B}, High expression of macrophage associated genes was associated with a better metastasis\hyp{}free survival (cohort 1, Kaplan\hyp{}Meier curve, p-value obtained by Logrank test, patients with metastasis at diagnosis have an event at $t=0$. These patients are included, because patients who develop metastases later on may as well have micrometastases at the time of diagnosis). Metastasis\hyp{}free survival curves for {\it HLA-DRA}, {\it CLEC5A}, and {\it FCGR2A} can be found in Supplemental Figure S2 ({\it available online}~\cite{ch4additional}). {\it C}, Double immunohistochemical staining of CD163 with the hematopoietic cell marker CD45 was performed and analyzed using spectral imaging microscopy. The pseudo\hyp{}IF image (pseudo\hyp{}IF) shows CD163\hyp{}positive cells in red, CD45\hyp{}positive cells in green, and colocalization of both markers in orange. Lack of expression of CD163 and CD45 on surrounding tumor cells (dark blue) and some single positive CD45 cells can be noted.}
	\label{fig4.1}
\end{figure}
%
In addition, we performed
double IHC for the hematopoietic cell marker
CD45, which is not expressed by osteosarcoma tumor cells,
and the macrophage marker CD163 or the macrophage\hyp{}associated
protein HLA-DR$\upalpha$ (Figure~\ref{fig4.1}C). We chose this
approach because no reliable osteosarcoma markers are
available~\cite{raymond2002conventional}. Our results confirmed that infiltrating hematopoietic
cells were the source of the macrophage\hyp{}associated
gene expression levels. Together, these data show that
osteosarcoma tumor cells do not express macrophage\hyp{}associated
genes, neither {\it in vitro} nor {\it in vivo}.

\subsection{Macrophage numbers in osteosarcoma biopsies
correlate with {\it CD14} gene expression levels and are
positively associated with localized disease and better
outcome (cohorts 2 and 3)}
To confirm the presence of TAMs in osteosarcoma, we
stained a tissue microarray containing 145 samples of 88
patients for the macrophage marker CD14 and counted the
number of positive cells per tissue microarray core (cohort
2; Figure~\ref{fig4.2}A).
%
%
\begin{figure}[htbp]
	\centering
%	\includegraphics[width=1.0\textwidth]{figs04/fig2bw.pdf}	% OBS! print version bw
	\includegraphics[width=1.0\textwidth]{figs04/fig2rgb.pdf}	% OBS! pdf version rgb
	\caption{{\it A}, Example of representative stainings of high\hyp{}grade osteosarcoma with high (left) versus low (right) levels of macrophage infiltration (CD14 staining) and vascular density (CD31 staining). {\it B}, High numbers of infiltrating macrophages (left, defined as the 3 upper quartiles, or more than 12 CD14\hyp{}positive cells per tissue array core) are associated with better OS (right, Logrank test p-value $=0.02$, cohort 2). Q1: lowest quartile, Q2, 3, 4, 3: highest quartiles.}
	\label{fig4.2}
\end{figure}
%
CD14 was chosen as opposed to CD68 because
the latter marker is not expressed by monocytes and often
shows cross\hyp{}reactivity with mesenchymal tissue ({\it data not
shown}). Number of CD14\hyp{}positive cells per tissue microarray
core correlated significantly with {\it CD14} mRNA expression
levels (14 samples overlap with gene expression
analysis, Spearman correlation coefficient 0.64, p-value $=0.01$). Similar to the gene expression data, there was a
trend for patients with primary localized disease to have
higher numbers of macrophages in pretreatment diagnostic
biopsies than patients with metastatic disease at presentation
(mean number of macrophages per core, 55 {\it vs} 27;
Mann\hyp{}Whitney U test p-value $=0.09$). Also, patients with
high macrophage counts at diagnosis tended to be less
likely to develop metastases within 5 years ($\chi^2$, p-value $=0.13$).

We subdivided this cohort into four quartiles based on
numbers of CD14\hyp{}positive cells to determine the group
with the best OS. No significant differences were found
between quartiles 2 and 4, but patients belonging to this
group had better OS than patients with low CD14 counts
(lowest quartile, or less than 12 CD14\hyp{}positive cells per
tissue array core; Figure~\ref{fig4.2}B, Logrank test p-value $=0.02$). In another
cohort of 16 patients, IF staining of CD14, CD163, and
HLA-DR$\upalpha$ was performed, again confirming a potential
prognostic value of high macrophage numbers (cohort 3,
Figure~\ref{fig4.3}, Logrank test p-value $=0.01$, Supplemental Figure S3 ({\it available online}~\cite{ch4additional})).
%
\begin{figure}[htbp]
	\centering
	\includegraphics[width=1.0\textwidth]{figs04/fig3rgb.pdf}	% pdf version also rgb
	\caption{{\it A}, Osteosarcoma samples are infiltrated with CD14 and CD163 single and double positive macrophages. Spectral imaging was used to reduce autofluorescence of osteosarcoma cells. In the composite image, CD14\hyp{}positive cells are represented in green, CD163\hyp{}positive cells are represented in red, and CD14/CD163 double positive cells are represented in yellow. Background autofluorescence of tumor cells is represented in gray. {\it B}, In an independent cohort of 16 patients (cohort 3), high macrophage infiltration as determined by IF CD14 staining was associated with significantly improved OS. p-values obtained using Logrank test, cutoff at the median.}
	\label{fig4.3}
\end{figure}
%

\subsection{Macrophages in osteosarcoma have both M1 and M2
characteristics}
To determine the phenotype of macrophages present in
osteosarcoma, we performed double IHC with CD14 and
either the M1\hyp{}associated marker HLA-DR$\upalpha$ or the M2\hyp{}associated
marker CD163. Not all CD163 and HLA-DR\hyp{}positive
infiltrating cells expressed CD14 (Figure~\ref{fig4.3}A and Supplementary
Figure S3A). The total number of macrophages as determined
by quantifying CD14\hyp{}positive macrophages was
associated with good survival (Figure~\ref{fig4.3}B), but the phenotype
of the macrophages (CD14/CD163 double positive versus
CD14/HLA-DR$\upalpha$ double positive) was not (Supplemental
Figure S3B ({\it available online}~\cite{ch4additional}); {\it data not shown}). Another M2 characteristic is
support of angiogenesis. The number of CD14\hyp{}positive
macrophages correlated with the number of CD31\hyp{}positive
vessels (Figure~\ref{fig4.2}A and Figure~\ref{fig4.4}), but vascularity did not correlate
with prognosis ({\it data not shown}).
%
\begin{figure}[htbp]
  \centering
  \begin{minipage}[b]{0.50\linewidth}
    \includegraphics[width=1\textwidth]{figs04/fig4bw.pdf}	% pdf version also bw
  \end{minipage}
    \hfill
  \begin{minipage}[b]{0.46\linewidth}
    \caption{Macrophage infiltration as determined by CD14\hyp{}positive cell count correlated with vascularity as determined by CD31\hyp{}positive vessel count. Data of all osteosarcoma samples (pre- and posttreatment primary tumor and metastatic samples, cohort 2) are shown. Q1: lowest quartile, Q2, 3, 4: three highest quartiles. Kruskal-Wallis test p-value $<0.0001$. $\ast$, Dunn's posttest p-value $<0.05$, $\ast\ast\ast$, Dunn's posttest p-value $<0.001$.}
     \label{fig4.4}
     \end{minipage}
\end{figure}
%

\subsection{Macrophage numbers in diagnostic biopsies may
predict histological response to chemotherapy and
macrophage number increases following
chemotherapy treatment}
There was a trend for high macrophage count (highest three
quartiles or more than 12 CD14\hyp{}positive cells per tissue
array core) in prechemotherapy diagnostic biopsies of the
primary tumor to predict for good histological response to
neoadjuvant chemotherapy (defined as more than 90\%
nonvital tumor tissue upon final resection), since 46\% of
patients with high macrophage numbers and 18\% of
patients with low macrophage numbers had a good histological
response (cohort 2; $\chi^2=0.09$). The prognostic
benefit of macrophage counts in osteosarcoma was not
independent of histological response using Cox proportional
hazard analysis. Macrophage numbers were higher
in postchemotherapy resections of the primary tumor than
in prechemotherapy biopsies (Supplemental Figure S4 ({\it available online}~\cite{ch4additional})).
Moreover, gene expression analysis showed upregulation
of macrophage\hyp{}associated probes in postchemotherapy
resections ($n=4$) as compared with prechemotherapy
biopsies ($n=79$, {\it data not shown}).

\section{Discussion}\label{discussion4}
OS of high-grade osteosarcoma patients with resectable
metastatic disease is poor at about 20\%~\cite{buddingh2010prognostic}. Mechanisms
for the development of metastases in osteosarcoma are
elusive. To identify genes that play a role in this process,
we performed genome\hyp{}wide expression profiling on
prechemotherapy biopsies of osteosarcoma patients. We
compared patients who developed clinically detectable
metastases within 5 years with patients who did not
develop metastases within this time frame (cohort 1).
About 20\% of genes overexpressed in patients without
metastases were macrophage\hyp{}associated, whereas an additional
25\% of genes had other immunological functions
({\it e.g.} in phagocytosis, complement activation or cytokine
production and response) but could still be attributed to
macrophages (Table 1 and Supplemental Table S4 ({\it available online}~\cite{ch4additional})). Thus,
in total, almost half of the DE genes belonged to one specific
process, {\it i.e.} macrophage function. Macrophage\hyp{}associated
genes were expressed by infiltrating hematopoietic cells
and not by osteosarcoma tumor cells (Figure~\ref{fig4.1}), indicating
a possible role for macrophages in preventing metastasis in
osteosarcoma. To confirm these findings, we quantified
infiltrating macrophages in two additional cohorts (cohorts 2
and 3) and found an association with better OS in both
cohorts.

The antimetastatic effect of TAMs in osteosarcoma is
remarkable because TAMs support tumor growth in a
substantial number of other cancers, which are mostly
tumors of epithelial origin. For example, macrophages
are associated with the angiogenic switch in breast cancer~\cite{lin2006macrophages}. We find an association between macrophage infiltration
and higher microvessel density, which suggests that
the influx of macrophages may support certain aspects of
tumor growth in osteosarcoma as well. However, in the
case of osteosarcoma, direct or indirect antitumor activity
of macrophages apparently outweighs their possible
tumor\hyp{}supporting effects. Macrophages can alter their phenotype
from M2 to M1 and become the tumor's foe instead
of its friend, given the right circumstances~\cite{hagemann2008re,sinha2005reduction,buhtoiarov2006macrophages}. The
TAMs that were identified in this study in osteosarcoma
had both M1 and M2 characteristics. The expression of
CD163 and the association with angiogenesis are M2
characteristics~\cite{lin2006macrophages,ojalvo2009high}. Some of the DE genes, such as
{\it MSR1} and {\it MS4A6A} are specific for M2 macrophages {\it in vitro}~\cite{martinez2006transcriptional}. Others, such as the proinflammatory cytokine {\it IL1B},
are more indicative of an M1 phenotype~\cite{mosser2008exploring}. How macrophages
inhibit osteosarcoma metastasis and whether a
balance between M1- and M2-type functions is responsible
is unknown.

In a multivariate regression model, the survival benefit
of high TAM numbers was at least partly dependent on
histological response to chemotherapy. Chemotherapy
can cause `immunogenic cell death' of cancer cells,
resulting in the release of endogenous danger signals~\cite{zitvogel2008immunological,kono2008dying}. The binding of these dangerous signals to
pattern recognition receptors on macrophages can skew
polarization of M2- to M1-type TAMs. The interaction
between dying tumor cells and resident TAMs may facilitate
clearance or inhibit outgrowth of metastatic tumor
cells. However, patients with localized disease at diagnosis
tended to have a larger macrophage infiltrate than
patients with metastatic disease at diagnosis (mean number
of macrophages per core 55 {\it vs} 27). At this point,
patients have not undergone chemotherapeutic treatment
yet and an interaction between chemotherapy and macrophages
can therefore not be responsible for the antimetastatic
effect of macrophages. Perhaps, the antimetastatic
effect of TAMs in these patients is due to the constitutive
presence of macrophages with an M1 phenotype. Alternatively,
the presence of macrophages might be a
reflection of a microenvironment not conducive for
metastasis.

Although preliminary analysis of a clinical trial investigating
the effect of treatment with the macrophage
activating agent MTP yielded conflicting results, subsequent
analysis revealed that treatment with MTP
improved 6-year OS from 70\% to 78\% in a cohort of
patients with primary localized disease~\cite{meyers2008osteosarcoma,meyers2005osteosarcoma}. Similar
results were obtained in canine osteosarcoma~\cite{kurzman1995adjuvant}. MTP
is a synthetic derivative of muramyl dipeptide (MDP), a
common bacterial cell wall component. Muropeptides
bind to intracellular pattern recognition receptors of
the nucleotide binding and oligomerization domain
(NOD)-like receptor (NLR) family, expressed
by macrophages~\cite{geddes2009unleashing}. In our study, 5 genes associated
with NLR family signaling and the associated `inflammasome'
were highly expressed in pretreatment biopsies of
patients who do not develop metastases. The DE genes
{\it NLRP3}, {\it NAIP}, {\it NLRC4}, and {\it PYCARD} are components of
the inflammasome, {\it LYZ} is a lysozyme that processes
bacterial cell wall peptidoglycan into MDP, a ubiquitous
natural analogue of MTP, and {\it IL1B} is the downstream
effector cytokine of the inflammasome pathway. Further
research is needed to clarify whether only patients with
high numbers of TAMs benefit from MTP treatment, or
whether MTP treatment is effective regardless of macrophage
number or activation status pretreatment. Also, it is
unknown whether treatment with agents promoting
macrophage migration or with other macrophage activating
agents like toll\hyp{}like receptor ligands or IFNs has a
similar beneficial effect on outcome.

Previous genome\hyp{}wide expression profiling studies in
osteosarcoma focused on identifying genes that predict
histological response to neoadjuvant chemotherapy~\cite{ochi2004prediction,salas2009molecular,man2005expression,mintz2005expression}.
As a consequence, the importance of macrophages in
controlling metastases was not recognized. However, we
previously compared gene expression profiles of osteosarcoma
biopsies and cultured mesenchymal stem cells
and determined which genes are expressed by tumor
stroma and not by tumor cells~\cite{cleton2009profiling}. There is a considerable
overlap between the stromal genes identified in
our previous study and the macrophage\hyp{}associated
genes identified in the present study (including HLA
class II genes as the most prevalent DE group of genes
and the macrophage\hyp{}associated genes {\it MSR1}, {\it MS4A6A},
and {\it FCGR2A}).

In conclusion, we showed the presence and clinical
significance of TAMs in pretreatment samples of high\hyp{}grade
osteosarcoma. TAMs in osteosarcoma are a heterogeneous
cell population with both M1 antitumor and M2 protumor
characteristics. Although the exact mechanism by which
macrophages exert their antimetastatic functions is still
unknown, this study provides an important biological
rationale for the treatment of osteosarcoma patients with
macrophage activating agents.

%%% references

\begin{small}
\begin{singlespace}
\bibliographystyle{unsrtnatshort}		% sorted as referenced, was unsrtnat, but unsrtnatshort gives shorter output
\bibliography{biblio}
\end{singlespace}
\end{small}

%\end{document}
% Marieke Kuijjer
% 2013-02-15
% chapter 05

	%\documentclass[12pt,b5paper]{book}
	%\setcounter{secnumdepth}{0}
	%\setcounter{tocdepth}{1}
	%\usepackage[hidelinks]{hyperref}

%\begin{document}

%%% title page

\chapter{IR/IGF1R signaling as potential target for treatment of high-grade osteosarcoma}\label{ch5}
\thispagestyle{empty}				%%% to remove page number from first page of chapter, must be placed after calling the chapter

\vfill

\vspace{0.5cm}
This chapter is based on the manuscript:
\underline{Kuijjer ML}, Peterse EFP, van den Akker BEWM, Briaire-de Bruijn IH, Serra M, Meza-Zepeda LA, Myklebost O, Hassan AB, Hogendoorn PCW, Cleton-Jansen AM. Accepted for publication in {\it BMC Cancer}

\newpage

%%% main document

\section{Abstract}\label{abstract5}
\textbf{Background}: High-grade osteosarcoma is an aggressive tumor most often developing in the long bones of adolescents, with a second peak in the 5\textsuperscript{th} decade of life. Better knowledge on cellular signaling in this tumor may identify new possibilities for targeted treatment.

\textbf{Methods}: We performed gene set analysis on previously published genome\hyp{}wide gene expression data of osteosarcoma cell lines ($n=19$) and pretreatment biopsies ($n=84$). We characterized overexpression of the insulin\hyp{}like growth factor receptor (IGF1R) signaling pathways in human osteosarcoma as compared with osteoblasts and with the hypothesized progenitor cells of osteosarcoma---mesenchymal stem cells. This pathway plays a key role in the growth and development of bone. Since most profound differences in mRNA expression were found at and upstream of the receptor of this pathway, we set out to inhibit IR/IGF1R using OSI-906, a dual inhibitor for IR/IGF1R, on four osteosarcoma cell lines. Inhibitory effects of this drug were measured by Western blotting and cell proliferation assays.

\textbf{Results}: OSI-906 had a strong inhibitory effect on proliferation of 3 of 4 osteosarcoma cell lines, with IC$_{50}$s below 100nM at 72hrs of treatment. Phosphorylation of IRS-1, a direct downstream target of IGF1R signaling, was inhibited in the responsive osteosarcoma cell lines.

\textbf{Conclusions}: This study provides an {\it in vitro} rationale for using IR/IGF1R inhibitors in preclinical studies of osteosarcoma.

\section{Background}\label{introduction5}
High-grade osteosarcoma is the most prevalent primary malignant bone tumor. The disease occurs most frequently in children and adolescents at the site where proliferation is most active, {\it i.e.} the metaphysis adjacent to the epiphyseal plate~\cite{raymond2002conventional}. The 5-year overall survival of osteosarcoma patients has raised from 10--20\% to about 60\% after the introduction of preoperative chemotherapy in the 1970s. However, about 45\% of all patients still die because of distant metastasis. No additional treatments have been found that can increase survival significantly, and administering higher doses of preoperative chemotherapy does not result in improved outcomes~\cite{lewis2007improvement,eselgrim2006dose}. Better knowledge on cellular signaling in high\hyp{}grade osteosarcoma may identify new possibilities for targeted treatment of this highly aggressive tumor.

We have previously described the roles of bone developmental pathways Wnt, TGF$\upbeta$\hyp{}BMP, and Hedgehog signaling in osteosarcoma, but unfortunately so far could not identify suitable targets for treatment~\cite{cai2010inactive,mohseny2012activities}. In addition to these signal transduction pathways, insulin\hyp{}like growth factor 1 receptor (IGF1R) signaling plays a key role in the growth and development of bone. Aberrant signaling of this pathway has been implicated in various cancer types, among others sarcomas~\cite{rikhof2009insulin,maki2010small}. Key players of insulin\hyp{}like growth factor (IGF) signaling are the ligands IGF1, IGF2, which are circulating polypeptides that can be expressed in endocrine, paracrine, and autocrine manners, and the tyrosine kinase receptor IGF1R, which forms homodimers, or hybrid receptors with the insulin receptor (IR)~\cite{pollak2012insulin}. IGF1R and IR/IGF1R hybrids are activated by both IGF1 and -2, which trigger autophosphorylation of IGF1R and subsequent downstream signal transduction. A second IGF receptor, IGF2R, can bind IGF2, but does not confer intracellular signaling, thereby diminishing the bioavailability of IGF2 to IGF1R~\cite{siddle2012molecular}. Autophosphorylation of IR/IGF1R receptors recruits the signaling proteins insulin receptor substrate (IRS) and Src homology 2 domain containing transforming protein (Shc) to the cell membrane, which get phosphorylated and subsequently activate the downstream PI3K/Akt and Ras/Raf/ERK signaling pathways, both of which are known to be important in cancer. These pathways ultimately act on several biological processes, such as transcription, proliferation, growth, and survival~\cite{siddle2012molecular,foulstone2005insulin,siddle2011signalling}. Interestingly, treatment targeted against IGF1R signaling has shown to be effective in a subset of Ewing sarcoma, another bone tumor that manifests at young age~\cite{subbiah2010targeted}.

The role of the IGF1R pathway in growth has been illustrated in studies of knockout mice. It was shown that IGF1 null mice are 40\% smaller than littermates, while IGF1R null mice are approximately 55\% smaller~\cite{liu1993mice}. In dogs, the size of different breeds was demonstrated to be dependent on IGF1 plasma levels~\cite{maki2010small}. Additionally, a specific IGF1 SNP haplotype was described to be common in small breed dogs and nearly absent in giant breeds~\cite{sutter2007single}. Interestingly, large and giant dog breeds are more prone to develop osteosarcoma~\cite{selvarajah2010prognostic}, which in dogs is biologically very similar to the human disease~\cite{kirpensteijn2008tp53}. Two recent studies on human osteosarcoma suggest a positive correlation between patient birth weight and height at diagnosis and the development of the disease~\cite{arora2011relationship,mirabello2011height}. Involvement of some members of IGF1R signaling in osteosarcoma has been described (as has been reviewed in Kolb {\it et al}.~\cite{kolb2009development}), but the activity of this pathway remains to be determined.

We have analyzed genome\hyp{}wide gene expression in high\hyp{}grade osteosarcoma cell lines and pretreatment biopsies, and observed significantly altered activity of genes involved in IGF1R signaling when compared to profiles of mesenchymal stem cells and osteoblasts. Specifically, upstream inhibitors of IGF1R signaling were found to be downregulated in osteosarcoma, and low expression of these genes correlated with worse event\hyp{}free survival. We inhibited IR/IGF1R signaling with the dual IR/IGF1R inhibitor OSI-906. This showed inhibition of phosphorylation of IRS-1 and of strong inhibition of proliferation in 3/4 osteosarcoma cell lines. Interestingly, the cell line which could not be inhibited with OSI-906, 143B, has a {\it KRAS} oncogenic transformation, which is a component of the Ras/Raf/ERK pathway, one of downstream effectors of IGF1R signaling. These results suggest that IR/IGF1R signaling may be an effective targeted for treatment of high\hyp{}grade osteosarcoma patients.

\section{Methods}\label{methods5}
\subsection{Cell culture}
The 19 high\hyp{}grade osteosarcoma cell lines that were used in this study were characterized and are described by Ottaviano {\it et al}.~\cite{ottaviano2010molecular}. The 12 mesenchymal stem cell and 3 osteoblast cultures were previously described~\cite{kuijjer2012identification}. MSCs have been previously~\cite{cleton2009profiling} characterized through FACS analysis and have been tested for their ability to be committed under proper conditions towards adipogenesis, chondrogenesis and osteogenesis as described in Bernardo {\it et al}.~\cite{bernardo2007human}. Osteoblast cultures were derived from MSCs which were treated to undergo osteogenic differentiation. Cell line DNA was short tandem repeat profiled to confirm cell line identity with use of the Cell ID system of Promega (Madison, WI). For Western blotting experiments, cells were maintained in RPMI 1640 (Invitrogen, Carlsbad, CA), supplemented with 10\% fetal bovine serum (F7524, Sigma-Aldrich, St. Louis, MO) and 1\% glutamax (Gibco 35050, Invitrogen, Carlsbad, CA).

\subsection{Microarray experiments, preprocessing, and data analysis}
For genome\hyp{}wide gene expression analysis, we used Illumina Human-6 v2.0 BeadChips. Microarray experiments and data preprocessing are described in Kuijjer {\it et al}.~\cite{kuijjer2012identification}. Previously deposited genome\hyp{}wide gene expression data of mesenchymal stem cells (MSCs) and osteoblasts can be found in the Gene Expression Ombinus (GEO accession numbers GSE28974 and GSE33382, respectively). Data from osteosarcoma cell lines have been published before~\cite{namlos2012modulation}, but since we normalized and processed all raw data together, we deposited normalized values in the Gene Expression Omnibus (GEO, accession number GSE42351, superseries accession GSE42352). Data from the 84 high\hyp{}grade osteosarcoma pretreatment biopsies have been previously published (GEO accession number GSE33382)~\cite{kuijjer2012identification}. Ethical guidelines of the individual European partner institutions were followed and samples and clinical data were handled in a coded fashion and stored in the EuroBoNeT biobank. We determined significant differential expression between osteosarcoma cell lines ($n=19$) and mesenchymal stem cells ($n=12$), and between osteosarcoma cell lines and osteoblasts ($n=3$) using Bioconductor~\cite{gentleman2004bioconductor} package {\it LIMMA}~\cite{smyth2004linear} in statistical language R~\cite{r2.15.0}. Probes with Benjamini and Hochberg false discovery rate\hyp{}adjusted p-values $<0.05$ were considered to be significant. Gene set analysis was performed on KEGG pathways~\cite{kanehisa2000kegg} (Release 63.0, July 1, 2012) using R-package globaltest~\cite{goeman2004global}. For each analysis, the top 15 significant KEGG pathways were returned. All returned pathways had a Benjamini and Hochberg false\hyp{}discovery rate\hyp{}corrected p-value $<1\cdot10^{-5}$. To visualize differential expression in the IGF1R pathway, we performed Core analyses using Ingenuity Pathways Analysis (IPA, Ingenuity Systems, \url{www.ingenuity.com}).

\subsection{Antibodies and reagents}
Rabbit monoclonal and polyclonal antibodies against IGF1R and IRS-1, respectively (both $1:1,000$) were obtained from Cell Signaling (Danvers, MA). Rabbit polyclonal antibody against phospho-IRS-1 (Y612, $1:1,000$) was purchased from Biosource, Invitrogen (Carlsbad, CA). A mouse monoclonal antibody against $\upalpha$-tubulin from Abcam (Cambridge, UK) was used as a loading control ($1:3,000$). Secondary antibodies (both $1:10,000$, BD Transduction Laboratories, Lexington, KY) were horseradish peroxidase (HRP) conjugated polyclonal goat\hyp{}anti\hyp{}rabbit IgG for components of the IR/IGF1R pathway, and HRP conjugated polyclonal goat\hyp{}anti\hyp{}mouse for $\upalpha$-tubulin. OSI-906 was purchased from Selleck Chemicals LLC (Houston, TX).

\subsection{Western blotting}
Osteosarcoma cell lines OHS, KPD, SAOS-2, and 143B were treated with 0.5\% DMSO or with 1$\upmu$M OSI-906 for 3hrs, and were subsequently lysed using Mammalian Protein Extraction Reagent (Thermo Scientific 78503), to which Halt Phosphatase and Protease Inhibitor Cocktails (Thermo Scientific 78420 and 78418, respectively) were added according to the manufacturer's protocol. Concentrations of cell lysates were determined using the BioRad DC Protein Assay Kit (Biorad, Hercules, CA). Per sample, 20$\upmu$g of protein was loaded on SDS-PAGE gels. Lysate of HepG2-A16 cells transfected with IR and stimulated with insulin, containing 10$\upmu$g of protein, was taken along as a positive control. Western blotting was performed as described by Schrage {\it et al}.~\cite{schrage2009kinome}.

\subsection{Proliferation assays}
OSI-906 was diluted in DMSO and stored at -20$^\circ$C. OHS, SAOS-2, KPD, and 143B cells were plated in 96 wells plates, using $4,000$, $2,000$, $12,000$, and $2,000$ cells per well, respectively. After 24hrs, OSI-906 was added in triplicate at different concentrations---0nM, 0.01nM, 0.1nM, 1nM, 10nM, 100nM, 1$\upmu$M, and 10$\upmu$M. The inhibitor was incubated for 72hrs and 96hrs, in different experiments. The results shown are representative results from at least three independent experiments. Cell proliferation reagent WST-1 (Roche) was incubated for 2hrs and subsequently measured using a Wallac 1420 VICTOR2 (Perkin Elmer, Waltham, MA). Data were analyzed in Graphpad Prism 5.0 (\url{www.graphpad.com}). Relative IC$_{50}$s were calculated using results from the different concentrations up to the highest dose where toxicity was not yet present.

\section{Results}\label{results5}
\subsection{Enrichment of IGF1R signaling in high\hyp{}grade osteosarcoma}
Genome\hyp{}wide gene expression data were of good quality for all cell lines. {\it LIMMA} analysis resulted in $7,891$ probes encoding for differentially expressed (DE) genes between osteosarcoma cell lines and MSCs, and $2,222$ probes encoding for DE genes between osteosarcoma cells and osteoblasts. We tested the global expression patterns of KEGG pathways using globaltest~\cite{goeman2004global} and determined the intersection of the pathways most significantly different in osteosarcoma cell lines as compared with MSCs, and of osteosarcoma cell lines as compared with osteoblasts. This approach resulted in five significantly affected pathways---insulin signaling pathway, oocyte meiosis, ubiquitin mediated proteolysis, progesterone\hyp{}mediated oocyte maturation, and glycerophospholipid metabolism. Details of the globaltest are shown in Table~\ref{tab5.1}.
%
%%% table tab5.1 %%% OBS! i had some problems with font size and spacing for a horizontal longtable (not for tab2.1, which is a vertical longtable), is ok
%%% so i put an enter before the table
%
\begin{myTable}
%\begingroup
%	\begin{singlespacing}
%{\small
%     \renewcommand{\arraystretch}{0.5}
		\begin{longtable}[c]{|>{\raggedright}p{1.55in} >{\raggedright}p{0.95in} >{\raggedright}p{0.85in} >{\raggedright}p{0.6in} >{\raggedright}p{0.6in} >{\raggedright}p{0.53in}|}
		\hline
		KEGG pathway & Analysis & adjP & Statistic & Expected & Std.dev\tabularnewline
		\hline
		\rule{-2.5pt}{1ex} Insulin signaling pathway & OScellvsOB \\ OScellvsMSC & $1.01\cdot10^{-7}$ \\ $3.07\cdot10^{-15}$ & $26.34$ \\ $35.12$ & $4.76$ \\ $3.33$ & $1.92$ \\ $1.78$\tabularnewline
		Oocyte meiosis & OScellvsOB \\ OScellvsMSC & $2.70\cdot10^{-7}$ \\ $5.04\cdot10^{-16}$ & $37.45$ \\ $53.70$ & $4.76$ \\ $3.33$ & $2.90$ \\ $2.84$\tabularnewline
		Ubiquitin mediated proteolysis & OScellvsOB \\ OScellvsMSC & $3.21\cdot10^{-7}$ \\ $5.04\cdot10^{-16}$ & $22.88$ \\ $37.99$ & $4.76$ \\ $3.33$ & $1.75$ \\ $1.89$\tabularnewline
		Progesterone-mediated oocyte maturation & OScellvsOB \\ OScellvsMSC & $7.16\cdot10^{-7}$ \\ $1.34\cdot10^{-15}$ & $34.26$ \\ $55.35$ & $4.76$ \\ $3.33$ & $2.71$ \\ $2.77$\tabularnewline
		Glycerophospholipid metabolism & OScellvsOB \\ OScellvsMSC & $1.40\cdot10^{-6}$ \\ $2.25\cdot10^{-15}$ & $27.13$ \\ $55.86$ & $4.76$ \\ $3.33$ & $2.25$ \\ $2.82$\tabularnewline
		\hline
		\caption{The top five significant pathways with aberrant expression in both osteosarcoma cell lines versus osteoblasts (OScellvsOB) and osteosarcoma cell lines versus mesenchymal stem cells (OScellvsMSC). adjP: FDR\hyp{}adjusted p-value, Statistic: test statistic of the globaltest, Expected: expected test statistic of the globaltest, Std.dev: standard deviation under the null hypothesis.}
		\label{tab5.1}
		\end{longtable}
%	\end{small}
%}
%\endgroup
%     \renewcommand{\arraystretch}{1.0}
\end{myTable}
%
IGF1R signaling is involved in three out of the five detected KEGG pathways (insulin signaling pathway, oocyte meiosis, and progesterone\hyp{}mediated oocyte maturation). Interestingly, a globaltest on mRNA expression of previously published pretreatment biopsies~\cite{kuijjer2012identification} compared with normal bones~\cite{namlos2012global} also returned insulin signaling as the most significantly affected pathway ({\it data not shown}). Notably, there is no specific IGF1R signaling pathway in the KEGG database~\cite{kanehisa2000kegg}. Because of the overrepresentation of IGF1R signaling, and because of its known role in cancer, we decided to study expression of members of this pathway in detail. 

\subsection{Differentially expressed genes of the IGF1R pathway}
To determine which genes have the most specific up- or downregulation in osteosarcoma, we combined lists of significantly differentially expressed genes of osteosarcoma cell lines ($n=19$) and a previously published set of osteosarcoma pretreatment biopsies ($n=84$, GEO accession GSE33382) in comparison with mesenchymal stem cells ($n=12$) and osteoblasts ($n=3$) by four\hyp{}way Venn analysis of all significantly affected probes with the same direction of fold change (upregulated or downregulated in all four analyses). We identified {\it IGFBP4} and {\it GAS6} as the most downregulated genes in osteosarcoma (average log fold changes of $-4.43$ and  $-4.29$, respectively). {\it IGFBP2} was also present in the top 20 results from this four\hyp{}way analysis. In addition, {\it IGFBP3} and {\it -7} were significantly downregulated, and {\it IGF2BP3} was significantly upregulated in three out of the four analyses. Both {\it IGFBP4} and {\it GAS6} show high variability in expression in osteosarcoma cell lines and biopsies (Figure~\ref{fig5.1}A).
%
\begin{figure}[htbp]
	\centering
	\includegraphics[width=1.0\textwidth]{figs05/fig1bw.pdf}	% pdf version also bw
	\caption{{\it A}, Normalized gene expression levels of {\it GAS6} and {\it IGFBP4} in osteosarcoma biopsies, cell lines, mesenchymal stem cells (MSCs), and osteoblasts (OB). Expression of both proteins is considerably higher in the controls (FDR\hyp{}adjusted p-value $<0.001$ for both genes in all four analyses). {\it B}, Kaplan\hyp{}Meier curves depicting metastasis\hyp{}free survival in years for 83 high\hyp{}grade osteosarcoma patients (for 1/84 patients, we did not have follow\hyp{}up data available), based on quartiles of mRNA expression of the genes of interest.}
	\label{fig5.1}
\end{figure}
%
Patients whose biopsies had very low expression of these genes had poor event\hyp{}free survival profiles (Logrank test for trend, p-value $=0.01$ for {\it IGFBP4} and p-value $=0.04$ for {\it GAS6}, Figure~\ref{fig5.1}B). To visualize mRNA expression of the IGF1R signaling pathway members, we used Ingenuity Pathways Analysis on {\it LIMMA} toptables from osteosarcoma cells as compared with mesenchymal stem cells and from osteosarcoma cells as compared with osteoblasts (Figure~\ref{fig5.2}). As can be seen in this figure, overlap of differentially expressed genes between these analyses was detected upstream of IGF1R.
%
\begin{figure}[htbp]
	\centering
	\includegraphics[width=1.0\textwidth]{figs05/fig2col.pdf}	% pdf version also col
	\caption{This figure shows the IGF1R signaling pathway, with significantly upregulated genes in red, downregulated genes in green, and genes that did not meet our criteria for significance in gray. The left part of the symbols shows the analysis of osteosarcoma cell lines as compared with mesenchymal stem cells, the right part as compared with osteoblasts. Most consensus in gene expression is found upstream IGF1R signaling, in the expression of the IGF binding proteins.}
	\label{fig5.2}
\end{figure}
%

\subsection{OSI-906 inhibits phosphorylation of IRS-1}
Gene expression levels of IGF1R and IRS-1 were validated at the protein level by Western blot analysis ({\it data not shown}). We used phosphorylated IRS-1 as a readout for IR/IGF1R signal transduction activity, as IRS-1 is a direct downstream target of these receptors. We performed Western blot analysis on cell lysates of OHS, KPD, SAOS-2, and 143B, using antibodies against IRS-1 and phosphorylated IRS-1, before and after treatment with OSI-906---a selective small molecule dual kinase inhibitor of both IR and IGF1R. An inhibition of intrinsic IRS-1 phosphorylation at Y612 was detected after treatment with OSI-906 in all cell lines (Figure~\ref{fig5.3}), indicating that this inhibitor could affect signaling downstream IGF1R in osteosarcoma cells.
%
\begin{figure}[htbp]
  \centering
  \begin{minipage}[b]{0.50\linewidth}
    \includegraphics[width=1\textwidth]{figs05/fig3bw.pdf}		% pdf version also bw
  \end{minipage}
    \hfill
  \begin{minipage}[b]{0.46\linewidth}
     \caption{Western blot of IRS-1 and p-IRS-1 of lysates of untreated (--) osteosarcoma cell lines OHS, KPD, SAOS-2, and 143B, and of these cells treated for 3hrs with 1$\upmu$M of OSI-906 (+).}
     \label{fig5.3}
     \end{minipage}
\end{figure}
%

\subsection{OSI-906 inhibits proliferation of 3 of 4 osteosarcoma cell lines}
In 3 of 4 osteosarcoma cell lines tested, inhibition with OSI-906 was dose\hyp{}dependent (Figure~\ref{fig5.4}).
%
\begin{figure}[htbp]
	\centering
	\includegraphics[width=1.0\textwidth]{figs05/fig4bw.pdf}	% pdf version also bw
	\caption{Osteosarcoma cell lines were inhibited with different concentrations of OSI-906, for 72 (gray line) or 96 (black line) hours. OHS ({\it A}), KPD ({\it B}), and SAOS-2 ({\it C}) showed a dose\hyp{}dependent inhibition, while 143B ({\it D}) did not respond to OSI-906.}
	\label{fig5.4}
\end{figure}
%
Except for a toxic response at the maximum dose of 10$\upmu$M ({\it data not shown}), there was no effect on 143B. Because of this toxicity, relative IC$_{50}$s were determined using measurements until 1$\upmu$M. OHS, SAOS-2, and KPD had an IC$_{50}$ of 25nM, 92nM, and 90nM at 72hrs, respectively, and of 37nM, 57nM, and 23nM at 96hrs of inhibition, respectively. At 1$\upmu$M OSI-906, approximately 60\% of proliferation of OHS, SAOS-2, and KPD cells was inhibited, while 143B proliferation was not inhibited (Figure~\ref{fig5.4}).

\section{Discussion}\label{discussion5}
Genome\hyp{}wide gene expression and subsequent gene set analysis on osteosarcoma cell lines and biopsies indicated increased insulin\hyp{}like growth factor signaling in high\hyp{}grade osteosarcoma as compared with the hypothesized osteosarcoma progenitors, which is currently the best control, since there is no benign precursor and no certainty of the normal counterpart of osteosarcoma. Because IGF1R signaling can be exploited as a therapeutic target, and osteosarcoma patients are in severe need of new therapies, we examined mRNA expression of members of this signaling pathway in detail. {\it IGFBP4} and {\it GAS6}, which code for proteins that inhibit IGF1R signaling, showed the highest significant downregulation (log fold changes $<-4$) in a four\hyp{}way analysis, in which osteosarcoma pretreatment biopsies or cell lines were compared with osteoblastic cultures ($n=3$) or MSCs ($n=12$). Insulin\hyp{}like growth factor binding proteins (IGFBPs) generally inhibit IGF1R signaling by competitively binding IGFs, but can under certain circumstances also stimulate IGF1R signaling~\cite{grimberg2000role}. IGFBP4 is a negative regulator of IGF signaling in various tissues, including bone~\cite{conover2008insulin}. GAS6, or growth arrest\hyp{}specific 6, was shown to inhibit the growth promoting effects of IGF signaling and to stimulate differentiation in the chondrogenic cell line ATDC5~\cite{hutchison2010scf}. Both {\it IGFBP4} and {\it GAS6} expression have previously been shown to be downregulated in osteosarcoma cell lines ({\it IGFBP4} in MG-63~\cite{scharla19931}, {\it GAS6} in MG-63 and SAOS-2 cells~\cite{shiozawa2010gas6}). Next to {\it GAS6} and {\it IGFBP4}, {\it IGFBP2} was also significantly downregulated in all four analyses, with log fold changes of approximately $-3$. IGFBP2 generally inhibits IGF action and may play a role in IGF2\hyp{}induced osteoblast differentiation~\cite{conover2008insulin}. {\it IGFBP3} was highly downregulated in three out of four analyses, and has been shown to elicit anticancer effects by inhibiting IGF1R signaling in Ewing sarcoma~\cite{benini2006insulin}. IGFBP7 activity has not yet been reported in sarcoma, but has been associated with {\it e.g.} hepatocellular carcinoma~\cite{chen2011insulin}. Interestingly, {\it IGF2BP3} was highly overexpressed in 3 of 4 analyses. This binding protein can bind IGF2 mRNA, thereby probably activating the translation of IGF2~\cite{liao2005rna}. Overexpression of {\it IGF2BP3} has been reported in several cancer types~\cite{schaeffer2010insulin,suvasini2011insulin}. Figure~\ref{fig5.2} shows that differential expression is most pronounced in upstream regulators of IGF1R, while downstream components, such as SHC and FOS, are slightly downregulated, although for most genes this only holds when compared with mesenchymal stem cells, and not with osteoblasts. This may be caused by negative feedback loops, triggered by the active IGF1R signaling pathway. These results suggest that, in osteosarcoma, the IGF1R signaling pathway can be inhibited at the level of the receptor. We therefore validated protein levels of IGF1R and of IRS-1, a direct downstream component of IGF1R and IR signaling using Western blotting. IGF1R and IRS-1 protein levels correlated fairly well with mRNA expression levels. Most importantly, phosphorylated IRS-1, which is a measure for pathway activity, was detected in all four osteosarcoma cell lines, indicating that IGF1R signaling is active in osteosarcoma, and is possibly regulated upstream of IGF1R. Accordingly, targeting this receptor may be an effective way to inhibit this pathway.

OSI-906 is a selective small molecule dual kinase inhibitor of both IR and IGF1R~\cite{mulvihill2009discovery}. We specifically chose to treat osteosarcoma cells with a dual inhibitor, because the insulin receptor can activate the same downstream signaling pathways as IGF1R, therefore providing cells a way to circumvent single inhibition of IGF1R. This has formerly been demonstrated in osteoblasts~\cite{fulzele2007disruption} and in Ewing sarcoma cells~\cite{garofalo2011efficacy}. In fact, this dual inhibitor has been shown to cause enhanced inhibition of the Akt signaling pathway when compared with a selective monoclonal antibody against IGF1R, which could inhibit IR/IGF1R hybrids, but not IR homodimers~\cite{buck2010compensatory}. OSI-906 is currently being tested by OSI Pharmaceuticals in a Phase III trial in adrenocortical carcinoma and in a Phase I/II clinical trial in ovarian cancer. Treatment of osteosarcoma cells with OSI-906 at physiological levels leads to decreased phosphorylation of IRS-1 at Y612. Inhibition of IRS-1 at Y612 after treatment with OSI-906 was previously reported by Buck {\it et al}. in direct complementation breast cancer cells for IGF1R-IGF2 and IR(A)-IGF2~\cite{buck2010compensatory}. Interestingly, we also detected a small shift in the size of p-IRS-1 on the Western Blot, indicating that multiple phosphorylation groups are removed after treatment with OSI-906. Surprisingly, total IRS-1 levels were highest in 143B, and were downregulated after treatment with OSI-906 in this cell line, although this had no effect on cell growth in this line, as opposed to the three others, which showed low IC$_{50}$s. Proliferation of 143B was only inhibited most likely unspecifically at high and toxic levels of the drug. The 143B cell line is a derivative of the osteosarcoma cell line HOS, transformed by a {\it KRAS} oncogene. Constitutive activation of the Ras/Raf/ERK pathway can explain why proliferation of this cell line cannot be inhibited by OSI-906. Of the cell lines that were responsive to OSI-906, KPD and OHS showed that treatment of 96hrs was most effective, while SAOS-2 already reached maximum inhibition at 72hrs.

IGF1R signaling has been previously modulated in sarcoma in preclinical and clinical models. Several phase I and II clinical trials including treatment with IGF1R monoclonal antibodies are currently being conducted in sarcoma, especially in Ewing sarcoma (an overview of these trials is given in Olmos {\it et al}.~\cite{olmos2010biological}). Monoclonal antibodies against IGF1R have modest activity against Ewing sarcoma, as was observed in a phase I/II study of figitumumab (partial response in 14.2\% of all subjects)~\cite{juergens2011preliminary} and in a phase II study using R1507 (complete/partial response rate of 10\%)~\cite{pappo2011r1507}. Results of a phase II study of ganitumab in subjects with Ewing sarcoma and desmoplastic small round cell tumors were published very recently, and reported clinical benefit in 17\% of all patients~\cite{tap2012phase}. Preclinically, treatment with different monoclonal antibodies against IGFR1 has been performed in osteosarcoma xenograft models, in which a response was detected in at least 60\% of all cases studied~\cite{kolb2010r1507,kolb2008initial,houghton2010initial}. However, no objective responses were observed in phase I trials testing monoclonal antibodies in osteosarcoma~\cite{juergens2011preliminary,bagatell2011pharmacokinetically,quek2010combination}, although 2 of 3 patients treated with R1507 had prolonged stable disease~\cite{bagatell2011pharmacokinetically}. Clinical data using dual IGF1R/IR inhibitors osteosarcoma is still very limited~\cite{desai2010phase}. Because resistance to highly specific IGF1R inhibitors may develop through IR~\cite{garofalo2011efficacy}, blocking both IGF1R and IR with a dual kinase inhibitor will most likely lead to better inhibition of downstream IRS-1 signaling. We thus expect clinical outcomes to improve for osteosarcoma patients treated with dual IGF1R/IR inhibitor OSI-906. The effects of combination of OSI-906 with chemotherapeutics in osteosarcoma still need to be assessed before such a treatment can be clinically tested.

Phosphorylated IRS could be used as a biomarker in order to determine whether patients would respond to IGF1R inhibition. Patients with tumors exhibiting an activating mutation in downstream pathways will most likely not respond to IGF1R inhibition. Further research needs to be performed in order to assess these candidate biomarkers for response to treatment. The IGF1R pathway acts on several biological mechanisms that promote tumor progression---mitogenesis, protection from apoptosis, malignant transformation, and metastasis~\cite{rikhof2009insulin}. It is therefore possible that inhibiting these pathways with a dual IR/IGF1R kinase inhibitor, such as OSI-906, may reduce tumor sizes, as well as osteosarcoma metastasis, the leading cause of death in these patients.

\section{Conclusions}\label{conclusions5}
Using gene set analysis of genome\hyp{}wide gene expression data of high\hyp{}grade osteosarcoma biopsies and cell lines, we detected an overrepresentation of IGF1R signaling. Specifically, different upstream inhibitors of IGF1R signaling, {\it e.g.} several IGF binding proteins, were downregulated. As this indicated the IGF1R receptor as a potential target for treatment of osteosarcoma, we set out to inhibit this receptor in four osteosarcoma cell lines. We used OSI-906, a selective small molecule dual kinase inhibitor of both IR and IGF1R, since the insulin receptor can activate the same downstream signaling pathways as IGF1R, thereby providing a way to circumvent single inhibition of IGF1R. Treatment with OSI-906 resulted in inhibition of phosphorylation of IRS-1 Y612, a direct downstream target of IGF1R, and in strong inhibition of proliferation in 3 of 4 osteosarcoma cell lines. The nonresponsive cell line, 143B, has a {\it KRAS} oncogenic transformation, and may therefore not respond to this treatment. In conclusion, we have shown that IGF1R signaling is active in osteosarcoma, and that dual inhibition of IR/IGF1R inhibits downstream signaling and proliferation of these cells. Responsiveness to this treatment may be evaluated by Western blotting against phosphorylated IRS. This study provides an {\it in vitro} rationale for using dual IR/IGF1R inhibitors in preclinical studies of osteosarcoma.

%%% references

\begin{small}
\begin{singlespace}
\bibliographystyle{unsrtnatshort}		% sorted as referenced, was unsrtnat, but unsrtnatshort gives shorter output
\bibliography{biblio}
\end{singlespace}
\end{small}

%\end{document}
% Marieke Kuijjer
% 2013-02-15
% chapter 06

	%\documentclass[12pt,b5paper]{book}
	%\setcounter{secnumdepth}{0}
	%\setcounter{tocdepth}{1}
	%\usepackage[hidelinks]{hyperref}

%%% OBS! 1 reference added: kuijjer2011mrna (at 9,14 instead of kuijjer2011identification)

%\begin{document}

%%% title page

\chapter{Kinome and mRNA expression profiling of osteosarcoma identifies genomic instability, and reveals Akt as potential target for treatment}\label{ch6}
\thispagestyle{empty}				%%% to remove page number from first page of chapter, must be placed after calling the chapter

\vfill

\vspace{0.5cm}
This chapter is based on the manuscript:
\underline{Kuijjer ML}, van den Akker BEWM, Hilhorst R, Mommersteeg M, Buddingh EP, Serra M, B{\"u}rger H, Hogendoorn PCW, Cleton-Jansen AM. {\it Submitted}

\newpage

%%% main document

\section{Abstract}\label{abstract6}
\textbf{Background}: High\hyp{}grade osteosarcoma is a primary malignant bone tumor mostly occurring in adolescents and young adults, with a second peak at middle age. Overall survival is approximately 60\%, and has not significantly increased since the introduction of neoadjuvant chemotherapy in the 1970s. The genomic profile of high\hyp{}grade osteosarcoma is complex and heterogeneous. Integration of different types of genome\hyp{}wide data may be advantageous in extracting relevant information from the large number of aberrations detected in this tumor.

\textbf{Methods}: We analyzed genome\hyp{}wide gene expression data of osteosarcoma cell lines, and integrated these data with a kinome screen. Data were analyzed in statistical language R, using {\it LIMMA} for detection of differential expression/phosphorylation. We subsequently used Ingenuity Pathways Analysis to determine deregulated pathways in both data types.

\textbf{Results}: Gene set enrichment indicated that pathways important in genomic stability are highly deregulated in these tumors, with many genes showing upregulation, which could be used as a prognostic marker, and with kinases phosphorylating peptides in these pathways. Akt and AMPK were identified as active and inactive, respectively. As these pathways have an opposite role on mTORC1 signaling, we set out to inhibit Akt with the allosteric Akt inhibitor MK-2206. This resulted in inhibition of proliferation of osteosarcoma cell lines U-2 OS and HOS, but not of 143B, which harbors a {\it KRAS} oncogenic transformation.

\textbf{Conclusions}: We identified both overexpression and hyperphosphorylation in pathways playing a role in genomic stability. Kinome profiling identified active Akt signaling, which could inhibit proliferation in 2/3 osteosarcoma cell lines. This study provides a rationale for further testing inhibitors of the PI3K/Akt/mTORC1 pathway in preclinical studies of osteosarcoma.

\section{Background}\label{introduction6}
High\hyp{}grade osteosarcoma is the most prevalent primary malignant bone tumor. Most frequently, the long bones of adolescents and young adults are affected, with a yearly incidence of approximately 5 cases per million per year~\cite{raymond2002conventional}. Patients are generally treated with high doses of neoadjuvant chemotherapy to prevent the outgrowth of micrometastases. In 15--25\% of all patients, however, metastatic disease is clinically detectable at diagnosis and despite the intensive treatment, 45\% of all patients develop distant metastases, the leading cause of death of osteosarcoma patients~\cite{bacci2006prognostic,buddingh2010prognostic}. The introduction of neoadjuvant chemotherapy in the 1970s has increased survival from 10--20\% to approximately 60\%. However, survival has reached a plateau, and new treatments are urgently needed~\cite{allison2012meta,anninga2011chemotherapeutic,hattinger2010emerging}. Osteosarcoma is an extremely genomically unstable tumor, with karyotypes harboring numerous numerical and structural changes~\cite{cleton2005central,szuhai2012molecular}. In addition, osteosarcoma genotypes show a considerable degree of heterogeneity, both intra- and intertumoral. Both the complex genotype and its heterogeneity render it difficult to determine which genomic alterations are important in osteosarcomagenesis, as not all alterations may lead to a difference in mRNA, protein levels, or enzyme activity in the tumor tissue. Integration of different data types is therefore of particular relevance for studying a heterogeneous tumor with a complex genomic profile such as osteosarcoma. Previously, genomic and expression data of osteosarcoma pretreatment biopsies have been integrated, in order to detect highly recurrent osteosarcoma driver genes. The list of driver genes obtained with this study was enriched in genes playing a role in genomic stability~\cite{kuijjer2012identification}. Yet, even though recurrent driver genes may provide knowledge on what pathways are affected that help tumor cells survive, such driver genes may not always be accessible as targets for treatment. This especially holds for pathways involved in genetic stability, since the damage is already done.

Oncogenic kinases are often active in tumor cells, and a number of kinases can be pharmacologically inhibited. Therapies targeting oncogenic kinases have provided promising results in inhibiting proliferation of cancer cells, and some kinases have been targeted in preclinical and clinical studies in childhood sarcomas (as reviewed in Wachtel {\it et al}.~\cite{wachtel2010targets}), {\it e.g.} IGF1R and mTOR~\cite{kolb2010r1507,chawla2012phase}. An unbiased approach to identify active kinases in cancer is to perform kinome\hyp{}wide screens. Such screens have previously been effectively used in other types of sarcoma and have led to the detection of specific targets for treatment~\cite{schrage2009kinome,willems2010kinome}. As combining the analysis of different data types using systems biology approaches can give a more complete impression of the state of a tumor cell, we set out to integrate genome\hyp{}wide gene expression data of osteosarcoma cell lines with kinome profiling data. Osteosarcoma cell lines are widely available and have been shown to be representative for the tumor of origin, both on a genome\hyp{}wide as on a functional level, and are therefore a good model to study osteosarcoma preclinically~\cite{kuijjer2011mrna,mohseny2011functional}. We previously have performed genome\hyp{}wide expression analysis on a panel of 19 osteosarcoma cell lines~\cite{ottaviano2010molecular}. In the present study, we compared expression profiles with the different putative progenitor cells of osteosarcoma---mesenchymal stem cells (MSCs) and osteoblasts---in order to define the common denominator pathways that are deregulated in osteosarcoma. Pathways with a role in genomic stability appeared to be enriched in overexpressed genes. By integrating expression data with a serine/threonine (Ser/Thr) kinome screen, we show that these pathways are enriched in hyperphosphorylation as well, confirming that genomic stability is highly deregulated in osteosarcoma, both on a transcriptional level and on phosphorylation activity.

In order to detect overactive kinases in osteosarcoma, which may be potential targets for treatment, we identified the most significant pathways in the kinome profiling data, which indicated active PI3K/Akt and inactive AMPK signaling. These pathways play an opposite role in mTORC1 signaling, with Akt promoting and AMPK inhibiting signal transduction~\cite{memmott2009akt}. We pharmacologically inhibited Akt in osteosarcoma cell lines, which reduced proliferation of 2/3 cell lines. In summary, this study describes integration of mRNA and phosphorylation data, and gives a rationale for treatment of osteosarcoma with inhibitors of the PI3K/Akt pathway.

\section{Methods}\label{methods6}
\subsection{Cell culture}
Osteosarcoma cell lines were previously characterized and described~\cite{ottaviano2010molecular}. Human bone\hyp{}marrow\hyp{}derived MSCs were obtained from two osteosarcoma patients, and were characterized and handled as described~\cite{cleton2009profiling}. For kinome profiling of osteosarcoma versus MSCs, cells were cultured in Dulbecco's Modified Eagle Medium (DMEM; Invitrogen, Carlsbad, CA), supplemented with 10\% fetal bovine serum (Greiner Bio-one, Frickenhausen, Germany), in order to eliminate differences in kinase activity caused by culture conditions. For inhibition experiments and kinome profiling of inhibition experiments, osteosarcoma cell lines 143B, U-2 OS, and HOS were maintained in RPMI 1640 supplemented with 10\% fetal calf serum (both from Invitrogen, Carlsbad, CA). The human pre-B acute lymphoblastic leukemia cell line NALM-6 cell line was kindly provided by Mw. N. Duinkerken (Department of Hematology, Leiden University Medical Center, the Netherlands), and was maintained in Iscove's Modified Dulbecco's Medium (IMDM) supplemented with GlutaMAX-1 (Life Technologies, Carlsbad, CA) and 10\% fetal bovine serum (Greiner Bio-one, Frickenhausen, Germany). All cells were regularly tested for mycoplasm and were genotyped using the Powerplex 1.2 system (Promega, Leiden, the Netherlands), as described previously~\cite{mohseny2011functional}.

\subsection{Cell lysates}
Kinome profiling was performed on osteosarcoma cell lines 143B and U-2 OS and on two MSCs---MSC001 and MSC006. Cells at 80\% confluence were washed twice with Phosphate buffered Saline and lysed with M-PER Mammalian Extraction Buffer, supplemented with Halt Phosphatase Inhibitor Cocktail and EDTA free Halt Protease Inhibitor Cocktail (Pierce Biotechnology, Rockford, IL), according to the manufacture's protocol. Cells were incubated on ice for at least 30 minutes before collecting the lysates and centrifuging these for 15 minutes at 4$^\circ$C at $>10,000\cdot g$. Protein concentration was measured using a detergent\hyp{}compatible Protein Assay (Bio-Rad Laboratories, Hercules, CA) according to the manufacturer's protocol. Samples were snap\hyp{}frozen and stored at -70$^\circ$C.

\subsection{Proliferation assays}
MK-2206 was dissolved in DMSO at a concentration of 10mM and stored at -20$^\circ$C. For 143B, U-2 OS, and HOS, $2,000$, $4,000$, and $2,000$ cells/well respectively, were plated in a 96-wells plate. NALM-6, a human pre-B acute lymphoblastic leukemia (ALL) cell line, was included as a positive control, as ALL cell lines have been shown to be highly sensitive to MK-2206~\cite{gorlick2012testing}. This cell line grows in suspension and was plated at $50,000$ cells/well. After 24hrs, MK-2206 was added in triplicate in different concentrations---0nM, 0.5nM, 1nM, 5nM, 10nM, 50nM, 100nM, 500nM, 1$\upmu$M, 5$\upmu$M, and 10$\upmu$M. For 143B and HOS, the effect of concentrations of 2, 3, 4, and 5nM was assessed as well. Cells were grown in the presence of inhibitor for 120hrs. Cell proliferation was determined by incubating the cells with reagent WST-1 (Roche, Basel, Switzerland) for 2hrs and subsequently measured using a Wallac 1420 VICTOR2 (Perkin Elmer, Waltham, MA). Data were analyzed in Graphpad Prism 5.01 (\url{www.graphpad.com}). Relative IC$_{50}$s were calculated using results from the different concentrations up to the highest dose where toxicity was not yet present. The results shown are representative results from at least three independent experiments.

\subsection{Genome\hyp{}wide gene expression profiling}
We analyzed our previously published data of osteosarcoma cell lines ($n=19$), MSCs ($n=12$), and osteoblasts ($n=3$) (GEO superseries, accession number GSE42352)~\cite{kuijjer2012identification}. Microarray data processing and quality control were performed in the statistical language R version 2.15~\cite{r2.15.0} as described previously~\cite{buddingh2011tumor}.

\subsection{Kinome profiling}
Kinome profiling was performed on 1$\upmu$g of cell lysate on the serine/threonine (Ser/Thr) Kinase PamChip\textregistered peptide microarrays (PamGene, 's-Hertogenbosch, the Netherlands) according to the manufacturer's protocol, essentially as described in Hilhorst {\it et al}.~\cite{hilhorst2013peptide}. This peptide microarray comprises 142 peptide sequences derived from human phosphorylation sites. Peptide phosphorylation is detected in time with a mixture of fluorescently labeled antiphosphoserine/threonine antibodies. We used at least three technical replicates for each MSC line, and four technical replicates for the osteosarcoma cell lines. Images were taken every 5 minutes, over the course of 60 minutes. Signal quantification on phosphorylated peptides was performed in BioNavigator software (PamGene International, 's Hertogenbosch, the Netherlands). Subsequently, data were normalized in R~\cite{smyth2004linear} using the {\it vsn} package~\cite{huber2002variance}. Median signals at 60 minutes of incubation with the cell lysates were analyzed in Bioconductor~\cite{gentleman2004bioconductor} package {\it arrayQualityMetrics}~\cite{kauffmann2009arrayqualitymetrics} to identify poor quality samples, which were removed from further analysis. Technical replicates of good quality were averaged. To determine whether these data were reproducible, we analyzed data from different cycles (0, 10, 20, 30, 40, 50, and 60 minutes incubation with cell lysates).

In the second kinome profiling experiment we compared lysates of untreated cells with lysates of cells treated with MK-2206. Different treatment durations and concentrations were used---no treatment, treatment for 5, 30, 180, and 960 minutes with 1$\upmu$M MK-2206, and treatment for 180 minutes with 10$\upmu$M of the drug. Kinome profiling was performed as described above, with the difference that we used 1--5 technical replicates per condition. Of this experiment, we analyzed signals at 30 minutes of incubation with the lysates.

\subsection{Statistical analyses of microarray data}
We performed {\it LIMMA} analysis~\cite{smyth2004linear} in order to determine differential mRNA expression between osteosarcoma cell lines ($n=19$) and control cell lines---MSCs ($n=12$) and osteoblasts ($n=3$) and to determine differential phosphorylation of peptides on the PamChip\textregistered microarray between osteosarcoma cell lines ($n=2$) and MSCs ($n=2$). We used a Benjamini and Hochberg False Discovery Rate (FDR) of 0.05 as cut-off for significance. Kinome profiling signals obtained for the different treatment conditions were analyzed in a paired approach, in which signals from untreated cells were subtracted from the signals from treated cells. For both kinome profiling experiments, we used a cut-off of 0.1 for the absolute log fold change (logFC). Heatmaps were generated using the function heatmap.2 of R package gplots.

\subsection{Pathway analysis}
In order to reveal pathways which were significantly affected on mRNA levels in osteosarcoma cell lines, we intersected the toptables obtained by {\it LIMMA} analysis of osteosarcoma cell lines versus MSCs and of osteosarcoma cell lines versus osteoblasts. Gene symbols for all probes were imported into the software Ingenuity Pathways Analysis (IPA, Ingenuity Systems, \url{www.ingenuity.com}), together with FDR adjusted p-values (adjP) and average logFCs. Only the gene symbols of probes that were both significantly upregulated or both significantly downregulated in osteosarcoma cell lines as compared with MSCs and with OBs (adjP $<0.05$) were selected to be considered as significantly differentially expressed in the IPA analysis. For differential phosphorylation, we imported the results from the {\it LIMMA} analysis on kinome profiling data, with a cut-off of 0.05 for adjusted p-value and a cut-off of 0.1 for logFC. The significance of the association between the data set and the canonical pathways was measured as described previously~\cite{mohseny2012activities}. Pathways with adjP $<0.05$ were considered to be significantly affected. In addition, transcription factor analyses were performed on gene expression data in IPA in order to predict activated or inhibited transcription factors based on expression of target genes, returning p-values (with a cut-off of 0.05 for significance) and regulation z-scores.

\section{Results}\label{results6}
\subsection{Genome\hyp{}wide gene expression profiling of high\hyp{}grade osteosarcoma cell lines}
We started by comparing gene expression signatures of 19 osteosarcoma cell lines, 12 MSC, and 3 osteoblast cultures using unsupervised hierarchical clustering. Two separate clusters were detected---one containing all tumor cell samples and one containing control samples. Within the control sample cluster, osteoblasts clustered separately from MSCs ({\it data not shown}). {\it  LIMMA} analysis resulted in $7,891$ probes encoding for differentially expressed (DE) genes between osteosarcoma cell lines and MSCs, and $2,222$ probes encoding for DE genes between osteosarcoma cells and osteoblasts. Intersecting of these gene lists showed $1,410$ probes that were significant in both analyses, of which $1,390$ were upregulated in both analyses, or downregulated in both analyses (Figure~\ref{fig6.1}). These probes, encoding for $1,312$ genes, were selected for subsequent pathways analysis, in order to determine commonly affected pathways in osteosarcoma tumor cells.
%
\begin{figure}[htbp]
  \centering
  \begin{minipage}[b]{0.50\linewidth}
%    \includegraphics[width=1\textwidth]{figs06/fig1bw.pdf}		% OBS! print version bw
   \includegraphics[width=1\textwidth]{figs06/fig1col.pdf}	% OBS! pdf version rgb
  \end{minipage}
    \hfill
  \begin{minipage}[b]{0.46\linewidth}
     \caption{Venn diagram showing the significant probes in the analysis of osteosarcoma cell lines {\it vs} MSC (vsMSC) and {\it vs} osteoblasts (vsOB), and the intersection of these significant probes with the subset of all probes (both significant and nonsignificant) that shows both up- or both downregulation in these two analyses (same sign). In total, $1,410$ probes are significant in both analyses, of which $1,390$ have the same sign of logFC.}
     \label{fig6.1}
     \end{minipage}
\end{figure}
%

\subsection{Gene expression is altered in pathways regulating genomic stability}
Pathway analyses on the $1,312$ differentially expressed genes resulted in 17 significantly affected pathways (Figure~\ref{fig6.2}).
%
\begin{figure}[htbp]
	\centering
%	\includegraphics[width=1.0\textwidth]{figs06/fig2bw.pdf}	% OBS! print version bw
	\includegraphics[width=1.0\textwidth]{figs06/fig2col.pdf}	% OBS! pdf version rgb
%	\caption{Stacked bar chart depicting all significantly affected pathways as identified by gene expression profiling of osteosarcoma cell lines, showing percentages of downregulated (gray), not significantly altered (light gray), and upregulated (dark gray) genes, and genes which were not present on the microarray (white). The -log(adjP) (-log(B-H) p-value) is plotted in black, and is above 1.3 for adjP $<0.05$.} % OBS! caption for bw print file
	\caption{Stacked bar chart depicting all significantly affected pathways as identified by gene expression profiling of osteosarcoma cell lines, showing percentages of downregulated (green), not significantly altered (gray), and upregulated (red) genes, and genes which were not present on the microarray (white). The -log(adjP) (-log(B-H) p-value) is plotted in orange, and is above 1.3 for adjP $<0.05$.}			%%% OBS! caption for rgb pdf file
	\label{fig6.2}
\end{figure}
%
Fourteen out of these 17 pathways play a direct or indirect role in genomic stability. Unsupervised hierarchical clustering of all cell line data and data from 84 osteosarcoma biopsies (GEO accession number GSE33382~\cite{kuijjer2012identification}) was performed on all DE genes present in these 17 significantly affected pathways, which resulted in a cluster of control cells and biopsies, and larger cluster of osteosarcoma cell lines and biopsies (Additional Figure~\ref{afig6.1}). Patients whose biopsies had expression profiles of these pathways similar to osteosarcoma cell lines showed worse metastasis\hyp{}free survival than patients with intermediate expression profiles, and than patients whose biopsies had expression profiles more similar to the control cultures, {\it i.e.} nontransformed primary mesenchymal cell cultures and osteoblast cultures (Logrank test for trend, p-value $=0.049$, Figure~\ref{fig6.3}).
%
\begin{figure}[htbp]
  \centering
  \begin{minipage}[b]{0.50\linewidth}
%    \includegraphics[width=1\textwidth]{figs06/fig3bw.pdf}		% OBS! print version bw
   \includegraphics[width=1\textwidth]{figs06/fig3col.pdf}	% OBS! pdf version rgb
  \end{minipage}
    \hfill
  \begin{minipage}[b]{0.46\linewidth}
     \caption{Kaplan\hyp{}Meier metastasis\hyp{}free survival analysis on data obtained from patient biopsies which clustered with osteosarcoma cell lines, biopsies clustering with control cell lines, and an intermediate group, based on gene expression of genes all present in the 17 significantly affected pathways (as in Additional Figure~\ref{afig6.1}). Logrank test for trend, p-value $=0.049$.}
     \label{fig6.3}
     \end{minipage}
\end{figure}
%
Transcription factors were predicted to be activated or inhibited based on expression of target genes are shown in IPA. The most activated transcription factor was {\it MYC}, while the most inactivated transcription factor was {\it TP53}.

\subsection{Kinome profiling of osteosarcoma cell lines}
To obtain more information on the activity of the pathways which showed aberrant mRNA expression, we integrated mRNA expression data with data obtained with kinase PamChip\textregistered peptide microarrays. These peptide microarrays were incubated with lysates of the osteosarcoma cell lines 143B and U-2 OS, and with lysates of two MSC cultures. Kinases present in the cell lysates can, in the presence of ATP, phosphorylate the peptides present on the microarray, which is detected by fluorescently labeled antibodies. We compared kinome profiling data at different incubation times by intersecting lists of differentially phosphorylated peptides between osteosarcoma cells and MSCs, obtained by {\it LIMMA} analyses, as shown in Figure~\ref{afig6.5}.
%
\begin{figure}[htbp]
  \centering
  \begin{minipage}[b]{0.50\linewidth}
%    \includegraphics[width=1\textwidth]{figs06/addfile5bw.pdf}		% OBS! print version bw
   \includegraphics[width=1\textwidth]{figs06/addfile5rgb.pdf}		% OBS! pdf version rgb
  \end{minipage}
    \hfill
  \begin{minipage}[b]{0.46\linewidth}
     \caption{Comparison of peptide phosphorylation at different time points. {\it LIMMA} analyses were performed on different time points, ranging from 0 to 60 minutes of incubation with cell lysates. Venn diagrams show overlap of significantly differentially phosphorylated peptides between the consecutive time points.}
     \label{afig6.5}
     \end{minipage}
\end{figure}
%
This data analysis demonstrated a large overlap in the detected differentially phosphorylated peptides, and a build-up of differentially phosphorylated peptides over time. Most peptides showed differential phosphorylation after 20 minutes of incubation with cell lysates. After 60 minutes of incubation on the peptide microarray, 49 peptides were detected to be significantly differentially phosphorylated between osteosarcoma cell lines and mesenchymal stem cells. These peptides are represented in Figure~\ref{fig6.4}.
%
\begin{figure}[htbp]
	\centering
	\includegraphics[width=1.0\textwidth]{figs06/fig4col.pdf}	% pdf version also rgb
	\caption{Supervised clustering of all 49 significantly differentially phosphorylated peptides identified by the comparison of two osteosarcoma cell lines with two MSC cultures. Peptides are sorted on logFC, from lower phosphorylation to higher phosphorylation in osteosarcoma cell lines. Orange: higher phosphorylation levels, blue: lower phosphorylation levels.}
	\label{fig6.4}
\end{figure}
%
As a reference, we performed an unsupervised hierarchical clustering including all technical replicates ({\it data not shown}), which showed that phosphorylation of peptides by cell lysates of most technical replicates was comparable.

\subsection{Altered phosphorylation in genomic stability pathways}
The significance of the 17 pathways that were returned from the pathway analysis on mRNA expression data was tested on kinome profiling results in IPA. In total, 7/17 pathways were significant in kinome profiling as well. These seven pathways were a subset of the 14 pathways with a known role in genomic stability. Most significantly differentially phosphorylated peptides in these seven pathways showed higher phosphorylation levels in osteosarcoma cell lines (Figure~\ref{fig6.5}), indicating that kinases affect phosphorylation of molecules playing a role in genomic stability.
%
\begin{figure}[htbp]
	\centering
%	\includegraphics[width=1.0\textwidth]{figs06/fig5bw.pdf}	% OBS! print version bw
	\includegraphics[width=1.0\textwidth]{figs06/fig5col.pdf}	% OBS! pdf version rgb
%	\caption{Stacked bar chart showing kinome profiling pathway analysis on the subset of pathways which were significant on gene expression profiling. Percentages of downregulated (gray), not significantly altered (light gray), and upregulated (dark gray) genes, and genes which were not present on the microarray (white) are shown. The -log(adjP) (-log(B-H) p-value) is plotted in black, and is above 1.3 for adjP $<0.05$.} % OBS! caption for bw print
	\caption{Stacked bar chart showing kinome profiling pathway analysis on the subset of pathways which were significant on gene expression profiling. Percentages of downregulated (blue), not significantly altered (gray), and upregulated (orange) genes, and genes which were not present on the microarray (white) are shown. The -log(adjP) (-log(B-H) p-value) is plotted in orange, and is above 1.3 for adjP $<0.05$.}			%%% OBS! caption for col pdf file
	\label{fig6.5}
\end{figure}
%

\subsection{PI3K/Akt and AMPK signaling in osteosarcoma}
Unsupervised pathway analysis on the kinome profiling results returned the IPA pathway PI3K/Akt signaling as the most significantly affected pathway in osteosarcoma cells (Figure~\ref{fig6.6}) and the AMPK pathway as second most significantly affected pathway.
%
\begin{figure}[htbp]
	\centering
	\includegraphics[width=1.0\textwidth]{figs06/fig6col.pdf}	% pdf version also in rgb
	\caption{The Akt signaling pathway in IPA. Blue: significantly lower, orange: significantly higher phosphorylation in osteosarcoma cell lines, gray, no significant difference in phosphorylation, white: no phosphorylation sites of the particular protein on the PamGene Ser/Thr chip. Blue lines indicate known downstream phosphorylation by the upstream kinase.}
	\label{fig6.6}
\end{figure}
%
Specifically, molecules directly downstream of Akt showed higher phosphorylation in osteosarcoma than in MSCs, while molecules downstream of AMPK showed lower phosphorylation levels. As these results indicate that Akt signaling is active in osteosarcoma and might be driving its high proliferative capacity, we set out to pharmacologically inhibit Akt using the compound MK-2206.

\subsection{MK-2206 inhibits proliferation of U-2 OS and HOS, but not of 143B}
We inhibited osteosarcoma and control cells for 120hrs with allosteric inhibitor MK-2206. Inhibition of the positive control leukemia cell line NALM-6, and of osteosarcoma cell line U-2 OS with MK-2206 was dose\hyp{}dependent, with IC$_{50}$s of 0.38$\upmu$M and 2.5$\upmu$M, and maximal responses of 94\% and 71\%, respectively (Figure~\ref{fig6.7}).
%
\begin{figure}[htbp]
	\centering
	\includegraphics[width=1.0\textwidth]{figs06/fig7bw.pdf}	% pdf version also bw
	\caption{Proliferation of osteosarcoma cell lines was inhibited with different concentrations of MK-2206, for 120hrs. NALM-6, U-2 OS, and HOS showed a dose\hyp{}dependent inhibition, while 143B did not respond.}
	\label{fig6.7}
\end{figure}
%
143B did not show any response at concentrations below 5$\upmu$M. Because 143B exhibits an oncogenic {\it KRAS} transformation, we assessed MK-2206 specificity on the parental cell line of 143B, HOS, which does not exhibit this transformation. HOS indeed responded similar to U-2 OS, with an IC$_{50}$ of 2.6$\upmu$M and maximal response of 62\%.

\subsection{Different phosphorylation patterns upon treatment with MK-2206}
As 143B and U-2 OS showed different sensitivities to MK-2206, we performed a paired analysis between kinome profiling data obtained from lysates of cells, which were treated with different concentrations of MK-2206, and for different treatment lengths. Overall, the phosphorylation patterns differed between both cell lines, and distances between treatment options within each cell line were smaller than between the cell lines (Figure~\ref{afig6.8}).
%
\begin{figure}[htbp]
  \centering
  \begin{minipage}[b]{0.65\linewidth}
    \includegraphics[width=1\textwidth]{figs06/addfile8rgb.pdf}		% print version rgb pdf version also rgb
  \end{minipage}
    \hfill
  \begin{minipage}[b]{0.31\linewidth}
     \caption{Unsupervised hierarchical clustering depicting the distances between data obtained from kinome profiling of cells treated with different concentrations of MK-2206 and for different time intervals. 1\_30: treatment of 30min with 1$\upmu$M of MK-2206, etc.} %
     \label{afig6.8}
     \end{minipage}
\end{figure}
%
We generated a heatmap of differential phosphorylation in the paired analysis of treated and untreated cells, depicting all peptides of the PamGene chip which are downstream of PI3K/Akt (Figure~\ref{fig6.8}). This figure shows that the inhibition pattern of MK-2206 is different in the two osteosarcoma cell lines, suggesting that other upstream kinases may be affected by inhibition of Akt with MK-2206 as well.
%
\begin{figure}[htbp]
	\centering
	\includegraphics[width=1.0\textwidth]{figs06/fig8col.pdf}	% pdf version also in rgb
	\caption{Unsupervised clustering depicting differential phosphorylation of peptides of the PI3K/Akt pathway by cell lysates treated with different concentrations of MK-2206 for different time intervals, as compared with untreated cells. Blue: logFC $<0$, orange: logFC $>0$. Different treatment options are shown in different shades of gray (from light to dark gray: 1$\upmu$M 5, 30, 180, and 960 minutes, and 10$\upmu$M 180 minutes of treatment with MK-2206. Light green: 143B, dark green: U-2 OS.}
	\label{fig6.8}
\end{figure}
%

\section{Discussion}\label{discussion6}
Osteosarcoma is a highly genomically unstable tumor. The identification of specific molecular targets that drive oncogenesis and that might be targets for therapy may thereby be hampered. Genome\hyp{}wide gene expression profiling of high\hyp{}grade osteosarcoma cell lines, in fact, showed an enrichment in differential expression in pathways important in genomic stability (Figure~\ref{fig6.2}), with a role in cell cycle and checkpoint regulation ({\it e.g.} p53 signaling, G1/S and G2/M checkpoint regulation), DNA damage response ({\it e.g.} ATM signaling, role of BRCA1 in DNA damage response), and purine/pyrimidine metabolism. Most significantly differentially expressed genes in these pathways were upregulated, for example {\it DNA-PK}, {\it BRCA1}, and {\it CDC25A}. Some downregulated genes were detected as well, such as {\it CDKN1A}, which has an inhibitory role on cell cycle progression, and genes downstream of {\it TP53} ({\it e.g.} {\it THBS1} and {\it SERPINE1}, encoding TSP1 and PAI-1, respectively). Interestingly, as shown by unsupervised clustering on expression levels of genes of these pathways, osteosarcoma pretreatment biopsies can have profiles more similar to those of osteosarcoma cell lines, or more similar to profiles of the control cells. The first is associated with poor, while the latter is associated with good metastasis\hyp{}free survival. Expression profiles can also be of an intermediate type, with intermediate metastasis\hyp{}free survival (Additional Figure~\ref{afig6.1}, Figure~\ref{fig6.3}). This suggests that deregulated genomic stability is a key driver of osteosarcomagenesis, as was already previously reported~\cite{kuijjer2012identification}. IPA transcription factor analysis showed that {\it MYC} is the most significantly activated (z-score of 6.294), and {\it TP53} the most significantly inactivated (z-score of -7.660) transcription factor. Other highly predicted activated transcription factors are {\it e.g.} {\it E2F1/2/3}, whereas {\it CDKN2A} and {\it RB1} were detected as inactivated. These different genes are known to be affected in osteosarcoma~\cite{cleton2005central,kuijjer2012identification,mohseny2010small}. The role of these genes in cell cycle progression further confirms the importance of these pathways in osteosarcoma.

As kinome\hyp{}wide screens have previously led to the detection of specific targets for treatment in other sarcoma types~\cite{schrage2009kinome,willems2010kinome}, we performed kinome profiling of osteosarcoma cell lysates. Since the pathways that were shown to be significantly affected on mRNA expression mostly contained Ser/Thr kinases, we selected a Ser/Thr peptide microarray---the Ser/Thr PamChip\textregistered. Pathway analysis on kinome profiling data showed that 50\% of the pathways that were significant on gene expression data were also significantly enriched in differential phosphorylation signals (Figure~\ref{fig6.5}). All significant peptides were higher phosphorylated in osteosarcoma cell lines, except for a peptide present in the gene {\it CREB1}. Since most of these peptides showed higher phosphorylation, we expect these pathways to be highly active, demonstrating higher cell cycling of the tumor cells, and deregulated responses to DNA damage.

We next determined the most significantly affected pathways in the kinome data from the entire IPA canonical pathways database, and detected deregulation of the PI3K/Akt and AMPK signaling pathways. Molecules downstream of Akt showed higher phosphorylation (Figure~\ref{fig6.6}), while downstream of AMPK, lower levels of phosphorylation were detected. Akt and AMPK act antagonistically to regulate mTOR signaling through inhibitory and activating phosphorylation of TSC2, respectively~\cite{memmott2009akt}. The Akt pathway is one of the most commonly affected pathways in cancer, with active PI3K/Akt signaling leading to excessive cell growth and proliferation~\cite{engelman2006evolution,manning2007akt}. Inhibition of this pathway by targeting mTOR with agents such as rapamycin is effective in some cancer types~\cite{guertin2007defining}. In a recent phase II trial in bone and soft tissue sarcomas, inhibition of mTOR with ridaforolimus resulted in better progression\hyp{}free survival~\cite{chawla2012phase}. Inhibiting mTOR can, however, also activate a strong negative feedback loop from S6K1 to enhance Akt signaling~\cite{engelman2006evolution,guertin2007defining}. It may, therefore, be more effective to inhibit Akt itself. Inhibition of Akt was recently tested in a panel of xenografts of different pediatric cancers, and was most effective in osteosarcoma, with significant differences in event\hyp{}free survival in 6/6 xenografts~\cite{gorlick2012testing}. In addition, AMPK activators suppress growth of cell lines of various tumor types~\cite{vakana2012targeting}.

We treated osteosarcoma cell lines with the allosteric Akt inhibitor MK-2206 (Selleck Chemicals LLC, Houston, TX). Inhibition of proliferation was dose\hyp{}dependent in U-2 OS (IC$_{50}$ $=2.5\upmu$M), but not in 143B (Figure~\ref{fig6.7}). Important to note is that active Akt signaling can be detected by kinome profiling in this cell line, but this does not necessarily imply that this pathway can also be fully inhibited, for example in the case that downstream actors in the same pathway cause a survival benefit for the cell line. As 143B is derived from the HOS cell line with a {\it KRAS} oncogenic transformation, we determined inhibitory effects of MK-2206 on HOS as well. HOS responded to MK-2206 in a similar manner as U-2 OS (IC$_{50}=2.6\upmu$M). This suggests that constitutive Ras/Raf/ERK signaling causes insensitivity to inhibition of the Akt pathway to MK-2206. Kinome profiling of cells treated with MK-2206 resulted in different phosphorylation patterns in 143B and U-2 OS of peptides of molecules in the PI3K/Akt pathway (Figure~\ref{fig6.8}). Differences between these cell lines were found in BAD Ser-99, of which phosphorylation was inhibited after treatment with MK-2206 in the responsive cell line U-2 OS, but stimulated in 143B, and in BAD Ser-118, where an opposite pattern was detected. BAD Ser-99 is the major site of Akt phosphorylation, while Ser-118 is the major site of PKA phosphorylation~\cite{hornbeck2004phosphosite}. Opposite patterns were also detected for TP53 Thr-18 and CDKN1A Thr-145/Ser-146, of which CDKN1A Thr-145 can also be directly phosphorylated by Akt. These results suggest that activity of other kinases may be affected by inhibition of Akt using MK-2206, or by MK-2206 itself. This depends on the cellular context, as we otherwise would not have expected to detect any differences in a paired analysis for the different conditions in each cell type.

An important finding of our studies is that the PI3K/Akt and AMPK signaling pathways were detected with kinome profiling, while mRNA expression profiling did not result in the identification of these pathways. This suggests that in osteosarcoma, these pathways are regulated by phosphorylation rather than by transcriptional activity. Gene expression and protein synthesis imply a long time commitment of a cell to potential activation of its synthesized proteins. Phosphorylation, on the other hand, provides a very rapid way to mobilize extra catalytic power for a short time, and allows fine\hyp{}tuning of the activation of a pathway to the needs of a cell. This difference in time scale emphasizes the importance of applying different platforms for the analysis of a complex tumor as high\hyp{}grade osteosarcoma.

\section{Conclusions}\label{conclusions6}
In summary, this study shows that genomic stability pathways are deregulated on both mRNA and kinome levels, with most significantly affected genes being upregulated and/or phosphorylated. Akt was detected as most probably overactive in osteosarcoma, as downstream peptides were hyperphosphorylated as compared with MSCs. Akt inhibitor MK-2206 could inhibit 2/3 osteosarcoma cell lines. Based on these results, we conclude that Akt inhibitors and other drugs inhibiting the PI3K/Akt/mTOR pathway could have an effect on survival of osteosarcoma tumor cells.

%%% references

\begin{small}
\begin{singlespace}
\bibliographystyle{unsrtnatshort}		% sorted as referenced, was unsrtnat, but unsrtnatshort gives shorter output
\bibliography{biblio}
\end{singlespace}
\end{small}

%%% appendix
% additional data file 3, now appendix figure 1
\begin{subappendices}
	\newpage
	\setcounter{figure}{0}
	\section{Additional Figures}
		\renewcommand{\figurename}{Additional Figure}
		%
		\begin{figure}[h]
		  \centering
%		    \includegraphics[width=1\textwidth]{figs06/addfile3bw.pdf}	% OBS! bw print, this corresponds to bw fig 3, should be rgb in pdf file
		    \includegraphics[width=1\textwidth]{figs06/addfile3rgb.pdf}	% OBS! rgb pdf, this corresponds to bw fig 3, should be rgb in pdf file
%		    \caption{Unsupervised hierarchical clustering of osteosarcoma cell line data (black bars), control cultures (MSC: dark gray bars, osteoblast: light gray bars), and data from osteosarcoma biopsies (blue bars) on mRNA expression levels of all DE genes present in the 17 significantly affected pathways as determined by IPA. The different clusters selected for Kaplan\hyp{}Meier analysis are shown in the upper dendrogram in different shades of gray, corresponding to the legend of Figure~\ref{fig6.3}. Red: upregulation, green: downregulation.} % OBS! bw print version
		    \caption{Unsupervised hierarchical clustering of osteosarcoma cell line data (black bars), control cultures (MSC: dark gray bars, osteoblast: light gray bars), and data from osteosarcoma biopsies (blue bars) on mRNA expression levels of all DE genes present in the 17 significantly affected pathways as determined by IPA. The different clusters selected for Kaplan\hyp{}Meier analysis are shown in the upper dendrogram in different shades of blue, corresponding to the legend of Figure~\ref{fig6.3}. Red: upregulation, green: downregulation.} %%% OBS! rgb version
		     \label{afig6.1}
		\end{figure}
		%
\end{subappendices}

%\end{document}
% Marieke Kuijjer
% 2013-02-15
% chapter 07

	%\documentclass[12pt,b5paper]{book}
	%\setcounter{secnumdepth}{0}
	%\setcounter{tocdepth}{1}
	%\usepackage[hidelinks]{hyperref}

%\begin{document}

%%% title page

\chapter{Identification of osteosarcoma driver genes by integrative analysis of copy number and gene expression data}\label{ch7}
\thispagestyle{empty}				%%% to remove page number from first page of chapter, must be placed after calling the chapter

\vfill

\vspace{0.5cm}
This chapter is based on the publication:
\underline{Kuijjer ML}, Rydbeck H, Kresse SH, Buddingh EP, Lid AB, Roelofs H, B{\"u}rger H, Myklebost O, Hogendoorn PCW, Meza-Zepeda LA, Cleton-Jansen AM. {\it Genes Chromosomes Cancer}. 2012 Jul;51(7):696-706

\newpage

%%% main document

\section{Abstract}\label{abstract7}
High-grade osteosarcoma is a tumor with a complex genomic profile, occurring primarily in adolescents with a second peak
at middle age. The extensive genomic alterations obscure the identification of genes driving tumorigenesis during osteosarcoma
development. To identify such driver genes, we integrated DNA copy number profiles (Affymetrix SNP 6.0) of 32 diagnostic
biopsies with 84 expression profiles (Illumina Human-6 v2.0) of high\hyp{}grade osteosarcoma as compared with its
putative progenitor cells, {\it i.e.} mesenchymal stem cells ($n=12$) or osteoblasts ($n=3$). In addition, we performed paired analyses between copy number and expression profiles of a subset of 29 patients for which both DNA and mRNA profiles were available. Integrative analyses were performed in Nexus Copy Number software and statistical language R. Paired analyses
were performed on all probes detecting significantly differentially expressed genes in corresponding {\it LIMMA} analyses. For
both nonpaired and paired analyses, copy number aberration frequency was set to $>35\%$. Nonpaired and paired integrative
analyses resulted in 45 and 101 genes, respectively, which were present in both analyses using different control sets. Paired
analyses detected $>90\%$ of all genes found with the corresponding nonpaired analyses. Remarkably, approximately twice as
many genes as found in the corresponding nonpaired analyses were detected. Affected genes were intersected with differentially expressed genes in osteosarcoma cell lines, resulting in 31 new osteosarcoma driver genes. Cell division related genes, such as {\it MCM4} and {\it LATS2}, were overrepresented and genomic instability was predictive for metastasis\hyp{}free survival, suggesting that deregulation of the cell cycle is a driver of osteosarcomagenesis.

\section{Introduction}\label{introduction7}
High-grade osteosarcoma is an aggressive primary
bone tumor, which mostly occurs during
adolescence, with a second peak at middle age,
at the metaphysis of long bones. The tumor is
characterized by aberrant production of osteoid
matrix and by very complex karyotypes~\cite{raymond2002conventional,cleton2005central}.
Since the introduction of DNA microarray technology,
recurrent DNA copy number changes in
human osteosarcoma tumor tissues have been
identified by comparative genomic hybridization
(CGH) and high\hyp{}density single nucleotide polymorphisms
(SNP) microarray analysis. There is a
general consensus about gain of chromosome
arms 6p, 8q, and 17p, but many additional regions
are reported as well~\cite{squire2003high,man2004genome,
atiye2005gene,yen2009identification,kresse2010evaluation}. The effects of copy number
alterations may be reflected by changes in expression
of genes in the affected chromosomal
regions. There are various publications on human
osteosarcoma gene expression, but few show robust
bioinformatics (as described by Kuijjer {\it et al}.~\cite{kuijjer2011mrna}). Often, small sample sizes and heterogeneity
within groups result in only a small number of
significant genes, on which usually no correction
for multiple testing is applied. Another problem
when studying osteosarcoma gene expression data
is the lack of an osteosarcoma benign precursor
lesion and its debated cell of origin---although it
becomes clearer that the mesenchymal stem cell
or its derivative is the progenitor of osteosarcoma~\cite{mohseny2009osteosarcoma,mohseny2011concise}. The disease seems to develop suddenly as
a full\hyp{}blown tumor, rendering it difficult to detect
early drivers of osteosarcomagenesis. We have
previously determined differential expression
related to specific clinical parameters~\cite{buddingh2011tumor,kuijjer2011mrna}. In addition, we
have compared osteosarcoma with osteoblastoma---a
benign tumor which develops at the
same site as osteosarcoma, but which does not
progress into the latter. This comparison of
human osteosarcoma with a control tissue showed
that cell cycle regulation is the most significantly
altered pathway in osteosarcoma~\cite{cleton2009profiling}.

There are advantages of integrating copy number
and expression data when aiming to identify
driver genes. First, copy number data analysis of
tumors with complex genomic profiles may return
numerous bystander or hitch\hyp{}hiker genes, as copy
number alterations may occur not only because
they are advantageous for the tumor but also as a
result of general genomic instability. Regions of
copy number alteration may therefore encompass
no driver gene at all, or may include additional
genes. Also, some genes with altered copy numbers
may not be expressed in the tumor due to
tissue\hyp{}specific expression. These aspects hamper
the identification of drivers of tumorigenesis,
especially when the number of recurrent genes in
such tumors is high. Second, at the mRNA level,
drivers affect downstream genes and switch on
feedback mechanisms, again rendering it difficult
to determine the real osteosarcoma drivers in a
pool of differentially expressed genes~\cite{lee2008integrative}. Integration of DNA copy number and
gene expression data filters out at least part of
such bystanders and of genes that act downstream
of drivers of tumorigenesis, because most
of these genes have altered copy numbers, but no
change in expression, or vice versa, while drivers
are both amplified and upregulated, or deleted
and downregulated. Particularly osteosarcoma is
genetically extremely unstable and therefore
genomic data analysis of this tumor type would
benefit from an approach that distinguishes driver
genes from the numerous more random genetic
events.

Nonpaired integrative analysis may be performed
by determining recurrent regions of copy
number alterations which have higher than
expected numbers of differentially expressed
genes. Paired integrative analysis is a more powerful
method, because the relationship between copy
number alterations and gene expression can be
inferred in each specific sample, instead of being
based on averaged quantities. A statistically correct
method for paired integrative analysis of these different
data types has not yet been defined. Paired
integrative analysis is usually performed by selecting
genes based on the correlation between gene
expression and copy number levels, such as is performed
by the recently published methods DR-Integrator~\cite{salari2010dr}
and Regularized dual
Canonical Correlation Analysis~\cite{soneson2010integrative}.
However, gains and losses may not necessarily
directly translate to the same quantity of change
in expression levels~\cite{lee2008integrative}, and important
genes may be overlooked this way. A method
where paired integrative analysis is detected for
specific chromosomal regions with altered genomic
and transcriptional status does exist~\cite{bicciato2009computational},
but this method is not optimal for tumors
such as osteosarcoma with highly unstable
genomes, since copy number values are normalized
to the mean copy number over each array, and this
mean value may be altered in such tumors. Two
methods, PARADIGM and CNAmet, combine different
types of data on a gene\hyp{}based level. In
PARADIGM, integration of different data types is
used to detect patient\hyp{}specific pathway activities~\cite{vaske2010inference}. CNAmet returns genes that
show differential expression between samples with
and without methylation and/or copy number
alteration~\cite{louhimo2011cnamet}. This
elegant approach may however hamper the identification
of genes that are regulated by other frequently
altered genes, such as {\it TP53} and {\it MDM2} in
osteosarcoma.

Aiming to identify osteosarcoma driver genes,
we performed both nonpaired and paired integrative
analyses on high\hyp{}grade osteosarcoma prechemotherapy
biopsy data. We combined results
from analyses as compared with different control
sets---mesenchymal stem cells (MSCs) and osteoblasts,
so that we did not exclude one of these
proposed progenitors as the cell of origin of osteosarcoma.
We show that the paired integrative
analysis returns more affected genes than the
nonpaired integrative analysis. There is an overrepresentation
of genes involved in genomic stability
in osteosarcoma samples. The identified
genes may be important drivers in
osteosarcomagenesis.

\section{Materials and methods}\label{methods7}
\subsection{Ethics statement}
All biological material was handled in a coded
fashion. Ethical guidelines of the individual European
partner institutions were followed and samples
and clinical data were handled in a coded
fashion and stored in the EuroBoNeT biobank.

\subsection{Patient material and cell lines}
Genome\hyp{}wide expression profiling was performed
on pretreatment diagnostic biopsies of 84
resectable high\hyp{}grade osteosarcoma patients from
the EuroBoNeT consortium (\url{www.eurobonet.eu}).
Clinicopathological details of these samples can
be found in Table~\ref{tab7.1}.
%
\begin{table}[htbp]
	\centering
	\small
		\begin{tabular}[c]{|lcc|}
		\hline
		Category & Patient characteristics & Number of biopsies (\%)\\
		\hline
		Institution & LUMC, Netherlands & 36 (42.9)\\
		& IOR, Italy & 12 (14.3)\\
		& LOH, Sweden & 3 (3.6)\\
		& Radiumhospitalet, Norway & 1 (1.2)\\
		& WWUM, Germany & 32 (38.1)\\
		Location primary tumor & Femur & 40 (47.6)\\
		& Tibia/Fibula & 28 (33.3)\\
		& Humerus & 11 (13.1)\\
		& Axial skeleton & 1 (1.2)\\
		& Unknown/other & 4 (4.8)\\
		Histological subtype & Osteoblastic & 52 (61.9)\\
		& Chondroblastic & 9 (10.7)\\
		& Fibroblastic & 7 (8.3)\\
		& Telangiectatic & 4 (4.8)\\
		& Minor subtype & 11 (13.1)\\
		& Unknown & 1 (1.2)\\
		Huvos grade & 1 or 2 & 38 (45.2)\\
		& 3 or 4 & 33 (39.3)\\
		& Unknown/NA & 14 (16.7)\\
		Metastasis at diagnosis & Yes & 14 (16.7)\\
		& No & 69 (82.1)\\
		& Unknown & 1 (1.2)\\
		Sex & Male & 54 (64.3)\\
		& Female & 29 (34.5)\\
		& Unknown & 1 (1.2)\\
		Age & $<20$ years & 64 (76.2)\\
		& $>=20$ years & 19 (22.6)\\
		& Unknown & 1 (1.2)\\
		\hline
		\end{tabular}
\caption{Clinicopathological details.}
	\label{tab7.1}  
\end{table}
%
Human bone\hyp{}marrow\hyp{}derived
MSCs were obtained from five osteosarcoma
patients and seven healthy donors. Osteoblasts
($n=3$) were derived from MSCs on
osteogenic differentiation. MSCs and osteoblasts
were characterized and handled as described~\cite{cleton2009profiling}. Copy number analysis
was performed on 32 pretreatment diagnostic
biopsies, of which 29 overlapped with the 84
samples used for expression analysis.

\subsection{Copy number microarray data analysis}
Affymetrix Genome\hyp{}Wide Human SNP 6.0
arrays (Affymetrix, Santa Clara, CA) were used
for SNP data analysis. Genomic DNA preparation,
labeling, hybridization, and scanning were
performed as described by Kresse {\it et al}.~\cite{kresse2010evaluation}.
Microarray data preprocessing was performed as
described previously~\cite{pansuriya2011genome}.
Hybridization quality was estimated by the genotype
call rate using the Birdseed genotype calling
algorithm in Genotyping Console (version 4.0,
Affymetrix). Samples of poor quality were
excluded from further analyses. We performed
copy number analysis in Nexus software version
5 (Biodiscovery, El Segundo, CA) using CNCHP
log ratio files generated by Genotyping Console
using 27 controls as a baseline, which is a subset
of the reference set of 29 samples which was
used by Pansuriya {\it et al}.~\cite{pansuriya2011genome}. We rejected two
samples based on results from the quality control
analysis in Genotyping Console. Circular Binary Segmentation (CBS)\hyp{}based SNPRank Segmentation
was used to identify aberrant genomic
regions. To be included as frequently aberrant, a
copy number alteration was called when detected
in at least 35\% of all cases. Correlation of copy
number alterations with clinical data was performed
in Nexus software, with correction for
multiple testing.

\subsection{Genome\hyp{}wide gene expression microarray data analysis}
Osteosarcoma tissue handling, RNA isolation,
synthesis of cDNA, cRNA amplification, hybridization
of cRNA onto the Illumina Human-6 v2.0
Expression BeadChips (Illumina, San Diego,
CA), and microarray data processing and quality
control in the statistical language R version 2.10~\cite{r2.10.0}
were performed
as described previously~\cite{buddingh2011tumor}.
High correlations between these microarray
data and corresponding qPCR results have
been demonstrated previously~\cite{buddingh2011tumor}.
Unsupervised hierarchical cluster analysis
was performed using R package pvclust with
default settings~\cite{suzuki2006pvclust}.

\subsection{Data deposition}
MIAME\hyp{}compliant copy number and gene
expression data have been deposited in the GEO
database (\url{www.ncbi.nlm.nih.gov/geo/}, superseries
accession number GSE33383).

\subsection{Detection of significantly differentially expressed genes}
We performed a {\it LIMMA} analysis~\cite{smyth2004linear}
in order to determine differential expression
between high\hyp{}grade osteosarcoma samples ($n=84$)
and control tissues---MSCs ($n=12$) and osteoblasts
($n=3$). Also, gene expression differences between
MSCs and osteoblasts were determined. We used a
Benjamini and Hochberg False Discovery Rate
(FDR) of 0.05 as cut-off for significance.

\subsection{Nonpaired integrative analysis}
Nonpaired integrative analysis was performed
by importing lists of differentially expressed
genes into the Copy Number module of Nexus
software version 5. Based on the length of the
gene list, Nexus software performs a Fisher's
exact test in order to determine whether the
number of differentially expressed genes in a
specific region with a significant copy number
alteration is larger than expected by chance.
Genes present in such regions of copy number
alteration with FDR\hyp{}adjusted p-values (Q-bounds
in Nexus software) $<0.05$ were returned from
this integrative analysis. We did not apply any
restrictions on the size of copy number aberrations.
A few small altered regions that did not
encompass an entire gene were detected, but
these regions did not return genes upon integration
with expression data. Nexus software only
reports genes which are both gained and overexpressed,
or both deleted and downregulated.

\subsection{Paired integrative analysis}
For the paired integrative analysis, copy number
data of all autosomal overlapping genes
between the copy number and gene expression
data were exported from Nexus software, and
converted into a binary matrix containing all
genes with a gain (1) and no gain (0), and a similar
binary matrix for losses. As in the nonpaired
integrative analysis, we did not apply any restrictions
on the size of copy number alterations.
Gene expression data of each probe for each sample
were normalized against average gene expression
values of the corresponding probes over all
control samples (either expression data from 12
MSCs or from three osteoblasts)---this was performed
by subtracting the average expression of
the control samples from the expression levels of
the sample of interest, since these are log\hyp{}transformed
expression values. For both analyses, only
genes that were significantly differentially
expressed between the 84 osteosarcoma samples
and the specific control set were analyzed, in
order to make sure that all genes returned from
the integrative analysis were significantly differentially
expressed. Subsequently, genes that overlapped
between the copy number binary matrices
and that matched the fold change of expression
(upregulation for genes with gains, and downregulation
for genes with losses) were returned as
two-way contingency tables using scripts in R.
Genes that were altered in two types of data
were further filtered by applying the sample recurrence
criterion of 35\%. This generated lists of
recurrent two-way altered genes. The odds ratios
for having both copy number and expression
changes were calculated for different combinations,
for instance gain and upregulation. We
used Bonferroni corrected Chi-square or Fisher's
exact p-values $<0.05$ to determine significance.

\subsection{Gene set enrichment}
GO term enrichment was tested using Bioconductor
package {\it topGO}~\cite{alexa2006improved}. Lists
of significantly affected genes were compared
with all genes eligible for the analysis. GO terms
with Fisher's exact p-values $<0.0001$, as calculated
by the {\it weight01} algorithm from {\it topGO}, were
defined significant.

\subsection{Genomic instability scores and survival analysis}
We calculated genomic instability scores for 83
(out of the 84) osteosarcoma biopsies (for one sample
no follow\hyp{}up data were available) and all controls,
as well as for two normal bone samples
(obtained from cancer patients at the Norwegian
Radium Hospital), 20 osteosarcoma xenografts
(Kresse {\it et al}., {\it unpublished data}, and Kuijjer {\it et al}.~\cite{kuijjer2011mrna}), 19 osteosarcoma cell lines~\cite{ottaviano2010molecular},
and the HeLa cervical cancer cell line. For
calculation of the genomic instability scores, we
refer to the article by Carter {\it et al}.~\cite{carter2006signature}. In short,
this method calculates per sample per probe the
expression of that particular probe minus the
mean expression of that probe over all samples.
For each sample, the sum of these values for all
probes present in the genomic instability signature
is calculated. This value is then compared
between all samples and thus gives a relative measure
of genomic instability. We used 24 genes of
the CIN25 signature, because for one gene no
probe was present on the Illumina v2.0 BeadChip.
For genes with multiple probes, we used the
probe that showed the highest variation in expression
levels. We determined metastasis\hyp{}free survival
using the Kaplan\hyp{}Meier method and performed a
Logrank test for trend using GraphPad Software
(La Jolla, CA, www.graphpad.com).

\section{Results}\label{results7}
\subsection{Recurrent chromosomal regions with copy number aberrations in high\hyp{}grade osteosarcoma}
Thirty two high\hyp{}grade osteosarcoma prechemotherapy
biopsies were hybridized to Af\-fy\-met\-rix
SNP 6.0 arrays in order to determine recurrent
copy number alterations. In total, 67 regions with
recurrent alterations were detected, of which 35
regions had copy number gain, and 32 copy number
loss (see Supporting Information Table 1 ({\it available online}~\cite{ch7additional})).
Recurrent gains were present on chromosome
arms 1p, 1q, 4p, 5p, 6p, and 8q, and losses on
chromosome arms 1p, 1q, 2q, 3q, 6q, 7q, 8p, 10p,
10q, 12p, 13q, 15q, 16p, and 16q. A genome\hyp{}wide
frequency plot of copy number alterations is
shown in Figure~\ref{fig7.1}.
%
\begin{figure}[htbp]
	\centering
%	\includegraphics[width=1.0\textwidth]{figs07/fig1bw.pdf}	% OBS! print version bw
	\includegraphics[width=1.0\textwidth]{figs07/fig1rgb.pdf}	% OBS! pdf version rgb
	\caption{Genome\hyp{}wide frequency plot of copy number alterations on chromosomes 1--22 in 32 high\hyp{}grade osteosarcoma prechemotherapy biopsies. Left of the chromosomes: loss, right: gain.}
	\label{fig7.1}
\end{figure}
%
No significant correlation was
detected for specific regions with copy number
alterations and clinical information (tested clinical
parameters are shown in Table~\ref{tab7.1}).

\subsection{Comparison of osteoblasts and MSCs}
Unsupervised hierarchical cluster analysis
resulted in separate clusters for biopsies and cell
lines. Within the cell line cluster, osteosarcoma
cell lines formed one subcluster, whereas MSCs
and osteoblasts formed a second subcluster (Supporting
Information Figure 1 ({\it available online}~\cite{ch7additional})). This indicates that
the control cell lines are more similar to one
another than to osteosarcoma cells. On the basis
of hierarchical clustering of gene expression data,
we cannot determine the cell of origin of osteosarcoma.
A total of $1,382$ genes were differentially
expressed between osteoblasts and MSCs. GO
term enrichment resulted in seven significant GO
terms, which are represented in Supporting Information
Figure 2 ({\it available online}~\cite{ch7additional}). In summary, GO term enrichment
showed differences in cellular structure,
proliferation, and apoptosis. Genes showing significant
differences between both control cell
types, however, can nonetheless be differentially
expressed between osteosarcoma samples and
both control cell types, thus can still be important
drivers of osteosarcomagenesis. We therefore set
out to select genes that showed differential
expression in osteosarcoma as compared with
both MSCs and osteoblasts.

\subsection{Gene expression signature of high\hyp{}grade osteosarcoma}
We detected $12,542$ and $2,939$ probes encoding
for genes that were significantly differentially
expressed between the 84 osteosarcoma biopsies
and MSCs and osteoblasts, respectively. MA
plots, showing log\hyp{}intensity ratios and log\hyp{}intensity
averages for both analyses, are depicted in
Supporting Information Figure 3 ({\it available online}~\cite{ch7additional}). A total of $1,679$
probes overlapped between both analyses, of
which $1,639$ were either up- or downregulated in
both. GO term analysis on the genes represented
by these $1,639$ probes showed an enrichment of
apoptosis and signal transduction genes. Antigen
processing and presentation, as well as angiogenesis
were also overrepresented (Supporting Information
Figure 4 ({\it available online}~\cite{ch7additional})).

\subsection{Paired integrative analysis is more sensitive than nonpaired integrative analysis}
Nonpaired integrative analysis was performed
on data from 32 samples hybridized on SNP
arrays and from 84 samples hybridized on gene
expression arrays, whereas paired analysis was
performed on a subset of 29 samples for which
both SNP and expression data were available. In
total, $16,105$ autosomal genes were represented
both on SNP and on gene expression arrays.
Nonpaired integrative analysis resulted in 253
significantly affected genes in osteosarcoma biopsies
versus mesenchymal stem cells, whereas 71
genes were detected when osteoblasts were used
as a control. A total of 45 genes were identified
in both analyses versus MSCs and versus osteoblasts
(Figure~\ref{fig7.2}).
%
\begin{figure}[htbp]
  \centering
  \begin{minipage}[b]{0.50\linewidth}
%    \includegraphics[width=1\textwidth]{figs07/fig2bw.pdf}		% OBS! print version bw
    \includegraphics[width=1\textwidth]{figs07/fig2rgb.pdf}	% OBS! pdf version rgb
  \end{minipage}
    \hfill
  \begin{minipage}[b]{0.46\linewidth}
    \caption{Venn diagram with numbers of affected genes in both nonpaired and paired analyses, and in osteosarcoma biopsies versus MSCs and versus osteoblasts. NP: nonpaired integrative analysis, P: paired integrative analysis, OB: analysis of osteosarcoma biopsies versus osteoblasts, MSC: analysis of osteosarcoma samples versus mesenchymal stem cells.}
     \label{fig7.2}
     \end{minipage}
\end{figure}
%
Of these 45 genes, 23 were also
detected in expression analyses of a panel of 19
osteosarcoma cell lines~\cite{ottaviano2010molecular}
versus MSCs and osteoblasts (Supporting Information
Figure 5A ({\it available online}~\cite{ch7additional})). For the paired integrative analyses,
we determined whether the number of genes
with gain combined with overexpression and with
loss combined with downregulation was higher
than expected per sample, based on the numbers
of copy number alterations and gene expression
changes in the whole genome. This was true for
most samples, as depicted in Figure~\ref{fig7.3}, in which the
odds ratios and significance of data dependencies
are shown.
%
\begin{figure}[htbp]
	\centering
	\includegraphics[width=1.0\textwidth]{figs07/fig3rgb.pdf}	% pdf version also rgb
	\caption{Dependence of gene copy number and gene expression data. The heatmaps depict odds ratios for the numbers of genes per sample which show gain and overexpression (overGain), gain and underexpression (underGain), loss and overexpression (overLoss), and loss and underexpression (underLoss). Chi-square tests, or, in case a group contained $<10$ genes, Fisher's exact tests, were performed in order to evaluate whether the number of genes reported from the integrative analysis was higher than expected by chance. $\ast$ Bonferroni\hyp{}corrected p-values $<0.05$. {\it A}, Osteosarcoma biopsies versus MSCs, {\it B}, versus osteoblasts.}
	\label{fig7.3}
\end{figure}
%
Paired integrative analysis resulted in
445 and 138 genes when compared with MSCs
and osteoblasts, respectively. A total of 101 genes
overlapped between these different analyses (Figure~\ref{fig7.2}), and of this set, 31 genes were also detected in
the cell line expression data (Supporting Information
Figure 5B ({\it available online}~\cite{ch7additional}), Table~\ref{tab7.2}).
%
\begin{table}[htbp]
	\centering
	\small
		\begin{tabular}[c]{|lcccc|}
		\hline
		Symbol & Cytoband & CNA & CNA freq (\%) & logFC\\
		\hline
		{\it CLCC1} & 1p13.3 & Gain & 41.4 & 1.24\\
		{\it MCM4} & 8q11.21 & Gain & 37.9 & 1.35\\
		{\it AKR1C3} & 10p15.1 & Loss & 37.9 & -1.94\\
		{\it AKR1C4} & 10p15.1 & Loss & 37.9 & -1.34\\
		{\it ARHGAP22} & 10q11.22 & Loss & 37.9 & -0.45\\
		{\it PGBD3} & 10q11.23 & Loss & 41.4 & -0.82\\
		{\it ARID5B} & 10q21.2 & Loss & 48.3 & -2.33\\
		{\it REEP3} & 10q21.3 & Loss & 48.3 & -0.51\\
		{\it HERC4} & 10q21.3 & Loss & 51.7 & -1.31\\
		{\it PBLD} & 10q21.3 & Loss & 48.3 & -0.29\\
		{\it RUFY2} & 10q21.3 & Loss & 48.3 & -0.20\\
		{\it KIAA1279} & 10q22.1 & Loss & 43.1 & -0.57\\
		{\it SRGN} & 10q22.1 & Loss & 43.1 & -2.26\\
		{\it AIFM2} & 10q22.1 & Loss & 44.8 & -0.52\\
		{\it CHST3} & 10q22.1 & Loss & 48.3 & -1.17\\
		{\it FAS} & 10q23.31 & Loss & 44.8 & -0.42\\
		{\it PCGF5} & 10q23.32 & Loss & 37.9 & -0.34\\
		{\it PPP1R3C} & 10q23.32 & Loss & 37.9 & -2.89\\
		{\it AVPI1} & 10q24.2 & Loss & 37.9 & -2.35\\
		{\it BLOC1S2} & 10q24.31 & Loss & 37.9 & -0.51\\
		{\it CASC2} & 10q26.11 & Loss & 44.8 & -0.18\\
		{\it FAM45A} & 10q26.11 & Loss & 39.7 & -0.78\\
		{\it ERCC6} & 13q11.23 & Loss & 41.4 & -0.52\\
		{\it WASF3} & 13q12.13 & Loss & 44.8 & -2.43\\
		{\it C13orf33} & 13q12.3 & Loss & 48.3 & -2.26\\
		{\it LHFP} & 13q14.11 & Loss & 48.3 & -1.89\\
		{\it WBP4} & 13q14.11 & Loss & 55.2 & -0.93\\
		{\it TSC22D1} & 13q14.11 & Loss & 58.6 & -1.39\\
		{\it RCBTB1} & 13q14.2 & Loss & 58.6 & -0.25\\
		{\it LATS2} & 13q21.11 & Loss & 44.8 & -0.96\\
		{\it DCUN1D3} & 16p12.3 & Loss & 37.9 & -1.39\\
		\hline
		\end{tabular}
\caption{Candidate osteosarcoma driver genes. All frequencies and fold changes are mean values of both integrative analyses---osteosarcoma biopsies versus MSCs and osteosarcoma biopsies versus osteoblasts. For genes for which more than one probe was present on the array, the probe with the highest fold change was used. Cytoband: UCSC cytogenetic band, CNA: copy number aberration, CNA freq: copy number aberration frequency (for $n=29$), logFC: log fold change in biopsies (negative means downregulation, positive means upregulation).}
	\label{tab7.2}
\end{table}
%
Hence, paired analyses
detected $>90\%$ of all genes found with corresponding
nonpaired analyses. In addition, approximately
twice as many genes as found in the
corresponding nonpaired analyses were detected
(Figure~\ref{fig7.2}, Supporting Information Figure 6 ({\it available online}~\cite{ch7additional})). Note that
in the paired analysis fewer samples are included.
Thus, paired analysis gives more robust results
despite the lower sample size. Changing the
threshold of FDR\hyp{}adjusted p-values in the nonpaired
integrative analysis from 0.05 to 0.15 ({\it data
not shown}) did not alter this ratio.

\subsection{Genomic instability genes play a role in osteosarcoma progression}
We calculated genome instability scores using
the method of Carter {\it et al}.~\cite{carter2006signature}, which compares
levels of gene expression of a previously
defined genomic instability signature between
samples in a dataset, for all osteosarcoma biopsies
and different control tissues and cell lines (Figure~\ref{fig7.4}A).
%
\begin{figure}[htbp]
	\centering
	\includegraphics[width=1.0\textwidth]{figs07/fig4bw.pdf}	% print version bw, pdf version bw
	\caption{Genomic instability scores and metastasis\hyp{}free survival. {\it A}, Genomic instability scores for high\hyp{}grade osteosarcoma biopsies, normal bone, osteosarcoma xenografts and cell lines, the HeLa cell line, and mesenchymal stem cells (MSC) and osteoblasts (OB), as calculated by the method of Carter {\it et al}.~\cite{carter2006signature}. {\it B}, Metastasis\hyp{}free survival Kaplan\hyp{}Meier curves for four quartiles of genomic instability scores. {\it C}, Metastasis\hyp{}free survival Kaplan\hyp{}Meier curves for the total amount of genes with copy number gains and losses, using a cut-off based on the median amount of genes per sample showing copy number aberration.}
	\label{fig7.4}
\end{figure}
%
The osteosarcoma biopsies showed highly
variable scores, whereas genomic instability scores
for the controls, normal bone, MSCs, and osteoblasts
were relatively low. High instability scores
were detected for osteosarcoma xenografts, cell
lines, and the HeLa cell line, in increasing order.
This signature predicted for metastasis\hyp{}free survival
in osteosarcoma samples as well (Figure~\ref{fig7.4}B),
with high scores correlating with shorter metastasis\hyp{}free survival (Logrank test for trend p-value $=0.0112$).
As expected, the total number of genes
with copy number gains or losses, which is a direct
measure of genomic instability from the SNP data,
was predictive for progression as well (Logrank
test p-value $=0.018$, Figure~\ref{fig7.4}C).

\subsection{Candidate osteosarcoma driver genes}
The 31 genes returned by the paired integrative
analysis on clinical samples that also were
differentially expressed in osteosarcoma cell lines
are shown in Table 2, together with their chromosomal
locations, aberration frequencies, and
log fold changes. A total of 22/31 genes have
been described to play a role in cancer. Interestingly,
one third of these 22 genes have a role in
cell cycle regulation, matching the importance of
cell cycle and replication in osteosarcomagenesis
as was found both using the genomic instability
scores of the expression data and the overall chromosomal
instability as detected in the copy number
data (Figure~\ref{fig7.4}).

\section{Discussion}\label{discussion7}
In this study, we report copy number and gene
expression alterations in high\hyp{}grade osteosarcoma
prechemotherapy biopsies, and then integrate
these data in order to detect osteosarcoma driver
genes. Copy number analyses, which were
obtained with high\hyp{}density SNP microarrays,
showed very high genomic instability in the osteosarcoma
biopsies. The pattern of aberrations is
in line with previous studies using aCGH and
SNP arrays, which show recurrent gains in chromosome
arms 1p, 6p, and 8q, and losses in chromosome
10. The previously reported recurrent
amplification on chromosome arm 17p~\cite{squire2003high,man2004genome,atiye2005gene,yen2009identification}
is not listed, because we used a
very strict cut-off for aberration frequency (35\%).
Aberration frequencies of 17\%~\cite{man2004genome}
and 26\%~\cite{yen2009identification} were previously found
on chromosome arm 17p, and a distinct amplification
in 17p with an aberration frequency of 21\%
can be seen in Figure 1. We chose such a high
cut-off for recurrent aberrations in order to enrich
for selected genetic events and exclude the
numerous haphazard alterations that can be
attributed to the high genomic instability of high\hyp{}grade
osteosarcoma. In addition, we previously
determined that this cut-off, as compared with
cut-offs of 15\% and 50\%, showed the most consistent
results in subsequent network and pathway
analyses on osteosarcoma cell line SNP data
({\it data not shown}). For genome\hyp{}wide gene expression
analyses, both MSCs and osteoblasts were
used as control cells, and we only considered
overlapping genes between both comparisons, in
order to make sure the affected genes were differentially
regulated in osteosarcoma when compared
with its putative progenitor cells. This
analysis identified a large number ($n=1,639$) of 
probes encoding for differentially expressed
genes. Many of these genes encode tissue type\hyp{}specific
proteins, as is shown in the GO term
enrichment analysis, and appear as upregulated in
osteosarcoma biopsies because the {\it in vitro} grown
control cells, MSCs and osteoblasts, lack surrounding
stroma and are nurtured under other
conditions. Antigen processing and presentation
as well as angiogenesis pathways were expected
to be upregulated, as macrophages and other
infiltrating cells are present in osteosarcoma tissue~\cite{buddingh2011tumor}, and as angiogenesis
plays a role in osteosarcoma progression~\cite{lee1999cell}.
Nevertheless, most stroma\hyp{}derived
gene expression is filtered out by integration with
copy number data, as this expression is not a
result of underlying copy number changes. In
addition to stroma\hyp{}related gene sets, GO term
analysis showed enrichment in apoptosis and signal
transduction genes, which are probably
altered in the osteosarcoma tumor cells and not
in the stroma. Because genes with concordant
changes in copy number and gene expression are
likely to be enriched in drivers of tumorigenesis,
we performed integrative analyses on both types
of data.

We found a remarkable increase in significant
differential genes in paired compared with nonpaired
analysis, {\it i.e.} 101 versus 45. In general,
paired integrative analysis was advantageous over
nonpaired integrative analysis, identifying roughly
twice as many genes, also when different aberration
frequency cut-offs or less stringent cut-offs
for significance were used in the nonpaired analysis.
Nonpaired analysis as performed in Nexus
software compares the number of differentially
expressed genes in a region of copy number aberration
with the expected number of differentially
expressed genes, which is based on the total
number of differentially expressed genes over the
whole genome. This method may be too rigorous,
because an altered copy number region may
encompass tissue\hyp{}specific genes, which may not
be expressed in the particular tumor tissue.
These genes then have altered copy number, but
no difference in expression. If an altered copy
number region contains a relatively large number
of such genes plus only a few candidate drivers,
the entire region will be removed from the output
of the analysis, which increases the amount
of false negatives. Moreover, in the cancer gene
expression profile, a large number of genes downstream
of drivers, {\it i.e.} directly or indirectly regulated
by drivers, or present in feedback loops will
be differentially expressed. This increases the
total number of differentially expressed genes,
which again lowers the chance that a specific
altered region is returned from the nonpaired
integrative analysis as significantly affected. Furthermore,
a single differentially expressed gene
in a certain region of copy number alteration may
still exert its driving function, and this driving
function usually does not depend on the proportion
of differentially expressed genes in the same
region. Because of this, and because our method
of paired integrative analysis is gene\hyp{}based and
not region\hyp{}based, we did not perform a correction
based on the total number of differentially
expressed genes when compared with the
affected copy number regions in the paired analysis
in R, and this may be an additional reason
why more genes are returned from the paired
analysis. However, in all samples, except for one
(L3438), the number of genes showing both copy
number alteration and differential expression was
higher than expected when compared with the
numbers of copy number alterations and differentially
expressed genes over the whole genome.
This was significant for the vast number of samples
(28/29, 23/29, 27/29, and 23/29, for combinations
gain and overexpression, loss and
underexpression in biopsies versus MSCs, and
gain and overexpression, loss and underexpression
in biopsies versus osteoblasts, respectively,
as shown in Figure~\ref{fig7.3}).

Genomic instability scores showed that the
instability in osteosarcoma tissues ranges from a
level comparable to that of the controls, to the
high instability levels of repeatedly passaged
tumors in xenografts and osteosarcoma cell lines.
We demonstrated both on copy number data, as
well as by applying a genomic instability gene
signature to genome\hyp{}wide gene expression data,
that high genomic instability in osteosarcoma is
correlated with poor metastasis\hyp{}free survival. This
suggests that genes playing a role in genomic
instability may be potent drivers of osteosarcoma
progression, as has been reported for various
other tumor types~\cite{carter2006signature}. Paired
integrative analysis confirmed this result, as one
third of the genes with a possible role in tumorigenesis
had a function connected to the cell
cycle. Of these genes, {\it MCM4} showed gain and
overexpression and was only detected by the
paired integrative analysis. {\it MCM4} is part of the
minichromosome maintenance complex, which
functions as a replication helicase, with a role in
maintaining genomic stability~\cite{aguilera2008genome}. This gene has been
reported overexpressed in various tumor types~\cite{freeman1999minichromosome,alison2002minichromosome,majid2010regulation}. Genes that were detected in both
nonpaired and paired analyses were all deleted
and underexpressed. {\it AVPI1}, or arginine vasopressin\hyp{}induced 1, may be involved in cell cycling~\cite{apweiler2011ongoing}. {\it ERCC6} is involved
in transcription\hyp{}coupled nucleotide excision
repair, which is a critical survival pathway protecting
against cancer~\cite{fousteri2008transcription}. {\it RCBTB1}, a candidate tumor suppressor,
was recently shown to have growth inhibitory activity
in osteosarcoma cells by regulating pathways
of DNA damage/repair and apoptosis~\cite{zhou2010clld7}.
{\it LATS2}, or large tumor suppressor
homolog 2, plays a critical role in centrosome
duplication, maintenance of mitotic fidelity,
and genomic instability~\cite{visser2010lats}.
Positive feedback between the p53 and Lats2 tumor
suppressors prevents tetraploidization~\cite{aylon2006positive},
which could be an initiating step in
osteosarcomagenesis, leading to genomic instability~\cite{ganem2007limiting,ganem2007tetraploidy}. Also, a role of Lats2 in quenching of the
increased genomic instability of H-Ras\hyp{}induced
transformation has been identified~\cite{aylon2006positive}. {\it DCUN1D3} encodes for a UVC\hyp{}responsive
protein involved in cell cycle progression and cell
growth~\cite{ma2008dcun1d3}. Additional candidate
genes with no direct role in cell cycle regulation
include for example genes with a role in apoptosis
({\it AIFM2}, {\it BLOC1S2}, {\it FAS}) and metabolism
({\it AKR1C3} and {\it -4}). Some previously reported
genes with a driver role in osteosarcoma were not
identified, mainly because our high cut-off for
recurrence. For example, {\it CDKN2A}, {\it MDM2}, and
{\it E2F2} had recurrence frequencies of 28\%, 17\%,
and 34\%, respectively (in the dataset of 29 samples).
{\it CDKN2A} and {\it MDM2} were not significantly
differentially expressed, but {\it E2F2} was consistently
significantly overexpressed with log fold
changes $>1.50$ in all analyses (biopsies and cell
lines as compared with different controls). {\it TP53}
and {\it RB1} aberrations were present in $>35\%$ of all
samples (38\% and 69\%, respectively). {\it TP53} was
significantly downregulated in biopsies as compared
with both controls, but not in the osteosarcoma
cell line dataset. {\it RB1} showed significant
downregulation when compared with MSCs, but
not with osteoblasts, indicating a difference
between these controls in {\it RB1} signaling. We set
our cut-off for recurrence to 35\% and only
selected genes present both in osteosarcoma
biopsies as well as in cell lines as compared with
two different control sets, in order to select for
the most important osteosarcoma drivers. Using
this method, we were able to detect previously
unreported driver genes.

In summary, we have shown that an individual
gene\hyp{}based paired integrative analysis of copy
number and gene expression data performs better
than a region\hyp{}based nonpaired analysis. Several
osteosarcoma candidate driver genes, especially
genes playing a role in cell cycle progression,
have been identified. Additional research, particularly
functional studies, should reveal whether
these genes are early or late drivers in
osteosarcomagenesis.

%%% references

\begin{small}
\begin{singlespace}
\bibliographystyle{unsrtnatshort}		% sorted as referenced, was unsrtnat, but unsrtnatshort gives shorter output
\bibliography{biblio}
\end{singlespace}
\end{small}

%\end{document}
% Marieke Kuijjer
% 2013-02-15
% chapter 06

	%\documentclass[12pt,b5paper]{book}
	%\setcounter{secnumdepth}{0}
	%\setcounter{tocdepth}{1}
	%\usepackage[hidelinks]{hyperref}

%\begin{document}

%%% title page

\chapter{Frequent loss of heterozygosity and amplification in high-grade osteosarcoma: analysis of recurrent tumor suppressor genes}\label{ch8}
\thispagestyle{empty}				%%% to remove page number from first page of chapter, must be placed after calling the chapter

\vfill

\vspace{0.5cm}
This chapter is based on the manuscript:
\underline{Kuijjer ML}, Liebelt F, Rydbeck H, Myklebost O, Meza-Zepeda LA, Szuhai K, Hogendoorn PCW, Cleton-Jansen AM. \emph{In preparation}

\newpage

%%% main document

\section{Abstract}\label{abstract8}
High-grade osteosarcoma is an aggressive primary bone tumor, with a peak incidence at adolescence. Osteosarcoma karyotypes are heterogeneous, and characterized by a high degree of genomic instability, rendering it difficult to detect recurrent driver genes in this tumor type. By performing Affymetrix SNP 6.0 microarray data analysis of 29 prechemotherapy biopsies, we observed that loss of heterozygosity (LOH) often cooccurs with copy number gains. We performed paired integrative analysis with genome\hyp{}wide expression data in order to determine which genes show differential expression in regions of LOH and gain, and integrated data from biopsies with data from 12 cell lines, which resulted in the identification of 29 recurrent genes with LOH, gain, and overexpression. We validated LOH of candidate tumor suppressor genes by Sanger sequencing and screened for mutations in candidate genes, as osteosarcoma cells may have selected for LOH and amplification of mutated tumor suppressors. We did not identify recurrent mutations, suggesting that these genes do not have a tumor driving function. Fluorescence in situ hybridization (FISH) analysis of candidate gene {\it XRCC6BP1} showed that this gene was present on homogeneous staining regions (HSRs) in 1/2 cell lines. As we detected a large number of recurrent candidate oncogenes by paired integrative analysis of LOH, gain, and overexpression, it may be valuable to determine whether these candidate oncogenes are present on HSRs in osteosarcoma.

\section{Background}\label{introduction8}
High-grade osteosarcoma is the most frequent primary malignant bone tumor, affecting roughly five persons in a population of one million each year~\cite{raymond2002conventional}. The tumor is highly aggressive, leading to distant metastases in approximately 45\% of all patients. Since the introduction of neoadjuvant chemotherapy in the 1970s, survival profiles have reached a plateau. In order to identify specific targets for therapy it is important to screen for recurrent driver genes in osteosarcoma. High\hyp{}grade osteosarcoma karyotypes are characterized by a high level of genomic instability, often harboring numerous numerical and structural changes, and high degree of aneuploidy~\cite{cleton2005central}. This results in many frequently affected genes in osteosarcoma, which may not all be important drivers, and thus renders it difficult to determine which genes are true drivers of osteosarcomagenesis. Integration of genomic and transcriptomic data will filter out most bystander and tissue\hyp{}specific genes, and can thereby result in a more specific list of candidate recurrent drivers. We previously detected novel osteosarcoma driver genes by integrating high\hyp{}throughput copy number and gene expression data~\cite{kuijjer2012identification}. Zygosity status can also be retrieved from SNP microarray data, which we describe in the present study.

Loss of heterozygosity (LOH) and allelic imbalance have been studied in osteosarcoma to quite some extent, and several recurrent regions have been described in detail~\cite{johnson2003determination,deshpande2006phc3,yen2009identification,kresse2009lsamp,pasic2010recurrent}. Smida {\it et al}. reported that the amount of LOH negatively correlated with survival~\cite{smida2010genomic}. This may be a readout of general genomic instability of the tumor, as for multiple human cancers~\cite{carter2006signature}, including osteosarcoma~\cite{kuijjer2012identification}, it has been shown that genomic instability is predictive for survival, but LOH of specific regions may play an important role in tumorigenesis of osteosarcoma. Loss of heterozygosity caused by the loss of one allele may cause downregulation of the transcript, which may especially be relevant for tumorigenesis when an affected tumor suppressor gene shows haploinsufficiency~\cite{berger2011continuum}. In osteosarcoma, {\it TP53}, {\it RB1}, and {\it PTEN} are frequently deleted~\cite{cleton2005central}, and these genes could drive tumorigenesis by haploinsufficiency, although one study found that the LOH state of {\it RB1} is not associated with prognosis~\cite{heinsohn2007determination}. {\it CDKN2A}, another tumor suppressor often affected in osteosarcoma, also shows hemizygous losses which may have a role in tumorigenesis, although small homozygous deletions in this gene are also seen~\cite{mohseny2010small}. {\it LSAMP} is frequently focally deleted in osteosarcoma, and may have a role as haploinsufficient tumor suppressor~\cite{yen2009identification,kresse2009lsamp,pasic2010recurrent}. Copy neutral LOH (CN-LOH), which is LOH without change in copy number, may play a role in tumorigenesis as well, as is for example shown in hematological malignancies~\cite{o2010copy}. In a study of osteosarcoma samples on Affymetrix 10 K 2.0 SNP arrays, it was reported that 28\% of LOH events result from CN-LOH~\cite{yen2009identification}. Regions of LOH accompanied by gains have not yet been discussed in high\hyp{}grade osteosarcoma, but have been described in other cancer types, {\it e.g.} in lung cancer~\cite{harris2011both} and triple-negative breast cancer~\cite{ha2012integrative}. Tumor cells could in theory select for a region of amplified LOH in case a mutated tumor suppressor with a gain\hyp{}of\hyp{}function or partial dominant negative function is affected, with deletion of the wild\hyp{}type gene and amplification and overexpression of the mutated gene. Another advantage for the tumor of stretches of LOH accompanied by gains is that tumor suppressors with inactivating mutations and oncogenes can have tumor\hyp{}promoting activities at the same time~\cite{bacolod2009emerging}.

In the present study, we analyzed high\hyp{}throughput SNP data of osteosarcoma pretreatment biopsies, and detected that LOH is often accompanied by copy number gains. By paired integrative analysis of LOH, copy number gain, and gene expression of osteosarcoma biopsy and cell line data, we identified 29 candidate driver genes, exhibiting both LOH and copy number gains. Gene set enrichment on genes in regions of LOH accompanied by gains and overexpression of the transcript returned pathways important in tumorigenesis and genomic instability. We validated a selection of candidate tumor suppressor genes by Sanger sequencing and Fluorescence In Situ Hybridization (FISH). Mutation analysis of a selection of candidate tumor suppressors did not reveal any recurrent mutations. Further studies need to be performed to determine the role of drivers in these regions.

\section{Methods}\label{methods8}
\subsection{SNP microarray data analysis}
Previously published Affymetrix Genome\hyp{}Wide Human SNP 6.0 arrays were used for SNP microarray data analysis of high\hyp{}grade osteosarcoma pretreatment biopsies (GEO accession number GSE33383) and of high\hyp{}grade osteosarcoma cell lines (GEO accession number GSE36003). Microarray data preprocessing was performed as described in Pansuriya {\it et al}.~\cite{pansuriya2011genome}. We previously described quality control and detection of aberrant regions~\cite{kuijjer2012identification}. We used recommended settings for detection of LOH and allelic imbalance in Affymetrix SNP 6.0 data---a minimum LOH length of 500 kb, a homozygous frequency threshold of 95\%, a homozygous value threshold of 0.8, and a heterozygous imbalance threshold of 0.4. High gain, gain, and losses were defined using log$_2$ ratio cut-offs of 0.6, 0.2, and -0.2, respectively, which are slightly more conservative cut-offs than recommended by the software (0.6, 0.18, and -0.18 for Affymetrix SNP 6.0 data). We selected 29 patients for which gene expression microarray data were available, so that we could perform a paired integrative analysis. Of the 19 cell lines, 12 passed our quality control (143B, HAL, HOS, IOR/MOS, IOR/OS10, IOR/OS15, IOR/SARG, KPD, MG-63, MNNG-HOS, OSA, and SAOS-2). For all cell lines, expression data was available. Aberration frequency cut-offs of 5\% (at least 2 samples out of 29) and of 15\% (at least 2 samples out of 12) were used to detect recurrent regions in biopsies and in cell lines, respectively.

\subsection{Genome\hyp{}wide gene expression microarray data analysis}
Genome\hyp{}wide gene expression Illumina Human-6 v2.0 microarray data were previously published (GEO accession number GSE33383 for biopsies, GSE42351 for osteosarcoma cell lines). Microarray data processing and quality control in the statistical language R version 2.14~\cite{r2.14.0} were performed as described previously~\cite{buddingh2011tumor}. Mesenchymal stem cells (MSCs, $n=12$) and osteoblasts ($n=3$) were used as control samples, as described by Kuijjer {\it et al}.~\cite{kuijjer2012identification} (GEO accession number GSE33383).

\subsection{Paired integrative analyses}
A detailed description of the paired integrative analysis can be found in Kuijjer {\it et al}.~\cite{kuijjer2012identification}. For this study, we generated different binary files, including all genes that showed both LOH and copy number loss (1) or not (0), and LOH and copy number gain (1) or not (0). Gene expression data were normalized against average gene expression of the corresponding probes over all control samples (MSCs or osteoblasts). Different from our previous study, we included all genes---not only the subset of genes with significant differential expression. Genes were determined to be affected when frequencies of recurrent aberrations were higher than 5\% and log fold changes $>1$. Finally, only overlapping genes between analyses with both control samples were considered of interest.

\subsection{GO term enrichment}
Gene set enrichment was performed using Bioconductor package {\it topGO}~\cite{alexa2006improved}. Lists of significantly affected genes were compared with all genes eligible for the analysis. GO terms with Fisher's exact p-values $<0.001$, as calculated by the {\it weight01} algorithm from {\it topGO}, were defined significant.

\subsection{Other statistical analyses}
Comparisons between the number of genes with both LOH and loss, and with both LOH and gain were performed using Pearson's chi-square test. All p-values were below $0.001$.

\subsection{Primer design}
Primers for PCR amplifications were designed with a universal M13 tail in order to be able to use one set of universal primers for all sequencing reactions (M13 tail forward 5'-TGTAAAACGACGGCCAGT-3' and reverse 3'-CAGGAAACAGCTATGACC-5'). In order to first validate the LOH detected with the SNP arrays, we selected Affymetrix SNP probes for {\it XRCC6BP1}, {\it RASD1}, and {\it LLGL1} according to the population frequencies of the specific SNPs. Population frequency data from Affymetrix validation studies (\url{www.Affymetrix.com}) were assessed in order to select probes with frequencies close to an even distribution (50\%/50\%). We determined the number of SNPs to be evaluated for each gene by minimizing the chance of false positive homozygosity to less than 5\%. Primers were designed up- and downstream of the SNPs using Primer 3 (\url{www.Primer3.com}) and the UCSC genome browser (\url{www.genome.ucsc.edu}). Primer sequences can be found in Additional Table~\ref{atab8.1}. For mutation analysis primers were designed for the exons of {\it XRCCBP1}, {\it PLEKHO1}, and {\it TCC19} using Primer 3 (\url{www.Primer3.com}) and the UCSC genome browser (\url{www.genome.ucsc.edu}) (Additional Table~\ref{atab8.2}).

\subsection{Sanger sequencing}
The procedure for PCR amplification is described in Rozeman {\it et al}.~\cite{rozeman2005absence}. The following PCR protocol was used: 5min at 95$^\circ$C, 3 cycles of 10sec at 95$^\circ$C and 10sec of 60$^\circ$C, followed by 10sec of 72$^\circ$C. Sequencing was performed at Macrogen (Macrogen Europe, Amsterdam, the Netherlands), Baseclear (Leiden, the Netherlands), and the Leiden Genome Technology Center (Leiden, the Netherlands). Sequences were analyzed with the Mutation Surveyor software, Softgenics (State College, PA) and Chromas software (Technelysium Pty Ltd, Helensvale, Australia).

\subsection{Fluorescent in situ hybridization (FISH)}
Metaphase preparations of osteosarcoma cell lines KPD and SAOS-2 and of a control cell line were obtained using colcemid as in Pajor {\it et al}.~\cite{pajor1998combined}. The BAC probe for {\it XRCC6BP1} (BAC/Fosmid ID RP1160O7) was ordered from the BacPac Resource Centre at Children's Hospital Oakland Research Institute (Oakland, CA) and was labeled with biotin\hyp{}16-2'\hyp{}deoxyuridine\hyp{}5'\hyp{}triphosphate (Bio-16-dUTP) using a Nick translation method. The centromere probe for chromosome 12~\cite{pajor1998combined} was labeled with digoxigenin\hyp{}11-dUTP. For immunodetection, the following antibodies were used: streptavidin\hyp{}Texas Red ($1:100$), mouse\hyp{}anti\hyp{}digoxin ($1:1,000$), goat\hyp{}anti\hyp{}streptavidin\hyp{}bio ($1:100$), rabbit\hyp{}anti\hyp{}mouse\hyp{}FITC ($1:1,000$), and goat\hyp{}anti\hyp{}rabbit\hyp{}FITC ($1:100$). FISH was scored by counting red and green probes in 50 metaphase and 50 interphase nuclei per cell line.

\section{Results}\label{results8}
\subsection{LOH and allelic imbalance in osteosarcoma biopsies}
We set out to determine recurrent LOH in high\hyp{}grade osteosarcoma. From SNP data analysis, we could demonstrate that LOH and allelic imbalance was detected less frequently than CN gains and losses (Figure~\ref{fig8.1}).
%
\begin{figure}[htbp]
	\centering
%	\includegraphics[width=1.0\textwidth]{figs08/fig1bw.pdf}	% OBS! print version bw
	\includegraphics[width=1.0\textwidth]{figs08/fig1rgb.pdf}	% OBS! pdf version rgb
%	\caption{This figure shows the distribution of frequencies of LOH (black, left of chromosomes) and allelic imbalance (black, right of chromosomes) on a background of frequencies of copy number losses (gray, left of chromosomes) and gains (gray, right of chromosomes) for the 29 osteosarcoma biopsies.} % OBS! for print version bw
	\caption{This figure shows the distribution of frequencies of LOH (blue) and allelic imbalance (purple) on a background of frequencies of copy number gains (green) and losses (red) for the 29 osteosarcoma biopsies.}	%%% OBS! for pdf version different text rgb
	\label{fig8.1}
\end{figure}
%
Recurrent LOH (frequency $>5\%$) was detected for 9.4\% of all analyzed genes. The highest percentage of recurrent LOH detected in this dataset was 35\%. Recurrent allelic imbalance was seen in 0.14\% (23 genes), while recurrent total allelic loss was detected in 0.16\% (25 genes, including {\it TP53} and {\it RB1}) of all analyzed genes, with highest recurrent frequencies of 7\% and 55\%, respectively.

\subsection{LOH is often accompanied by copy number gains }
On average, 0.05\% of all genes show LOH and loss of DNA. A chi-square test demonstrated that LOH and loss occurred less frequent than expected (log odds $-1.55$, p-value $<0.0001$). Also copy neutral LOH occurred less frequent than expected (log odds $-1.75$, p-value $<0.0001$). Based on a comparison of the allelic ratio overview of the genome with CN gains and losses, LOH appears to often cooccur with gain at the other allele (Figure~\ref{fig8.1}). On average, 1.10\% of all genes show LOH and gain at the other allele in the same sample. Chi-square test verified that LOH accompanied by copy number gains indeed occurred more frequently than expected (log odds $2.44$, p-value $<0.0001$).

\subsection{Integration of LOH, gain, and differential expression}
In order to identify genes present in regions of LOH and gain (LOH-gain) which also were differentially expressed and hence may have a tumor driving function, we performed paired integrative analyses of LOH-gain and expression in the dataset of osteosarcoma pretreatment biopsies. Paired integrative analysis of these biopsies as compared with MSCs resulted in 148 up- and 17 downregulated genes in combination with LOH-gain, while the analysis where osteoblasts were used as a control resulted in 135 up- and 9 downregulated genes in combination with LOH-gain. Of these affected genes, 114 upregulated and 5 downregulated genes overlapped between both analyses. 

The same approach was taken for the analysis of high\hyp{}grade osteosarcoma cell line data. This analysis returned 137 up- and 44 downregulated genes in combination with LOH-gain in osteosarcoma compared with MSCs, and 134 up- and 35 genes when compared with osteoblasts. In total, 97 upregulated genes and 20 downregulated genes overlapped. Of the 119 genes being over- and underexpressed together with LOH-gain in osteosarcoma biopsies as compared with MSCs and osteoblasts, 29 showed recurrent LOH-gain in combination with significant differential overexpression in both analyses of osteosarcoma cell lines (Figure~\ref{fig8.2}A).
%
\begin{figure}[htbp]
  \centering
%    \includegraphics[width=1\textwidth]{figs08/fig2bw.pdf}		% OBS! print version bw
   \includegraphics[width=1\textwidth]{figs08/fig2rgb.pdf}	% OBS! pdf version rgb
    \hfill
     \caption{Depicted are the numbers of returned genes from the paired integrative analysis on LOH with gain and {\it A}, upregulation or {\it B}, downregulation, as compared with MSCs or osteoblasts (OB), in both osteosarcoma biopsies and cell lines.}
     \label{fig8.2}
\end{figure}
%
No genes showing LOH and CN gain together with downregulation overlapped (Figure~\ref{fig8.2}B).

\subsection{Involvement of cell cycle pathways}
GO term enrichment was performed on the 29 affected genes obtained with the analysis of biopsies and cell lines. This resulted in three significant GO terms---S-phase of mitotic cell cycle (GO:0000084), double\hyp{}strand break repair via non\hyp{}homologous end\hyp{}joining (GO:0006303), and M/G1 transition of mitotic cell cycle (GO:0000216), including genes such as {\it CDK4}, {\it MCM4}, and {\it XRCC6BP1} (Table~\ref{tab8.1}). Literature review (\url{www.genecards.org}) indicated that 17/29 genes may have an oncogenic role and 2/29 a tumor suppressive role.
%
	\begin{table}[htbp]
		\centering
		\small
		\begin{tabular}[c]{|l p{1.85in} rrrr p{1.1in}|} % OBS! 4.5in is entire page length % 4.2in gives same length as table 3.2
			\hline
			GO ID & Term & Ann & Sign & Exp & weight01 & Genes affected\tabularnewline
			\hline
			0000084 & S phase of mitotic cell cycle & 114 & 4 & $0.26$ & $0.00013$ & {\it CDK4}, {\it MCM4}, {\it PSMB4}, {\it PSMD4}\\
			0006303 & double\hyp{}strand break repair via NHEJ & 10 & 2 & $0.02$ & $0.00023$ & {\it PRKDC}, {\it XRCC6BP1}\\
			0000216 & M/G1 transition of mitotic cell cycle & 67 & 3 & $0.16$ & $0.00049$ & {\it MCM4}, {\it PSMB4}, {\it PSMD4}\\
			\hline
		\end{tabular}
		\caption{GO terms significantly enriched for genes with LOH-gain and overexpression. GO ID: GO-term ID, Term: GO term, Ann: number of annotated genes, Sig: number of significant genes, Exp: number of genes expected to be significant, weight01: p-value obtained with {\it weight01} algorithm, NHEJ: non\hyp{}homologous end\hyp{}joining.}
		\label{tab8.1}
	\end{table}
%

\subsection{Validation by Sanger sequencing}
Tumor suppressor genes which are present in a region of LOH and gain may be particularly interesting, because a tumor cell could select for a mutant allele with a partial dominant negative or altered function. We therefore set out to identify mutations in tumor suppressor genes present in these regions of LOH and gain. Yet, false positive regions of LOH may be returned from SNP data analysis in regions of high CN amplification as a technical artifact. Hence, we validated regions of LOH and gain by Sanger sequencing. For validation, we selected the candidate tumor suppressor gene {\it XRCC6BP1}. In addition, we chose to validate the gene {\it LLGL1}, which showed LOH and gain in the SNP data of cell line IOR/OS15 and allelic imbalance and gain in cell line IOR/SARG. We also validated {\it RASD1}, which showed recurrent LOH and gain, but downregulation when compared with osteoblasts. We validated LOH in the cell lines which showed LOH and gain in the particular genes in the SNP data analysis. The selected genes harbored homozygous as well as heterozygous SNPs when analyzed on normal blood donor DNA. Sequencing of the selected SNPs in and around {\it XRCC6BP1} in the cell lines KPD and SAOS-2, and of the SNPs in and around {\it RASD1} in cell lines 143B, HOS, and IOR/OS15 and the diagnostic biopsy L2613 revealed only homozygous SNPs. For {\it LLGL1}, we detected homozygosity in cell line IOR/OS15, but heterozygosity in cell line IOR/SARG, which was detected as allelic imbalance in SNP microarray data analysis. The probabilities for obtaining false positive results in the Sanger sequencing validation were $0.001$, $0.037$, and $0.019$ for {\it XRCC6BP1}, {\it RASD1}, and {\it LLGL1}, respectively. These findings therefore confirm the detection of homozygosity by the SNP microarray data analysis.

\subsection{Nature of the amplification}
The copy number state of the {\it XRCC6BP1} locus in the affected cell lines was analyzed by FISH (Figure~\ref{fig8.3}).
%
\begin{figure}[htbp]
  \centering
  \begin{minipage}[b]{0.50\linewidth}
    \includegraphics[width=1\textwidth]{figs08/fig3rgb.pdf}		% print version pdf version both rgb
  \end{minipage}
    \hfill
  \begin{minipage}[b]{0.46\linewidth}
     \caption{FISH depicting {\it A}, a control metaphase cell, {\it B}, a metaphase cell of KPD with arrows indicating examples of HSRs, and {\it C}, a metaphase cell of SAOS-2 with arrows indicating {\it XRCC6BP1} alleles not located on chromosome 12. Green: probe for the centromere of chromosome 12, red: BAC probe for {\it XRCC6BP1}.}
     \label{fig8.3}
     \end{minipage}
\end{figure}
%
For KPD, 42/50 metaphase cells had four copies of chromosome 12, of which two were negative for the {\it XRCC6BP1} probe. In addition, these cells showed more than ten homogeneous staining regions (HSR) for {\it XRCC6BP1}. 8/50 cells had only two copies of chromosome 12, of which one harbored {\it XRCC6BP1}, and showed 5--10 HSRs per cell. In 50/50 SAOS-2 metaphases, in contrast, we detected two copies of chromosome 12 harboring the {\it XRCC6BP1} locus, and two additional chromosomes without a chromosome 12 centromere, but with signals for {\it XRCC6BP1}. These results do not prove homozygosity of the locus, but do illustrate the different levels of amplification in the different cell lines. These amplifications identified with FISH corresponded to results from the SNP data analysis, as we detected a high gain in KPD and a normal gain in SAOS-2.

\subsection{Mutation analysis of selected genes}
We performed Sanger sequencing for the entire coding region of {\it XRCC6BP1}, one of the two candidate tumor suppressor genes with recurrent LOH-gain and overexpression, but did not identify any mutation in this gene, indicating that the wild\hyp{}type allele is amplified in the osteosarcoma cell lines KPD and SAOS-2, which harbor the region of LOH and gain. We therefore expanded our list of candidate genes with tumor suppressor genes showing upregulation when compared with MSCs only (2/11 genes may have a tumor suppressive function), and when compared with osteoblasts only (3/5 genes may have a tumor suppressive function). The coding region of {\it PLEKHO1}, which showed overexpression only when compared with osteoblasts, was sequenced and analyzed for mutations in cell lines which showed this aberration---HOS, IOR/MOS, and IOR/SARG. No mutations could be identified, but we were not able to sequence the first exon of this gene. Mutation analysis of {\it TTC19}, which showed overexpression compared to MSCs, revealed a point mutation in exon 7 in the cell line IOR/OS10, but no mutations in the other affected cell lines (143B, HOS, IOR/OS15, IOR/SARG) and in two additionally analyzed diagnostic biopsies showing LOH and gain (L3437, L3469) were detected, indicating that the mutation in {\it TTC19} is not recurrent in osteosarcoma. The mutation, R274G, however, was predicted to be possibly damaging with a score of $0.752$ (sensitivity of $0.85$, specificity of $0.92$) by PolyPhen-2~\cite{adzhubei2010method}.

\section{Discussion}\label{discussion8}
By analysis of high\hyp{}grade osteosarcoma high\hyp{}throughput genomic data, we demonstrated that, in osteosarcoma, recurrent LOH happens less frequently than copy number aberrations such as gains and losses. Interestingly, we found that LOH was more often accompanied by copy number gains than expected by chance. Tumor suppressor genes showing overexpression in recurrent regions of gain and LOH may be drivers if these genes harbor mutations leading to a gain\hyp{}of\hyp{}function or partial dominant negative function. We thus screened for mutations in candidate tumor suppressor genes in these regions. Of the 29 genes that were recurrently affected in all comparisons, two may have a possible tumor suppressive role---{\it XRCC6BP1} and {\it PRKDC}. We performed mutation analysis for {\it XRCC6BP1}, or XRCC6 binding protein 1 / Ku70 binding protein 3, which is involved in non\hyp{}homologous end\hyp{}joining (NHEJ) of DNA double strand breaks~\cite{yang2011genetic}. {\it XRCC6BP1} has been reported to be amplified and overexpressed in an alternatively spliced isoform in human gliomas, which may interfere with the normal function of the DNA-PK complex~\cite{fischer2001kub3}. In regions of LOH-gain in osteosarcoma cell lines, {\it XRCC6BP1} did not harbor any recurrent mutations. We did not screen for mutations in {\it PRKDC}, or DNA-PK catalytic subunit, a Ser/Thr kinase which also plays a role in NHEJ~\cite{chan2002autophosphorylation}, because of its size (over 13kb). We did analyze two additional genes, {\it PLEKHO1} and {\it TTC19}, which showed recurrent LOH and gain, but which were overexpressed only in comparison with osteoblasts, or only in comparison with MSCs, respectively. No mutations in {\it PLEKHO1} (pleckstrin homology domain containing, family O member 1), a gene with a role in regulation of the actin cytoskeleton~\cite{canton2005pleckstrin} with an inhibitory effect on PI3K/Akt signaling~\cite{tokuda2007casein}, were detected, but we were unable to sequence exon 1 of this gene. We did detect a mutation in {\it TTC19} (tetratricopeptide repeat domain 19), of which the protein is reported to be involved in oxidative phosphorylation in mitochondria~\cite{ghezzi2011mutations}, and which may play a role in cytokinesis as well~\cite{sagona2010ptdins}. The mutation detected results in a arginine to glycine substitution at codon 274 of the TTC19 protein and was predicted as possibly damaging. However, the mutation was only found in 1/6 samples analyzed, and is therefore not recurrent. A shortcoming of these mutation analyses is that only exons were sequenced, and mutations in {\it e.g.} intronic regions may also affect the protein function, for example by affecting alternative splicing. We thus cannot exclude that these genes harbor any recurrent mutations.

A weakness of this study is that we did not have paired control samples available for the SNP microarray data analysis. Using paired control samples helps in avoiding the false\hyp{}positive regions of LOH which are detected when comparing tumor samples with an independent set of controls, because of patient\hyp{}specific inherited segments of homozygosity~\cite{heinrichs2010snp}. A second limitation in the analysis of these data is that high\hyp{}grade osteosarcoma is extremely genomically unstable. Copy number aberrations that are returned by data analysis are relative changes against the background copy number state of the samples, and therefore true copy number states are not uncovered~\cite{gardina2008ploidy}. Regions of copy number gain could, {\it e.g.} in a tetraploid background, consequently represent even higher gains than what is expected based on assumption of a near\hyp{}diploid background. In the case of an unbalanced gain, the detection of the other allele may be low, which can lead to the detection of a false\hyp{}positive region of LOH. Because of these considerations, we validated a selection of genes in regions of recurrent LOH-gain by Sanger sequencing. All regions we tested were indeed detected as homozygous for the all SNPs. However, in a highly amplified region, Sanger sequencing may also not be sensitive enough to detect the sequence of an allele of which only one copy is present. We therefore cannot conclude that these regions are actually homozygous, although for regions of low amplification this is probably the case.

FISH analysis of {\it XRCC6BP1} revealed four copies of chromosome 12 in both cell lines, of which two harbored the {\it XRCC6BP1} locus and two not. In addition, cell line KPD showed numerous homozygous staining regions. These results could be an indication of LOH and amplification of the other allele with intrachromosomal HSRs, but allele\hyp{}specific FISH should be performed to clarify whether these are true cases of LOH. Nevertheless, FISH validated the copy number states that were detected by SNP microarray data analysis, as the gain detected in SAOS-2 was represented by four copies of the gene, and the high gain detected in KPD was represented by $>10$ copies of the {\it XRCC6BP1} locus in FISH, thereby confirming the detection algorithm for copy number gain we used was appropriate for these samples. Chromothripsis, or chromosome scattering, is reported to be present in bone tumors with a frequency of at least 25\%~\cite{stephens2011massive}. In chromothripsis, part of the genome generally oscillates between two states, with the higher copy number state retaining heterozygosity and the lower copy number state showing LOH. The regions of LOH-gain we detected in the osteosarcoma SNP data could represent chromothripsis, since it would be possible that small oscillating regions were not detected due to the density of the probes targeting SNPs on the microarray and the detection algorithm. Such regions may be returned as larger regions of CN gain harboring LOH. A characteristic of chromothripsis is the presence of double minutes---small circular extra-chromosomal DNA fragments, which may be highly amplified in the tumor cell, and which frequently harbor oncogenes in cancer cells~\cite{forment2012chromothripsis}. The HSRs which were detected in cell line KPD may represent chromosomal integration of double minutes, especially considering 17/29 genes detected by our analysis are possible oncogenes. It would thus be interesting to characterize whether the regions that we detected are recurrent HSRs or double minutes, and what the function of these oncogenes is in tumorigenesis of osteosarcoma.

%%% references

\begin{small}
\begin{singlespace}
\bibliographystyle{unsrtnatshort}		% sorted as referenced, was unsrtnat, but unsrtnatshort gives shorter output
\bibliography{biblio}
\end{singlespace}
\end{small}

%%% appendix
% supplemental table 1 and 2 included as additional table 1 and 2
\begin{subappendices}
	\newpage
	\setcounter{table}{0}
	\section{Additional Tables}
		\renewcommand{\tablename}{Additional Table}
		%
\begin{center}
\begin{singlespacing}
   \hvFloat[
    nonFloat=true,
    capPos=r, % caption to the right
    capWidth=w, % caption has size of object
    capAngle=90,
    objectAngle=90,
]{table}{\small
    \begin{tabular}{|ccccc|}
	\hline
    Gene & SNP ID & Population frequencies (\%) & Forward primer & Reverse primer\\
    \hline
		{\it XRCC6BP1} & SNP\_A\_8453712 & 45/55 & GCAAGAACCTGTCTCTGAAAAA & GCTGCATAGTATTCCGTGGTG\tabularnewline
		& SNP\_A\_4194973 & 45/55 & AGACAGTGGTGCAGCTGAGA & GACCACACGGGCTGTTTTAT\tabularnewline
		& SNP\_A\_1846441 & 50/50 & AACCCCAGAGAAAAACACCA & GTCCCAAGATGCATTGCTTT\tabularnewline
		& SNP\_A\_2291513 & 45/55 & TCTGGCAGTAATGTGGTGGT & ATTGCTGCTAAGCCAAGGAC\tabularnewline
		& SNP\_A\_2063148 & 45/55 & TCTGAGCCTAAAACCCAGGA & GCTGGGCAGCTGACTCTAAT\tabularnewline
		& SNP\_A\_8461882 & 41/59 & AAGCAGGGAAACAGGCTACC & CACACCACAGCTGCAGAATC\tabularnewline
		& SNP\_A\_8379343 & 48/52 & GGGCTGATGTGGTCTAGGAG & CCCTGCACAGATGTCTACCC\tabularnewline
		& SNP\_A\_1818707 & 50/50 & AGGTGGGAATATGAAGTTCAGTG & GAGCCACAAGGGTGAGAAGT\tabularnewline
		& SNP\_A\_8352439 & 46/54 & TGAATCCTGCCTTTCCCATA & CATCCATATAGTTGCTGAAATGC\tabularnewline
		& SNP\_A\_8552537 & 47/53 & CCATGAACCTTTTGGAAGGA & CTTCATGATGATGGAAGCTCTG\tabularnewline
		{\it RASD1} & SNP\_A\_2173899 & 25/75 & CCTCCCTCCTGCTTCTTCTT & TGATCAGTGACAACCATCACA\tabularnewline
		& SNP\_A\_8383260 & 27/73 & CAAGTGTCCATTGCCTGATG & GTGTCCGGCTTCTCTCACTC\tabularnewline
		& SNP\_A\_8307109 & 27/74 & TTGATGCCATCTCTCAGCAC & AGTGTCCCCAGCAGTGTCTT\tabularnewline
		& SNP\_A\_1785268 & 44/56 & TTCCAAGGAGCTGGAAGTTG & AGGCACCTTATCCCTTCTCC\tabularnewline
		& SNP\_A\_2186302 & 10/90 & GCCTTGGTTGTCTCATTTTTG & GCCCTTACCAGTCCATTCCT\tabularnewline
		& SNP\_A\_8609757 & 25/75 & GATTTGCAAGGTGTGAGACG & GTGGGAAATTTCAACCCAGA\tabularnewline
		& SNP\_A\_1953953 & 20/80 & GTTAGTGGCCCACCCACTTT &  CCAAATGAAGCCAGGGTCTA\tabularnewline
		{\it LLGL1} & SNP\_A\_8649983 & 30/70 & AAGTTTGGCCTGAAGCTGTG & TCAGCTCCGTGTGTCTATGG\tabularnewline
		& SNP\_A\_8287520 & 40/60 & GGGAAGGTCCTGGATTTGTT & CAGGCATGTGAGGTATGTGG \tabularnewline
		& SNP\_A\_8651023 & 20/80 & CTGTGCATAGGCAGGGTTG & CTGGGTTGGTACTCCCCTTT\tabularnewline
		& SNP\_A\_8688092 & 40/60 & " & " \tabularnewline
		& SNP\_A\_1919461 & 15/85 & TCCCAAAGTGCTGGGATTAC & GCAAGGAAATGGCTGTGGTA\tabularnewline
		& SNP\_A\_8330336 & 40/60 & ATCATACCACTGCCCTCCAG & CCAGACTCATGGATGCAGAA\tabularnewline
		& SNP\_A\_2043066 & 20/80 & " & "\tabularnewline
		& SNP\_A\_4253141 & 20/80 & " & "\tabularnewline
    \hline
    \end{tabular}%
}%
[Caption]{Primer sequences to validate LOH.}{atab8.1}
\end{singlespacing}
\end{center}
%
\begin{landscape}
%	\begin{table}[htbp]
		\centering
		\small
		\begin{singlespacing}
		\begin{longtable}[c]{|cccc|}
		\hline
		Gene & Exon & Forward primer & Reverse primer\tabularnewline
		\hline
		{\it PLEKHO1} & 2 & TGAAAACCTTTCCGAAGTGG & GCAGATGAGATGGGGGTAGA\tabularnewline
		& 3 & CCTCTCCTCAGGCTTCCTCT & TGCCTGGAAAGAAGGAAATG\tabularnewline
		& 4 & GGTGAGGCACCACCTCTAAA & TGGTGGAGGAGCGAGTAAAC\tabularnewline
		& 5 & TGTCCAGTAAATCCCCTTGC & CCTAATGGGCGCTGAATAAA\tabularnewline
		& 6.1 & CTGATGACAGGTTCCCCACT & AAGGTCGGGAGAGACTGCTT\tabularnewline
		& 6.2 & CTGAGAGCTTTCGGGTTGAC & TCCAATTCGATGATGCCTCT\tabularnewline
		& 6.3 & AGGTTCAGGGACTGGGAGAT & TACGAGGGGCATATGGAAAG\tabularnewline
		{\it XRCC6BP1} & 1 & CGGGAGGGAGGTTACCTTT & CAGACCCATTCTGTGGAACC\tabularnewline
		& 2 & CGCCTCAACCTCCTAAAGTG & GTTTTCAGCAGCCAGACCTC\tabularnewline
		& 3 & TATGGTGGAGGGTCTCCTGA & GCATGTGGAAGATGCTCAAA\tabularnewline
		& 4 & TCACTGAACTTCTTTTATTTTGGTG & CCGAAATTCAAGACTAAGGTAGAA\tabularnewline
		& 5 & GAGCATGAGCGTTTATTTCTTTT & ACACTCTGGAGGGGAAGTGA\tabularnewline
		& 6.1 & GAGCCTCATACTTTTCCTTCTTTTT & TGCCTTGGAGTTTAAAGCAG\tabularnewline
		& 6.2 & AAACAGAGAAGACTGTGATTCTAGC & CAACAGCTCAATAAGTATCCTACAATG\tabularnewline
		{\it TTC19} & 1.1 & CAACTGCGCTGTACCGTAAAT & CAGGATCCTCCACAGGTAGG\tabularnewline
		& 1.2 & GGCAACACTACGGCCATC & AGCTCAGGAGCCGGAACAT\tabularnewline
		& 1.3 & AGGGCGAGACGGAGTGAC & GAAGGGGCTCTGAGGTCAT\tabularnewline
		& 2 & GATGACCTCAGAGCCCCTTC & TAGAGTCGGAAAAGCCTGGA\tabularnewline
		& 3 & CAGTTGGGATGTACAGTTGCAT & CCAACCTTCCTCATCAGTGG\tabularnewline
		& 4 & TTGAGGGTGAAAGCAAAAGG & TCCCTTGAAGCTACTCCTTCAT\tabularnewline
		& 5 & GGGGCCCAATTAAAAGAAAA & CTCCACCTTTCCTGACCAAA\tabularnewline
		& 6 & CCACCGTCAGTCTGGAAGTT & GACACCCAATTTCTGGGAGA\tabularnewline
		& 7 & CAACAAGAGCGAAACTCCATC & GAAAAGGCAATGCCCAGATA\tabularnewline
		& 8 & TGGGTCCTGGTAACAACCAT & GGACCATCTGCTGATCCTGT\tabularnewline
		& 9 & TTGGATGCACTCCACATTAAA & CTTGCCCTCCCTACATACCA\tabularnewline
		& 10 & CACCAGCTTGTCGCTTCATA & ATGCCCAGAAAACTCCAGTG\tabularnewline
		\hline
		\caption{Primer sequences for mutation analysis.}
		\label{atab8.2}
		\end{longtable}
%	\end{table}
		\end{singlespacing}
\end{landscape}
%
\end{subappendices}

%\end{document}
% Marieke Kuijjer
% 2013-02-15
% chapter 09

	%\documentclass[12pt,b5paper]{book}
	%\setcounter{secnumdepth}{0}
	%\setcounter{tocdepth}{1}
	%\usepackage[hidelinks]{hyperref}

%\begin{document}

%%% title page

\chapter{Concluding remarks and future perspectives}\label{ch9}
\thispagestyle{empty}				%%% to remove page number from first page of chapter, must be placed after calling the chapter

%\vfill

\newpage
%%% main document
%%% first paragraph should not be an indent, so i set noident{} to this paragraph
\noindent{High-grade osteosarcoma is a primary bone tumor with complex genetic alterations, for which targeted therapy is lacking. The aim of this thesis was to use high\hyp{}throughput molecular data analysis of high\hyp{}grade osteosarcoma specimens and model systems, in order to learn more on osteosarcomagenesis and to find possible ways to inhibit this process. {\bf Chapter~\ref{ch1}} and {\bf Chapter~\ref{ch2}} give an introduction and literature review on microarray data analysis of high\hyp{}grade osteosarcoma.}

In {\bf Chapter~\ref{ch3}} we provide a rationale for the use of model systems for osteosarcoma. It describes differential expression for the clinical parameters sex, tumor location (femur, humerus, fibula/tibia), response to preoperative chemotherapy (poor responders, or Huvos grade 1--2, versus good responders, or Huvos grade 3--4), and histological subtype (osteoblastic, chondroblastic, fibroblastic osteosarcoma). Importantly, as we describe in a previous study performed by our group~\cite{cleton2009profiling}, no significantly differentially expressed genes were detected between poor and good responders to preoperative chemotherapy, even though a substantial amount of tumor samples was analyzed (see also Chapter~\ref{ch2}). Several publications do report differences between poor and good responders, but used relatively small sample sizes, and did not apply correction for multiple testing. An analysis of gene expression profiles of the three described histological subtypes showed that these differed significantly. Sets of fibroblastic- and of chondroblastic osteosarcoma\hyp{}specific genes were determined, and were enriched in genes with a role in cellular growth and proliferation and in the chondroid extracellular matrix, respectively. Using nearest shrunken centroids classification, an expression signature consisting of 24 probes that could predict for histological subtype was generated. This profile was validated on an independent dataset of osteosarcoma and control samples. Interestingly, this prediction profile was able to classify histological subtypes of the primary tumor from which the tested osteosarcoma xenografts and cell lines were derived, even though such material often lacks extracellular matrix. This implicates that the mRNA expression profiles of these model systems are representative for the primary tumor, and favor the use of osteosarcoma xenografts and cell lines in studying osteosarcoma biology. This is of particular importance given the rarity of this tumor as well as the difficulties to obtain adequate tissue.

%
\section{Targets for treatment of high-grade osteosarcoma}\label{targets9}
Different comparative analyses of the various types of data that were available led to the discovery of a number of particular ways to target this tumor. In {\bf Chapter~\ref{ch4}}, we compared patients who developed metastases within five years with patients who did not develop metastases within this time frame. The list of significantly differentially expressed genes was enriched in macrophage\hyp{}associated genes expressed by infiltrating cells (approximately 50\% of all genes), which were all overexpressed in patients who did not develop metastases. Tumor\hyp{}associated macrophages (TAM) of both the M1- (antitumor) and M2 (protumor)\hyp{}type were quantified with IHC in additional cohorts. The total count of M1- and M2-type macrophages was significantly correlated with a better overall survival. This is in contrast with most epithelial tumor types, which often show a correlation between (M2-type) infiltrating macrophages and poor survival. However, mesenchymal tumor cells may not need the guidance of the infiltrate to metastasize, and tumor\hyp{}associated macrophages may have a more antitumorigenic role in osteosarcoma~\cite{cleton2012immunotherapy}. Moreover, macrophages are plastic cells, and it could be that M2-type macrophages polarize towards M1-type macrophages after chemotherapy due to the release of danger signals by the dying tumor cells. The results of Chapter~\ref{ch4} provide a rationale for adjuvant treatment of high\hyp{}grade osteosarcoma patients with macrophage\hyp{}activating and recruiting agents, such as liposomal muramyl tripeptide phosphatidylethanolamine (L-MTP-PE). This drug has been previously shown to increase overall survival in canine and in human osteosarcoma~\cite{cleton2012immunotherapy,kager2010review}, although interpretation of the results obtained from the latter study has been difficult, due to the 2x2 factorial design; standard adjuvant chemotherapy treatment plus L-MTP-PE and/or ifosfamide. This study showed that two additional drugs did not show significant interaction (p-value $=0.101$) and therefore the treatment arms were pooled. A significant difference was then found for overall survival, but not for event\hyp{}free survival (EFS). In an unpooled analysis, EFS for patients treated with L-MTP-PE and ifosfamide was significantly improved when compared with patients treated with ifosfamide alone, but EFS arms of patients with only the standard adjuvant chemotherapy and patients who received L-MTP-PE without ifosfamide were not significantly different~\cite{kager2010review}. Further testing of this drug in osteosarcoma is therefore necessary, and this will, in the near future, be initiated in a phase II study in patients with metastatic and/or relapsed osteosarcoma.

In addition to differences between high\hyp{}grade osteosarcoma tumors with different clinical features, we studied common gene expression changes between sets of tumors and control samples. In {\bf Chapter~\ref{ch5}}, mRNA expression in osteosarcoma cell lines was compared with expression in osteosarcoma progenitors. Global pathway analyses pointed to differences in mRNA expression of the IGF1R pathway. Specifically genes negatively regulating this pathway upstream the IGF1 receptor showed downregulation in osteosarcoma, of the highest degree ({\it i.e.} the highest negative fold changes in the dataset). We therefore hypothesized that this pathway can be inhibited at the receptor level, and that this may inhibit growth of these tumors. Osteosarcoma cell lines were treated with a dual kinase inhibitor OSI-906, which inhibits both the insulin receptor (IR) and IGF1R, as IR can take over downstream signaling in case IGF1R is blocked, thereby inducing resistance to single IGF1R targeting~\cite{fulzele2007disruption,garofalo2011efficacy}. Inhibition with OSI-906 resulted in an inhibition of proliferation of 3/4 osteosarcoma cells, and may therefore be a promising drug for treatment in addition to adjuvant chemotherapy. Other pathways with a role in bone development, namely canonical Wnt signaling~\cite{cai2010inactive} and TGF$\upbeta$/BMP signaling~\cite{mohseny2012activities}, have been reported to play a role in osteosarcomagenesis. Because of the role of IGF1R signaling in bone development and growth, it is not surprising that this pathway is deregulated in osteosarcoma. Notably, in a recent case--parent study, two SNPs in the growth hormone (GH)/IGF1 pathway (in {\it IGF2R} and {\it IGFALS}) were significantly associated with osteosarcoma incidence~\cite{musselman2012case}. Interestingly, one of the very few genes which were overexpressed in patients developing metastases within 5 years (Chapter~\ref{ch4}) was the growth hormone receptor (GHR), which was also frequently amplified (in 34\% of all samples). IGF1 synthesis is largely dependent on growth hormone signaling~\cite{khandwala2000effects}, and an association between osteosarcoma and height/growth has been reported. The specific roles of the GH/IGF1 axis in osteosarcoma tumor growth and metastasis remain to be elucidated. The osteosarcoma cell line panel shows a variable expression of {\it GHR}, and includes four cell lines with high expression of {\it GHR}. These cell lines could be utilized to further experimentally examine this pathway in osteosarcoma.

{\bf Chapter~\ref{ch6}} described Ser/Thr kinome profiling analysis of two osteosarcoma cell lines using a peptide microarray. Although it is not yet possible to directly infer what kinase caused differential phosphorylation of the identified peptides, by pathway analysis we detected hyperphosphorylation directly downstream of Akt, pointing to active PI3K/Akt signaling. We treated osteosarcoma cells with MK-2206, an Akt inhibitor, which inhibited proliferation of 2 out of 3 cell lines. Inhibition of the PI3K/Akt signaling pathway may therefore also be a possible target for treatment of these tumors.

The effects of IGF1R and Akt inhibitors on osteosarcoma cell proliferation should be studied further, in order to determine why these cells stop proliferating after treatment (this may for example be ascribed to induction of apoptosis). In addition, the drugs need to be tested in combination with chemotherapy, in order to check for synergy and also in order to rule out toxicity of a combined treatment, as targeted treatment using a single drug against these signal transduction pathways will probably not be able eliminate all osteosarcoma tumor cells, and patients may develop resistance to targeted treatment. In addition, if targeted treatments will be used to treat osteosarcoma patients, the genomic and mutational status of associated pathway players should be determined, because patients with downstream aberrations may be insensitive to treatment, as was shown in Chapter~\ref{ch6} for the 143B cell line, which is insensitive to Akt inhibitor MK-2206, most probably due to its oncogenic transformation of {\it KRAS}.

%
\section{Integrative analysis and genomic instability}\label{integrative9}
A main finding of this thesis regards genomic instability, which appears to be affected in all data types studied---the genome, transcriptome, and kinome. In {\bf Chapter~\ref{ch6}}, an integrative analysis shows that pathways with a role in genomic stability are enriched for overexpressed genes, as well as for hyperphosphorylation of peptides, implying that not only gene expression, but also kinase activity is deregulated in these pathways.

In addition to complementing two different data types as was performed in Chapter~\ref{ch6}, integrative analysis can also be performed by applying intersections of both data types. This is useful in the analysis of copy number and gene expression data, because differently from kinase activity and transcript expression, the influence of DNA copy number of a specific gene on the mRNA expression levels of that particular gene is more direct than the influence of kinase activity on gene expression, since kinases usually act quite far upstream of transcription factors in a specific pathway. Different methods for performing integrative analyses on gene expression and copy number data exist, as is described in {\bf Chapter~\ref{ch7}} of this thesis. We tested the performance of two of such methods---a nonpaired and a paired analysis---on the high\hyp{}grade osteosarcoma dataset. In the nonpaired analysis, genes showing significant differential expression as compared with the control samples were returned when present in recurrent regions with copy number alterations that contained a higher number of significantly differentially expressed than expected by chance. For the paired analysis, a new approach was developed in statistical language R. This method determined cooccurrence of copy number changes and significant differential expression. A comparison of both methods on osteosarcoma data illustrated that the paired analysis returned more genes with biological relevance over a larger number of regions, even though fewer samples were used for this analysis, since complete RNA expression--DNA copy number pairs were available for 29/32 cases. By using a conservative approach and by combining the results of different paired analyses, we identified 31 candidate osteosarcoma drivers with high frequency of occurrence and significant differences in expression. While most of these genes were not yet reported in osteosarcoma, more than two\hyp{}thirds of the genes have been described to play a role in cancer. A large number of our candidate genes had a role in cell cycle regulation, stressing the possible role of genomic instability in driving osteosarcoma progression. This was further evaluated by calculating genomic instability scores, which showed that higher genomic instability correlated with poorer metastasis\hyp{}free survival. In addition, a negative correlation between the total amount of copy number aberrations and metastasis\hyp{}free survival was detected. We determined correlation with metastasis\hyp{}free survival, and not overall survival, because of the limited follow\hyp{}up available. Metastasis\hyp{}free survival, however, highly correlates with overall survival, as only a small percentage of patients with resectable metastases survive.

Some issues can be raised with regard to the selection of the method we applied to identify candidate driver genes. For determining recurrent copy number aberrations, we used a cut-off for frequency. This will result in the detection of mostly broad events ({\it e.g.} the amplification of an entire chromosome arm). Focal events may be detected as well, but this method of analysis does not directly pinpoint specific targets (alterations with selective benefits) of recurrent copy number aberrations. Focal events are most often determined by the identification of the minimal common region of overlap of a copy number aberration, but this approach is prone to misidentification of the driver gene, especially when the recurrence frequency of the driver event is low~\cite{mermel2011gistic2}, which is one of the reasons why we did not take this approach. Even though broad events do have a higher prevalence in most cancer types (low\hyp{}level aberrations affecting a chromosome arm or an entire chromosome), determining focal alterations may have great power to identify important genes in cancer~\cite{beroukhim2010landscape}. Importantly to note is also that both types of events may have different biological consequences. A method exist which can distinguish broad and focal events from each other and from background ({\it i.e.} passenger) events (GISTIC, Beroukhim {\it et al}.~\cite{beroukhim2007assessing}). It would be therefore interesting to perform this method to the osteosarcoma data set, in order to also identify significant recurrent focal aberrations. Results obtained with this analysis could return less frequently occurring aberrations, which are specifically selected for in a subset of the tumors. Using our integrative method, which does not filter out passenger copy number alterations statistically (as is done in GISTIC), but which integrates copy number with expression data, we were able to identify those significantly differentially expressed genes of which a large part (at least 35\%) of all tumors could be explained by an underlying copy number aberration. This approach gives us more information on osteosarcomagenesis in general ({\it i.e.} genes are affected in a high percentage of osteosarcoma samples). In addition to our list of new candidate osteosarcoma drivers, determining significant focal copy number events could provide us with some additional possible targets for treatment of a subset of osteosarcoma patients.

Candidate drivers need to be validated in an experimental setting. For the detected amplified and overexpressed genes this can be done by shRNA studies, but the effects of deleted tumor suppressors are more challenging to validate, especially because affecting a single gene will probably not be sufficient to stop tumor growth in cells with large amounts of aberrations. Because the majority of osteosarcomas do not have a known benign or less malignant precursor lesion, it is difficult to study tumor evolution, and to discriminate between early and late events in tumorigenesis of osteosarcoma. The detection of early drivers is important---it will reveal the first steps a normal cell takes in order to become tumorigenic, and these findings may be used in for example diagnostics. Genomic instability appears to play a major role in osteosarcoma, and in at least 25\%~\cite{stephens2011massive}, but probably in total approximately 50\% of all high\hyp{}grade osteosarcomas, this can be explained by chromothripsis. However, how chromothripsis exactly occurs, and what happens in the other half of osteosarcomas is yet unknown. By using the right model systems, these mechanisms can be studied. This is for example ongoing in studies which make use of the injection of different passages of transformed MSCs in mice and zebrafish~\cite{mohseny2012osteosarcomamodels,mohseny2012osteosarcomazebrafish} upon which their tumorigenicity can be assessed. Furthermore, conditional transgenic mouse models are useful tools to follow osteosarcomagenesis from the normal cell to the fully malignant tumor.

Finally, in {\bf Chapter~\ref{ch8}}, loss\hyp{}of\hyp{}heterozygosity (LOH) calls were integrated with copy number calls and expression. This approach identified a high cooccurrence of LOH and copy number gains. Such regions may harbor mutated tumor suppressor genes. Because the detected LOH may have been a technical artifact, we validated the LOH of a subset of tumor suppressors by Sanger sequencing and fluorescence in situ hybridization (FISH). Sanger sequencing may not be sensitive enough to pick up LOH in regions of high amplification, but using FISH we confirmed regions of low level gains. In these regions, the LOH which was detected with both the SNP microarray and Sanger sequencing was probably not an artifact. Mutation analysis of a subset of genes did, however, not detect any recurrent mutations. In order to improve the accuracy of mapping regions of LOH in osteosarcoma, an analysis which includes paired tumor--control samples should be performed.

FISH analysis detected homozygous staining regions (HSRs) in samples with high level gains. Oncogenes may be present in HSRs, and could be associated with chromothripsis. It will therefore be of interest to perform FISH on these regions of high amplification, to validate whether these oncogenes are indeed highly amplified. Targeting a specific oncogene that is highly amplified could be beneficial in osteosarcoma, as the tumor may be addicted to such an oncogene. In osteosarcoma, screening with high\hyp{}throughput methods for such events may detect possible patient\hyp{}specific treatment options.

%
\section{Summary}\label{summary9}
In summary, by high\hyp{}throughput data analysis of pretreatment biopsies of a relatively large, homogeneous cohort of osteosarcoma patients, which was collected as a collaborative effort by EuroBoNeT, we discovered a protective role of macrophages against the development of metastases. In addition, the IR/IGF1R and PI3K/Akt signaling pathways were discovered as potential targets for treatment. By integrative genomic analyses, the genomic complexity of this tumor was confirmed, and a correlation of genomic complexity with metastasis\hyp{}free survival was identified. A conservative integrative approach to filter out passenger genes from driving events resulted in a list of mostly new, highly frequent candidate drivers in osteosarcoma. Most convincingly, genes playing a role in the maintenance of genomic stability have a considerable driving role in osteosarcomagenesis, as these pathways were affected in all three data types studied (mRNA expression, copy number data, and the kinome screen). Figure~\ref{fig9.1} summarizes the results obtained from these studies.
%
\begin{figure}[h] % OBS! changed [htbp]
	\centering
	\includegraphics[width=1.0\textwidth]{figs09/fig1bw.pdf}	% pdf version also bw
	\caption{Schematic overview of the results described in this thesis. {\it A}, Genomic instability and low macrophage count in the primary tumor are associated with poor prognosis and IGF1R and Akt signaling pathways are active in osteosarcoma. {\it B}, Low genomic instability scores are associated with better prognosis. Possible ways to intervene tumor progression are activation of macrophages with L-MTP-PE and inhibition of IR/IGF1R and Akt signaling with OSI-906 and MK-2206, respectively.}
	\label{fig9.1}
\end{figure}
%
This thesis provides the first steps in unraveling the genomic and transcriptomic landscape of this highly genomically unstable tumor. The research of osteosarcoma genomics at an even higher resolution (Next Generation Sequencing) will, together with the proposed future studies discussed in this chapter, help to better understand this highly genomically unstable tumor, and will provide indispensable knowledge on cancer evolution, diagnostics, prognostics, and targeted therapies.

%%% references

\begin{small}
\begin{singlespace}
\bibliographystyle{unsrtnatshort}		% sorted as referenced, was unsrtnat, but unsrtnatshort gives shorter output
\bibliography{biblio}
\end{singlespace}
\end{small}

%\end{document}
% Marieke Kuijjer
% 2013-02-15
% chapter 09

	%\documentclass[12pt,b5paper]{book}
	%\setcounter{secnumdepth}{0}
	%\setcounter{tocdepth}{1}
	%\usepackage[hidelinks]{hyperref}

	\fancyhead[LO]{\bfseries\ Chapter 10}		% instead of Chapter \thechapter
	\fancyhead[RE]{\bfseries\ Chapter 10}		% instead of Chapter \thechapter
 
%\begin{document}


%%% title page

%%% main document
\chapter{Nederlandse samenvatting}\label{ch10}
\thispagestyle{empty}				%%% to remove page number from first page of chapter, must be placed after calling the chapter

\selectlanguage{dutch} % for dutch hyphenation
Het hooggradig osteosarcoom is een maligne primaire bottumor, welke met name bij adolescenten en jonge volwassenen voorkomt op de plaats waar tijdens de pubertijd snelle botgroei plaatsvindt. Het is een zeer agressieve tumor, welke in 45\% van de pati\"enten uitzaait, meestal naar de longen. De 5-jaars overleving van osteosarcoom pati\"enten is ongeveer 60--70\%. Behandeling van het hooggradig osteosarcoom bestaat uit chemotherapie en operatieve verwijdering van de tumor. Een gerichte behandeling tegen specifiek osteosarcoom cellen, zoals dit bijvoorbeeld bestaat in de vorm van tamoxifen tegen oestrogeenreceptor\hyp{}positieve borsttumoren, bestaat niet. Osteosarcoom tumor cellen hebben vele en complexe afwijkingen in het DNA. In dit proefschrift is zogenaamde high\hyp{}throughput moleculaire data analyse gebruikt, om genoomwijd het osteosarcoom op verschillende niveaus, zoals DNA en mRNA, te kunnen bestuderen, met als doel meer over deze tumor te weten te komen en eventuele gerichte behandelingen tegen het osteosarcoom te kunnen identificeren. In de inleidende hoofdstukken van dit proefschrift, {\bf hoofdstuk~\ref{ch1}} en {\bf hoofdstuk~\ref{ch2}}, worden de verschillende microarray technieken---SNP-, genexpressie- en kinoomprofilering---besproken die gebruikt zijn in dit proefschrift, en wordt in een literatuuronderzoek een samenvatting gegeven van de tot nu toe gepubliceerde studies waarin dit soort technieken gebruikt zijn om het osteosarcoom te bestuderen.

{\bf Hoofdstuk~\ref{ch3}} betreft mRNA expressie data analyse van pre\hyp{}operatieve osteosarcoom biopten en modellen van het osteosarcoom, zoals cellijnen en diermodellen. Dit hoofdstuk beschrijft differenti\"ele expressie tussen osteosarcoom biopten van verschillende groepen pati\"enten, bijvoorbeeld geslacht, de locatie van de primaire tumor in het lichaam van de pati\"ent, de reactie op de pre\hyp{}operatieve chemotherapie en het histologische subtype van de tumor. Een opmerkelijke bevinding is dat er geen verschil in expressie bestaat tussen biopten van pati\"enten met een goede of slechte reactie op pre\hyp{}operatieve chemotherapie. In tegenstelling tot een aantal publicaties waarin wel verschillen in genexpressie beschreven worden, is in onze studie gecorrigeerd voor herhaaldelijk testen---een statistische methode die toegepast moet worden als meerdere hypotheses worden getest, zoals het geval is bij het analyseren van microarray data. De meest voorkomende histologische subtypes van het conventionele osteosarcoom---osteoblastisch, chondroblastisch en fibroblastisch osteosarcoom---vertoonden onderling verschillende genexpressieprofielen. Het expressieprofiel van het fibroblastaire osteosarcoom was verrijkt met genen die een rol spelen bij groei en proliferatie, terwijl het profiel specifiek voor het chondroblastaire osteosarcoom verrijkt was met genen die een rol spelen bij de chondro\"ide extracellulaire matrix van deze tumorcellen. Met behulp van een classificatiemethode werd een profiel van 24 probes (die met bepaalde genen corresponderen) ontwikkeld, welke het histologische subtype van de pre\hyp{}operatieve biopten kon bepalen. Dit profiel werd vervolgens toegepast op genexpressie data verkregen uit osteosarcoomcellen en diermodellen en kon de histologische subtypes van de originele tumor waaruit deze modellen waren ontstaan correct classificeren. Deze modellen hebben minder of geen extracellulaire matrix, de eigenschap waarop de verschillende histologische subtypes van het osteosarcoom van elkaar onderscheiden worden. Dit impliceert dat genexpressieprofielen van deze osteosarcoommodellen nog steeds representatief zijn voor de primaire tumor waaruit deze ontwikkeld zijn, en is daarom een beweegreden om deze modellen te gebruiken in onderzoek naar het osteosarcoom indien er niet genoeg primair materiaal beschikbaar is.

\section{Gerichte therapie\"en tegen het hooggradig osteosarcoom}\label{gerichte10}
Met behulp van verschillende analyses van verscheidene datasets zijn een aantal specifieke manieren ontdekt om deze tumor te bestrijden. In {\bf hoofdstuk~\ref{ch4}} zijn genexpressieprofielen van twee groepen pati\"enten met elkaar vergeleken---pati\"enten welke binnen 5 jaar uitzaaiingen ontwikkelden en pati\"enten bij wie in een periode van 5 jaar geen uitzaaiingen gevonden werden. De lijst van genen die significant verschilden in expressie was verrijkt met macrofaag\hyp{}geassocieerde genen, welke door tumor infiltrerende cellen tot expressie gebracht werden. Deze genen vertoonden allen overexpressie in de pati\"entgroep zonder metastasen. Het totale aantal tumor\hyp{}geassocieerde macrofagen van type M1 (anti\hyp{}tumor) en type M2 (pro\hyp{}tumor) associeerde in aanvullende cohorten met een betere prognose. De resultaten van hoofdstuk~\ref{ch4} verschaffen een reden om osteosarcoompati\"enten naast chemotherapie tevens met macrofaag\hyp{}activerende/aantrekkende middelen te behandelen. Een voorbeeld hiervan is het medicijn liposomal muramyl tripeptide phosphatidylethanolamine (L-MTP-PE), wat in een fase III trial de prognose van pati\"enten met osteosarcoom kon verbeteren.

Naast het testen van verschillen in genexpressie tussen groepen van hooggradig osteosarcoom biopten met verschillende klinische karakteristieken, is genexpressie van het osteosarcoom ook vergeleken met controles. In {\bf hoofdstuk~\ref{ch5}} zijn verschillen in genexpressie bepaald tussen osteosarcoomcellijnen en voorlopercellen van het osteosarcoom. Een globale pathway analyse wees op verschillen in de IGF1R signaaltransductieroute, welke een rol speelt bij botgroei. Negatieve regulatoren van de IGF1 receptor waren sterk downgereguleerd in de osteosarcoomcellen, wat zou kunnen duiden op een verhoogde activiteit van deze signaalstransductieroute. Deze route werd daarom vervolgens in osteosarcoomcellen ge\"inhibeerd met kinaseremmer OSI-906, welke niet alleen IGF1R, maar ook de insuline receptor kan remmen, wat noodzakelijk is om resistentie tegen remming van IGF1R tegen te gaan. Inhibitie met OSI-906 resulteerde in verlaagde proliferatie in drie van de vier geteste osteosarcoomcellijnen. OSI-906 zou derhalve een veelbelovend geneesmiddel kunnen zijn voor de behandeling van het osteosarcoom, naast de gebruikelijke chemotherapie.

In {\bf hoofdstuk~\ref{ch6}} wordt Serine/Threonine kinoomprofilering van twee osteosarcoomcellijnen beschreven. Voor deze studie is een peptide microarray gebruikt, welke peptides bevat die gefosforyleerd kunnen worden door kinases die aanwezig zijn in de cellysaten. Met behulp van een pathway analyse werd ontdekt dat de PI3K/Akt signaaltransductieroute verrijkt was met hyperfosforylering van moleculen direct downstream van Akt. Deze signaaltransductieroute speelt een rol bij celdeling en celgroei en kan geremd worden met verschillende medicijnen. Osteosarcoomcellijnen werden behandeld met de Akt remmer MK-2206, welke proliferatie van twee van de drie geteste cellen kon remmen. Inhibitie van de PI3K/Akt signaaltransductieroute is dan ook een mogelijke manier om deze tumor gericht te kunnen behandelen.

\section{Genomische instabiliteit}\label{genomische10}
Een tweede bevinding van dit proefschrift betreft genomische instabiliteit, welke zijn stempel drukt op verschillende niveaus---DNA, mRNA en kinaseactiviteit---in de tumorcel. De ge\"integreerde analyse beschreven in {\bf hoofdstuk~\ref{ch6}}, toegepast op signaaltransductieroutes van verschillende types data, laat zien dat signaaltransductieroutes die een rol spelen in het behouden van genomische stabiliteit verrijkt zijn in zowel overexpressie van genen als hyperfosforylering van peptiden. 
Ge\"integreerde analyse kan ook uitgevoerd worden op genniveau, door naar genen te kijken die aangedaan zijn in de te bestuderen data types. Deze methode kan gebruikt worden bij het combineren van DNA en genexpressie data, omdat de hoeveelheid DNA een directe invloed heeft op de expressie van het betreffende gen. In {\bf hoofdstuk~\ref{ch7}} worden twee methoden voor ge\"integreerde analyse besproken---de gepaarde en de ongepaarde ge\"integreerde analyse---en toegepast op data van osteosarcoom biopten en cellijnen. In de ongepaarde analyse zijn alle genen meegenomen die significant differenti\"ele expressie vertonen ten opzichte van de controle celculturen. Significant differentieel tot expressie komende genen die zich bevinden in gebieden met genomische veranderingen, welke een hoger aantal van zulke genen bevatten dan verwacht, zijn hier kandidaatgenen. In de gepaarde analyse wordt van alle genen die significant differentieel tot expressie komen bepaald of deze in hetzelfde weefsel of in dezelfde cellijn tevens in een gebied van amplificatie of deletie liggen. Dit zijn vervolgens kandidaat kanker\hyp{}drijvende genen. Met gepaarde analyse werden meer kandidaatgenen gedetecteerd in de bestudeerde osteosarcoomdataset dan met de ongepaarde analyse. Een conservatieve aanpak waarin alleen kandidaatgenen meegenomen werden die zowel in biopten als cellijnen, en vergeleken met verschillende controle celculturen gevonden waren, resulteerde in een lijst van 31 osteosarcoom kandidaatgenen, welke in minstens 35\% van alle osteosarcomen aangedaan waren. Het overgrote deel van deze kandidaatgenen was voorheen nog niet ontdekt in osteosarcoma, en meer dan twee derde van de genen speelt een mogelijke rol in kanker. Een groot aantal van deze kandidaatgenen is belangrijk voor reguleren van bijvoorbeeld de celcyclus, wat aangeeft dat het verlies van genomische stabiliteit een rol speelt in het osteosarcoom. Deze bevinding werd verder ge\"evalueerd door het berekenen van bepaalde genomische instabiliteit scores, waaruit bleek dat hogere genomische instabiliteit gecorreleerd was met slechtere prognose. Tevens werd een negatieve correlatie tussen het totale aantal DNA kopieveranderingen en prognose waargenomen.

Tenslotte is in {\bf hoofdstuk~\ref{ch8}} loss\hyp{}of\hyp{}heterozygosity (LOH) data ge\"integreerd met DNA kopie en genexpressie data. LOH gaat veelal gepaard met DNA amplificatie. Zulke gebieden zouden tumor suppressor genen kunnen bevatten, welke door bijvoorbeeld een dominant negatieve mutatie voordeel kunnen hebben van de amplificatie. Door hoge amplificatie kan echter vals positieve LOH gedetecteerd worden, vandaar dat LOH voor een aantal tumor suppressor genen gevalideerd werd met behulp van Sanger sequencing en fluorescence in situ hybridization (FISH). Sanger sequencing detecteerde geen heterozygositeit in deze genen, maar deze methode is wellicht \'{o}\'{o}k niet sensitief genoeg om heterozygositeit te kunnen detecteren in het geval van een hoge amplificatie van \'{e}\'{e}n allel. Met behulp van FISH werden zowel hoge amplificatie als lage amplificatieniveaus in verschillende samples weergegeven, welke overeenkwamen met de SNP data. LOH in lage amplificatieniveaus zijn waarschijnlijk geen artifact, en tumor suppressorgenen die in deze gebieden liggen zouden interessante mutatie kunnen bevatten. Mutatieanalyse van een aantal tumor suppressorgenen wees echter niet op mutaties in deze genen. Zogenaamde homozygous staining regions (HSRs) werden met behulp van de FISH techniek gedetecteerd in osteosarcoomcellen met hoge amplificatieniveaus. Deze hoog geamplificeerde gebieden zijn wellicht geassocieerd met chromothripsis (het uiteenvallen van chromosomen) en bevatten mogelijk oncogenen waartegen gerichte, tumor\hyp{}specifieke therapie\"en gebruikt zouden kunnen worden.

\section{Conclusie}\label{conclusie10}
In dit proefschrift is `high\hyp{}throughput' data analyse beschreven van een relatief grote microarray dataset bestaande uit osteosarcoom pre\hyp{}operatieve biopten, cellijnen en xenotransplantaten, welke beschikbaar werd gesteld door het FP6 netwerk EuroBoNeT. Met behulp van data analyse is gevonden dat macrofaag\hyp{}activerende middelen en IGF1R en Akt remmers mogelijke adjuvante therapie\"en kunnen zijn in de behandeling van het osteosarcoom. Ge\"integreerde genomische analyses wezen op de genomische complexiteit van deze tumor, en de rol van genomische instabiliteit ten opzichte van agressiviteit van het osteosarcoom. Een conservatieve benadering om zogenaamde passagiergenen weg te filteren, zodat genen die een belangrijke rol spelen bij tumorgenese gedetecteerd kunnen worden resulteerde in een lijst van 31 kandidaatgenen die frequent in het hooggradig osteosarcoom aangedaan zijn op DNA en expressieniveau, waaronder meerdere genen die een rol spelen bij het behouden van genomische stabiliteit. Met dit proefschrift zijn de eerste stappen voor het in kaart brengen van het genoom- en transcriptoomlandschap van het hooggradig osteosarcoom in gang gezet. Deze informatie is onmisbaar voor het begrijpen van deze zeer genomisch instabiele tumor en voor het ontwikkelen van diagnostische/prognostische methoden en nieuwe therapie\"en.
\selectlanguage{english} % rest of the document: english
%
%
\makeatletter\@openrightfalse %%% OBS! this causes rest of the chapters to start immediately (overrides openright to openany)
%
\newpage
\renewcommand{\chaptername}{}		% so that the chapter will not get a number
\renewcommand{\thechapter}{}			% so that the chapter will not get a number
\chapter{Curriculum Vitae}
\thispagestyle{empty}				%%% to remove page number from first page of chapter, must be placed after calling the chapter
Marieke Lydia Kuijjer was born on February 5, 1982, in Zaanstad, the Netherlands. She attended preuniversity school (gymnasium) at the Sint Adelbert College in Wassenaar. After graduating with honors in 2000, she enrolled in the Bachelor's programme Mathematics and Statistics at Leiden University. She switched to Biomedical Sciences in 2003, and received her Propaedeutics with distinction. During the Bachelor's phase of her studies, Marieke followed an Erasmus exchange program at Karolinska Institutet in Stockholm, Sweden. For her Bachelor's thesis, she studied protein dynamics using confocal microscopy under the supervision of C.R. Jost, PhD, at the department of Molecular Cell Biology, Leiden University Medical Center. During her Master's programme, she received two scholarships to follow Master's courses at Karolinska Institutet. As a trainee at the department of Human Genetics, Leiden University Medical Center, she integrated microarray data of a panel of cancer cell lines under the supervision of J.M. Boer, PhD. For her Master's thesis, she studied Wnt signaling in bone formation at the department of Molecular Cell Biology of the same institute, under the supervision of D.J. de Gorter, PhD, and P. ten Dijke, PhD. Marieke received her Master of Science degree in December 2008 and started her PhD education in the same month. The results obtained during her PhD education are described in this thesis. In May 2013, Marieke started as a postdoctoral fellow in the laboratory of J. Quackenbush, PhD, at the department of Biostatistics and Computational Biology of the Dana-Farber Cancer Institute, Boston (MA), USA.
\newpage
%
%
\renewcommand{\chaptername}{}		% so that the chapter will not get a number
\renewcommand{\thechapter}{}			% so that the chapter will not get a number
\chapter{List of publications}
\thispagestyle{empty}				%%% to remove page number from first page of chapter, must be placed after calling the chapter
\begin{itemize} % to make a bulleted list
	\item Pahl JHW, Santos SJ, \underline{Kuijjer ML}, Boerman GH, Sand LGL, Szuhai K, Cleton-Jansen AM, Egeler RM, Bov{\'e}e JVGM, Schilham MW, Lankester AC. Expression of the immune regulation antigen CD70 in osteosarcoma. {\it Submitted}
	\item Buddingh EP, Ruslan SEN, Reijnders CMA, \underline{Kuijjer ML}, Roelofs H, Hogendoorn PCW, Egeler RM,  Cleton-Jansen AM, Lankester AC. Mesenchymal stromal cells derived from healthy donors and osteosarcoma patients do not transform during long\hyp{}term culture. {\it Submitted}
	\item Kansara E, Leong HS, Lin DM, Popkiss S, Pang P, Garsed DW, Walkley CR, Cullinane C, Ellul J, Haynes NM, Hicks R, \underline{Kuijjer ML}, Cleton-Jansen AM, Hinds PW, Smyth MJ, Thomas DM. Senescence\hyp{}related immunologic functions of the RB1 tumor suppressor in radiation\hyp{}induced osteosarcoma. {\it Submitted}
	\item \underline{Kuijjer ML}, van den Akker BEWM, Hilhorst R, Mommersteeg M, Buddingh EP, Serra M, B{\"u}rger H, Hogendoorn PCW, Cleton-Jansen AM. Kinome and mRNA expression profiling of high\hyp{}grade osteosarcoma identifies genomic instability, and reveals Akt as potential target for treatment. {\it Submitted}
	\item de Vos van Steenwijk PJ, Ramwadhdoebe TH, Goedemans R, Doorduijn E, van den Ham JJ, Gorter A, van Hall T, \underline{Kuijjer ML}, van Poelgeest MIE, van der Burg SH, Jordanova ES. Tumor infiltrating CD14 positive myeloid cells work side by side with T cells to prolong the survival in patients with cervical carcinoma. Accepted for publication in {\it International Journal of Cancer}
	\item \underline{Kuijjer ML}, Peterse EFP, van den Akker BEWM, Briaire-de Bruijn IH, Serra M, Meza-Zepeda LA, Myklebost O, Hassan AB, Hogendoorn PCW, Cleton-Jansen AM. IR/IGF1R signaling as potential target for treatment of high\hyp{}grade osteosarcoma. Accepted for publication in {\it BMC Cancer}
	\item \underline{Kuijjer ML}, Cleton-Jansen AM, Hogendoorn PCW. Genome\hyp{}wide analyses on high\hyp{}grade osteosarcoma; making sense of a most genomically unstable tumor. \emph{Review}. {\it International Journal of Cancer}. 2013 Feb 20;Epub
	\item Naml{\o}s HM, Meza-Zepeda LA, Bar{\o}y T, {\O}stensen IHG, Kresse SH, \underline{Kuijjer ML}, Serra M, B{\"u}rger H, Cleton-Jansen AM, Myklebost O. Modulation of the Osteosarcoma Expression Phenotype by MicroRNAs. {\it PloS One}. 2012;7(10):e48086
	\item Mohseny AB, Cai Y, \underline{Kuijjer ML}, Xiao W, van den Akker B, de Andrea CE, Jacobs R, ten Dijke P, Hogendoorn PCW, Cleton-Jansen AM. The activities of Smad and Gli mediated signalling pathways in high\hyp{}grade conventional osteosarcoma. {\it European Journal of Cancer}. 2012 Dec;48(18):3429--3438
	\item Lenos K, Grawenda AM, Lodder K, \underline{Kuijjer ML}, Teunisse AF, Repapi E, Grochola LF, Bartel F, Hogendoorn PCW, Wuerl P, Taubert H, Cleton-Jansen AM, Bond GL, Jochemsen AG. Alternate splicing of the p53 inhibitor HDMX offers a superior prognostic biomarker than p53 mutation in human cancer. {\it Cancer Research}. 2012 Aug 15;72(16):4074--84
	\item \underline{Kuijjer ML}, Rydbeck H, Kresse SH, Buddingh EP, Lid AB, Roelofs H, B{\"u}rger H, Myklebost O, Hogendoorn PCW, Meza-Zepeda LA, Cleton-Jansen AM. Identification of osteosarcoma driver genes by integrative analysis of copy number and gene expression data. {\it Genes, Chromosomes and Cancer}. 2012 Jul;51(7):696--706
	\item Pansuriya TC, van Eijk R, d'Adamo P, van Ruler MA, \underline{Kuijjer ML}, Oosting J, Cleton-Jansen AM, van Oosterwijk JG, Verbeke SL, Meijer D, van Wezel T, Nord KH, Sangiorgi L, Toker B, Liegl-Atzwanger B, San-Julian M, Sciot R, Limaye N, Kindblom LG, Daugaard S, Godfraind C, Boon LM, Vikkula M, Kurek KC, Szuhai K, French PJ, Bov{\'e}e JVGM. Somatic mosaic IDH1 and IDH2 mutations are associated with enchondroma and spindle cell hemangioma in Ollier disease and Maffucci syndrome. {\it Nature Genetics}. 2011 Nov 6;43(12):1256--1261
	\item \underline{Kuijjer ML}, Naml{\o}s HM, Hauben EI, Machado I, Kresse SH, Serra M, Llombart-Bosch A, Hogendoorn PCW, Meza-Zepeda LA, Myklebost O, Cleton-Jansen AM. mRNA expression profiles of primary high\hyp{}grade central osteosarcoma are preserved in cell lines and xenografts. {\it BMC Medical Genomics}. 2011 Sep 20;4:66
	\item Buddingh EP{\small $\dagger$}, \underline{Kuijjer ML}{\small $\dagger$}, Duim RA, B{\"u}rger H, Agelopoulos K, Myklebost O, Serra M, Mertens F, Hogendoorn PCW, Lankester AC, Cleton-Jansen AM. Tumor\hyp{}infiltrating macrophages are associated with metastasis suppression in high\hyp{}grade osteosarcoma: a rationale for treatment with macrophage activating agents. {\it Clinical Cancer Research}. 2011 Apr 15;17(8):2110--2119
\begin{small}
$\dagger$Shared first authorship
\end{small}
\end{itemize}
%
%
\newpage
\renewcommand{\chaptername}{}		% so that the chapter will not get a number
\renewcommand{\thechapter}{}			% so that the chapter will not get a number
\chapter{Acknowledgments}
It is a great pleasure to give respect to those who made this thesis possible. I owe sincere gratitude to my promotor, prof. dr. P.C.W. Hogendoorn, and to my copromotor, dr. A.M. Cleton-Jansen. Pancras, I enjoyed working in your group, which has been an excellent environment to start my scientific career. Anne-Marie, your enthusiasm and knowledge were always stimulating, and I thank you for giving me freedom to find my own research direction. All colleagues from the department of Pathology are acknowledged, in particular the technicians, who provided me with excellent technical support. I am indebted to my roommates for their patience and support and especially to Cathelijn, Jolieke, Sara, and Wei for their friendship. I dearly thank Lisanne Vijfhuizen, Frauke Liebelt, and Elleke Peterse; I enjoyed to work with you and wish you all the best with your careers. Special thanks to all EuroBoNeT partners and to PamGene for the good collaborations. An honorable mention goes to my family and friends for their understanding and support. Finally, I thank Alessandro Marin for supporting me in every possible way, and for believing in me. {\it Ti amo}.
\thispagestyle{empty}				%%% to remove page number from first page of chapter, must be placed after calling the chapter


%%%%% OBS!! you need to have chapter 10 in the fancyhdr

\end{document}