% Marieke Kuijjer
% 2013-02-15
% chapter 02

	%\documentclass[12pt,b5paper]{book}
	%\setcounter{secnumdepth}{0}
	%\setcounter{tocdepth}{1}
	%\usepackage[hidelinks]{hyperref}

%\begin{document}

%%% title page

\chapter{Genome\hyp{}wide analyses on high-grade osteosarcoma: making sense of a genomically most unstable tumor}\label{ch2}
\thispagestyle{empty}				%%% to remove page number from first page of chapter, must be placed after calling the chapter

\vfill

\vspace{0.5cm}
This chapter is based on the review:
\underline{Kuijjer ML}, Hogendoorn PCW, Cleton-Jansen AM. \emph{Int J Cancer}. 2013 Feb 22;Epub

\newpage

%%% main document

%
\section{Abstract}\label{abstract2}
High-grade osteosarcoma is an extremely genomically unstable tumor. This, together with other challenges, such as the
heterogeneity within and between tumor samples, and the rarity of the disease, renders it difficult to study this tumor on a
genome\hyp{}wide level. Now that most laboratories change from genome\hyp{}wide microarray experiments to Next\hyp{}Generation
Sequencing it is important to discuss the lessons we have learned from microarray studies. In this review, we discuss the
challenges of high\hyp{}grade osteosarcoma data analysis. We give an overview of microarray studies that have been conducted so far on both osteosarcoma tissue samples and cell lines. We discuss recent findings from integration of different data types,
which is particularly relevant in a tumor with such a complex genomic profile. Finally, we elaborate on the
translation of results obtained with bioinformatics into functional studies, which has lead to valuable findings, especially
when keeping in mind that no new therapies with a significant impact on survival have been developed in the past decades.

%
\section{Introduction}\label{introduction2}
\subsection{High-grade osteosarcoma, a rare, genomically complex and unstable tumor}
High-grade osteosarcoma is the most prevalent primary malignant
bone tumor. The disease occurs most often in children
and adolescents and is the sixth leading cause of death
in children under the age of 15 years. Notwithstanding,
osteosarcoma is a rare disease, with an incidence of five to
ten new cases per $1,000,000$ per year~\cite{fletcher1994cytogenetic,raymond2002conventional}. Osteosarcoma is
composed of extremely genomically complex and unstable
mesenchymal tumor cells, generally exhibiting both complex
clonal and numerous nonclonal aberrations~\cite{fletcher1994cytogenetic}, which are characterized
by the direct production of osteoid~\cite{raymond2002conventional,helman2003mechanisms}. The tumor is
highly aggressive, with distant metastases developing in
approximately 45\% of all patients~\cite{pakos2009prognostic} although patients are
treated with intensive neoadjuvant treatment consisting of
high doses of multiple chemotherapeutic drugs. Better surgery
has improved survival slightly but no other significant
improvement has been made since decades, and increasing
dose or the administration of more than three chemotherapeutic
regimens does not increase overall
survival~\cite{lewis2007improvement,eselgrim2006dose,anninga2011chemotherapeutic}. Hence, new
therapeutics are seriously needed. Studying the tumor
biology and pathology in a systematic manner can result in a
better understanding of osteosarcomagenesis and can potentially
identify new targets for treatment.

\subsection{Caveats and challenges}
Several challenges and caveats are encountered when studying
a rare, highly genomically unstable tumor on a genome\hyp{}wide
level. The first challenge is apparent when collecting
osteosarcoma tumor samples. Osteosarcoma is a rare disease
and therefore often large interinstitutional efforts have to be
achieved to collect the substantial amount of samples that is
needed for analyses in computational biology. For most purposes,
studying osteosarcoma pretreatment diagnostic biopsies
is preferred over using resection material of the primary
tumor. Presurgery chemotherapy causes substantial necrosis,
even in poor responders, thereby rendering the tissue unsuitable
for high quality nucleic acid retrieval. Moreover, biopsies
are more representative of the state of the tumor before any
treatment as chemotherapy changes the distribution of subclones
present in the primary tumor, and can cause clonal
evolution~\cite{ding2012clonal}. Biopsies are taken to establish a histopathological
diagnosis, and are unfortunately often very small and not
always available for research. In addition, material is often
collected retrospectively, which can introduce heterogeneity
owing to, for example, different treatment procedures, unless
patients are collected who have been enrolled in the same
clinical trial. Thus, the collection of clinical data and the
grouping of clinical parameters have to be carried out very
carefully. For a rare entity such as osteosarcoma, collaborations
are indispensable to collect significant cohorts, an
example of this being the European Network of Excellence
EuroBoNeT, in which various European institutes collaborated
to collect a large, homogeneous set of, among other
bone tumors, high\hyp{}grade osteosarcoma biopsies.

Primary osteosarcoma is subdivided into numerous different
low- and high\hyp{}grade subtypes~\cite{mohseny2008bone}. In this review, we concentrate
on high\hyp{}grade conventional osteosarcoma, which is by
far the most prevalent variant. Although there is often intratumor
heterogeneity, high\hyp{}grade conventional osteosarcoma
can be grouped into various histological subtypes, based on
the produced extracellular matrix of the tumor~\cite{mohseny2008bone}. Osteoblastic,
chondroblastic and fibroblastic osteosarcoma are the most
common histological subtypes of high\hyp{}grade conventional osteosarcoma.
Some correlation of the distinct histological subtypes
to specific clinical outcomes has been observed~\cite{hauben2002does,hauben2006clinico} and
it may thus be difficult to collect a homogeneous set of samples.
In fact, often it is not clearly described which exact histological
subtypes are used in a specific study, and in what
percentages these subtypes are present in the data set. In
addition, the subclassification is hindered by the occurrence
of mixed cases containing two different matrix types. Nonetheless,
a concordance of 98\% has been found between the
histological subtype of osteosarcoma biopsies and the corresponding
resections~\cite{hauben2002does}.

A general problem in studying tumor cell biology is that
the true cell of origin is often not defined, rendering it difficult
to select a representative control tissue or control cells.
Osteosarcoma cells are osteoblast\hyp{}like cells of mesenchymal
origin. Of the different histological subtypes that exist, multiple
subtypes can be present within a single tumor. Considering
the differentiation capacity of the mesenchymal stem cell
(MSC), this cell type is the most probable candidate for being
the osteosarcoma progenitor~\cite{mohseny2011concise,bovee2003skeletogenesis}. It was recently found that
osteosarcoma tumors can be spontaneously formed when
mouse MSCs are transferred into mice~\cite{tolar2006sarcoma,mohseny2009osteosarcoma} and zebrafish~\cite{mohseny2012osteosarcomazebrafish,mohseny2012osteosarcomamodels}.
This does, however, not exclude osteoblasts as putative progenitor
cells, as osteoblasts might redifferentiate into the
primitive osteoblast\hyp{}like tumor cells of osteosarcoma.

\subsection{Osteosarcoma models}
As collecting fresh frozen osteosarcoma tumor samples can
be a challenge, performing analyses on data derived from osteosarcoma
cell lines or xenografts may be a good alternative~\cite{mohseny2012osteosarcomamodels}. Osteosarcoma cell lines are frequently used in
biological studies, because they generally grow fast and are
easy to maintain in culture and hence osteosarcoma cell lines
are easily available. One caveat of using cell cultures is that
slight differences in culture conditions, for example the percentage
of cells in the culture dish or flask, or the medium
that is used, can lead to significant differences in protein
expression or signal transduction pathway activities, and
these specific conditions may differ per cell line. Using a
large panel of cell lines cultured under standard settings can
overcome this problem. Cell culture may furthermore
introduce additional mutations and genomic aberrations in
the cell genome, because of selection based on the {\it in vitro}
conditions~\cite{weinberg2007cells}, but in general, cell lines are reported to
adequately represent the tumor from which they are derived.
{\it In vitro}, they preserve the genetic aberrations of the parent
tumor, while acquiring additional locus\hyp{}specific alterations~\cite{greshock2007cancer}.

A panel of 19 osteosarcoma cell lines was recently characterized
genetically by MLPA on 38 tumor suppressor gene
loci~\cite{ottaviano2010molecular}. A screen for {\it TP53} mutations, {\it MDM2} amplification,
{\it CDKN2A/B} deletion and genomic deletions of 38 additional
tumor suppressor genes was performed on these cell lines. As
three cell lines of this panel---HOS, 143B and MNNG-HOS---
have common ancestry, we report the following percentages
based on 17 cell lines. Homozygous deletion of the
{\it CDKN2A/B} locus was detected in 35\%, whereas hemizygous
deletion of this locus was found in 24\% of osteosarcoma cell
lines. An additional homozygous deletion was found for
{\it TP73} in one cell line. Mutation in {\it TP53} was detected in 41\%,
whereas {\it MDM2} amplification was detected in 17\% of cell
lines. These percentages are higher than those in osteosarcoma
tumor tissues that are reported in the previously published
literature~\cite{cleton2005central}, which may be explained by an advantage
for primary tumor cells harboring such mutations to be effectively
immortalized, or by the acquisition of additional mutations
owing to long\hyp{}term culture. {\it MDM2} amplification and
{\it TP53} mutations were mutually exclusive in this cell line
panel. This has also been observed in osteosarcoma tumor
data~\cite{overholtzer2003presence}. The differentiation capacity of this cell line panel has
been determined as well~\cite{mohseny2011functional}. All 19 cell lines were able to differentiate
toward at least one of the three tested---osteoblastic,
chondroblastic and adipocytic---lineages. Most cell lines
(14/19) could differentiate to at least two lineages, whereas
3/19 cell lines had full differentiation capacity.

{\it In vivo} osteosarcoma model systems include transplantation
of a human tumor in mice~\cite{mayordomo2010tissue,kresse2011preclinical}, subcutaneous or orthotopically
injections of osteosarcoma cells or late\hyp{}passage
transformed MSCs into mice~\cite{mohseny2009osteosarcoma}
or zebrafish~\cite{mohseny2012osteosarcomazebrafish}. Transgenic
mouse models of osteosarcoma can be developed by overexpression
of {\it c-fos}~\cite{ruther1989c}, or conditional inactivation of {\it TP53} and
{\it RB1}~\cite{walkley2008conditional}. These different models have been shown to resemble
osteosarcoma phenotypically~\cite{mohseny2009osteosarcoma,mohseny2012osteosarcomazebrafish,mohseny2011functional,mayordomo2010tissue,ruther1989c,walkley2008conditional,kuijjer2011mrna,kresse2011preclinical}. For example, subcutaneous
and intramuscular injection of osteosarcoma cells
in nude mice resulted in high\hyp{}grade sarcoma, resembling
tumors which produced osteoid~\cite{mohseny2011functional} for 8/19 cells from the
above\hyp{}described panel. The {\it in vivo} lineage\hyp{}specific differentiation
capacity of these cells, however, was limited, reflecting
the importance of stomal or microenvironmental stimulation
for this process.

As with cell lines, xenograft tumor cells may acquire additional
changes owing to selection, and often, xenografts lose
matrix after several passages~\cite{mayordomo2010tissue}. This will probably not have a
significant effect on genomic profiles, but does influence
expression and methylation patterns. High\hyp{}resolution microassay\hyp{}based array comparative genomic hybridization
(aCGH) including nine osteosarcoma patient--xenograft pairs
showed that genomes of human tumors transplanted into
immunodeficient mice, which were repeatedly passaged in
new mice, where comparable to genomes of their tumor of
origin, with the acquisition of only a small number additional
significant changes in the xenograft genomes~\cite{kresse2011preclinical}. Different
microarray studies have shown that osteosarcoma cell lines
and xenografts resemble the primary tumor from which they
are derived. Gene expression profiling of a subset of the
EuroBoNeT cell line panel, for which the original histological
subtype of the primary tumor was known, and of osteosarcoma
xenografts and pretreatment biopsies showed that,
despite the lower amounts of matrix, histological subtypespecific
mRNA signatures are retained in these model
systems, and therefore may be a useful tool for expression
analysis (Chapter~\ref{ch3},~\cite{kuijjer2011mrna}). Despite the similarities between genome and
expression profiles of the model systems described above and
the tumors of origin, the absence (cell lines) or lower
amounts (xenografts) of stromal cells and extracellular
matrix, the absence of interaction with the immune system
(cell lines and some xenograft models) and the higher degree
of clonality remain important limitations for studying tumor
biology using these model systems.

\subsection{Genome wide profiling to study osteosarcoma}
In the next sections, we describe different methods to analyze
specific types of microarray data, and give examples of
how results from bioinformatics can be translated into functional
studies. This review is not aiming to give a comprehensive
overview of all genome\hyp{}wide studies on
osteosarcoma, but rather illustrates and summarizes the
major findings on DNA/RNA microarray reports. A summary
of these findings is provided in Table~\ref{tab2.1}.
%
%%% OBS! references in this table are placed there by hand, as it would be difficult to place them here, but use text numbering
\afterpage{
	 \clearpage% flush all other floats
	 \ifodd\value{page}
%	\else% uncomment this else to get odd/even instead of even/odd
	\expandafter\afterpage% put it on the next page if this one is odd
	\fi
    {
\begin{landscape}
	\centering
	\footnotesize
	\begin{singlespacing}
		\begin{longtable}[c]{|>{\raggedright}p{0.7in} >{\raggedright}p{0.6in} >{\raggedright}p{1.35in} >{\raggedright}p{1.2in} >{\raggedright}p{2.0in} >{\raggedright}p{3.0in}|}
		\hline
		Analysis & Data type & Study & Osteosarcoma samples & Comparison & Pathway/genes\tabularnewline
		\hline
		Single-way & mRNA & Kuijjer {\it et al}.$^{28}$ & 76 B, 13 X, 18 C & Histological subtypes & NF$\upkappa$B in fibroblastic, chondroid\hyp{}matrix\hyp{}associated genes in chondroblastic osteosarcoma. Primary tumor expression signatures are preserved in model systems\tabularnewline
		& & Buddingh {\it et al}.$^{30}$ & 53 B & Metastasis\hyp{}free survival & Macrophage\hyp{}associated genes correlate
with better MFS\tabularnewline
		& & Su {\it et al}.$^{31}$ & 3 C, 5 X & Capacity to metastasize & {\it IGFBP5} downregulation correlates with metastasis\tabularnewline
		& & Naml{\o}s {\it et al}.$^{33}$ & 12 B/T, 11 M & Tumor sample type & Immunological processes and chemokine
pattern upregulated in metastases\tabularnewline
		& & Cleton-Jansen {\it et al}.$^{32}$ \\ Kuijjer {\it et al}.$^{28}$ & 25 B \\ 69 B & Response to chemotherapy & No significant differential expression\tabularnewline
		& & Cleton-Jansen {\it et al}.$^{32}$ & 25 B & Control samples (osteoblastoma, MSC, osteoblast) &  Cell\hyp{}cycle regulation, DNA replication pathways\tabularnewline
		& & Sadikovic {\it et al}.$^{42}$ & 6 B & Control sample (osteoblast) &  DNA replication network\tabularnewline
		& & Kuijjer {\it et al}.$^{43}$ & 84 B & Control samples (MSC, osteoblast) &  Apoptosis, signal transduction\tabularnewline
		& & Kansara {\it et al}.$^{44}$ & 5 C & Treatment with demethylating agent &  WIF1 methylation and downregulation\tabularnewline
		& miRNA & Jones {\it et al}.$^{45}$ & 18 B & Control samples (normal bone) &  miR-16 Downregulation, miR-27a association with metastasis\tabularnewline
		& CN & Kresse {\it et al}.$^{25}$ & 9 T/M and their derived xenografts &  Tumor sample type & Xenografts are representative for primary tumors although some additional aberrations are observed\tabularnewline
		& & Squire {\it et al}.$^{50}$ \\ Man {\it et al}.$^{51}$ \\ Atiye {\it et al}.$^{52}$ \\ Yang {\it et al}.$^{53}$ \\ Kresse {\it et al}.$^{54}$ \\ Kuijjer {\it et al}.$^{43}$ \\ Lockwood {\it et al}.$^{55}$ \\ Yen {\it et al}.$^{56}$ \\ Smida {\it et al}.$^{58}$ \\ Pasic {\it et al}.$^{59}$ & 9B \\ 48 B/T/M \\ 22 C/TS/R \\ 20 B \\ 36 TS/M/X, 20 C \\ 32 B \\ 22 TS \\ 42 TS/R/M/C \\ 45 B \\ 27 B & Control samples & Overall high level of aneuploidy, which seems nonrandom. Regions described by three or more studies are gains on 1p, 6p, 8q, 12q and 17p and losses on 2q, 3q, 6q, 10, 13q and 17p\tabularnewline
		& & Kuijjer {\it et al}.$^{43}$ \\ Smida {\it et al}.$^{58}$ & 32 B \\ 45 B &  Metastasis/event\hyp{}free survival & Genomic alterations are prognostic predictors\tabularnewline
		& & Yen {\it et al}.$^{56}$ & 23 TS, 14 R/M & Tumor sample type & Identified deletions/amplifications which differ between TS and R/M\tabularnewline
		& & Kresse {\it et al}.$^{54}$ \\ Yen {\it et al}.$^{56}$ \\ Pasic {\it et al}.$^{59}$ & 36 TS/M/X, 20 C \\ 42 TS/R/M/C \\ 27 B & Control samples & Frequent deletion of {\it LSAMP}\tabularnewline
		& & Yang {\it et al}.$^{53}$ & 20 B & Control samples & Enrichment of VEGF pathway\tabularnewline
		Integrative & CN, mRNA & Kuijjer {\it et al}.$^{43}$ & 29 B & Control samples (MSC, osteoblast) & Set of 31 candidate drivers enriched in genes with a role in genomic instability\tabularnewline
		& & Lockwood {\it et al}.$^{55}$ & 22 TS, 8 X & Control samples (normal tissues) & Amplification and overexpression of cyclin E3\tabularnewline
		& miRNA, mRNA & Jones {\it et al}.$^{45}$ & 14 B & Control samples (normal bone) & Transcriptional regulation, cell cycle control and cancer signaling\tabularnewline
		& & Naml{\o}s {\it et al}.$^{46}$ & 19 C & Control samples (normal bone) & Pairs of miRNAs with 26 mRNAs\tabularnewline
		& CN, mRNA, methylation & Sadikovic {\it et al}.$^{48}$ & 2 C & Control sample (osteoblast) & Hypomethylation of genes connected to c-Myc\tabularnewline
		& & Sadikovic {\it et al}.$^{42}$ & 5 B & Control sample (osteoblast) & RUNX2 amplification and overexpression, {\it DOCK5} and {\it TNFRSF10A/D} loss and underexpression, hypomethylation, gain and overexpression of histone cluster 2 genes\tabularnewline
		& & Kresse {\it et al}.$^{49}$ & 19 C & Control samples (osteoblast, normal bone) & 350 genes with two aberration types, including {\it RUNX2} and {\it DLX5} amplification and overexpression\tabularnewline
		\hline
		\caption{Overview of genome\hyp{}wide data analyses in high\hyp{}grade osteosarcoma. The table gives an overview of single\hyp{}way and integrative analyses described in this review. For each study, the sample type and sample size is given under Osteosarcoma samples column and the comparison which is made in the bioinformatics analysis, for example, comparison with control tissue, is shown in the Comparison column. Several studies used different sample types in one group. When this was done, these sets are shown in the table as combined into one group as well. Groups of different sample types which have been used in separate analyses are shown as different groups. Not always, it is clear whether na{\"i}ve tumor biopsies, untreated primary tumor resections or resections of treated primary tumors were used. For such studies, we have used the abbreviation TS (for tumor sample). B: na{\"i}ve tumor biopsies, T: resections of primary tumors, R: resections of recurrences, M: metastatic resections, C: cell lines, X: xenografts.}
		\label{tab2.1}
		\end{longtable}
	\end{singlespacing}
\end{landscape}
}
}
%
With the
purpose to review bioinformatic analyses on osteosarcoma,
we only review studies where at least three samples were
included, and only refer to articles where robust statistical
analyses have been applied.

%
\section{Single platform analyses of osteosarcoma genome\hyp{}wide data}\label{single2}
\subsection{Different approaches for single\hyp{}way analyses}
In a typical supervised genome\hyp{}wide data analysis, significant
differences, for example significantly differentially expressed
genes/miRNAs or differential methylation, are determined
between two or more groups of samples. These groups can
exist of different clinical parameters, of tumor samples and
their nontumorigenic counterpart or of experimentally
induced and noninduced samples as shown in Figure~\ref{fig2.1}.
%
\begin{figure}[htbp]
  \centering
    \includegraphics[width=1\textwidth]{figs02/fig1bw.pdf}	% pdf version also bw
    \caption{Different supervised comparisons in genome\hyp{}wide data
analysis. Flow chart describing single\hyp{}way bioinformatic analyses
that are most typically performed on genome\hyp{}wide data. For mRNA,
miRNA and methylation data analysis, the comparative analysis
usually exists of tumor samples versus nontumorigenic counterparts,
of different groups of tumor samples, defined by clinical parameters
or samples which are experimentally altered compared to
samples which are not, although tumor samples of a specific
group are also sometimes compared to a pool of all samples (not
illustrated in this figure). Copy number data are most often compared
to a reference set, which may be an in-house, or a public
reference set, and which does not have to consist of the nontumorigenic
counterpart of the tumor that is studied. Additional comparative
analyses may determine the differences between different
subgroups within the samples that are studied. Although for
mRNA, miRNA and methylation data, often significant differential
expression/methylation is returned by statistical tests, for copy
number data researchers mostly look at frequency of the aberration
in the studied groups.}
     \label{fig2.1}
\end{figure}
%
Copy
number profiling data are analyzed somewhat differently, as
copy number profiles of tumor samples do not necessarily
have to be compared to their specific nontumorigenic counterparts,
but can be compared to, for example, a public reference
set, such as HapMap samples~\cite{gibbs2003international}. Usually, a cutoff for
frequency is used to determine whether an amplification or
deletion is recurrently present in a specific region. Unsupervised
analysis, on the other hand, can give information on
quality of the data, and on whether there are certain subgroups
within the tumor samples that behave differently.

Each of these distinct ways to analyze genome\hyp{}wide data
has been applied to high\hyp{}grade osteosarcoma data sets. An
overview of these different approaches in osteosarcoma on
gene expression, microRNA (miRNA), methylation and copy
number data is given in the following paragraphs. Functional
verification of the results obtained with these studies will be
discussed in a later section of this review.

\subsection{Genome\hyp{}wide gene expression data, comparison of clinical parameters}
Comparisons between different clinical subgroups of osteosarcoma
have resulted in a prediction profile that can classify
the main histological subtypes of conventional high\hyp{}grade osteosarcoma
in biopsy material, but also in cell lines and in
osteosarcoma xenografts (Chapter~\ref{ch3},~\cite{kuijjer2011mrna}). Protein interaction networks illustrated
that chondroid matrix\hyp{}associated proteins were overexpressed
in chondroblastic osteosarcoma, whereas NF$\upkappa$B--STAT5
signaling showed higher expression in fibroblastic osteosarcoma.
The absence of a specific network for osteoblastic
osteosarcoma indicates that the features of the main osteoblast\hyp{}like
cell and of the osteoid matrix are present in tumors
of all three main histological subtypes.

A second example of a comparison between different clinical
parameters is the comparison of samples with different
outcomes in event\hyp{}free survival or overall survival. It is important
to note that when designing an analysis for such a
study, a uniform set of clinical follow\hyp{}up parameters should
be employed, instead of directly comparing patients with or
without metastases, or patients who are alive or deceased. In
one study, differential expression was determined between biopsy
material of patients developing metastases within 5
years and patients who did not develop metastases within
this time frame. This study demonstrated that, in osteosarcoma,
an expression profile associated with macrophages correlated
with better overall survival (Chapter~\ref{ch4},~\cite{buddingh2011tumor}). To identify genes
playing a role in metastasis, comparisons between osteosarcoma
cell lines that can or cannot metastasize upon passaging
into mice have also been made. A recent study identified
downregulation of {\it IGFBP5}, or insulin\hyp{}like growth factor
binding protein 5, in the metastatic cell line MG63.2 and in
tumors derived from this cell line~\cite{su2011insulin}. Interestingly, this gene
was also significantly downregulated in our analysis, comparing
osteosarcoma biopsies with control tissues~\cite{cleton2009profiling}). Metastasis
progression can be studied by comparing metastatic resections
to the primary tumor. This has been performed in one
study, where higher expression of genes involved in immunological
processes was detected in the metastasis samples~\cite{namlos2012global}.
This may correlate with our findings that more CD14\textsuperscript{+} cells
are present in metastatic lesions than in pretreatment
biopsies~\cite{buddingh2011tumor}.

Another important clinical parameter that has been studied
in human osteosarcoma is response to chemotherapy,
which is predictive for overall survival~\cite{bacci1998predictive,huvos1991bone,bacci2006prognostic}. Differentially
expressed genes discriminating between good and poor responders
to chemotherapy have been detected by different
groups, but with little consensus in the gene lists. Most studies
did not use robust statistics with correction for multiple
testing, a shortcoming that is too often seen in biomedical
research~\cite{dupuy2007critical}. When differential expression was determined
between poor and good responders in two studies where correction
for multiple testing was applied (Chapter~\ref{ch3},~\cite{kuijjer2011mrna}, and~\cite{cleton2009profiling}), no significant
genes were detected although larger sample sizes and homogeneous
data sets were used (17 poor {\it vs} 8 good responders
in Cleton-Jansen {\it et al}.~\cite{cleton2009profiling} and 36 poor {\it vs} 33 good responders
in Kuijjer {\it et al}.~\cite{kuijjer2011mrna}). Although these sample sizes are not comparable
to what is often used for studying less rare tumor
types, the distribution of the nonadjusted p-values did not
show any trend for the lower p-values to be more prevalent
(Additional Figure~\ref{afig2.1}). This indicates that in a comparison
between two groups no effect is detected, and increasing
sample size will not lead to a significant increase in power~\cite{van2009relative}. A
major issue with comparing responders with nonresponders in
gene expression analysis is that resistance to chemotherapy may
be caused by the alteration of a single gene. A specific gene
causing resistance in a subset of samples will not be picked up
by a comparison of responders and nonresponders~\cite{borst2010predictive}.

In human osteosarcoma xenografts, significant differential
expression has been detected between good and poor responders
to single chemotherapeutic agents~\cite{bruheim2009gene}. A pitfall of this
study, however, was that the studied sample set included xenografts
derived from biopsies, resections as well as from metastases.
Surviving cells of pretreated tumors are resistant to
chemotherapy. Thus, the differences in gene expression
between poor and good responders to these chemotherapeutic
agents may actually reflect an effect of presurgery therapy.
It was indeed demonstrated that xenografts of these
implanted pretreated tumors often responded poorly to multiple
chemotherapeutic agents~\cite{bruheim2004human}.

\subsection{Genome\hyp{}wide gene expression data, comparison with control tissues}
mRNA expression levels in osteosarcoma samples can also be
compared to expression in control tissues. The control tissues
that have been used for this purpose are normal bone, osteoblastoma,
osteoblasts, MSCs, or, for example, a pool of different
cell lines. One comparison of high\hyp{}grade osteosarcoma
biopsy specimens with control samples is described in Cleton-Jansen {\it et al}.~\cite{cleton2009profiling}, who made different comparisons of 25 osteosarcoma
biopsies with five osteoblastomas, with five MSCs
and with five osteoblast cultures. Gene set enrichment
detected cell\hyp{}cycle regulation and DNA replication pathways
as the most significantly affected pathways in osteosarcoma.
A DNA replication network was also identified in an analysis
of gene expression microarrays of six osteosarcoma biopsies
as compared to one osteoblast culture although a caveat of
this study is the small sample size of the control set
($n=1$)~\cite{sadikovic2009identification}. A larger set of osteosarcoma biopsies ($n=84$) was
compared to 12 MSCs and separately with three osteoblast
cultures (Chapter~\ref{ch7},~\cite{kuijjer2012identification}). Intersection of the differentially expressed genes in
both analyses identified antigen processing and presentation
as well as angiogenesis as significantly different between tumor
samples and control cell lines, most probably because of
the amount of stroma present in the tumor samples. In addition,
altered apoptosis and signal transduction were detected.

\subsection{Genome\hyp{}wide gene expression data, experimentally induced differences}
We give a final example of genome\hyp{}wide gene expression
analyses in osteosarcoma, which is experimentally induced
differential expression. This is, for example, reported in the
study by Kansara {it et al}.~\cite{kansara2009wnt}, who compared a set of five human
osteosarcoma cells treated with a demethylating agent to
untreated cells, after having shown that demethylating agents
can induce growth arrest and differentiation in osteosarcoma.
The list of candidate genes was then filtered for expression in
human osteoblasts and loss of expression in primary osteosarcomas.
This screen identified {\it WIF1}, a Wnt inhibitory factor,
as a candidate tumor suppressor in osteosarcoma.

\subsection{microRNA expression data}
Several studies have been published, describing miRNA
microarray data analysis on osteosarcoma tissues or cell lines
as compared to osteoblasts or normal bone, but in most studies
no robust statistics were applied. Jones {\it et al}.~\cite{jones2012mirna} and
Naml{\o}s {\it et al}.~\cite{namlos2012modulation} published the only miRNA microarray studies
in which false discovery rate corrections were applied. In
the article by Jones {\it et al}.~\cite{jones2012mirna}, miRNA expression was compared
between 18 osteosarcoma resections or biopsies and 12 normal
bone samples, which lead to the detection of a downregulated
tumor suppressive miRNA and of a prometastatic
miRNA (these miRNAs will be discussed in the Integrative % OBS! add link here and to section
analyses section). Naml{\o}s {\it et al}.~\cite{namlos2012modulation} compared miRNA expression
in 19 osteosarcoma cell lines with expression in normal
bone ($n=4$) and integrated these results with mRNA expression
data. Results from this study will therefore be discussed
in the Integrative analyses section. Sarver {\it et al}.~\cite{sarver2010s} published % OBS! add link here and to section
an online accessible Sarcoma miRNA expression database
(S-MED), which includes 15 osteosarcoma samples and six
normal bone samples.

\subsection{Genome\hyp{}wide methylation data}
Only three studies have been published so far on genome\hyp{}wide
methylation in high\hyp{}grade osteosarcoma~\cite{sadikovic2009identification,sadikovic2008vitro,kresse2012integrative}. These
studies describe an integrative analysis with different data
types, without presenting conclusions on specific genes, or on
results obtained with gene set enrichment on single\hyp{}way
methylation analyses although Kresse {\it et al}.~\cite{kresse2012integrative} found overall
more hypermethylation in osteosarcoma cell lines than hypomethylation.
We will discuss the results from these studies
under the Integrative analysis section of this review. % OBS! add link here and to section

\subsection{Genomic copy number data}
The genomic instability of high\hyp{}grade osteosarcoma, which is
more pronounced in this tumor than in many other tumor
types, hampers the identification of specific genomic regions.
Several array comparative genomic hybridization (aCGH)
studies~\cite{kresse2011preclinical,squire2003high,man2004genome,atiye2005gene,yang2011genetic,kresse2009lsamp,lockwood2011cyclin} and single\hyp{}nucleotide polymorphism (SNP)
microarray studies~\cite{kuijjer2012identification,yen2009identification,kresse2010evaluation,smida2010genomic,pasic2010recurrent} on osteosarcoma specimens have
been published. Copy number profiles clearly show that
high\hyp{}grade osteosarcoma samples are characterized by a high
level of aneuploidy, and that there is heterogeneity between
different tumor samples. There is a general consensus about
copy number alterations for some regions, such as gains on
chromosome arms 6p, 8q and 17p, which have been detected
by classical karyotyping and conventional CGH as well~\cite{raymond2002conventional,lau2003frequent},
but it is difficult to directly compare studies as the definition
of a recurrent alteration varies.

In three separate studies, a focal deletion of the region
3q13.31, which harbors a putative tumor suppressor gene,
{\it LSAMP}, was detected~\cite{kresse2009lsamp,yen2009identification,pasic2010recurrent}. siRNA\hyp{}mediated silencing of
{\it LSAMP} promoted proliferation of normal osteoblasts~\cite{pasic2010recurrent}, and
low expression of the gene was associated with poor overall
survival in one of these studies~\cite{kresse2009lsamp}. Gene set enrichment on an
aCGH study showed an enrichment of amplified genes of the
VEGF signaling pathway of which {\it VEGFA} amplification also
correlated with poor prognosis, showing that gene set enrichment
on copy number data can identify pathways associated
with tumorigenesis~\cite{yang2011genetic}.

Copy number profiles of osteosarcoma cell lines roughly
resemble profiles of tumor biopsies, but show an increased
overall aneuploidy (Kuijjer {\it et al}., {\it unpublished data}) and
increased expression of genomic instability genes (Chapter~\ref{ch7},~\cite{kuijjer2012identification}). Also, as
described above, genomic profiles of xenografts are highly
similar to primary tumors although some deviations may
occur owing to additional genetic alterations during passaging,
or owing to general tumor progression~\cite{kresse2011preclinical}. In some data
sets, specific copy number alterations have not been detected
for different clinical groups of interest~\cite{kuijjer2012identification} although both a high
degree of genomic alterations and a loss of heterozygosity
were found to be associated with poor event\hyp{}free survival~\cite{kuijjer2012identification,smida2010genomic}.
Yen {\it et al}.~\cite{yen2009identification} found that specific aberrations were more frequent
in recurrences and metastases than in primary
tumors---deletion of 6q14.1-q22.31 and 8p23.2-p12 and
amplification of 8q21.12-q24.3 and 17p12---and vice versa---Xp11.22 gain and 13q31.3 deletion~\cite{yen2009identification}.

%
\section{Integrative analyses}\label{integrative2}
For high-grade osteosarcoma, integration of different data
types is of specific importance. Integrative analysis can narrow
down the large lists of significantly affected genes to a
gene list containing the major tumor driver genes. An integrative
approach on copy number and gene expression data,
for example, typically returns a more specific list of driver
genes because passenger- and tissue\hyp{}specific genes will be
largely eliminated~\cite{lee2008integrative}. Different methods exist for the integration
of different types of data. Figure~\ref{fig2.2} shows an overview of
direct dependencies between copy number, methylation,
miRNA and mRNA data.
%
\begin{figure}[htbp]
  \centering
  \begin{minipage}[b]{0.50\linewidth}
    \includegraphics[width=1\textwidth]{figs02/fig2bw.pdf}	% pdf version also bw
  \end{minipage}
    \hfill
  \begin{minipage}[b]{0.46\linewidth}
    \caption{Flow chart showing
direct dependencies between different data types, which can be
utilized for the interpretation of integrative analyses. Arrow\hyp{}headed
and bar\hyp{}headed lines show positive and negative influences,
respectively. DNA copy number positively affects miRNA and mRNA
copies, whereas miRNA expression can cause downregulation of
target mRNAs, and DNA methylation can inhibit transcription.}
     \label{fig2.2}
     \end{minipage}
\end{figure}
%
Comparison of data can be performed
nonpaired or paired, and by determining correlation
or cooccurrence.

Cooccurring genomic alterations and gene expression
changes have been recently determined to identify putative
driver genes in high\hyp{}grade osteosarcoma (Chapter~\ref{ch7},~\cite{kuijjer2012identification}). A paired integrative
analysis of 29 pretreatment biopsies returned a list of 31
genes with recurrence frequency of at least $35\%$, which showed an
overall significant upregulation as compared to control cell
lines in case of a gain, and downregulation in case of a deletion.
Genes affecting genomic stability were overrepresented,
which may point to a role of this process in osteosarcoma.
Nonpaired analysis on the same series, but extended with
more cases in both the SNP and the gene expression data
sets resulted in a smaller set of significantly affected genes,
with substantial overlap with the list of genes detected by the
paired analysis, thereby showing that the paired analysis was
more powerful on this data set. This is especially of interest
for the data analysis of osteosarcoma pretreatment biopsies
because these samples are rare. By performing a paired analysis,
fewer samples can be used. Nonpaired integrative analysis
of high\hyp{}level amplifications in 22 osteosarcoma specimens
with gene expression data of eight osteosarcoma xenografts
as compared to 19 normal tissue controls identified 43 genes
with high\hyp{}level amplification and overexpression in osteosarcoma.
{\it CCNE1}, the gene encoding for cyclin E1, showed correlation
of copy number levels and gene expression in an
additional panel of ten osteosarcoma cell lines, and therefore
could play an oncogenic role in osteosarcoma~\cite{lockwood2011cyclin}.

miRNA expression data can be integrated with mRNA
expression data to determine whether the miRNAs of interest
affect mRNA expression of their target genes. This is
generally performed by correlation of expression levels, as
was performed by Baumhoer {\it et al}.,~\cite{baumhoer2012microrna} Naml{\o}s {\it et al}.~\cite{namlos2012modulation} and Jones {\it et al}.~\cite{jones2012mirna} (discussed above). The latter subsequently performed
pathway analysis on target genes of the detected differentially
expressed miRNAs, which illustrated the effects
of these miRNAs on transcriptional regulation, cell\hyp{}cycle
control and known cancer signaling pathways~\cite{jones2012mirna}. In the
study by Naml{\o}s {\it et al}.~\cite{namlos2012modulation}, cell line miRNA data were integrated
with mRNA targets which were significant in both
osteosarcoma pretreatment biopsies and cell lines. Among
the inversely correlated miRNA/mRNA pairs, miRNAs regulating
{\it TGFBR2}, {\it IRS1}, {\it PTEN} and PI3K subunits were
detected. Methylation data are also typically integrated with
mRNA expression data to evaluate the effect of the methylation
on gene expression, but few studies described two\hyp{}way
comparisons of methylation and mRNA microarray data in
osteosarcoma. Kresse {\it et al}.~\cite{kresse2012integrative} detected hypermethylation and
underexpression of chemokine ligand 5 ({\it CXCL5}) by two\hyp{}way
comparison in both osteosarcoma cell lines and tumor
samples.

Integration of more than two different data types is
reported by Sadikovic {\it et al}.~\cite{sadikovic2008vitro} in two articles, where copy
number, methylation and gene expression data were integrated.
In one of these articles, the authors described cooccurrent
epigenetic, genomic and gene expression changes
in two osteosarcoma cell lines as compared to an osteoblast
culture, and detected a region of gain on chromosome 8q
encompassing the {\it c-MYC} oncogene, which was also detected
in a network analysis, confirming overexpression and hypomethylation
of genes connected to {\it c-MYC}. In the second article,
the authors used the same integrative approach to
perform a three\hyp{}way analysis on five osteosarcoma pretreatment
biopsies, and to compare gene regulation networks of
single\hyp{}way analyses including more samples. In this way, a
number of candidate genes were characterized, including
{\it RUNX2}, a transcription factor involved in osteoblastic differentiation~\cite{sadikovic2009identification}. A
shortcoming of both studies, however, is that
as a control for methylation and mRNA expression in
osteosarcoma, material from only one osteoblastic culture
was used. Another integrative analysis on copy number,
methylation and mRNA data reported 350 genes, showing
two types of aberrations ({\it e.g.} gain and overexpression, or hypermethylation
and underexpression). This set of genes was
enriched in genes with a function in skeletal system development
and extracellular matrix remodeling, such as {\it RUNX2}
and {\it DLX5}~\cite{kresse2012integrative}.

%
\section{Translating bioinformatics into functional studies}\label{translating2}
\subsection{Functional validation of candidate genes}
Several of the candidate tumor suppressor genes and oncogenes
that have been identified with microarray studies have
been functionally validated. {\it IGFBP5} was significantly downregulated
in metastatic cell lines and derivative tumors as compared
to nonmetastatic cell lines, and also showed lower
protein expression in metastatic lesions than in primary tumor
samples of osteosarcoma patients. The effects of overexpression
or knockdown of {\it IGFBP5} on cell proliferation,
migration, wound healing and invasion confirmed the role of
this IGF\hyp{}binding protein in preventing metastasis, which was
furthermore validated in a xenograft model~\cite{su2011insulin}.

The candidate tumor suppressor gene {\it WIF1} was found to
regulate differentiation and suppress cell growth {\it in vitro}.
{\it WIF1} knockout mice developed radiation\hyp{}induced osteosarcoma
earlier than their littermate controls~\cite{kansara2009wnt}. From miRNA
expression profiling studies, miR-16 was validated as a tumor
suppressive miRNA, whereas miR-27a was validated as a
prometastatic miRNA, using colony formation assays, and
wound healing and invasion assays, respectively. Overexpression
of these miRNAs {\it in vivo} resulted in smaller tumors for
miR-16, and in higher numbers of pulmonary metastases for
miR-27a~\cite{jones2012mirna}.

\subsection{Functional validation of pathway activity and enriched gene sets}
Pathways important in the development of bone biology have
been returned from gene expression analysis as compared to
controls. Genes upstream canonical Wnt signaling were, for
example, found to be downregulated as compared to osteoblasts~\cite{cleton2009profiling}. A
subsequent functional study, where nuclear $\upbeta$-catenin
staining was determined on osteosarcoma biopsies, and
Wnt luciferase activity and mRNA expression of the specific
downstream Wnt target gene Axin2 were measured in cell
lines, illustrated that canonical Wnt signaling is indeed often
downregulated in osteosarcoma~\cite{cai2010inactive}. Loss of canonical Wnt signaling
causes failure to commit to differentiation of MSCs, as
has been reported in malignant fibrous histiocytoma (also
undifferentiated pleomorphic sarcoma), which could be
reprogrammed by re\hyp{}establishing Wnt signaling~\cite{matushansky2007derivation}. Also in osteosarcoma,
reactivation of the Wnt signaling pathway with a
GSK3$\upbeta$ inhibitor triggered a more differentiated phenotype,
or a reduced proliferation capacity, depending on the osteosarcoma
cell line~\cite{cai2010inactive}.

These results seem contradictive to the finding that {\it WIF1}
can inhibit cell growth and increase differentiation in osteosarcoma
cells~\cite{kansara2009wnt}. A possible explanation for this discrepancy is
that {\it WIF1} inhibits both canonical and noncanonical Wnt signaling~\cite{malinauskas2011modular},
whereas GSK3$\upbeta$ also plays a role in additional signal
transduction pathways, such as NF$\upkappa$B signaling~\cite{tang2012glycogen}. However,
the role of Wnt signaling remains contradictory, as this pathway
was recently described to be active in multiple sarcoma
subtypes, which also included osteosarcoma~\cite{vijayakumar2011high}. The use of
different methods to assess active Wnt signaling may be the
cause for the discrepancies between these studies.

TGF-$\upbeta$/BMP signaling was found to be affected in osteosarcoma
by pathway analysis on mRNA expression data. Activity
of these pathways was validated by immunohistochemistry of
phosphorylated Smad1 and Smad2, and nuclear staining of
these intracellular effectors was detected in 70\% of all osteosarcoma
samples. Cases with very low or absent phosphorylated
Smad2 had worse overall survival. {\it In vitro} pathway modulation
did not affect proliferation or differentiation, but lower TGF$\upbeta$/BMP
activity might affect the prevention of metastasis in
these patients~\cite{mohseny2012activities}.

The macrophage signature that was prominent upon comparing
mRNA profiles of metastatic and nonmetastatic osteosarcoma
was confirmed by qPCR and immunohistochemistry,
and it was shown in additional cohorts that the sum of M1
and M2 types of macrophages was predictive for better overall
survival (Chapter~\ref{ch4},~\cite{buddingh2011tumor}). Treating patients with macrophage\hyp{}activating
agents may reduce metastases of osteosarcoma~\cite{cleton2012immunotherapy}. This is corroborated
by clinical trials in dogs and humans, where treatment
with mifamurtide, a macrophage\hyp{}activating agent, has
been reported to positively affect overall survival~\cite{kurzman1995adjuvant,meyers2008osteosarcoma}.

%
\section{Conclusions and future directions}\label{conclusions2}
In this review, we have presented and discussed the results of
studies on high\hyp{}grade osteosarcoma material using bioinformatic
analysis on microarray data of three or more samples.
Although studying such a very heterogeneous and genomically
unstable tumor remains challenging, and sample sizes
are often small owing to the rarity of the disease, structured
microarray data analysis has provided interesting results and
has given further insight into the biology and progression of
osteosarcoma. This information could not have been obtained
from functional studies only. Studying copy number aberrations,
differential expression, and epigenetics in a genome\hyp{}wide
manner and subsequent integration leads to new
hypotheses regarding tumor development and progression,
which can subsequently be validated in functional studies.
This provides a motivation to take the study of high\hyp{}grade
osteosarcoma to the next level, and to analyze this tumor
into further detail using Next Generation Sequencing methods,
such as whole\hyp{}genome, exome or transcriptome sequencing.
Whole\hyp{}genome sequencing has recently been performed
in a study of different cancer types, which showed that a subset
of osteosarcomas (three out of nine) undergo chromothripsis---a
single catastrophic genomic instability event,
resulting in hundreds of genomic rearrangements~\cite{stephens2011massive}. This may
explain the sudden onset of osteosarcoma and the complexity
and heterogeneity of the osteosarcoma genome. Next Generation
Sequencing will provide us with many forms of new information.
In addition to copy number changes, mutations,
translocations, unannotated genes, splicing variants, and so
on, can be detected in a high\hyp{}throughput manner. Transcriptome
sequencing exhibits higher sensitivity and increased
dynamic range than mRNA expression microarray data,
thereby providing higher power for the detection of differential
gene expression~\cite{oshlack2010rna}. Now that the first Next Generation
Sequencing studies including large numbers of high\hyp{}grade osteosarcoma
are ongoing or being planned, it is important to
reflect on the previous genome\hyp{}wide studies in osteosarcoma.
When keeping in mind the lessons we have learned on study
design in microarray data analysis---using a sufficient amount
of samples, defining homogeneous groups, and analyzing the
data with robust statistics---we will be given new opportunities
in unraveling the biology of this complex disease and in
providing future clinical trials with robust data to incorporate
into novel therapeutic strategies.

%%% references

\begin{small}
\begin{singlespace}
\bibliographystyle{unsrtnatshort}		% sorted as referenced, was unsrtnat, but unsrtnatshort gives shorter output
\bibliography{biblio}
\end{singlespace}
\end{small}

%%% appendix
% supporting information figure 1, now appendix figure 1
\begin{subappendices}
	\newpage
	\setcounter{figure}{0}
	\section{Additional Figures}
		\renewcommand{\figurename}{Additional Figure}
		%
		\begin{figure}[h]
		  \centering
		    \includegraphics[width=1\textwidth]{figs02/supfig1bw.pdf}	% pdf version also bw
		    \caption{This figure illustrates a histogram of nonadjusted, moderated p-values and the empirical
cumulative distribution of p-values for the studies of {\it A}, Cleton-Jansen {\it et al}.~\cite{cleton2009profiling} and {\it B}, Kuijjer {\it et al}.~\cite{kuijjer2011mrna}, both describing no significant difference in mRNA expression in pretreatment biopsies of patients with poor versus good response to chemotherapy. The figures were generated using Bioconductor package {\it SSPA}~\cite{van2009relative}.}
		     \label{afig2.1}
		\end{figure}
		%
\end{subappendices}

%\end{document}