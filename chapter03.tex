% Marieke Kuijjer
% 2013-02-15
% chapter 03

	%\documentclass[12pt,b5paper]{book}
	%\setcounter{secnumdepth}{0}
	%\setcounter{tocdepth}{1}
	%\usepackage[hidelinks]{hyperref}

%\begin{document}

%%% title page

\chapter{mRNA expression profiles of primary high-grade
central osteosarcoma are preserved in cell lines
and xenografts}\label{ch3}
\thispagestyle{empty}				%%% to remove page number from first page of chapter, must be placed after calling the chapter

\vfill

\vspace{0.5cm}
This chapter is based on the publication:
\underline{Kuijjer ML}, Naml{\o}s HM, Hauben EI, Machado I, Kresse SH, Serra M, Llombart-Bosch A, Hogendoorn PCW, Meza-Zepeda LA, Myklebost O, Cleton-Jansen AM. {\it BMC Med Genomics}. 2011 Sep 20;4:66

\newpage

%%% main document

\section{Abstract}\label{abstract3}
\textbf{Background}: Conventional high-grade osteosarcoma is a primary malignant bone tumor, which is most prevalent
in adolescence. Survival rates of osteosarcoma patients have not improved significantly in the last 25 years. Aiming
to increase this survival rate, a variety of model systems are used to study osteosarcomagenesis and to test new
therapeutic agents. Such model systems are typically generated from an osteosarcoma primary tumor, but undergo
many changes due to culturing or interactions with a different host species, which may result in differences in
gene expression between primary tumor cells, and tumor cells from the model system. We aimed to investigate
whether gene expression profiles of osteosarcoma cell lines and xenografts are still comparable to those of the
primary tumor.

\textbf{Methods}: We performed genome\hyp{}wide mRNA expression profiling on osteosarcoma biopsies ($n=76$), cell lines ($n=13$), and xenografts ($n=18$). Osteosarcoma can be subdivided into several histological subtypes, of which
osteoblastic, chondroblastic, and fibroblastic osteosarcoma are the most frequent ones. Using nearest shrunken
centroids classification, we generated an expression signature that can predict the histological subtype of
osteosarcoma biopsies.

\textbf{Results}: The expression signature, which consisted of 24 probes encoding for 22 genes, predicted the histological
subtype of osteosarcoma biopsies with a misclassification error of 15\%. Histological subtypes of the two
osteosarcoma model systems, {\it i.e.} osteosarcoma cell lines and xenografts, were predicted with similar
misclassification error rates (15\% and 11\%, respectively).

\textbf{Conclusions}: Based on the preservation of mRNA expression profiles that are characteristic for the histological
subtype we propose that these model systems are representative for the primary tumor from which they are
derived.

\section{Background}\label{introduction3}
Conventional high\hyp{}grade osteosarcoma is the most frequent
primary malignant bone tumor, with a peak
occurrence in children and adolescents and a second
peak in patients older than 40 years. It is a highly
genetically unstable tumor, of which karyotypes often
show aneuploidy, high level amplification and deletion,
and translocations~\cite{cleton2005central}. No precursor lesion is known,
although part of the osteosarcomas in patients over 40
years is secondary, and is induced by radiation, chemicals,
or by an underlying history of Paget's disease of
bone~\cite{raymond2002conventional}. The leading cause of death of osteosarcoma
patients are distant metastases, which despite aggressive
chemotherapy regimens still develop in approximately
45\% of all patients~\cite{buddingh2010prognostic}. Overall survival of osteosarcoma
patients has increased from 10--20\% before the introduction
of preoperative chemotherapy in the 1970s, to
about 60\%~\cite{rozeman2006pathology}. However, survival has reached a plateau,
and treating with higher doses of chemotherapy does
not lead to better overall survival~\cite{lewis2007improvement}.

Osteosarcoma is a heterogeneous tumor type, which
can be subdivided into various subtypes~\cite{mohseny2008bone}. Conventional
high\hyp{}grade osteosarcoma is the most common
subtype, and can be further subdivided in different histological
subtypes, of which osteoblastic (50\%), chondroblastic
(25\%), and fibroblastic osteosarcoma (25\%) are
the most frequent ones. Other subtypes of conventional
high\hyp{}grade osteosarcoma, such as chondromyxoid
fibroma\hyp{}like, clear cell, epitheliod, sclerosing, and giant
cell rich osteosarcoma, are extremely rare~\cite{raymond2002conventional}. Often,
osteosarcoma tissue contains a mixture of morphologically
differing cell types, and the classification is based
on the most dominant type~\cite{hauben2002does}. The three main histological
subtypes have different survival profiles. Patients
with fibroblastic osteosarcoma have a significantly better
response to preoperative chemotherapy, which is a
known predictor for improved survival, than do osteoblastic
osteosarcoma patients~\cite{huvos1991bone}. Although patients with
chondroblastic osteosarcoma are relatively poor responders
to preoperative chemotherapy~\cite{hauben2002does,bacci1998predictive}, which is probably
caused by the impermeability of the chondroid
areas of the tumor, there is a trend for these patients to
have better 5-year survival profiles~\cite{hauben2002does}, but also a higher
risk for late relapse~\cite{hauben2006clinico}.

The search for new (targeted) therapies to treat osteosarcoma
is ongoing~\cite{hattinger2010emerging}. Because the disease is relatively
rare, large efforts need to be done in order to collect a
considerable amount of patient samples. Moreover,
material is usually scarce due to necrosis in resections
of the primary tumor, which is mostly present in tumors
of patients who respond fairly well to neoadjuvant chemotherapy.
No necrosis is present in prechemotherapy
biopsies, but these are often very small and are not
readily available for research because they are needed
for diagnosis. Because of these limitations, model systems
are widely used to study osteosarcomagenesis and
for preclinical testing of candidate drugs. Osteosarcoma
cell lines, especially SAOS-2 and U-2 OS are frequently
used as model systems, remarkably not only to study
osteosarcoma, but all types of {\it in vitro} cell biological
processes, as these cell lines grow fast and are relatively
easy to transfect. Recently, the EuroBoNeT (\url{www.eurobonet.eu}) osteosarcoma panel of 19 cell lines was
characterized, which allows us to study osteosarcoma in
a high\hyp{}throughput manner~\cite{ottaviano2010molecular}. This panel of osteosarcoma
cell lines has been shown to resemble osteosarcoma
phenotypically and functionally~\cite{mohseny2011functional}. Other
established model systems include xenografts from primary
tumors or osteosarcoma cell lines in immunodeficient
nude mice, which subsequently develop into
tumors resembling osteosarcoma~\cite{mohseny2011functional,mayordomo2010tissue,kresse2011preclinical}. Osteosarcomagenesis
can also be induced in mice by radiation or
orthotopically implanting chemical carcinogens~\cite{jones2011osteosarcomagenesis}. We
have previously shown that DNA copy number profiles
of xenografts resemble those of the corresponding primary
tumor, although some significant changes for
osteosarcoma were observed~\cite{kresse2011preclinical}.

Established cancer cell lines are often thought not to
be representative for the originating primary tumor.
Since there could have been a selection for their propensity
to grow in culture, they lack interaction with
stroma and may have acquired additional mutations in
culture~\cite{weinberg2007biology}. Xenografts do have tumor--host interactions,
but can lose matrix as well after several passages. It is
not clear whether such changes in matrix composition
of xenografts are caused by the tumor cells, or by
changes in mouse stroma~\cite{mayordomo2010tissue}. Despite these biological
differences, model systems are useful for studying signal
transduction pathways important in tumor biology, of
which mRNA expression, as measured by qPCR or
using gene expression microarrays, is frequently used as
a readout. It is therefore highly important to determine
whether gene expression levels of these model systems
are comparable to those of the corresponding primary
tumors, which we aimed to do in this study. We performed
gene expression analysis on a panel of 76 conventional
high\hyp{}grade osteosarcoma pretreatment
biopsies. We set out to recapitulate representative
expression profiles from primary untreated osteosarcoma
biopsies in corresponding models {\it i.e.} cell lines
and xenografts. We could demonstrate that both model
systems still express genes that are characteristic for the
specific histological subtype of the primary tumor. We
therefore endorse that, despite biological differences,
both xenografts and cell lines are representative model
systems for studying mRNA expression in high\hyp{}grade
osteosarcoma. Specific models may be identified that
would be appropriate to use for studies of specific subgroups
of osteosarcoma.

\section{Methods}\label{methods3}
\subsection{Ethics statement}
All biological material was handled in a coded fashion.
Ethical guidelines of the individual European partners
were followed and samples and clinical data were stored
in the EuroBoNet biobank. For xenograft experiments,
informed consent and sample collection were approved
by the Ethical Committee of Southern Norway (Project
S-06132) and the Institutional Ethical Committee of
Valencia University.

\subsection{Patients cohorts}
Genome\hyp{}wide expression profiling was performed on
pretreatment diagnostic biopsies of 76 resectable highgrade
osteosarcoma patients from the EuroBoNet consortium
(\url{www.eurobonet.eu}). Clinicopathological
details of these 76 samples can be found in Table~\ref{tab3.1}.
Samples with a main histological subtype ($n=66$) were
selected for subsequent subtype analyses. These 66 samples
included 50 osteoblastic, 9 chondroblastic, and 7
fibroblastic osteosarcomas. Five additional osteosarcoma
biopsies (1 chondroblastic and 4 osteoblastic osteosarcomas),
12 mesenchymal stem cell (MSC) and 3 osteoblast
cultures, and 12 chondrosarcoma biopsies were used for
validation.
%
\begin{landscape}
	\begin{table}[htbp]
		\centering
		\small
		\begin{tabular}[c]{|p{4.5in}cccc|} % OBS! 4.5in is entire page length % 4.2in gives same length as table 3.2
			\hline
			Category & Patient characteristics & Biopsies (\%) & Cell lines (\%) & Xenografts (\%)\\
			\hline
			Total nr of samples & & 76 (100) & 13 (100) & 18 (100)\\
			Institution & LUMC, Netherlands & 29 (38.2) & 0 (0) & 0 (0)\\
			& IOR, Italy & 11 (14.5) & 7 (53.8) & 0 (0)\\
			& LOH, Sweden & 3 (3.9) & 0 (0) & 0 (0)\\
			& Radiumhospitalet, Norway & 1 (1.3) & 3 (23.1) & 12 (66.7)\\
			& UV, Spain & 0 (0) & 0 (0) & 6 (33.3)\\
			& WWUM, Germany & 32 (42.1) & 0 (0) & 0 (0)\\
			& Other & 0 (0) & 3 (23.1) & 0 (0)\\
			Origin & Biopsy & 76 (100) & 0 (0) & 0 (0)\\
			& Resection & 0 (0) & 7 (53.8) & 11 (61.1)\\
			& Metastasis & 0 (0) & 3 (23.1) & 1 (5.6)\\
			& Unknown & 0 (0) & 3 (23.1) & 6 (33.3)\\
			Location of primary tumor & Femur & 36 (47.4) & 0 (0) & 10 (55.6)\\
			& Tibia/fibula & 26 (34.2) & 0 (0) & 2 (11.1)\\
			& Humerus & 10 (13.2) & 0 (0) & 2 (11.1)\\
			& Axial skeleton & 1 (1.3) & 0 (0) & 1 (5.6)\\
			& Unknown/other & 3 (3.9) & 13 (100) & 3 (16.7)\\
			Histological subtype & Osteoblastic & 50 (65.8) & 9 (69.2) & 15 (83.3)\\
			& Chondroblastic & 9 (11.8)&  0 (0) & 3 (16.7)\\
			& Fibroblastic & 7 (9.2) & 4 (30.8) & 0 (0)\\
			& Minor & 10 (13.2) & 0 (0) & 0 (0)\\
			Histological response to preoperative chemotherapy in the primary tumor & Good response & 33 (43.4) & 0 (0) & 0 (0)\\
			& Poor response & 36 (47.4) & 0 (0) & 0 (0)\\
			& Unknown/NA & 7 (9.2) & 13 (100) & 18 (100)\\
			Sex & Male & 52 (68.4) & 9 (69.2) & 9 (50)\\
			& Female & 24 (31.6) & 4 (30.8) & 3 (16.7)\\
			& Unknown & 0 (0) & 0 (0) & 6 (33.3)\\
			\hline
		\end{tabular}
		\caption{Clinicopathological details of patients with conventional high\hyp{}grade osteosarcoma, including all patients from the biopsy, cell line, and xenograft datasets.}
		\label{tab3.1}
	\end{table}
\end{landscape}
%

\subsection{Osteosarcoma cell lines}
Out of the EuroBoNeT panel of 19 cell lines, 13 cell
lines were recorded to belong to a main histological
subtype. This set of 13 cell lines contained 4 cell lines
derived from fibroblastic, and 9 cell lines derived from
osteoblastic osteosarcomas. The 13 osteosarcoma cell
lines IOR/MOS, IOR/OS10, IOR/OS14, IOR/OS15,
IOR/OS18, IOR/OS9, IOR/SARG, KPD, MG-63, MHM,
OHS, OSA, and ZK-58 were maintained in RPMI 1640
(Invitrogen, Carlsbad, CA, USA) supplemented with 10\%
fetal calf serum and 1\% Penicillin/Streptomycin (Invitrogen)
as previously described~\cite{ottaviano2010molecular}. Clinical details of the
tissue from which these cell lines were derived are
shown in Table~\ref{tab3.1} and are described previously~\cite{ottaviano2010molecular}.

\subsection{Osteosarcoma xenografts}
The osteosarcoma xenograft model is described in
Kresse {\it et al}.~\cite{kresse2011preclinical}. In short, human tumors were
implanted directly from patient samples and successively
passaged subcutaneously in nude mice. Eighteen different
xenografts were used, of which 3 were derived from
chondroblastic, and 15 from osteoblastic osteosarcomas.
Clinical data on primary tumor samples and xenograft
passages that were used are shown in Table~\ref{tab3.1}.

\subsection{Determination of histological subtypes}
Histological subtyping was performed by two pathologists
(PCWH, EH) on hematoxylin and eosin (HE)
stained slides of all biopsies and of all primary tumors
from which the osteosarcoma cell lines and xenografts
were derived. Osteoblastic, chondroblastic, and fibroblastic
osteosarcoma samples were selected for further
study. Other subtypes (anaplastic, chondromyxoid
fibroma\hyp{}like, fibroblastic MFH-like, giant cell rich, pleomorphic,
and sclerosing osteosarcoma) were excluded
because these subtypes are rare.

\subsection{RNA isolation, cDNA synthesis, cRNA amplification, and Illumina Human-6 v2.0 Expression BeadChip hybridization}
Osteosarcoma and xenograft tissue was handled as previously
described~\cite{buddingh2011tumor}. Osteosarcoma cell lines were prepared
as in Ottaviano {\it et al}.~\cite{ottaviano2010molecular}. RNA isolation,
synthesis of cDNA, cRNA amplification, and hybridization
of cRNA onto the Illumina Human-6 v2.0 Expression
BeadChips were performed as previously described~\cite{buddingh2011tumor}.

\subsection{Microarray data preprocessing}
Microarray data processing and quality control were
performed using the statistical language R~\cite{r2.10.0} as
described previously~\cite{buddingh2011tumor}. MIAME\hyp{}compliant data have
been deposited in the GEO database (\url{www.ncbi.nlm.nih.gov/geo/}, accession number GSE30699). High
correlations between these microarray data and corresponding
qPCR results have been demonstrated previously~\cite{buddingh2011tumor}.

\subsection{Detection of significantly differentially expressed genes}
We performed {\it LIMMA} analyses~\cite{smyth2004linear} in order to determine
differential expression for the following clinical
parameters: sex (52 male {\it vs} 24 female), tumor location
(36 femur, 10 humerus, 26 fibula/tibia), response to preoperative
chemotherapy (36 poor responders, or Huvos
grade 1--2, {\it vs} 33 good responders, or Huvos grade 3--4),
and histological subtype (an analysis comparing
50 osteoblastic, 9 chondroblastic, and 7 fibroblastic
osteosarcomas). Genes that play a role in metastasis\hyp{}free
survival are described in Buddingh {\it et al}.~\cite{buddingh2011tumor}. Probes
with Benjamini and Hochberg False discovery rate\hyp{}adjusted
p-values (adjP) $< 0.05$ were considered to be
significantly differentially expressed.

\subsection{Prediction analysis}
The gene expression profile was generated on the dataset
of biopsies using Bioconductor~\cite{gentleman2004bioconductor} package {\it pamr}~\cite{tibshirani2002diagnosis}. Internal cross\hyp{}validation was performed 50 times.
A threshold was selected where the error rate of the
prediction profile was minimal. The minimum error rate
was representative of 50 independent simulations. In
order to minimize optimization bias~\cite{wood2007classification}, we validated
the profile on an independent dataset of osteosarcoma
biopsies ($n=5$), containing 1 chondroblastic
osteosarcoma and 4 osteoblastic osteosarcomas. In addition,
we applied the profile on datasets containing positive
controls---mesenchymal stem cells (MSC, $n=12$),
osteoblasts ($n=3$), and chondrosarcoma biopsies ($n=12$, previously published in~\cite{hallor2009genomic}, GEO accession number
GSE12532). We subsequently applied the validated prediction
profile to two independent datasets, the first
consisting of gene expression data of osteosarcoma cell
lines, the second of xenografts. Expression of the probes
that composed the prediction profile was verified using
a {\it LIMMA} analysis, comparing chondroblastic,
fibroblastic, and osteoblastic osteosarcoma biopsy
samples.

\subsection{Gene set enrichment}
Network analysis was performed using Ingenuity Pathways
Analysis (IPA, Ingenuity Systems, \url{www.ingenuity.com}). For both chondroblastic\hyp{}specific and
fibroblastic\hyp{}specific analyses, data for all reference
sequences containing expression values and FDR\hyp{}adjusted
p-values were uploaded into the application.
Each identifier was mapped to its corresponding object
in Ingenuity's Knowledge Base. An adjP cut-off of 0.05
was set to select genes whose expression was significantly
differentially regulated. The Network Eligible
molecules were overlaid onto a global molecular network
developed from information contained in Ingenuity's
Knowledge Base. Networks of Network Eligible
Molecules were then algorithmically generated based on
their connectivity. GO term enrichment was tested
using Bioconductor package {\it topGO}~\cite{alexa2006improved}. Lists of significantly
affected genes were compared with all genes eligible
for the analysis. GO terms with Fisher's exact p-values
$< 0.001$, as calculated by the {\it weight} algorithm
from {\it topGO}, were defined significant.

\section{Results}\label{results3}
\subsection{Histological subtypes of osteosarcoma biopsies have different gene expression profiles}
We determined differential expression for different clinical
parameters. Of all comparisons of clinical parameters
only histological subtypes appeared to give a sufficient
number of differentially expressed genes to design a prediction
profile. {\it LIMMA} analyses resulted in one location\hyp{}specific differentially expressed gene: {\it HOXD4},
which was overexpressed in tumors at the humerus versus
at fibula/tibia and femur. Between tumors from
male and female patients, 18 genes were significantly
differentially expressed, all belonging to X- and Y\hyp{}chromosome\hyp{}specific genes, which are not considered as
representative for osteosarcoma, yet this finding validates
the analysis. No significantly affected genes were
returned with regards to response to preoperative chemotherapy.
To determine differential expression
between the three main histological subtypes, we
excluded all samples with unknown or rare subtypes.
This resulted in a dataset of 66 conventional high\hyp{}grade
osteosarcoma biopsies with a main histological subtype.
Using a {\it LIMMA} analysis, we determined $1,338$
significantly differentially expressed genes (adjP $<0.05$)
that were specific for a certain main histological subtype
(depicted in a Venn diagram in Figure~\ref{fig3.1}).% figure title is `Subtype\hyp{}specific genes'
%
\begin{figure}[htbp]
  \centering
  \begin{minipage}[b]{0.50\linewidth}
    \includegraphics[width=1\textwidth]{figs03/fig1bw.pdf}		% OBS! print version bw
%   \includegraphics[width=1\textwidth]{figs03/fig1rgb.pdf}	% OBS! pdf version rgb
  \end{minipage}
    \hfill
  \begin{minipage}[b]{0.46\linewidth}
%			%%% OBS! caption for fig3.1 print version:
     \caption{Venn diagram representing numbers of fibroblastic-, \mbox{chondroblastic-,} and osteoblastic\hyp{}specific differentially expressed genes (shaded regions) obtained with {\it LIMMA} analysis, considering chondroblastic versus osteoblastic (chondro {\it vs} osteo), fibroblastic versus osteoblastic (fibro {\it vs} osteo), and chondroblastic versus fibroblastic (chondro {\it vs} fibro) analyses. Subtype\hyp{}specific genes are genes that are either both upregulated or both downregulated in the subtype of interest in the different comparisons.}
%			%%% OBS! caption for fig3.1 pdf version:
%    \caption{Venn diagram representing numbers of fibroblastic- (green), chondroblastic- (red), and osteoblastic (blue)\hyp{}specific differentially expressed genes obtained with {\it LIMMA} analysis, considering chondroblastic versus osteoblastic (chondro {\it vs} osteo), fibroblastic versus osteoblastic (fibro {\it vs} osteo), and chondroblastic versus fibroblastic (chondro {\it vs} fibro) analyses. Subtype\hyp{}specific genes are genes that are either both upregulated or both downregulated in the subtype of interest in the different comparisons.}
     \label{fig3.1}
     \end{minipage}
\end{figure}
%
%
%
%
A subtype\hyp{}specific
probe was defined as a probe that had the same
sign of log fold change in both analyses, {\it e.g.} the probe
was upregulated in chondroblastic samples versus osteoblastic,
and in chondroblastic versus fibroblastic
samples.

\subsection{Gene set enrichment shows specific sets of genes are affected in fibroblastic and chondroblastic osteosarcoma}
Network analysis using IPA showed that fibroblastic
osteosarcoma\hyp{}specific networks most\-ly had a role in cellular
growth and proliferation, which was also the most
significant biological function as detected by IPA (see
Additional File 1 ({\it available online}~\cite{ch3additional}) for all affected networks and biological
functions). The most significant network is illustrated in
Additional File 2A ({\it available online}~\cite{ch3additional}) and shows that mRNA of various
genes with a connection to the NF-$\upkappa$B pathway and
STAT5A signaling are upregulated in fibroblastic osteosarcoma
biopsies, as compared with both osteoblastic
and chondroblastic osteosarcoma. The most significant
network specific for the chondroblastic subtype consisted
of genes important in skeletal connective tissue
development and function (Additional File 2B ({\it available online}~\cite{ch3additional})), and
shows that, also on the gene expression level, chondroblastic
osteosarcoma is mainly distinguished from osteoblastic
and fibroblastic osteosarcoma based on the
composition of the extracellular matrix of the tumor
(Additional File 1 ({\it available online}~\cite{ch3additional}) shows all affected networks and biological
functions).

Results from network analysis were confirmed using
{\it topGO}, a gene set enrichment approach analyzing the significance
of GO terms in a specific dataset. These analyses
resulted in two significant fibroblastic specific GO terms
in osteosarcoma: regulation of tyrosine phosphorylation of
Stat5 protein (GO:0042522, p-value $=4.8\cdot10^{-4}$) and regulation of
survival gene product expression (GO:0045884, p-value $=8.2\cdot10^{-
4}$). Significantly differentially expressed genes from both
GO terms partly overlap the fibroblastic osteosarcoma\hyp{}specific
network detected with IPA. Two GO terms were
significant in the chondroblastic\hyp{}specific analysis as well:
skeletal system development (GO:0001501), and extracellular
matrix organization (GO:0030198), which strengthen
the results found in the IPA network analyses. GO term
subgraphs of the five most significant GO terms for both
analyses are shown in Additional File 3 ({\it available online}~\cite{ch3additional}).

Gene set enrichment on genes specific for osteoblastic
osteosarcoma was not performed, because only one
osteoblastic osteosarcoma\hyp{}specific probe was detected
that distinguishes the osteoblastic subtype from fibroblastic
and chondroblastic. This probe matches to
{\it UNQ1940}, or {\it FAM180A}, a protein\hyp{}coding gene with a
yet unknown function.

\subsection{Generation and validation of the prediction profile}
Because we could not directly compare subtype\hyp{}specific
genes between our different model systems due to small
sample sizes, we generated a profile that could predict
the histological subtype of osteosarcoma. The prediction
profile was generated on 66 high\hyp{}grade conventional
osteosarcoma prechemotherapy biopsies, using nearest
shrunken centroids classification. Optimal control of
error rate in the prediction profile was found at delta
thresholds of $4.9-5.1$ (Figure~\ref{fig3.2}A), where merely 10 out
of 66 samples (15\%) in the training set were wrongly
assigned to a specific histological subtype.
%
\begin{figure}[htbp]
	\centering
	\includegraphics[width=1.0\textwidth]{figs03/fig2bw.pdf}	% OBS! print version bw
%	\includegraphics[width=1.0\textwidth]{figs03/fig2rgb.pdf}	% OBS! pdf version rgb
	\caption{{\it A}, Illustration of training the {\it pamr} prediction profile on osteosarcoma biopsies. At thresholds of $4.9-5.1$, the misclassification error rate was minimal. {\it B}, True versus predicted values from the nearest shrunken centroid fit. {\it C}, Probabilities of each biopsy to belong to any of the three histological subtypes. Samples are separated (dotted lines) based on their true subtypes. Cross\hyp{}validated probabilities for each histological subtype are shown on the y-axis, so that for every sample three open circles are present (black, gray, and light gray circles for \mbox{osteo-,} chondro-, and fibroblastic osteosarcoma, respectively). A sample is classified into a specific subtype if the probability to belong to that specific subtype is higher than the probabilities to belong to the other subtypes. {\it D}, The FDR plotted against different thresholds of the prediction profile. At a threshold of 5.0, 24 genes are included in the prediction profile. These 24 genes have an FDR $< 5\%$.} % OBS! bw caption for print file
	% 	\caption{{\it A}, Illustration of training the {\it pamr} prediction profile on osteosarcoma biopsies. At thresholds of $4.9-5.1$, the misclassification error rate was minimal. {\it B}, True versus predicted values from the nearest shrunken centroid fit. {\it C}, Probabilities of each biopsy to belong to any of the three histological subtypes. Samples are separated (dotted lines) based on their true subtypes. Cross\hyp{}validated probabilities for each histological subtype are shown on the y-axis, so that for every sample three open circles are present (blue, red, and green circles for \mbox{osteo-,} chondro-, and fibroblastic osteosarcoma, respectively). A sample is classified into a specific subtype if the probability to belong to that specific subtype is higher than the probabilities to belong to the other subtypes. {\it D}, The FDR plotted against different thresholds of the prediction profile. At a threshold of 5.0, 24 genes are included in the prediction profile. These 24 genes have a FDR $< 5\%$.} % OBS! rgb caption for pdf file
	\label{fig3.2}
\end{figure}
%
This error
rate was representative for a set of 50 simulations,
which resulted in error rates between 13.5\% and 15\%.
Subtype\hyp{}specific error rates were 22\%, 43\%, and 10\% for
chondroblastic, fibroblastic, and osteoblastic subtypes,
respectively (Figure~\ref{fig3.2}B). Probabilities of each sample to
belong to any of the three histological subtypes are
shown in Figure~\ref{fig3.2}C. At a threshold delta of 5.0, the prediction
profile consisted of 24 probes encoding for 22
genes. All genes were below a FDR threshold of 5\% (Figure~\ref{fig3.2}D). Expression of the 24 probes of the profile were
verified in a {\it LIMMA} analysis which was corrected
for multiple testing. All 24 probes were confirmed
to be significantly differentially expressed in the
{\it LIMMA} analysis as well. Results from {\it pamr} and
{\it LIMMA} analyses are shown in Table~\ref{tab3.2}. A supervised
heatmap depicting expression of the 24 probes in all
samples is shown in Additional File 4 ({\it available online}~\cite{ch3additional}). The prediction
profile was validated at threshold delta of 5.0 in an independent
dataset of osteosarcoma biopsies and positive
controls. Histological subtypes of biopsies had a prediction
error of 0\% (0/5). Mesenchymal stem cells and
osteoblasts all fitted in the osteoblastic group, while
11/12 chondrosarcoma samples were best corresponding to
the group of chondroblastic osteosarcoma. The remaining
chondrosarcoma sample was a dedifferentiated
chondrosarcoma and was predicted in the fibroblastic
group, probably because of the high amount of spindle
cells present in the biopsy. Additional File 5 ({\it available online}~\cite{ch3additional}) shows prediction
probabilities for each subtype of these additional
datasets.
%
\subsection{A prediction profile based on osteosarcoma biopsy data can predict histological subtypes of cell lines and xenografts}
Unsupervised clustering of all biopsies, xenografts, and
cell lines demonstrated that xenografts and cell lines
show different overall expression profiles from most
biopsies, and that there are no subtype\hyp{}specific clusters
based on overall expression (Additional File 6 ({\it available online}~\cite{ch3additional})). In order
to determine whether histological subtypes of cell lines
and xenografts could be predicted as well with the 24-gene prediction profile, we applied this profile to two
independent datasets. In the first dataset, consisting of
osteosarcoma cell line data, 2 out of 13 samples (15\%,
Figure~\ref{fig3.3}A) were wrongly classified.
%
These samples were
MG63, a cell line derived from a fibroblastic osteosarcoma,
which was subtyped as being osteoblastic, and
IOR/OS18, derived from an osteoblastic osteosarcoma,
which was subtyped by the prediction profile as a fibroblastic
osteosarcoma. Interestingly, HOS, HOS-MNNG,
and HOS-143B, all cell lines derived from the HOS cell
line, which is derived from fibroblastic and epithelial
osteosarcoma and therefore was not added to our set of
13 osteosarcoma cell lines, were all predicted as fibroblastic
osteosarcoma ({\it data not shown}). Two out of 18
xenograft samples (11\%, Figure~\ref{fig3.3}B) were wrongly
classified. One of these samples was OKx, a xenograft
derived from a chondroblastic tumor, which was subtyped
as an osteoblastic osteosarcoma. The other sample
was KPDx, a xenograft derived from an osteoblastic
tumor, which was subtyped as fibroblastic. The KPD cell
line was subtyped rightly as an osteoblastic
osteosarcoma.
%
\begin{landscape}
	\begin{table}[htbp]
		\centering
		\small
		\begin{tabular}{|l l >{\raggedleft}p{0.8in} >{\raggedleft}p{0.8in} >{\raggedleft}p{0.8in} l l l l|}
			\hline
			probeID & symbol & {\it LIMMA} logFC C{\it vs}F & {\it LIMMA} logFC C{\it vs}O & {\it LIMMA} logFC F{\it vs}O & {\it LIMMA} adjP & {\it pamr} chondro\hyp{}score & {\it pamr} fibro\hyp{}score & {\it pamr} osteo\hyp{}score\\
			\hline
			5910377 & {\it ACAN} & 2.42 & 2.24 & -0.18 & 0.0000 & 0.9294 & 0 & -0.0147\\
			3390678 & {\it NFE2L3} & -1.74 & -0.02 & 1.71 & 0.0000 & 0 & 0.9184 & 0\\
			1990523 & {\it COL9A1} & 3.49 & 3.01 & -0.47 & 0.0000 & 0.6011 & 0 & 0\\
			360553 & {\it SCRG1} & 4.55 & 3.51 & -1.04 & 0.0001 & 0.4571 & 0 & 0\\
			3310368 & {\ itID3} & 1.87 & -0.29 & -2.16 & 0.0003 & 0 & -0.4053 & 0\\
			10561 & {\it ITIH5L} & 0.68 & 0.65 & -0.04 & 0.0001 & 0.295 & 0 & 0\\
			5050110 & {\it MGC34761} & 0.93 & 0.83 & -0.09 & 0.0002 & 0.2818 & 0 & 0\\
			4780368 & {\it ACAN} & 1.34 & 1.19 & -0.14 & 0.0004 & 0.2716 & 0 & 0\\
			7150719 & {\it COL2A1} & 4.82 & 4.36 & -0.47 & 0.0016 & 0.183 & 0 & 0\\
			3830341 & {\it LYRM1} & -1.23 & -0.18 & 1.06 & 0.0007 & 0 & 0.1677 & 0\\
			3990500 & {\it MATN4} & 1.96 & 1.69 & -0.27 & 0.0012 & 0.151 & 0 & 0\\
			4280370 & {\it POPDC3} & -0.88 & -0.08 & 0.80 & 0.0009 & 0 & 0.0909 & 0\\
			6520487 & {\it UNQ830} & 4.10 & 2.90 & -1.20 & 0.0016 & 0.0817 & 0 & 0\\
			2850202 & {\it COL11A2} & 1.37 & 1.10 & -0.27 & 0.0014 & 0.0735 & 0 & 0\\
			4220452 & {\it C11ORF41} & -0.89 & -0.03 & 0.86 & 0.0011 & 0 & 0.0721 & 0\\
			4560091 & {\it COL9A3} & 1.14 & 1.21 & 0.07 & 0.0018 & 0.0698 & 0 & 0\\
			5890452 & {\it LOC652881} & 0.43 & 0.37 & -0.06 & 0.0001 & 0.0666 & 0 & 0\\
			3990259 & {\it PPP2R2B} & -1.00 & 0.10 & 1.10 & 0.0016 & 0 & 0.0603 & 0\\
			5340392 & {\it MAN2A1} & -1.42 & -0.22 & 1.20 & 0.0018 & 0 & 0.0477 & 0\\
			3360139 & {\it DLX5} & 1.84 & -0.20 & -2.04 & 0.0033 & 0 & -0.0358 & 0\\
			2630762 & {\it C14ORF78} & -1.07 & 1.45 & 2.52 & 0.0011 & 0 & 0 & -0.0307\\
			3460037 & {\it UNQ1940} & 0.44 & 1.71 & 1.27 & 0.0018 & 0 & 0 & -0.0219\\
			6110722 & {\it COL2A1} & 1.22 & 1.44 & 0.22 & 0.0032 & 0.0087 & 0 & 0\\
			6980164 & {\it ALPL} & 2.52 & -0.67 & -3.19 & 0.0038 & 0 & -0.0036 & 0\\
			\hline
		\end{tabular}
		\caption{Comparison of the 24 genes obtained with {\it pamr} prediction with a {\it LIMMA} analysis between the three different histological subtypes (CvsF: chondroblastic {\it vs} fibroblastic, CvsO: chondroblastic {\it vs} osteoblastic, FvsO: fibroblastic {\it vs} osteoblastic osteosarcoma), for which log fold changes (logFC) are shown for the different coefficients of the analysis. Note that the adjP shows the significance for the whole {\it LIMMA} analysis, and does not reflect the adjPs per subanalysis.}
		\label{tab3.2}
	\end{table}
\end{landscape}
%
%
\begin{figure}[htbp]
	\centering
	\includegraphics[width=1.0\textwidth]{figs03/fig3bw.pdf}	% OBS! print version bw
%	\includegraphics[width=1.0\textwidth]{figs03/fig3rgb.pdf}	% OBS! pdf version rgb
	\caption{Probabilities of {\it A}, cell lines and {\it B}, xenografts to belong to any of the three histological subtypes. For an explanation of what is represented by these graphs, see Figure~\ref{fig3.2}{\it C}.}
	\label{fig3.3}
\end{figure}
%

\section{Discussion}\label{discussion3}
In this study, we aimed to compare gene expression
profiles of osteosarcoma biopsies with cell lines and
xenografts, in order to investigate whether these model
systems are representative for the primary tumor. We
have determined differential gene expression for different
clinical parameters important in high\hyp{}grade osteosarcoma
on a dataset consisting of 76 conventional
high\hyp{}grade osteosarcoma samples. Importantly, pretreatment
biopsies were used instead of resected specimens,
because preoperative chemotherapy may cause
tumor necrosis in responsive patients, thus altering gene
expression and hampering the generation of high quality
mRNA. We intended to generate a gene expression profile
that could not only predict a specific clinical parameter
in biopsies, but in osteosarcoma cell lines and
xenografts as well. The metastasis/survival profile is
described previously and may serve as a tool to predict
prognosis and as a target for therapy~\cite{buddingh2011tumor}. However,
since most of the genes associated with osteosarcoma
metastasis were macrophage associated, and no stroma
or infiltrate is present in cell lines, this profile could not
be applied to osteosarcoma cell lines. We therefore
compared gene expression profiles of these different
sample sets based on other clinical parameters. No significant
differentially expressed genes were found
between poor and good responders to chemotherapy.
Several reports on genome\hyp{}wide expression profiling in
osteosarcoma have been published describing detection
of differential expression between poor and good
responders of preoperative chemotherapy~\cite{mintz2005expression,ochi2004prediction,man2005expression,salas2009molecular}. However,
the cohorts used in these studies are all relatively
small ($n=$ 13--30), and, more importantly, the reported
p-values were not corrected for multiple testing in these
studies. Remarkably, only two of the genes that were
found to correlate with response to chemotherapy in
these studies overlap, and one of these two genes was
upregulated in poor responders in one study, whereas it
was upregulated in good responders in the other study~\cite{mintz2005expression,salas2009molecular}. Another report described differential expression
between a metastatic and a nonmetastatic cell line, for
which metastatic capacity correlates with response to
chemotherapy~\cite{walters2008identification}. In that particular study, four genes
out of 252 were found to overlap with a patient study
by Mintz {\it et al}.~\cite{mintz2005expression}. However, the up- and downregulation
of these four genes were not consistent between
the two studies. We clearly show in a large cohort that
there are no differences between these groups of
patients, as the most significant probe had an adjP of
0.9998. These results are in line with our previous findings
obtained by analyzing an osteosarcoma cohort on a
different platform~\cite{cleton2009profiling}. The parameter `histological subtypes'
resulted in a considerable number of differentially
expressed genes. Our prediction profile is not directly
applicable to other platforms, but there is no real need
to have a prediction profile for primary osteosarcoma
histological subtype, since pathologists are very well able
to assess this on an HE-section, even on a biopsy, with
a concordance of 98\% between histological subtype of
biopsies and corresponding resections~\cite{hauben2002does}. Yet, we here
show a quite important use of this profile, {\it i.e.} to determine
the histological subtype of cell lines and xenografts.
{\it In vitro} 2\hyp{}dimensional growing cells lack extra
cellular matrix formation, which is the characteristic feature
to distinguish histological subtypes in high\hyp{}grade
central osteosarcoma.

The gene expression profiles as detected by analyzing
osteosarcoma biopsy data show a large number of subtype\hyp{}specific differentially expressed genes. In particular,
fibroblastic osteosarcoma differed most from the two
other main subtypes. Using gene set enrichment, we
detected a network of genes upregulated in fibroblastic
osteosarcoma, with a role in cellular growth and proliferation,
and connection to the NF-$\upkappa$B pathway. This
may be a readout of the high cellularity and low matrix
composition of fibroblastic osteosarcoma in comparison
with osteoblastic and chondroblastic osteosarcoma~\cite{raymond2002conventional}.
GO term enrichment analysis confirmed these results.
These pathways may explain why it is comparatively
easy to culture fibroblastic osteosarcoma cells, which
also may explain why the percentage of fibroblastic
osteosarcoma is relatively high in our cell line dataset
(31\%, compared to 11\% in the biopsy dataset). Next to
this link to cellular growth and proliferation, the most
significant network with fibroblastic\hyp{}specific upregulated
genes showed a connection to the immune system. GO
analysis of the five most significant GO terms pointed
to involvement of the immune system as well (GO term
GO:0006955, p-value $=3.9\cdot10^{-3}$, see Additional File 3 ({\it available online}~\cite{ch3additional}) for GO
term subgraphs). Forty\hyp{}four genes in this GO term were
significant, of which 43 were upregulated in fibroblastic
osteosarcoma. An elevated immune response might be
the reason why patients with fibroblastic osteosarcoma
tend to have better survival profiles, as a proinflammatory
environment has an important role in osteosarcoma
metastasis\hyp{}free survival. This profile is different from the
previously found macrophage\hyp{}specific profile which was
associated with better metastasis\hyp{}free survival of osteosarcoma
patients~\cite{buddingh2011tumor}. The overrepresentation of pathways
involved in chondrogenesis in the chondroblastic
profile is in line with the high amount of chondroid
matrix in this subtype. We only detected one osteoblastic\hyp{}specific gene, {\it UNQ1940}, or {\it FAM180A}, with a yet
unknown function. Since 50 osteoblastic osteosarcoma
samples were compared with only 9 chondroblastic and
7 fibroblastic osteosarcoma samples, we suggest that
fibroblastic and chondroblastic osteosarcoma have specific
characteristics that distinguishes these tumors from
osteoblastic osteosarcoma, and that the latter does not
have such an extra feature in comparison with chondro- and
fibroblastic osteosarcoma.

Our histological subtype prediction profile consists of
24 probes encoding for 22 genes, all with a specific
score which depends on the significance of each gene.
The genes that make up the chondroblastic\hyp{}specific part
of this expression profile are mostly chondroid matrixassociated
genes, such as {\it ACAN}, {\it COL2A1}, and {\it MATN4},
and are all upregulated in chondroblastic osteosarcoma.
Fibroblastic\hyp{}specific genes that make up the prediction
profile are up- or downregulated. An example of a gene
upregulated in fibroblastic osteosarcoma is {\it NFE2L3}, a
transcription factor which heterodimerizes with small
musculoaponeurotic fibrosarcoma factors and for which
a protective role was suggested in hematopoietic malignancies~\cite{chevillard2011nfe2l3}. {\it DLX5}, a transcription factor involved in
bone formation, is downregulated in fibroblastic osteosarcoma,
and reflects the lower amounts of matrix present
in fibroblastic osteosarcoma. No known function is
yet available for the two osteoblastic\hyp{}specific genes. The
misclassification error of the prediction profile in the
training set of biopsies was 15\%. Cell lines and xenografts
were predicted with misclassification errors of
15\% and of 11\%, respectively. It seems most likely that
these prediction errors are inherent to the error rate of
the prediction profile, which is also 15\%. Thus, because
these misclassification errors are in the same range, we
suggest that gene expression of these model systems is
highly similar to gene expression of the tumor from
which they are derived. This is especially of interest for
studies in cell lines, since no stroma is present on which
subtyping can be performed, but repeatedly passaged
xenografts often lose stroma as well. Most genes of the
prediction profile are matrix\hyp{}associated genes. Even
though these cell lines do not secrete any matrix, and
xenografts have diminished amounts of matrix, we can
still detect histological subtype markers on an mRNA
level, and are able to distinguish different histological
subtypes of cell lines and xenografts using this profile.

\section{Conclusions}\label{conclusions3}
As osteosarcoma xenografts and cell lines still express
histological subtype\hyp{}specific mRNAs that are characteristic
of the primary tumor, these model systems are
representative for the primary tumor from which they
are derived, and will be useful tools for studying mRNA
expression and pathways important in high\hyp{}grade
osteosarcoma.

%%% references

\begin{small}
\begin{singlespace}
\bibliographystyle{unsrtnatshort}		% sorted as referenced, was unsrtnat, but unsrtnatshort gives shorter output
\bibliography{biblio}
\end{singlespace}
\end{small}

%\end{document}