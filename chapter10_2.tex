% Marieke Kuijjer
% 2013-02-15
% chapter 09

	%\documentclass[12pt,b5paper]{book}
	%\setcounter{secnumdepth}{0}
	%\setcounter{tocdepth}{1}
	%\usepackage[hidelinks]{hyperref}

	\fancyhead[LO]{\bfseries\ Chapter 10}		% instead of Chapter \thechapter
	\fancyhead[RE]{\bfseries\ Chapter 10}		% instead of Chapter \thechapter
 
%\begin{document}


%%% title page

%%% main document
\chapter{Nederlandse samenvatting}\label{ch10}
\thispagestyle{empty}				%%% to remove page number from first page of chapter, must be placed after calling the chapter

\selectlanguage{dutch} % for dutch hyphenation
Het hooggradig osteosarcoom is een maligne primaire bottumor, welke met name bij adolescenten en jonge volwassenen voorkomt op de plaats waar tijdens de pubertijd snelle botgroei plaatsvindt. Het is een zeer agressieve tumor, welke in 45\% van de pati\"enten uitzaait, meestal naar de longen. De 5-jaars overleving van osteosarcoom pati\"enten is ongeveer 60--70\%. Behandeling van het hooggradig osteosarcoom bestaat uit chemotherapie en operatieve verwijdering van de tumor. Een gerichte behandeling tegen specifiek osteosarcoom cellen, zoals dit bijvoorbeeld bestaat in de vorm van tamoxifen tegen oestrogeenreceptor\hyp{}positieve borsttumoren, bestaat niet. Osteosarcoom tumor cellen hebben vele en complexe afwijkingen in het DNA. In dit proefschrift is zogenaamde high\hyp{}throughput moleculaire data analyse gebruikt, om genoomwijd het osteosarcoom op verschillende niveaus, zoals DNA en mRNA, te kunnen bestuderen, met als doel meer over deze tumor te weten te komen en eventuele gerichte behandelingen tegen het osteosarcoom te kunnen identificeren. In de inleidende hoofdstukken van dit proefschrift, {\bf hoofdstuk~\ref{ch1}} en {\bf hoofdstuk~\ref{ch2}}, worden de verschillende microarray technieken---SNP-, genexpressie- en kinoomprofilering---besproken die gebruikt zijn in dit proefschrift, en wordt in een literatuuronderzoek een samenvatting gegeven van de tot nu toe gepubliceerde studies waarin dit soort technieken gebruikt zijn om het osteosarcoom te bestuderen.

{\bf Hoofdstuk~\ref{ch3}} betreft mRNA expressie data analyse van pre\hyp{}operatieve osteosarcoom biopten en modellen van het osteosarcoom, zoals cellijnen en diermodellen. Dit hoofdstuk beschrijft differenti\"ele expressie tussen osteosarcoom biopten van verschillende groepen pati\"enten, bijvoorbeeld geslacht, de locatie van de primaire tumor in het lichaam van de pati\"ent, de reactie op de pre\hyp{}operatieve chemotherapie en het histologische subtype van de tumor. Een opmerkelijke bevinding is dat er geen verschil in expressie bestaat tussen biopten van pati\"enten met een goede of slechte reactie op pre\hyp{}operatieve chemotherapie. In tegenstelling tot een aantal publicaties waarin wel verschillen in genexpressie beschreven worden, is in onze studie gecorrigeerd voor herhaaldelijk testen---een statistische methode die toegepast moet worden als meerdere hypotheses worden getest, zoals het geval is bij het analyseren van microarray data. De meest voorkomende histologische subtypes van het conventionele osteosarcoom---osteoblastisch, chondroblastisch en fibroblastisch osteosarcoom---vertoonden onderling verschillende genexpressieprofielen. Het expressieprofiel van het fibroblastaire osteosarcoom was verrijkt met genen die een rol spelen bij groei en proliferatie, terwijl het profiel specifiek voor het chondroblastaire osteosarcoom verrijkt was met genen die een rol spelen bij de chondro\"ide extracellulaire matrix van deze tumorcellen. Met behulp van een classificatiemethode werd een profiel van 24 probes (die met bepaalde genen corresponderen) ontwikkeld, welke het histologische subtype van de pre\hyp{}operatieve biopten kon bepalen. Dit profiel werd vervolgens toegepast op genexpressie data verkregen uit osteosarcoomcellen en diermodellen en kon de histologische subtypes van de originele tumor waaruit deze modellen waren ontstaan correct classificeren. Deze modellen hebben minder of geen extracellulaire matrix, de eigenschap waarop de verschillende histologische subtypes van het osteosarcoom van elkaar onderscheiden worden. Dit impliceert dat genexpressieprofielen van deze osteosarcoommodellen nog steeds representatief zijn voor de primaire tumor waaruit deze ontwikkeld zijn, en is daarom een beweegreden om deze modellen te gebruiken in onderzoek naar het osteosarcoom indien er niet genoeg primair materiaal beschikbaar is.

\section{Gerichte therapie\"en tegen het hooggradig osteosarcoom}\label{gerichte10}
Met behulp van verschillende analyses van verscheidene datasets zijn een aantal specifieke manieren ontdekt om deze tumor te bestrijden. In {\bf hoofdstuk~\ref{ch4}} zijn genexpressieprofielen van twee groepen pati\"enten met elkaar vergeleken---pati\"enten welke binnen 5 jaar uitzaaiingen ontwikkelden en pati\"enten bij wie in een periode van 5 jaar geen uitzaaiingen gevonden werden. De lijst van genen die significant verschilden in expressie was verrijkt met macrofaag\hyp{}geassocieerde genen, welke door tumor infiltrerende cellen tot expressie gebracht werden. Deze genen vertoonden allen overexpressie in de pati\"entgroep zonder metastasen. Het totale aantal tumor\hyp{}geassocieerde macrofagen van type M1 (anti\hyp{}tumor) en type M2 (pro\hyp{}tumor) associeerde in aanvullende cohorten met een betere prognose. De resultaten van hoofdstuk~\ref{ch4} verschaffen een reden om osteosarcoompati\"enten naast chemotherapie tevens met macrofaag\hyp{}activerende/aantrekkende middelen te behandelen. Een voorbeeld hiervan is het medicijn liposomal muramyl tripeptide phosphatidylethanolamine (L-MTP-PE), wat in een fase III trial de prognose van pati\"enten met osteosarcoom kon verbeteren.

Naast het testen van verschillen in genexpressie tussen groepen van hooggradig osteosarcoom biopten met verschillende klinische karakteristieken, is genexpressie van het osteosarcoom ook vergeleken met controles. In {\bf hoofdstuk~\ref{ch5}} zijn verschillen in genexpressie bepaald tussen osteosarcoomcellijnen en voorlopercellen van het osteosarcoom. Een globale pathway analyse wees op verschillen in de IGF1R signaaltransductieroute, welke een rol speelt bij botgroei. Negatieve regulatoren van de IGF1 receptor waren sterk downgereguleerd in de osteosarcoomcellen, wat zou kunnen duiden op een verhoogde activiteit van deze signaalstransductieroute. Deze route werd daarom vervolgens in osteosarcoomcellen ge\"inhibeerd met kinaseremmer OSI-906, welke niet alleen IGF1R, maar ook de insuline receptor kan remmen, wat noodzakelijk is om resistentie tegen remming van IGF1R tegen te gaan. Inhibitie met OSI-906 resulteerde in verlaagde proliferatie in drie van de vier geteste osteosarcoomcellijnen. OSI-906 zou derhalve een veelbelovend geneesmiddel kunnen zijn voor de behandeling van het osteosarcoom, naast de gebruikelijke chemotherapie.

In {\bf hoofdstuk~\ref{ch6}} wordt Serine/Threonine kinoomprofilering van twee osteosarcoomcellijnen beschreven. Voor deze studie is een peptide microarray gebruikt, welke peptides bevat die gefosforyleerd kunnen worden door kinases die aanwezig zijn in de cellysaten. Met behulp van een pathway analyse werd ontdekt dat de PI3K/Akt signaaltransductieroute verrijkt was met hyperfosforylering van moleculen direct downstream van Akt. Deze signaaltransductieroute speelt een rol bij celdeling en celgroei en kan geremd worden met verschillende medicijnen. Osteosarcoomcellijnen werden behandeld met de Akt remmer MK-2206, welke proliferatie van twee van de drie geteste cellen kon remmen. Inhibitie van de PI3K/Akt signaaltransductieroute is dan ook een mogelijke manier om deze tumor gericht te kunnen behandelen.

\section{Genomische instabiliteit}\label{genomische10}
Een tweede bevinding van dit proefschrift betreft genomische instabiliteit, welke zijn stempel drukt op verschillende niveaus---DNA, mRNA en kinaseactiviteit---in de tumorcel. De ge\"integreerde analyse beschreven in {\bf hoofdstuk~\ref{ch6}}, toegepast op signaaltransductieroutes van verschillende types data, laat zien dat signaaltransductieroutes die een rol spelen in het behouden van genomische stabiliteit verrijkt zijn in zowel overexpressie van genen als hyperfosforylering van peptiden. 
Ge\"integreerde analyse kan ook uitgevoerd worden op genniveau, door naar genen te kijken die aangedaan zijn in de te bestuderen data types. Deze methode kan gebruikt worden bij het combineren van DNA en genexpressie data, omdat de hoeveelheid DNA een directe invloed heeft op de expressie van het betreffende gen. In {\bf hoofdstuk~\ref{ch7}} worden twee methoden voor ge\"integreerde analyse besproken---de gepaarde en de ongepaarde ge\"integreerde analyse---en toegepast op data van osteosarcoom biopten en cellijnen. In de ongepaarde analyse zijn alle genen meegenomen die significant differenti\"ele expressie vertonen ten opzichte van de controle celculturen. Significant differentieel tot expressie komende genen die zich bevinden in gebieden met genomische veranderingen, welke een hoger aantal van zulke genen bevatten dan verwacht, zijn hier kandidaatgenen. In de gepaarde analyse wordt van alle genen die significant differentieel tot expressie komen bepaald of deze in hetzelfde weefsel of in dezelfde cellijn tevens in een gebied van amplificatie of deletie liggen. Dit zijn vervolgens kandidaat kanker\hyp{}drijvende genen. Met gepaarde analyse werden meer kandidaatgenen gedetecteerd in de bestudeerde osteosarcoomdataset dan met de ongepaarde analyse. Een conservatieve aanpak waarin alleen kandidaatgenen meegenomen werden die zowel in biopten als cellijnen, en vergeleken met verschillende controle celculturen gevonden waren, resulteerde in een lijst van 31 osteosarcoom kandidaatgenen, welke in minstens 35\% van alle osteosarcomen aangedaan waren. Het overgrote deel van deze kandidaatgenen was voorheen nog niet ontdekt in osteosarcoma, en meer dan twee derde van de genen speelt een mogelijke rol in kanker. Een groot aantal van deze kandidaatgenen is belangrijk voor reguleren van bijvoorbeeld de celcyclus, wat aangeeft dat het verlies van genomische stabiliteit een rol speelt in het osteosarcoom. Deze bevinding werd verder ge\"evalueerd door het berekenen van bepaalde genomische instabiliteit scores, waaruit bleek dat hogere genomische instabiliteit gecorreleerd was met slechtere prognose. Tevens werd een negatieve correlatie tussen het totale aantal DNA kopieveranderingen en prognose waargenomen.

Tenslotte is in {\bf hoofdstuk~\ref{ch8}} loss\hyp{}of\hyp{}heterozygosity (LOH) data ge\"integreerd met DNA kopie en genexpressie data. LOH gaat veelal gepaard met DNA amplificatie. Zulke gebieden zouden tumor suppressor genen kunnen bevatten, welke door bijvoorbeeld een dominant negatieve mutatie voordeel kunnen hebben van de amplificatie. Door hoge amplificatie kan echter vals positieve LOH gedetecteerd worden, vandaar dat LOH voor een aantal tumor suppressor genen gevalideerd werd met behulp van Sanger sequencing en fluorescence in situ hybridization (FISH). Sanger sequencing detecteerde geen heterozygositeit in deze genen, maar deze methode is wellicht \'{o}\'{o}k niet sensitief genoeg om heterozygositeit te kunnen detecteren in het geval van een hoge amplificatie van \'{e}\'{e}n allel. Met behulp van FISH werden zowel hoge amplificatie als lage amplificatieniveaus in verschillende samples weergegeven, welke overeenkwamen met de SNP data. LOH in lage amplificatieniveaus zijn waarschijnlijk geen artifact, en tumor suppressorgenen die in deze gebieden liggen zouden interessante mutatie kunnen bevatten. Mutatieanalyse van een aantal tumor suppressorgenen wees echter niet op mutaties in deze genen. Zogenaamde homozygous staining regions (HSRs) werden met behulp van de FISH techniek gedetecteerd in osteosarcoomcellen met hoge amplificatieniveaus. Deze hoog geamplificeerde gebieden zijn wellicht geassocieerd met chromothripsis (het uiteenvallen van chromosomen) en bevatten mogelijk oncogenen waartegen gerichte, tumor\hyp{}specifieke therapie\"en gebruikt zouden kunnen worden.

\section{Conclusie}\label{conclusie10}
In dit proefschrift is `high\hyp{}throughput' data analyse beschreven van een relatief grote microarray dataset bestaande uit osteosarcoom pre\hyp{}operatieve biopten, cellijnen en xenotransplantaten, welke beschikbaar werd gesteld door het FP6 netwerk EuroBoNeT. Met behulp van data analyse is gevonden dat macrofaag\hyp{}activerende middelen en IGF1R en Akt remmers mogelijke adjuvante therapie\"en kunnen zijn in de behandeling van het osteosarcoom. Ge\"integreerde genomische analyses wezen op de genomische complexiteit van deze tumor, en de rol van genomische instabiliteit ten opzichte van agressiviteit van het osteosarcoom. Een conservatieve benadering om zogenaamde passagiergenen weg te filteren, zodat genen die een belangrijke rol spelen bij tumorgenese gedetecteerd kunnen worden resulteerde in een lijst van 31 kandidaatgenen die frequent in het hooggradig osteosarcoom aangedaan zijn op DNA en expressieniveau, waaronder meerdere genen die een rol spelen bij het behouden van genomische stabiliteit. Met dit proefschrift zijn de eerste stappen voor het in kaart brengen van het genoom- en transcriptoomlandschap van het hooggradig osteosarcoom in gang gezet. Deze informatie is onmisbaar voor het begrijpen van deze zeer genomisch instabiele tumor en voor het ontwikkelen van diagnostische/prognostische methoden en nieuwe therapie\"en.
\selectlanguage{english} % rest of the document: english
%
%
\makeatletter\@openrightfalse %%% OBS! this causes rest of the chapters to start immediately (overrides openright to openany)
%
\newpage
\renewcommand{\chaptername}{}		% so that the chapter will not get a number
\renewcommand{\thechapter}{}			% so that the chapter will not get a number
\chapter{Curriculum Vitae}
\thispagestyle{empty}				%%% to remove page number from first page of chapter, must be placed after calling the chapter
Marieke Lydia Kuijjer was born on February 5, 1982, in Zaanstad, the Netherlands. She attended preuniversity school (gymnasium) at the Sint Adelbert College in Wassenaar. After graduating with honors in 2000, she enrolled in the Bachelor's programme Mathematics and Statistics at Leiden University. She switched to Biomedical Sciences in 2003, and received her Propaedeutics with distinction. During the Bachelor's phase of her studies, Marieke followed an Erasmus exchange program at Karolinska Institutet in Stockholm, Sweden. For her Bachelor's thesis, she studied protein dynamics using confocal microscopy under the supervision of C.R. Jost, PhD, at the department of Molecular Cell Biology, Leiden University Medical Center. During her Master's programme, she received two scholarships to follow Master's courses at Karolinska Institutet. As a trainee at the department of Human Genetics, Leiden University Medical Center, she integrated microarray data of a panel of cancer cell lines under the supervision of J.M. Boer, PhD. For her Master's thesis, she studied Wnt signaling in bone formation at the department of Molecular Cell Biology of the same institute, under the supervision of D.J. de Gorter, PhD, and P. ten Dijke, PhD. Marieke received her Master of Science degree in December 2008 and started her PhD education in the same month. The results obtained during her PhD education are described in this thesis. In May 2013, Marieke started as a postdoctoral fellow in the laboratory of J. Quackenbush, PhD, at the department of Biostatistics and Computational Biology of the Dana-Farber Cancer Institute, Boston (MA), USA.
\newpage
%
%
\renewcommand{\chaptername}{}		% so that the chapter will not get a number
\renewcommand{\thechapter}{}			% so that the chapter will not get a number
\chapter{List of publications}
\thispagestyle{empty}				%%% to remove page number from first page of chapter, must be placed after calling the chapter
\begin{itemize} % to make a bulleted list
	\item Pahl JHW, Santos SJ, \underline{Kuijjer ML}, Boerman GH, Sand LGL, Szuhai K, Cleton-Jansen AM, Egeler RM, Bov{\'e}e JVGM, Schilham MW, Lankester AC. Expression of the immune regulation antigen CD70 in osteosarcoma. {\it Submitted}
	\item Buddingh EP, Ruslan SEN, Reijnders CMA, \underline{Kuijjer ML}, Roelofs H, Hogendoorn PCW, Egeler RM,  Cleton-Jansen AM, Lankester AC. Mesenchymal stromal cells derived from healthy donors and osteosarcoma patients do not transform during long\hyp{}term culture. {\it Submitted}
	\item Kansara E, Leong HS, Lin DM, Popkiss S, Pang P, Garsed DW, Walkley CR, Cullinane C, Ellul J, Haynes NM, Hicks R, \underline{Kuijjer ML}, Cleton-Jansen AM, Hinds PW, Smyth MJ, Thomas DM. Senescence\hyp{}related immunologic functions of the RB1 tumor suppressor in radiation\hyp{}induced osteosarcoma. {\it Submitted}
	\item \underline{Kuijjer ML}, van den Akker BEWM, Hilhorst R, Mommersteeg M, Buddingh EP, Serra M, B{\"u}rger H, Hogendoorn PCW, Cleton-Jansen AM. Kinome and mRNA expression profiling of high\hyp{}grade osteosarcoma identifies genomic instability, and reveals Akt as potential target for treatment. {\it Submitted}
	\item de Vos van Steenwijk PJ, Ramwadhdoebe TH, Goedemans R, Doorduijn E, van den Ham JJ, Gorter A, van Hall T, \underline{Kuijjer ML}, van Poelgeest MIE, van der Burg SH, Jordanova ES. Tumor infiltrating CD14 positive myeloid cells work side by side with T cells to prolong the survival in patients with cervical carcinoma. Accepted for publication in {\it International Journal of Cancer}
	\item \underline{Kuijjer ML}, Peterse EFP, van den Akker BEWM, Briaire-de Bruijn IH, Serra M, Meza-Zepeda LA, Myklebost O, Hassan AB, Hogendoorn PCW, Cleton-Jansen AM. IR/IGF1R signaling as potential target for treatment of high\hyp{}grade osteosarcoma. Accepted for publication in {\it BMC Cancer}
	\item \underline{Kuijjer ML}, Cleton-Jansen AM, Hogendoorn PCW. Genome\hyp{}wide analyses on high\hyp{}grade osteosarcoma; making sense of a most genomically unstable tumor. \emph{Review}. {\it International Journal of Cancer}. 2013 Feb 20;Epub
	\item Naml{\o}s HM, Meza-Zepeda LA, Bar{\o}y T, {\O}stensen IHG, Kresse SH, \underline{Kuijjer ML}, Serra M, B{\"u}rger H, Cleton-Jansen AM, Myklebost O. Modulation of the Osteosarcoma Expression Phenotype by MicroRNAs. {\it PloS One}. 2012;7(10):e48086
	\item Mohseny AB, Cai Y, \underline{Kuijjer ML}, Xiao W, van den Akker B, de Andrea CE, Jacobs R, ten Dijke P, Hogendoorn PCW, Cleton-Jansen AM. The activities of Smad and Gli mediated signalling pathways in high\hyp{}grade conventional osteosarcoma. {\it European Journal of Cancer}. 2012 Dec;48(18):3429--3438
	\item Lenos K, Grawenda AM, Lodder K, \underline{Kuijjer ML}, Teunisse AF, Repapi E, Grochola LF, Bartel F, Hogendoorn PCW, Wuerl P, Taubert H, Cleton-Jansen AM, Bond GL, Jochemsen AG. Alternate splicing of the p53 inhibitor HDMX offers a superior prognostic biomarker than p53 mutation in human cancer. {\it Cancer Research}. 2012 Aug 15;72(16):4074--84
	\item \underline{Kuijjer ML}, Rydbeck H, Kresse SH, Buddingh EP, Lid AB, Roelofs H, B{\"u}rger H, Myklebost O, Hogendoorn PCW, Meza-Zepeda LA, Cleton-Jansen AM. Identification of osteosarcoma driver genes by integrative analysis of copy number and gene expression data. {\it Genes, Chromosomes and Cancer}. 2012 Jul;51(7):696--706
	\item Pansuriya TC, van Eijk R, d'Adamo P, van Ruler MA, \underline{Kuijjer ML}, Oosting J, Cleton-Jansen AM, van Oosterwijk JG, Verbeke SL, Meijer D, van Wezel T, Nord KH, Sangiorgi L, Toker B, Liegl-Atzwanger B, San-Julian M, Sciot R, Limaye N, Kindblom LG, Daugaard S, Godfraind C, Boon LM, Vikkula M, Kurek KC, Szuhai K, French PJ, Bov{\'e}e JVGM. Somatic mosaic IDH1 and IDH2 mutations are associated with enchondroma and spindle cell hemangioma in Ollier disease and Maffucci syndrome. {\it Nature Genetics}. 2011 Nov 6;43(12):1256--1261
	\item \underline{Kuijjer ML}, Naml{\o}s HM, Hauben EI, Machado I, Kresse SH, Serra M, Llombart-Bosch A, Hogendoorn PCW, Meza-Zepeda LA, Myklebost O, Cleton-Jansen AM. mRNA expression profiles of primary high\hyp{}grade central osteosarcoma are preserved in cell lines and xenografts. {\it BMC Medical Genomics}. 2011 Sep 20;4:66
	\item Buddingh EP{\small $\dagger$}, \underline{Kuijjer ML}{\small $\dagger$}, Duim RA, B{\"u}rger H, Agelopoulos K, Myklebost O, Serra M, Mertens F, Hogendoorn PCW, Lankester AC, Cleton-Jansen AM. Tumor\hyp{}infiltrating macrophages are associated with metastasis suppression in high\hyp{}grade osteosarcoma: a rationale for treatment with macrophage activating agents. {\it Clinical Cancer Research}. 2011 Apr 15;17(8):2110--2119
\begin{small}
$\dagger$Shared first authorship
\end{small}
\end{itemize}
%
%
\newpage
\renewcommand{\chaptername}{}		% so that the chapter will not get a number
\renewcommand{\thechapter}{}			% so that the chapter will not get a number
\chapter{Acknowledgments}
It is a great pleasure to give respect to those who made this thesis possible. I owe sincere gratitude to my promotor, prof. dr. P.C.W. Hogendoorn, and to my copromotor, dr. A.M. Cleton-Jansen. Pancras, I enjoyed working in your group, which has been an excellent environment to start my scientific career. Anne-Marie, your enthusiasm and knowledge were always stimulating, and I thank you for giving me freedom to find my own research direction. All colleagues from the department of Pathology are acknowledged, in particular the technicians, who provided me with excellent technical support. I am indebted to my roommates for their patience and support and especially to Cathelijn, Jolieke, Sara, and Wei for their friendship. I dearly thank Lisanne Vijfhuizen, Frauke Liebelt, and Elleke Peterse; I enjoyed to work with you and wish you all the best with your careers. Special thanks to all EuroBoNeT partners and to PamGene for the good collaborations. An honorable mention goes to my family and friends for their understanding and support. Finally, I thank Alessandro Marin for supporting me in every possible way, and for believing in me. {\it Ti amo}.
\thispagestyle{empty}				%%% to remove page number from first page of chapter, must be placed after calling the chapter


%%%%% OBS!! you need to have chapter 10 in the fancyhdr